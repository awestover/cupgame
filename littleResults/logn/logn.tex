\documentclass{article}[11pt]
% \usepackage[subtle]{savetrees}
\usepackage[left=1in, right=1in, top=1in, bottom=1in]{geometry}

\usepackage{amsthm}
\usepackage{amssymb}
\usepackage{amsmath}
\usepackage{mathtools}

\usepackage{fancyhdr}
\pagestyle{fancy}
\lhead{Alek Westover}
\rhead{}
\usepackage{hyperref}

\DeclareMathOperator{\E}{\mathbb{E}}
\DeclareMathOperator{\Var}{\text{Var}}
\DeclareMathOperator{\img}{Im}
\DeclareMathOperator{\polylog}{\text{polylog}}
\DeclareMathOperator{\st}{\text{ such that }}
\newcommand{\norm}[1]{\left\lVert#1\right\rVert}
\newcommand{\interior}[1]{%
  {\kern0pt#1}^{\mathrm{o}}%
}

%** VECTOR NOTATION
\newcommand{\mb}{\mathbf}
\newcommand{\x}{\mathbf{x}}
\newcommand{\y}{\mathbf{y}}
\newcommand{\z}{\mathbf{z}}
\newcommand{\f}{\mathbf{f}}
\newcommand{\n}{\mathbf{n}}
\newcommand{\p}{\mathbf{p}}
\renewcommand{\k}{\mathbf{k}}
\renewcommand{\d}{\mathrm{d}} %straight d for integrals
\newcommand{\De}{\Delta}
\renewcommand{\Re}{\mathrm{Re}}
\renewcommand{\Im}{\mathrm{Im}}
\newcommand{\ran}{\mathrm{ran}}

%** SETS
\newcommand{\set}[1]{\mathbb{#1}}
\newcommand{\curly}[1]{\mathcal{#1}}
\newcommand{\goth}[1]{\mathfrak{#1}}
\newcommand{\setof}[2]{\left\{ #1\; : \;#2 \right\}}
\newcommand{\cc}{\subseteq\subseteq}
\newcommand{\R}{\set{R}}
\newcommand{\C}{\set{C}}
\newcommand{\Z}{\set{Z}}
\newcommand{\D}{\curly{D}}
\renewcommand{\S}{\set{S}}
\newcommand{\T}{\set{T}}


\newcommand{\contr}[0]{\[ \Rightarrow\!\Leftarrow \]}
\newcommand{\defeq}{\vcentcolon=}
\newcommand{\eqdef}{=\vcentcolon}

\newtheorem{fact}{Fact}
\newtheorem{definition}{Definition}
\newtheorem{remark}{Remark}
\newtheorem{proposition}{Proposition}
\newtheorem{lemma}{Lemma}
\newtheorem{corollary}{Corollary}
\newtheorem{theorem}{Theorem}

\usepackage{mdframed}
\newmdtheoremenv{q}{Question}


\author{Alek Westover}
\title{ A filling strategy that achieves backlog $\Omega(\log n)$ in the $p$-processor cup game on $n$ cups against an arbitrary emptier}

\begin{document}
% \begin{center}
% \begin{Large}
%   Alek Westover \\
%   \vspace{2mm}
%   A filling strategy that achieves backlog $\Omega(\log n)$ in the $p$-processor cup game on $n$ cups.
% \end{Large}
% \end{center}
% \thispagestyle{empty}
\maketitle

The cup game is a classic problem in computer science that models work scheduling.
In the cup game on $n$ cups, a \textbf{filler} and an \textbf{emptier} take turns adding and
removing water from the cups. On each round the filler will distribute some new
amount of water among the cups, and the emptier will remove some amount of water
from some of the cups. The filler can distribute the water however it wants (as
long as it places at most $1$ water in each cup), but the emptier has an added
``discretization constraint": it can only remove water from some cups. The
problem is to analyze how well each player can do, that is, how much water can
the filler force to be in the fullest cup, and what is the upper bound on this
fill that an appropriate emptying strategy can guarantee?

Kuszmaul previously studied a variant of this problem called the ``vanilla
multiprocessor cup-game" in which the filler distributes $p$ units of water
among the cups (with at most $1$ going to any particular cup on any particular
round), and the emptier choses $p$ cups to remove at most $1$ unit of water
from. Note that there is no resource augmentation in this variant of the game.
Kuszmaul showed that in the $p$-processor cup game on $n$ cups, a greedy
emptying strategy achieves backlog bounded above by $O(\log n)$. He also
provided a construction by which the filler could achieve backlog $\Omega(\log
(n-p))$. For $p \le n/2$ the upper-bound and lower-bound have no gap between
them (asymptotically). However, for $p > n/2$ these could potentially be
different. It was widely thought that the $O(\log n)$ upper-bound could be
reduced to $O(\log(n-p))$ for $p>n/2$. We will prove the following lemma, which
asserts that (surprisingly!) this is not the case!

\begin{lemma}
  There exists a filling strategy for the $p$-processor cup game on $n$ cups
  that achieves backlog $\Omega(\log n)$ against any emptier.
\end{lemma}
\begin{proof}
Kuszmaul's construction shows that there is a filling strategy that achieves
backlog $\Omega(\log (n-p))$.  If $p \le n - \sqrt{n}$, $n-p \ge \sqrt{n}$, so $\log
(n-p) \ge \frac{1}{2}\log (n)$, hence $\Omega(\log(n-p)) = \Omega(\log(n))$.
Thus the result holds for $p \le n - \sqrt{n}$; we proceed to consider the case
where $p > n-\sqrt{n}$

Let $H_n = 1/1+1/2+\cdots +1/n$ denote the $n$-th harmonic number.  We will
show that there is a filling strategy such that if $S_t(1) < \frac{1}{2} (H_n -
H_{n-p+1})$, then $\text{tot}(S_{t+n}) \ge \text{tot}(S_t) + \frac{1}{2}$.
Leveraging this result, the filler can repeatedly increase the total fill by a
constant amount. The filer can achieve backlog greater than $\frac{1}{2}(H_n -
H_{n-p+1})$ by repeating this strategy $n\cdot (H_n-H_{n-p+1})$ times, with the
added strategy that if the backlog ever becomes greater than
$\frac{1}{2}(H_n-H_{n-p+1})$ in this process the filler stops, and proceeds
to repeatedly put $1$ unit of water in the fullest cup, thus guaranteeing
backlog $\Omega(\log n)$ \footnote{ Note that $\frac{1}{2} (H_n - H_{n-p+1}) \ge
\Omega(\log n)$ because $n - p < \sqrt{n}$ so $H_{n-p+1} <
H_{\lceil\sqrt{n}\rceil + 1} < \frac{1}{2} H_n + 100$, hence $H_n - H_{n-p+1} > H_n -
\frac{1}{2}H_n- 100 = \frac{1}{2} H_n - 100 \ge \Omega(\log n)$.  } forever
after. If the backlog is less than $\frac{1}{2} (H_n-H_{n-p+1})$ at every step of
this process, then the total fill must have increased by at least $\frac{1}{2}
$ per each repetition of the strategy, for a total of $\frac{1}{2}n \cdot
(H_{n} - H_{n-p+1})$ increase. This however implies that at the end (denoted
$t_f$) $\text{av}(S_{t_f}) \ge \frac{1}{2} (H_n-H_{n-p+1})$, so $S_{t_f}(1) \ge
\frac{1}{2} (H_n - H_{n-p+1})$, and again we will have achieved backlog
$\Omega(\log n)$ by the end of the process. 

We proceed to outline the filling strategy that is able to increase the total
fill by at least $1/2$ in less than $n$ rounds (if the backlog starts out below
$\frac{1}{2}(H_n - H_{n-p+1})$.

Let $S$ denote the set of cups. The filler will maintain a set $U \subset S$
throughout the algorithm. The algorithm's procedure will ensure that once a cup
enters $U$ its fill never decreases for the rest of the process (the filler will
maintain the fill of these cups by placing $1$ unit of water in them each time). 
Furthermore, $|U|$ will increase by $n-p$ at each itteration of the process.
$U$ is initialized to $\emptyset$. For each of $\lfloor(n-2) / (n-p)\rfloor$ steps the filler will
\begin{itemize}
  \item Distribute $p - |U|$ water equally among the cups in $S\setminus U$ (thus, each such cup recieves $\frac{p-|U|}{n-|U|}$ fill)
  \item Distribute $|U|$ water equally among the cups in $U$ (thus, each such cup receives $1$ fill)
\end{itemize}

Then the emptier must chose $p$ cups to empty from, and hence $n-p$ cups to \textbf{neglect}.
Let $N$ be the set of neglected cups on a given step ($|N| = n-p$).
The emptier appends all cups in $N \setminus U$ to $U$, and then appends the
$(n-p) - |N\setminus U |$ fullest cups in $S \setminus U$ to $U$.  Thus the
number of elements in $U$ at the start of step $i$ (using $0$-indexing) is 
$(n-p)\cdot i$.

Now, note that the fill of any cup in $S \setminus U$ at the end of round $i$
has increased by $\frac{p-(n-p)\cdot i}{n-(n-p)\cdot i}$, and then decreased by
$1$ on this round. This is a net change of $-\frac{n-p}{n-(n-p)\cdot i}$.

At the end of these $\lfloor (n-2) / (n-p) \rfloor$ rounds $|U| \le n-2$, so
there are at least 2 cups in $S \setminus U$.
\textbf{We claim that the cups finally in $S\setminus U$ now have $0$ fill.}
On the $i$-th round of the filler's process the fill of these cups in $S
\setminus U$ has a net decrease of $\frac{n-p}{n-|U|} = \frac{n-p}{n-(n-p)\cdot i}$. 
Now consider the total amount that the fill of a cup that is in $S\setminus U$ at the end
have decreased since the start of this process:
$$\sum_{i=0}^{\lfloor(n-2) / (n-p)\rfloor - 1} \frac{n-p}{n-(n-p)\cdot i}.$$

For $p = n-1$ this sum is easily reckognizable as a difference of harmonic
numbers: $$\sum_{i=0}^{n-3}\frac{1}{n-i} = \frac{1}{n} + \frac{1}{n-1} + \cdots
+ \frac{1}{3} =  H_{n} - H_{n-p+1}.$$

We can achieve this as a lower bound for other values of $p$ too by lower
bounding a difference of ``strided harmonic numbers" with a difference of
harmonic numbers.
In particular, we can lower-bound the $i$-th term in the sum by:
$$\frac{n-p}{n-i(n-p)} = \sum_{j=0}^{n-p-1} \frac{1}{n-i(n-p)} \ge \sum_{j=n-i(n-p)}^{n-(i-1)(n-p)-1}\frac{1}{j}.$$
Adding these up we get a lower bound on the amount that the fill decreases in these cups:
$$\sum_{i=0}^{\lfloor(n-2) / (n-p)\rfloor - 1} \sum_{j=n-(i-1)(n-p)-1}^{n-i(n-p)}\frac{1}{j}.$$
This is now a difference of harmonic numbers. In particular, when $i=0,
j=n-(i-1)(n-p)-1$ we get the smallest term in the sum $1/(n+(n-p-1))$,
and when $i = \lfloor(n-2) / (n-p)\rfloor - 1, j=n-i(n-p)$ we get the largest
term in the sum. This largest term is greater than or equal to $n - ((n-2) /
(n-p)-1) (n-p) = n - (n-2 -(n-p)) = n-p+2$
Thus, the loss is lower bounded by $H_{n+(n-p-1)} - H_{n-p+1} \ge H_{n} - H_{n-p+1}$.

Because the backlog started less than $\frac{1}{2} (H_n - H_{n-p+1})$ by assumption, these cups must
now have $0$ fill.  Hence we have 2 cups with $0$ fill, so the final step of
the filler's procedure is to add $1/2$ fill to these 2 cups, then the total
fill increases by $1/2$ by necessity, as the emptier must ``waste" a unit of
fill on emptying one of these. That is, the amount of fill removed by the
emptier is at most $p-1/2$, which is less than the $p$ fill that the filler added this round. 
Note that total fill is monotonically increasing, a simple
consequence of the discretization constraint ($p$ fill is added to the system
each round, and no more than $p$ is emptier), so the fact that total fill
increases on this round implies that it increased over the whole process by at least $\frac{1}{2}$, as desired.
\end{proof}


\end{document}

