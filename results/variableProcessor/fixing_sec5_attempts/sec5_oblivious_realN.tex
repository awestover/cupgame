Now we show that we can force a constant fraction of the cups to
have high fill; using \cref{lem:obliviousManyUnknownCups} and
exploiting the greedy-like nature of the emptier we can get a
known cup with high fill (we show this in
\cref{prop:obliviousBase}).
\begin{lemma}
  \label{lem:obliviousManyUnknownCups}
  Let $\Delta \le O(1)$, let $h \le O(1)$ with $h \ge
  16+16\Delta$, let $n$ be at least a sufficiently large constant
  determined by $h$ and $\Delta$, and let $R \le \poly(n)$.
  Let $M\gg n$ be very large.
  % we will be setting $M = 2^{\log^2 (real n)}$
  % which is huge, so yes it does hurt our running time
  % but running time will still be quasi-polynomial
  Consider an $R$-flat cup configuration in the
  $M$-skip-emptyings $M$-extra-emptyings
  variable-processor cup game on $n$ cups.
  Let $A, B$ be disjoint subsets of the cups with $|AB| = n$.
  Over the course of the algorithm $B$ will give some cups to
  $A$, but $|A|$ will always satisfy $|A| \ll |B|$, and $|A|$
  will eventually be $\Theta(n)$.

  There is an oblivious filling strategy that either achieves
  mass at least $M$ in the cups, or makes an unknown
  set of $\Theta(n)$ cups in $A$ have fill at least $h +
  \mu_{\min}$, where $\mu_{\min}$ is the minimum value that the
  average fill of $AB$ attains throughout the process, with
  probability at least $1-2^{-\Omega(n)}$ in running time
  $\poly(M)$ against a $\Delta$-greedy-like emptier while also
  guaranteeing that $\mu(B) \ge -h/2 + \mu_{\min}$. Furthermore, throughout
  the filler's process the backlog never exceeds the average
  fill by more than $R_\Delta + 4h$.
\end{lemma}
\begin{proof}
  We refer to $A$ as the \defn{anchor} set, and $B$ as the
  \defn{non-anchor} set. Let $n_A = \Theta(n)$ be small enough to
  satisfy
  \begin{equation}
    \label{eq:chooseBmuchbiggerthanA}
    n_A \le (n - n_A) / (2e^{2h+1} + 1).
  \end{equation}
  The filler initializes $A$ to $\varnothing$, and $B$ to be all
  of the cups. Over the course of the algorithm $B$ will give
  away $n_A$ cups to $A$. Note that $|B| \ge n-n_A \gg n_A \ge |A|$.

We denote by \randalg the oblivious filling
strategy given by \cref{prop:obliviousTerribleProbability}. 
We denote by \flatalg the oblivious filling
strategy given by \cref{lem:greedylikeisflat}.
We say that the filler \defn{applies} a filling strategy
\genericalg to a set of cups $D \subseteq B$ if the filler uses
\genericalg on $D$ while placing $1$ unit of fill in each anchor cup. 

We now describe the filler's strategy.

The filler starts by flattening the cups, i.e. using \flatalg on
all of the cups for $\poly(M)$ rounds (setting $p=n/2$). After
this the filling strategy always places $1$ unit of water into
each anchor cup on every round. The filler performs a series of
$n_A$ \defn{donation-processes}, which are
procedures that the filler uses to get a new cup---which will
sometimes have high fill---in the anchor set. On each
donation-process the filler applies \randalg many times to
arbitrarily chosen constant-size sets $D \subset B$ with $|D| =
\ceil{e^{2h+1}}$. The number of times that the filler applies
\randalg is chosen at the start of the donation-process, chosen
uniformly at random from $[m]$ ($m = \poly(M)$ to be specified).
At the end of each donation-process, the filler does a
\defn{donation}: the filler takes the cup given by \randalg in
$B$ evicts it from $B$ and adds it to $A$. After performing a
donation the filler must increase $p$ by $1$ so that $p=|A| + 1$.
Before each application of \randalg the filler flattens $B$ by
applying \flatalg to $B$ for $\poly(M)$ rounds. 

We remark that this construction is similar to the construction
in \cref{lem:adaptiveAmplification}, but has a major difference
that substantially complicates the analysis: in the adaptive
lower bound construction the filler halts after achieving the
desired average fill in the anchor set, whereas the oblivious
filler cannot halt but rather must rely on the emptier's
greediness to guarantee that each application of \randalg has
constant probability of generating a cup with high fill.

We proceed to analyze our algorithm.

Without loss of generality we assume that the emptier does not do
this neglect the anchor set $M$ times without decreasing the
fills of an anchor cup in between the rounds on which the anchor
set is neglected; if the emptier chooses to neglect that much,
then the anchor cups will have achieved mass $M$, so the claim in
\cref{lem:obliviousManyUnknownCups} is already fulfilled. 

First note that the initial flattening makes the cups
$R_\Delta$-flat by \cref{lem:greedylikeisflat}. In particular,
note that the flattening happens in the $(n/2)$-processor
$M$-extra-emptyings $M$-skip-emptyings variable-processor cup
game on $n$ cups.

We say that a property of the cups has \defn{always} held if the
property has held since the start of the first donation-process;
i.e. from now on we only consider rounds after the initial
flattening.

We say that the emptier \defn{neglects} the anchor set on a round
if it does not empty from each anchor cup. We say that an
application of \randalg to $D\subset B$ is
\defn{non-emptier-wasted} if the emptier does not neglect the
anchor set during any round of the application of \randalg. We
define $d = \sum_{i=2}^{|D|} 1/i$ (recall that $|D| =
\ceil{e^{2h+1}}$). We say that an application of \randalg to $D$
is \defn{lucky} if it achieves backlog at least $\mu(B) -
R_\Delta + d$; note that by
\cref{prop:obliviousTerribleProbability} if we condition on an
application of \randalg where $B$ started $R_\Delta$-flat being
non-emptier-wasted then the application has at least a $1/|D|!$
chance of being lucky.

Now we prove several important bounds on fills of cups in $A$ and $B$.
\begin{clm}
  \label{clm:allflatteningsworkbyM}
  All applications of \flatalg make $B$ be $R_\Delta$-flat and
  $B$ is always $(R_\Delta + d)$-flat.
\end{clm}
\begin{proof}
  Given that the application of \flatalg immediately prior to an application
  of \randalg made $B$ be $R_\Delta$-flat, by
  \cref{prop:obliviousTerribleProbability} we have that $B$ will
  stay $(R_\Delta + d)$-flat during the application of \randalg. 
  Given that the application of \randalg immediately prior to an
  application of \flatalg resulted in $B$ being $(R_\Delta
  + d)$-flat, we have that $B$ remains $(R_\Delta + d)$-flat
  throughout the duration of the application of \flatalg by
  \cref{lem:greedylikeisflat}. Given that $B$ is $(R_\Delta +
  d)$-flat before a donation occurs $B$ is clearly still $(R_\Delta +
  d)$-flat after the donation, because the only change to $B$ during
  a donation is that a cup is removed from $B$ which cannot increase
  the fill-range of $B$.
  Note that $B$ started $R_\Delta$-flat at the beginning of the
  first donation-process before of the initial flattening of all
  the cups before the first donation-process.
  Note that if an application of \flatalg begins with $B$ being
  $(R_\Delta + d)$-flat, then by considering the flattening to
  happen in the $(|B|/2)$-processor $M$-extra-emptyings
  $M$-skip-emptyings cup game we ensure that it makes $B$ be
  $R_\Delta$-flat.
  Hence we have by induction that $B$ has always been $(R_\Delta
  + d)$-flat and that all flattening processes have made $B$ be
  $R_\Delta$-flat. 
  % start, flatalg, randalg, flatalg, randalg, ..., flatalg
  % randalg, donation, flatalg, randalg, flatalg, randalg, ..., donation,
  % ..., donation, ..., right now
\end{proof}

Now we aim to show that $\mu(B)$ is never too low, which we need
in order to establish that every non-emptier-wasted lucky
application of \randalg gets a cup with high fill. Interestingly
in order to lower bound $\mu(B)$ we first must upper bound
$\mu(B)$, which by greediness and flatness of $B$ gives an upper
bound on $\mu(A)$ which we use to get a lower bound on $\mu(B)$.

\begin{clm}
  \label{clm:muBdoesntgettoobig}
  We have always had
  $$\mu(B) \le 2 + \mu(A B).$$
\end{clm}
\begin{proof}
  There are two ways that $\mu(B)-\mu(A B)$ can increase: \\
  \textbf{Case 1:}
  The emptier could empty from $0$ cups in $B$ while emptying
  from every cup in $A$. \\
  \textbf{Case 2:}
  The filler could evict a cup with fill lower than $\mu(B)$ from
  $B$ at the end of a donation-process. \\

  Note that cases are exhaustive, in particular note that if the
  emptier skips more than $1$ emptying then $\mu(B) - \mu(AB)$
  must decrease because $|A|\approx |AB|$, in particular
  \eqref{eq:chooseBmuchbiggerthanA}, as opposed to in Case 1
  where $\mu(B) - \mu(AB)$ increases.

  In Case 1, because the emptier is $\Delta$-greedy-like,
  $$\min_{a\in A} \fil(a) > \max_{b\in B} \fil(b) - \Delta.$$
  Thus $\mu(B) \le \mu(A) + \Delta$. As $|B| \gg |A|$, in
  particular by \eqref{eq:chooseBmuchbiggerthanA}, this can be
  loosened to $\mu(B) \le 1 + \mu(A B)$.

  Consider the final round on which $B$ is skipped while $A$ is
  not skipped (or consider the first round if there is no such
  round).

  From this round onwards the only increase to $\mu(B) - \mu(A
  B)$ is due to $B$ evicting cups with fill well below $\mu(B)$.
  We can upper bound the increase of $\mu(B) - \mu(A B)$ by the
  increase of $\mu(B)$ as $\mu(A B)$ is strictly increasing.

  \todo{this is overly pessimistic, we just proved that B is
  flat. if you decrease the pessimism here, you should be able to
choose $n_A$ bigger, which is nice.}
  The cup that $B$ evicts at the end of a
  donation-process has fill at least $\mu(B) - R_\Delta -
  (|D|-1)$, as the running time of \randalg is $|D|-1$, and
  because $B$ starts $R_\Delta$-flat by
  \cref{clm:allflatteningsworkbyM}. Evicting a cup
  with fill $\mu(B) - R_\Delta - (|D| -1)$ from $B$ changes
  $\mu(B)$ by $(R_\Delta + |D| - 1) / (|B|-1)$ where $|B|$ is the
  size of $B$ before the cup is evicted from $B$. Even if this
  happens on each of the $n_A$ donation-processes $\mu(B)$ cannot
  rise higher than $n_A (R_\Delta + |D|-1) / (n-n_A)$ which by
  design in choosing $|B| \gg |A|$, as was done in
  \eqref{eq:chooseBmuchbiggerthanA}, is at most $1$.

  Thus it always is the case that $\mu(B) \le 2 + \mu(A B).$

\end{proof}

The upper bound on $\mu(B)$ along with the guarantee that $B$ is
flat allows us to bound the highest that a cup in $A$ could rise
by greediness, which in turn upper bounds $\mu(A)$ which in turn
lower bounds $\mu(B)$. In particular we have
\begin{clm}
  \label{clm:muBgreaterthanminushover2}
  We always have
  $$\mu(B) \ge -h/2 + \mu_{\min}.$$
\end{clm}
\begin{proof}
  By \cref{clm:muBdoesntgettoobig} and \cref{clm:allflatteningsworkbyM} 
  we have that no cup in $B$ ever has fill greater than
  $u_B = \mu(A B) + 2 + R_\Delta + d$. 

  Let $u_A = u_B + \Delta + 1$. We claim that the backlog in $A$
  never exceeds $u_A$.

  Consider how high the fill of a cup $c \in A$ could be.
  If $c$ came from $B$ then when it is donated
  to $A$ its fill is at most $u_B < u_A$. Otherwise, $c$
  started with fill at most $R_\Delta < u_A$. Now consider how
  much the fill of $c$ could increase while being in $A$. Because
  the emptier is $\Delta$-greedy-like, if a cup $c\in A$ has fill
  more than $\Delta$ higher than the backlog in $B$ then $c$ must
  be emptied from, so any cup with fill at least $u_B + \Delta =
  u_A - 1$ must be emptied from, and hence $u_A$ upper bounds the
  backlog in $A$. 

  Of course an upper bound on backlog in $A$ also serves as
  an upper bound on the average fill of $A$ as well, i.e.
  $\mu(A) \le u_A$. 
  Rearranging the expression 
  $$|B|\mu(B) + |A|\mu(A) = |AB|\mu(AB)$$
  we have
  \begin{align*}
    &\mu(B) \\
           &= -\frac{|A|}{|B|} \mu(A) + \frac{|A B|}{|B|}\mu(A B) \\
           &\ge -(\mu(AB) + 3+R_\Delta+d+\Delta) \frac{|A|}{|B|} + \frac{|AB|}{|B|}\mu(AB)\\
           &= -(3+R_\Delta+d + \Delta) \frac{|A|}{|B|} + \mu(AB)\\
           &\ge -h/2 + \mu(AB)
  \end{align*}
  where the final inequality follows because $\mu(AB) \ge 0$, and
  $|B|\gg |A|$, in particular by \eqref{eq:chooseBmuchbiggerthanA}.

  Of course $\mu(AB) \ge \mu_{\min}$ so we have
  $$\mu(B) \ge -h/2 + \mu_{\min}.$$

\end{proof}

Now we show that we can at least a constant fraction of the
donation-processes succeed with exponentially good probability.
\begin{clm}
  There exists choice of $m = \poly(M)$ such that with probability at least
  $1-2^{-\Omega(n)}$, the filler achieves fill at least
  $h+\mu_{\min}$ in at
  least $\Theta(n)$ of the cups in $A$. 
\end{clm}
\begin{proof}
  If the emptier was not allowed to neglect the anchor set ever
  and use extra-emptyings in $B$ then the claim would be true as
  each application of \randalg would unconditionally succeed with
  constant probability, so a Chernoff bound would give that
  $\Theta(n)$ of the donation-processes donate a cup with fill at
  least $\mu(B) - R_\Delta + d \ge h + \mu_{\min}$, where the inequality
  follows from \cref{clm:muBgreaterthanminushover2} which asserts
  that $\mu(B) \ge -h/2 + \mu_{\min}$, and from the facts $d\ge 2h$
  and $h \ge 16(1+\Delta)$.
  However, the emptier is allowed to neglect the anchor set, and
  in fact the emptier can choose to neglect the anchor set
  conditional on the filler's progress during \randalg. 

  We can lower bound the probability of getting $\Theta(n)$ cups
  with fills all at least $h + \mu_{\min}$ by considering an augmented emptier
  that is allowed to interfere with $M$ applications of \randalg
  per donation-process that only interferes with applications of
  \randalg that would otherwise donate a cup with fill $h +
  \mu_{\min}$ into $A$. 
  The optimal strategy for such an emptier, given our filler's
  strategy of randomly choosing which application to donate a cup
  on, is to simply interfere with the first $M$ applications of
  \randalg that without interference would have achieved a cup
  with fill $h$. The filler sets $m = 4M|D|!$. Conditional on
  the emptier not interfering, each of these applications of
  \randalg has at least a $1/|D|!$ chance of getting a cup with
  fill $h$. Hence, by a Chernoff bound with exponentially good
  probability at least $2M$ of the applications of \randalg have
  the potential to donate a cup with fill $h+\mu_{\min}$ to $A$, if the
  emptier does not interfere. The filler chooses an application
  uniformly at random from all $m$ applications on which to
  donate a cup. With probability at least $1/|D|!$ this is on an
  application where the filler could get a cup with fill
  $h+\mu_{\min}$ in
  $A$ if the emptier does not interfere, and with probability at
  least $1/2$ the emptier does not interfere on this application
  of \randalg, because the emptier can interfere on at most $M$
  of the applications of \randalg. 

  Against this augmented emptier whether or not 
  donation-processes achieve a cup with fill $h+\mu_{\min}$ in $A$ are
  independent events. As each happens with at least constant
  probability, by a Chernoff bound there is exponentially high
  probability that at least a constant fraction of them succeed.

  Note that we used the Chernoff bound $\Theta(n)$; by a union
  bound there is exponentially good probability that all of the
  desired events occur.

\end{proof}

We now analyze the running time of the filling strategy.
There are $|A|$ donation-processes. Each donation-process
consists of $\poly(M)$ applications of \randalg, which each take
constant time, and $\poly(M)$
applications of \flatalg, which each take $\poly(M)$ time.
Thus overall the algorithm takes $\poly(M)$ time, as desired.
  
\end{proof}

Finally, using \cref{lem:obliviousManyUnknownCups} we can show in
\cref{prop:obliviousBase} that an oblivious filler can achieve
constant backlog. We remark that \cref{prop:obliviousBase} plays a
similar role in the proof of the lower bound on backlog as
\cref{prop:adaptiveBase} does in the adaptive case, but is vastly
more complicated to prove (in particular,
\cref{prop:adaptiveBase} is trivial, whereas we have already
proved several lemmas and propositions as preparation for the
proof of \cref{prop:obliviousBase}).
\begin{proposition}
  \label{prop:obliviousBase}
  Let $H \le O(1)$, let $\Delta \le O(1)$, let $n$ be at
  least a sufficiently large constant determined by $H$ and
  $\Delta$, and let $R \le \poly(n)$. 
  Let $M \gg n$ be very large.
  Consider an $R$-flat cup configuration in the
  $M$-skip-emptyings $M$-extra-emptyings variable-processor cup
  game on $n$ cups with average fill $\mu_0$.
  Given this configuration, if the emptier does not use any
  extra-emptyings, an oblivious filler can either
  achieve mass $M$ among the cups, or achieve fill at least $\mu_0 + H$
  in a chosen cup in running time $\poly(M)$ against a
  $\Delta$-greedy-like emptier with probability at least $1-2^{-\Omega(n)}.$
  Furthermore, throughout the filling strategy the backlog never
  exceeds the average fill by more than $R_\Delta + H\cdot
  64(1+\Delta)$.
\end{proposition}
\begin{proof}
  The filler starts by performing the procedure detailed in
  \cref{lem:obliviousManyUnknownCups}, using $h = H\cdot
  16(1+\Delta)$. Because by assumption the emptier is not using
  extra-emptyings the average fill of the set of all cups is at
  least $\mu_0$ throughout the process.
  Let the number of cups which must now exist with
  fill $h+\mu_0$ be of size $nc = \Theta(n)$.

  The filler sets $p=1$, i.e. uses a single processor. Now the
  filler exploits the emptier's greedy-like nature to to get fill
  $H$ in a chosen cup $c_0$. Specifically, for $(5/8)h$ rounds
  the filler places $1$ unit of fill into $c_0$. Because the
  emptier is greedy-like it must empty from the $nc$ cups in $A$
  with fill at least $h+\mu_0$ until $c_0$ has large fill. Over
  $(5/8)h$ rounds the cups in $A$ cannot have their fill decrease
  below $(3/8)h \ge h/8 + \Delta + \mu_0$. Hence, any cups with fills
  less than $h/8+\mu_0$ must not be emptied from during these rounds.
  The fill of $c_0$ started as at least $-h/2+\mu_0$ as $\mu(B) \ge
  -h/2+\mu_0$. After $(5/8)h$ rounds $c_0$ has fill at least
  $h/8+\mu_0$,
  because the emptier cannot have emptied $c_0$ until it attained
  fill $h/8+\mu_0$, and if $c_0$ is never emptied from then it achieves
  fill $h/8+\mu_0$.

  Thus the filling strategy achieves backlog $h/8 +\mu_0 \ge H +
  \mu_0$ in $c_0$, a known cup, as desired.

  Now consider how much the backlog could exceed the average fill
  by during this process. By \cref{lem:obliviousManyUnknownCups}
  during the process that gets many unknown cups to have high
  fill the backlog never exceeds the average fill by more than
  the desired quantity. Clearly during the final steps of the
  process, where the filler is just adding fill to $c_0$ the
  emptier, being $\Delta$-greedy-like, will not let the backlog
  increase once $c_0$ has fill at least $\Delta$ above the
  average fill.

\end{proof}
Let $k_\Delta = 64(1+\Delta)$.
