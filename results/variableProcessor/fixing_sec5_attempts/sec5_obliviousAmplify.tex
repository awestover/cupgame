Next we prove the \defn{Oblivious Amplification Lemma}. The same
idea of using a function multiple times on subsets of the cups
drives both \cref{lem:obliviousAmplification} and
\cref{lem:adaptiveAmplification};
however the Oblivious Amplification Lemma is much more difficult to prove.
\begin{lemma}[Oblivious Amplification Lemma]
  \label{lem:obliviousAmplification} 
  Let $M$ be very large.
  Let $0 < \delta \ll 1/2$ be a constant
  parameter. Let $\Delta \le O(1)$, $R, R' \ge R_\Delta$. Let
  $\alg{f}$ be an oblivious filling strategy that, conditional on
  the emptier not using extra-emptyings, either achieves mass $M$
  or achieves backlog $f(n)+\mu_0$ in the $M$-skip-emptyings
  $M$-extra-emptyings variable-processor cup game
  on $n$ cups with probability at least
  $1-2^{-\Omega(n)}-1/\poly(M)$ in
  running time $T(n)$ when given a $R$-flat cup
  configuration with average fill $\mu_0$ against a $\Delta$-greedy-like emptier.
  Furthermore, let $\alg{f}$ guarantee that the cups are always
  $(R_\Delta + k_\Delta f(n))$-flat throughout this process.

  There exists an oblivious filling strategy $\alg{f'}$ that,
  conditional on the emptier not using extra-emptyings,
  either achieves mass $M$ or achieves backlog $f'(n)$
  satisfying
  $$f'(n) \ge (1-\delta) (f(\floor{(1-\delta)n})-R_\Delta) + f(\ceil{\delta n}) + \mu_0$$
   and $f'(n) \ge f(n)$, in the $M$-skip-emptyings
  $M$-extra-emptyings variable-processor cup
  game on $n$ cups with probability at least
  $1-2^{-\Omega(n)}-1/\poly(M)$
  in running time $$T'(n) \le M\cdot n\cdot
  T(\floor{(1-\delta)n}) + T(\ceil{\delta n})$$ when given a
  $R'$-flat cup configuration against a $\Delta$-greedy-like
  emptier.
  Furthermore, $\alg{f'}$ guarantees that the cups are always
  $(R_\Delta + k_\Delta f'(n))$-flat throughout this process.
\end{lemma}
\begin{proof}
  First we condition on the emptier not using extra emptying
  and show that we can achieve the desired result in this case
  (good backlog with good probability);
  at the end of the proof we will consider the case where the
  emptier does use extra emptyings, and bound the fill-range in
  that case. 

  The algorithm defaults to using $\alg{f}$ on all the cups if 
  $$f(n) \ge (1-\delta) (f(\floor{(1-\delta)n})-R_\Delta) +
  f(\ceil{\delta n}).$$
  In this case our strategy trivially results in the desired
  backlog in the desired running time. In the rest of the proof we
  consider the case where we cannot simply fall back on $\alg{f}$
  to achieve the desired backlog.

  We refer to $A$ as the \defn{anchor} set and $B$ as the
  \defn{non-anchor} set. Let $n_A = \ceil{\delta n}$, $n_B =
  \floor{(1-\delta)n}$. The filler initializes $A$ to
  $\varnothing$, and $B$ to be all of the cups. Over the course
  of $\alg{f'}$ $B$ will donate $n_A$ cups to $B$; note that 
  we always have $|B| \ge n_B$, $|A| \le n_A$ with equality
  achieved after $n_A$ donations.

  We denote by \flatalg the oblivious filling strategy given in
  \cref{lem:greedylikeisflat}. We say that the filler
  \defn{applies} an algorithm \genericalg to $B$ if it uses
  \genericalg on $B$ while placing $1$ unit of fill in each
  anchor cup.

  We now describe the filler's strategy.

  At a high level the filler's strategy is as follows:\\
  \textbf{Step 1:} Using $\alg{f}$ repeatedly on $B$, achieve a
  cup with fill $\mu(B) + f(|B|)$ in $B$ and then donate this cup into $A$. \\
  \textbf{Step 2:} Use $\alg{f}$ once on $A$ to obtain a cup in
  $A$ with fill $\mu(A) + f(|A|)$.\\

  We now describe in detail how to achieve Step 1, which is
  complicated by the fact that the emptier may attempt to prevent
  the filler from achieving high fill in a cup in $B$, and
  further by the fact that the filler, being oblivious, cannot
  know if the emptier has done this. However, we show that by
  repeatedly applying $\alg{f}$ enough times, because of the
  limit on the number of extra-emptyings we can mitigate the
  emptier's power.

  The filler starts by flattening the cups, i.e. using \flatalg
  on all of the cups for $\poly(M)$ rounds (setting $p=n/2$).
  After this the filling strategy always places $1$ unit of fill
  into each anchor cup on every round. The filler performs a
  series of $n_A$ \defn{donation-processes}, which are procedures
  that the filler uses to get a new cup---which will sometimes
  have high fill---in the anchor set. On each donation-process
  the filler applies $\alg{f}$ many times to $B$. The number of
  times that the filler applies $\alg{f}$ is chosen at the start
  of the donation-process, chosen uniformly at random from $[m]$
  ($m = \poly(M)$ to be specified). At the end of each
  donation-process, the filler does a \defn{donation}: the filler
  takes the cup given by $\alg{f}$ in $B$, evicts it from $B$ and
  adds it to $A$. After performing a donation the filler must
  increase $p$ by $1$ so that $p=|A|+1$. Before each application
  of $\alg{f}$ the filler flattens $B$ by applying \flatalg to $B$
  for $\poly(M)$ rounds.

  We proceed to analyze our algorithm.

Without loss of generality we assume that the emptier does not do
this neglect the anchor set $M$ times without decreasing the
fills of an anchor cup in between the rounds on which the anchor
set is neglected; if the emptier chooses to neglect that much,
then the anchor cups will have achieved mass $M$, so the claim in
\cref{lem:obliviousManyUnknownCups} is already fulfilled. 

First note that the initial flattening makes the cups
$R_\Delta$-flat by \cref{lem:greedylikeisflat}. In particular,
note that the flattening happens in the $(n/2)$-processor
$M$-extra-emptyings $M$-skip-emptyings variable-processor cup
game on $n$ cups.

We say that a property of the cups has \defn{always} held if the
property has held since the start of the first donation-process;
i.e. from now on we only consider rounds after the initial
flattening.

We say that the emptier \defn{neglects} the anchor set on a round
if it does not empty from each anchor cup. We say that an
application of $\alg{f}$ to $B$ is
\defn{non-emptier-wasted} if the emptier does not neglect the
anchor set during any round of the application of $\alg{f}$. 

Now we prove several important bounds on fills of cups in $A$ and $B$.
\begin{clm}
  \label{clm:allflatteningsworkbyM}
  All applications of \flatalg make $B$ be $R_\Delta$-flat and
  $B$ is always $(R_\Delta + f(\ceil{(1-\delta)n)})$-flat.
\end{clm}
\begin{proof}
  Given that the application of \flatalg immediately prior to an application
  of \randalg made $B$ be $R_\Delta$-flat, by
  \cref{prop:obliviousTerribleProbability} we have that $B$ will
  stay $(R_\Delta + d)$-flat during the application of \randalg. 
  Given that the application of \randalg immediately prior to an
  application of \flatalg resulted in $B$ being $(R_\Delta
  + d)$-flat, we have that $B$ remains $(R_\Delta + d)$-flat
  throughout the duration of the application of \flatalg by
  \cref{lem:greedylikeisflat}. Given that $B$ is $(R_\Delta +
  d)$-flat before a donation occurs $B$ is clearly still $(R_\Delta +
  d)$-flat after the donation, because the only change to $B$ during
  a donation is that a cup is removed from $B$ which cannot increase
  the fill-range of $B$.
  Note that $B$ started $R_\Delta$-flat at the beginning of the
  first donation-process before of the initial flattening of all
  the cups before the first donation-process.
  Note that if an application of \flatalg begins with $B$ being
  $(R_\Delta + d)$-flat, then by considering the flattening to
  happen in the $(|B|/2)$-processor $M$-extra-emptyings
  $M$-skip-emptyings cup game we ensure that it makes $B$ be
  $R_\Delta$-flat.
  Hence we have by induction that $B$ has always been $(R_\Delta
  + d)$-flat and that all flattening processes have made $B$ be
  $R_\Delta$-flat. 
  % start, flatalg, randalg, flatalg, randalg, ..., flatalg
  % randalg, donation, flatalg, randalg, flatalg, randalg, ..., donation,
  % ..., donation, ..., right now
\end{proof}

Now we aim to show that $\mu(B)$ is never too low, which we need
in order to establish that every non-emptier-wasted lucky
application of \randalg gets a cup with high fill. Interestingly
in order to lower bound $\mu(B)$ we first must upper bound
$\mu(B)$, which by greediness and flatness of $B$ gives an upper
bound on $\mu(A)$ which we use to get a lower bound on $\mu(B)$.

\begin{clm}
  \label{clm:muBdoesntgettoobig}
  We have always had
  $$\mu(B) \le 2 + \mu(A B).$$
\end{clm}
\begin{proof}
  There are two ways that $\mu(B)-\mu(A B)$ can increase: \\
  \textbf{Case 1:}
  The emptier could empty from $0$ cups in $B$ while emptying
  from every cup in $A$. \\
  \textbf{Case 2:}
  The filler could evict a cup with fill lower than $\mu(B)$ from
  $B$ at the end of a donation-process. \\

  Note that cases are exhaustive, in particular note that if the
  emptier skips more than $1$ emptying then $\mu(B) - \mu(AB)$
  must decrease because $|A|\approx |AB|$, in particular
  \eqref{eq:chooseBmuchbiggerthanA}, as opposed to in Case 1
  where $\mu(B) - \mu(AB)$ increases.

  In Case 1, because the emptier is $\Delta$-greedy-like,
  $$\min_{a\in A} \fil(a) > \max_{b\in B} \fil(b) - \Delta.$$
  Thus $\mu(B) \le \mu(A) + \Delta$. As $|B| \gg |A|$, in
  particular by \eqref{eq:chooseBmuchbiggerthanA}, this can be
  loosened to $\mu(B) \le 1 + \mu(A B)$.

  Consider the final round on which $B$ is skipped while $A$ is
  not skipped (or consider the first round if there is no such
  round).

  From this round onwards the only increase to $\mu(B) - \mu(A
  B)$ is due to $B$ evicting cups with fill well below $\mu(B)$.
  We can upper bound the increase of $\mu(B) - \mu(A B)$ by the
  increase of $\mu(B)$ as $\mu(A B)$ is strictly increasing.

  The cup that $B$ evicts at the end of a
  donation-process has fill at least $\mu(B) - R_\Delta -
  (|D|-1)$, as the running time of \randalg is $|D|-1$, and
  because $B$ starts $R_\Delta$-flat by
  \cref{clm:allflatteningsworkbyM}. Evicting a cup
  with fill $\mu(B) - R_\Delta - (|D| -1)$ from $B$ changes
  $\mu(B)$ by $(R_\Delta + |D| - 1) / (|B|-1)$ where $|B|$ is the
  size of $B$ before the cup is evicted from $B$. Even if this
  happens on each of the $n_A$ donation-processes $\mu(B)$ cannot
  rise higher than $n_A (R_\Delta + |D|-1) / (n-n_A)$ which by
  design in choosing $|B| \gg |A|$, as was done in
  \eqref{eq:chooseBmuchbiggerthanA}, is at most $1$.

  Thus it always is the case that $\mu(B) \le 2 + \mu(A B).$

\end{proof}

The upper bound on $\mu(B)$ along with the guarantee that $B$ is
flat allows us to bound the highest that a cup in $A$ could rise
by greediness, which in turn upper bounds $\mu(A)$ which in turn
lower bounds $\mu(B)$. In particular we have
\begin{clm}
  \label{clm:muBgreaterthanminushover2}
  We always have
  $$\mu(B) \ge -h/2 + \mu_{\min}.$$
\end{clm}
\begin{proof}
  By \cref{clm:muBdoesntgettoobig} and \cref{clm:allflatteningsworkbyM} 
  we have that no cup in $B$ ever has fill greater than
  $u_B = \mu(A B) + 2 + R_\Delta + d$. 

  Let $u_A = u_B + \Delta + 1$. We claim that the backlog in $A$
  never exceeds $u_A$.

  Consider how high the fill of a cup $c \in A$ could be.
  If $c$ came from $B$ then when it is donated
  to $A$ its fill is at most $u_B < u_A$. Otherwise, $c$
  started with fill at most $R_\Delta < u_A$. Now consider how
  much the fill of $c$ could increase while being in $A$. Because
  the emptier is $\Delta$-greedy-like, if a cup $c\in A$ has fill
  more than $\Delta$ higher than the backlog in $B$ then $c$ must
  be emptied from, so any cup with fill at least $u_B + \Delta =
  u_A - 1$ must be emptied from, and hence $u_A$ upper bounds the
  backlog in $A$. 

  Of course an upper bound on backlog in $A$ also serves as
  an upper bound on the average fill of $A$ as well, i.e.
  $\mu(A) \le u_A$. 
  Rearranging the expression 
  $$|B|\mu(B) + |A|\mu(A) = |AB|\mu(AB)$$
  we have
  \begin{align*}
    &\mu(B) \\
           &= -\frac{|A|}{|B|} \mu(A) + \frac{|A B|}{|B|}\mu(A B) \\
           &\ge -(\mu(AB) + 3+R_\Delta+d+\Delta) \frac{|A|}{|B|} + \frac{|AB|}{|B|}\mu(AB)\\
           &= -(3+R_\Delta+d + \Delta) \frac{|A|}{|B|} + \mu(AB)\\
           &\ge -h/2 + \mu(AB)
  \end{align*}
  where the final inequality follows because $\mu(AB) \ge 0$, and
  $|B|\gg |A|$, in particular by \eqref{eq:chooseBmuchbiggerthanA}.

  Of course $\mu(AB) \ge \mu_{\min}$ so we have
  $$\mu(B) \ge -h/2 + \mu_{\min}.$$

\end{proof}

Now we show that we can at least a constant fraction of the
donation-processes succeed with exponentially good probability.
\begin{clm}
  There exists choice of $m = \poly(M)$ such that with probability at least
  $1-2^{-\Omega(n)}$, the filler achieves fill at least
  $h+\mu_{\min}$ in at
  least $\Theta(n)$ of the cups in $A$. 
\end{clm}
\begin{proof}
  If the emptier was not allowed to neglect the anchor set ever
  and use extra-emptyings in $B$ then the claim would be true as
  each application of \randalg would unconditionally succeed with
  constant probability, so a Chernoff bound would give that
  $\Theta(n)$ of the donation-processes donate a cup with fill at
  least $\mu(B) - R_\Delta + d \ge h + \mu_{\min}$, where the inequality
  follows from \cref{clm:muBgreaterthanminushover2} which asserts
  that $\mu(B) \ge -h/2 + \mu_{\min}$, and from the facts $d\ge 2h$
  and $h \ge 16(1+\Delta)$.
  However, the emptier is allowed to neglect the anchor set, and
  in fact the emptier can choose to neglect the anchor set
  conditional on the filler's progress during \randalg. 

  We can lower bound the probability of getting $\Theta(n)$ cups
  with fills all at least $h + \mu_{\min}$ by considering an augmented emptier
  that is allowed to interfere with $M$ applications of \randalg
  per donation-process that only interferes with applications of
  \randalg that would otherwise donate a cup with fill $h +
  \mu_{\min}$ into $A$. 
  The optimal strategy for such an emptier, given our filler's
  strategy of randomly choosing which application to donate a cup
  on, is to simply interfere with the first $M$ applications of
  \randalg that without interference would have achieved a cup
  with fill $h$. The filler sets $m = 4M|D|!$. Conditional on
  the emptier not interfering, each of these applications of
  \randalg has at least a $1/|D|!$ chance of getting a cup with
  fill $h$. Hence, by a Chernoff bound with exponentially good
  probability at least $2M$ of the applications of \randalg have
  the potential to donate a cup with fill $h+\mu_{\min}$ to $A$, if the
  emptier does not interfere. The filler chooses an application
  uniformly at random from all $m$ applications on which to
  donate a cup. With probability at least $1/|D|!$ this is on an
  application where the filler could get a cup with fill
  $h+\mu_{\min}$ in
  $A$ if the emptier does not interfere, and with probability at
  least $1/2$ the emptier does not interfere on this application
  of \randalg, because the emptier can interfere on at most $M$
  of the applications of \randalg. 

  Against this augmented emptier whether or not 
  donation-processes achieve a cup with fill $h+\mu_{\min}$ in $A$ are
  independent events. As each happens with at least constant
  probability, by a Chernoff bound there is exponentially high
  probability that at least a constant fraction of them succeed.

  Note that we used the Chernoff bound $\Theta(n)$; by a union
  bound there is exponentially good probability that all of the
  desired events occur.

\end{proof}

We now analyze the running time of the filling strategy.
There are $|A|$ donation-processes. Each donation-process
consists of $\poly(M)$ applications of \randalg, which each take
constant time, and $\poly(M)$
applications of \flatalg, which each take $\poly(M)$ time.
Thus overall the algorithm takes $\poly(M)$ time, as desired.
  
\end{proof}

Finally, using \cref{lem:obliviousManyUnknownCups} we can show in
\cref{prop:obliviousBase} that an oblivious filler can achieve
constant backlog. We remark that \cref{prop:obliviousBase} plays a
similar role in the proof of the lower bound on backlog as
\cref{prop:adaptiveBase} does in the adaptive case, but is vastly
more complicated to prove (in particular,
\cref{prop:adaptiveBase} is trivial, whereas we have already
proved several lemmas and propositions as preparation for the
proof of \cref{prop:obliviousBase}).
\begin{proposition}
  \label{prop:obliviousBase}
  Let $H \le O(1)$, let $\Delta \le O(1)$, let $n$ be at
  least a sufficiently large constant determined by $H$ and
  $\Delta$, and let $R \le \poly(n)$. 
  Let $M \gg n$ be very large.
  Consider an $R$-flat cup configuration in the
  $M$-skip-emptyings $M$-extra-emptyings variable-processor cup
  game on $n$ cups with average fill $\mu_0$.
  Given this configuration, if the emptier does not use any
  extra-emptyings, an oblivious filler can either
  achieve mass $M$ among the cups, or achieve fill at least $\mu_0 + H$
  in a chosen cup in running time $\poly(M)$ against a
  $\Delta$-greedy-like emptier with probability at least $1-2^{-\Omega(n)}.$
  Furthermore, throughout the filling strategy the backlog never
  exceeds the average fill by more than $R_\Delta + H\cdot
  64(1+\Delta)$.
\end{proposition}
\begin{proof}
  The filler starts by performing the procedure detailed in
  \cref{lem:obliviousManyUnknownCups}, using $h = H\cdot
  16(1+\Delta)$. Because by assumption the emptier is not using
  extra-emptyings the average fill of the set of all cups is at
  least $\mu_0$ throughout the process.
  Let the number of cups which must now exist with
  fill $h+\mu_0$ be of size $nc = \Theta(n)$.

  The filler sets $p=1$, i.e. uses a single processor. Now the
  filler exploits the emptier's greedy-like nature to to get fill
  $H$ in a chosen cup $c_0$. Specifically, for $(5/8)h$ rounds
  the filler places $1$ unit of fill into $c_0$. Because the
  emptier is greedy-like it must empty from the $nc$ cups in $A$
  with fill at least $h+\mu_0$ until $c_0$ has large fill. Over
  $(5/8)h$ rounds the cups in $A$ cannot have their fill decrease
  below $(3/8)h \ge h/8 + \Delta + \mu_0$. Hence, any cups with fills
  less than $h/8+\mu_0$ must not be emptied from during these rounds.
  The fill of $c_0$ started as at least $-h/2+\mu_0$ as $\mu(B) \ge
  -h/2+\mu_0$. After $(5/8)h$ rounds $c_0$ has fill at least
  $h/8+\mu_0$,
  because the emptier cannot have emptied $c_0$ until it attained
  fill $h/8+\mu_0$, and if $c_0$ is never emptied from then it achieves
  fill $h/8+\mu_0$.

  Thus the filling strategy achieves backlog $h/8 +\mu_0 \ge H +
  \mu_0$ in $c_0$, a known cup, as desired.

  Now consider how much the backlog could exceed the average fill
  by during this process. By \cref{lem:obliviousManyUnknownCups}
  during the process that gets many unknown cups to have high
  fill the backlog never exceeds the average fill by more than
  the desired quantity. Clearly during the final steps of the
  process, where the filler is just adding fill to $c_0$ the
  emptier, being $\Delta$-greedy-like, will not let the backlog
  increase once $c_0$ has fill at least $\Delta$ above the
  average fill.

\end{proof}






















  \clearpage
  \todo{
  For each cup in $A$ the filler performs a procedure called a
  \defn{swapping-process}. Let $A_0$ be initialized to
  $\varnothing$; during each swapping-process the filler will get
  some cup in $B$ to have high fill (with very good probability),
  and then swap this cup into $A$, and place the cup in $A_0$ too.
  We say that the filler \defn{applies}
  $\alg{f}$ to $B$ if it follows the filling strategy $\alg{f}$ on
  $B$ while placing $1$ unit of fill in each anchor cup; during a
  swapping-process the filler repeatedly applies $\alg{f}$ to $B$,
  flattening $B \cup (A\setminus A_0)$, which results in $B$
  being $R_\Delta$-flat as well, before each application.
  We say that the emptier \defn{neglects} the anchor set on a
  round if the emptier does not empty from every anchor cup on
  this round. The mass of the anchor set increases by at least
  $1$ each round that the anchor set is neglected. An application
  of $\alg{f}$ to $B$ is said to be \defn{successful} if $A$ is
  never neglected during the application of $\alg{f}$ to $B$. We
  say that a swapping-process is \defn{successful} if the application of
  $\alg{f}$ on which the filler swaps a cup into $A$ is a
  successful application of $\alg{f}$.

  \todo{
  Let $\mu_\Delta = 2R_\Delta + \Delta$; the emptier, being
  $\Delta$-greedy-like, cannot neglect the anchor set more than
  $n\delta\mu_\Delta$ times. Thus, by making each
  swapping-process consist of $n^{\eta}$ applications of $\alg{f}$
  to $B$ and then choosing a single application among these
  (uniformly at random) after which to swap a cup into $A$ (and
  we also place the cup in $A_0$; $A_0$ consists of all cups in
  $A$ that were swapped into $A$ from $B$), we guarantee that
  with probability at least $n\delta\mu_\Delta/n^{\eta}$ this swap
  occurs at the end of a successful application of $\alg{f}$ to $B$. 
}

  If an application of $\alg{f}$ is successful, then with
  probability at least $1-2^{-\Omega(n)}$ it generates a cup with
  fill $f(|B|) + \mu(B)$ in $B$, because equal resources were put
  into $B$ on each round while $\alg{f}$ was used, and the cup
  state started as $R_\Delta$-flat and
  hence also started as $M$-flat (as $M\ge R_\Delta$).

  Now we aim to show that $\mu(A)$ is large; we do so by showing
  that $\mu(B)$ is small (i.e. very negative). Because the
  probability of an application of $\alg{f}$ being successful is
  only $1-1/\poly(n)$, which is in particular not as good as the
  $1-2^{-\Omega(n)}$ that we will guarantee, we will not be able
  to actually assume that every such application of $\alg{f}$ is
  successful. However, (as we will show later) we can guarantee
  that at least a constant fraction $\phi$ of the
  swapping-processes are successful with
  exponentially good probability.

  The filler swaps $|A|$ cups into $B$. 
  Consider how $\mu(B \cup A\setminus A_0)$ changes when a new
  cup is swapped into $A$ and placed in $A_0$. Let the initial value
  of $\mu(B \cup A\setminus A_0)$ be $\mu_0$. Say that
  initially $|A_0| = i$ (i.e. $i$ swapping-processes have occured
  so far). If the swapping-process is successful then the swapped cup has
  fill at least $\mu_0 - R_\Delta + f(|B|)$. Hence the new
  average fill of $B \cup A\setminus A_0$ after the swap is
  $$\frac{\mu_0\cdot (n-i) - (\mu_0 - R_\Delta + f(|B|))}{n-i-1} =
  \mu_0 - \frac{f(|B|) - R_\Delta}{n-i-1}.$$
  This recurrence relation allows us to find the value of
  $\mu(B \cup A\setminus A_0) = \mu(B)$ after $|A|$ swapping
  processes (i.e. once $A\setminus A_0 = \varnothing$):
  $$\mu(B) \le -\sum_{i=1}^{|A|\phi} \frac{f(|B|)-R_\Delta}{n-i}.$$
  Now we bound $H_{n-1} - H_{n-|A|\phi-1}$ where $H_i$ is the $i$-th harmonic number.
  Using the fact that 
  $$H_n = \ln n + \gamma + 1/(2n) - 1/(12 n^2) + 1/(120 n^4) - \ldots$$
  we have,
  \begin{align*}
    &H_{n-1} - H_{n-|A|\phi-1}\\
  &\ge \ln \frac{n-1}{n-|A|\phi-1} - \frac{1}{2(n-|A|\phi-1)}\\
  &\ge \ln \frac{n}{n-|A|\phi} - \frac{1}{n}\\
  &= \ln \frac{n}{n-\ceil{\delta n}\phi} - \frac{1}{n}\\
  &\ge \ln \frac{1}{1-\delta\phi} - \frac{1}{n}\\
  &\ge \delta\phi - \frac{1}{n}.
  \end{align*}

  Hence we have, 
  \begin{equation}
    \label{eq:nastyobliviousamplificationlemmastep1backlog}
  \mu(A) \ge
  \frac{(1-\delta)}{\delta}\paren{\delta\phi-\frac{1}{n}}(f(|B|)-R_\Delta).
  \end{equation}

  % {\color{red} 
  % so we're going to go for a new amplification lemma here that looks something like 

  % $$f'(n) \ge (1-\delta)^4 f(\floor{(1-\delta)n}) + f(\ceil{\delta n}).$$
  % In order to get this we choose $\phi \ge 1-\delta$ and 
  % make sure that $n\ge \delta^2$ and that $f(|B|) \ge
  % R_\Delta/\delta$ (note: this requires getting more than $1$
  % backlog in the base case, but still constant, so it's fine).

  % The asymptotic analysis still works out; it looks basically
  % like this: $$(1-\delta)^4c((1-\delta)n)^{1-\varepsilon} + c(\delta
  % n)^{1-\varepsilon} \ge cn^{1-\varepsilon}(1-(5-\varepsilon)\delta +
  % \delta^{1-\varepsilon}) \ge cn^{1-\varepsilon}$$ for sufficiently
  % small $\delta$.

  % This is pretty much what has to happen. It's not so bad.
  % so long as $f(\floor{(1-\delta)n}) \ge R_\Delta/\delta$ and $n\ge 1/\delta^2$
  % }

  % For sake of simplicity, assume for a moment that the cups in
  % $A$ start having $0$ fill, and that the emptier never
  % neglects $A$. Then, each swapping-process results in a cup
  % with fill $\mu(B)+ f(|B|)$ being swapped from $B$ with a cup
  % in $A$ that has $0$ fill; hence here the average fill of $B$
  % decreases from $\mu(B)$ to 
  % $$\frac{|B|-1}{|B|} \mu(B) + f(|B|) / |B|.$$
  % We start with $\mu(B)=0$, and list a sequence of lower bounds for $\mu(B)$ after a few swaps into $A$:
  % $$0, -\frac{f(|B|)}{|B|}, -\frac{f(|B|)}{|B|} \left(\frac{|B|-1}{|B|} +1 \right),$$
  % $$-\frac{f(|B|)}{|B|} \left(\left(\frac{|B|-1}{|B|}\right)^2 + \frac{|B|-1}{|B|} +1 \right).$$
  % Continuing on for $|A|$ swaps we find that by the end of this process $\mu(B)$ is at most 
  % $$-\frac{f(|B|)}{|B|}\left( \frac{\left(\frac{|B|-1}{|B|}\right)^{|A|}- 1}{\frac{|B|-1}{|B|} - 1} \right) \ge -\frac{|A|}{|B|}f(|B|).$$
  % Hence every cup ever swapped into $A$ has fill at least
  % least
  % \begin{align*}
  % -\frac{|A|}{|B|}f(|B|) + f(|B|) &\\
  % &= -\frac{\ceil{\delta n}}{\floor{(1-\delta) n}} f(|B|) + f(|B|) \\
  % &\ge (1-\delta/(1-\delta)) f(|B|) \\
  % &= h.
  % \end{align*}

  % If, in which case the mass
  % transfered from $B$ to $A$ would be $\delta n f((1-\delta) n)$.
  % In order for their to be an increase in the difference of the
  % average fills of $A$ and $B$ by this amount $B$ would have had
  % to contribute $|A|/n = \delta$ of the difference, with $A$
  % contributing $|B|/n=(1-\delta)$ of the difference. Hence the
  % average fill of $A$ would have actually only increased by
  % $(1-\delta) f((1-\delta)n)$.

  Now we establish that we can guarantee that $\phi |A|$ of the
  $|A|$ swapping-process succeed for any choice of $\phi =
  \Theta(1)$ by sufficiently large choice of $\eta$, i.e. by performing
  enough applications of $\alg{f}$ within each swapping-process.
  Recall that by construction of $\mu_\Delta$ the emptier cannot
  neglect the anchor set on more than $n\delta \mu_\Delta$
  applications of $\alg{f}$ to $B$. 
  %There are $n^\eta |A| \ge n^{\eta+1}\delta$ applications of $\alg{f}$ to $B$. 

  Let $X_i$ be the random variable that indicates the event that
  the $i$-th swapping-process was not successful; note that the
  $X_i$ are independent, because the filler's random choices of
  which applications of $\alg{f}$ within each swapping-process on
  which to swap a cup into the anchor set are independent.
  We have, for any constant $\phi$,
  % {\color{red} OK, so this part is a bit messed up. It's the right idea, but as it stands it's not doing so good. Specifically, here is what's messed up with what I'm doing here: a) events aren't independent, b) emptier can force a specific swapping-process to fail with higher probability. maybe a and b are fixable by bloating $\eta$.}
  \begin{align*}
  \Pr\left[\left|\frac{1}{|A|}\sum_{i=1}^{|A|}X_i - \frac{n\delta\mu_\Delta}{n^{\eta}}\right| \ge 1-2\phi \right] 
  &\le 2e^{-2|A|(1-2\phi)^2} \\
  &\le 2^{-\Omega(n)}.
  \end{align*}
  By appropriately large choice for $\eta \le O(1)$, $$n\delta\mu_\Delta / n^\eta
  \le \phi$$ no matter how small $w \ge \Omega(1)$ is chosen. In particular this
  implies that $$\Pr\left[\sum_{i=1}^{|A|} X_i \ge |A|(1-\phi)\right] \ge 1-2^{\Omega(n)}.$$
  That is, with exponentially good probability $|A|\phi$ of the swapping processes succeed.
  Taking a union bound over all applications of $\alg{f}$ we have
  that there is exponentially good probability that all
  applications of $\alg{f}$ succeeded. Thus, with exponentially
  good probability, by \eqref{eq:nastyobliviousamplificationlemmastep1backlog}, Step 1 achieves
  backlog $$(1-\delta)(\phi-1/(\delta n)(f(\floor{(1-\delta)n}-R_\Delta)$$

  % This is essentially the backlog that we aimed to achieve in $A$, however, 
  % It is almost clear that the desired backlog is achieved; if every swapping
  % process succeeded then we would achieve fill $(1-\delta)
  % f((1-\delta)\delta^\ell n)$ in each cup in the anchor set at each level of
  % recursion hence achieving backlog $$(1-\delta)\sum_{\ell=0}^L
  % f(n\delta^\ell(1-\delta))$$ overall. However each swapping process has some
  % (very small) probability of failing; we computed probability of failure this
  % to be at most $\delta \mu_\Delta / n^\eta.$ Consider the probability that
  % more than a constant fraction $w = \Theta(1)$ of the $s = \sum_{\ell=0}^L
  % n\delta^{\ell+1}$ swapping-processes fail. Let $X_i$ be the random variable
  % indicating whether the $i$-th swapping-process succeeds (note: this is
  % swapping-processes on all levels of recursion), and let $X=\sum_{i=1}^s X_i$.
  % Clearly $\E[X] = s(1-\delta\mu_\Delta/n^\eta)$. Success of the
  % swapping-processes are not independent events: a swapping-process is in-fact
  % more likely to succeed given that previous swapping-processes have failed.
  % Hence we can upper bound the probability of more than a $w$-fraction of the
  % swapping-processes failing by a Chernoff Bound: $$\Pr\left(\frac{1}{s}X \ge
  % \frac{1}{s}\E[X] - w/2\right) \ge 1-2e^{-s w^2/2} \ge 1-2^{\Omega(n)}$$ By
  % appropriately large choice for $\eta \le O(1)$, $$\delta\mu_\Delta /n^\eta
  % \le w/2$$ no matter how small $w \ge \Omega(1)$ is chosen. In particular this
  % implies that $\Pr[X \ge s(1-w)] \ge 1-2^{\Omega(n)}$.

  % Now we will define $\phi$ such that success of $s(1-w)$ of the
  % swapping-processes guarantees backlog $$\phi \cdot (1-\delta) \sum_{\ell=0}^L
  % f(n\delta^\ell(1-\delta)).$$ In the worst case the failed swapping-processes
  % bring very negative cups into the anchor-set, potentially as negative as
  % $-\delta f((1-\delta)\delta^\ell n)$ on the $\ell$-th level of recursion.
  % However, clearly this is equivalent to removing at most $2$ cups worth of
  % mass from the anchor set. Overall we thus remove at most $2w$ cups worth of
  % mass from the anchor set. Hence choosing $\phi = 1-2w$ works.
  % Noting that the constant $w > 0$ was arbitrary we have that $\phi$ can be
  % made any constant arbitrarily close to $1$.

  % In order to achieve this backlog however, not only does the filler need to be
  % able to swap over $s(1-w)$ cups on rounds where the emptier neglects the
  % anchor set, but no applications of $f$ can fail; failure happens with
  % probability $2^{-n\delta^\ell(1-\delta) q}$ for an application of $f$ to
  % $n\delta^\ell(1-\delta)$ cups. Taking a union bound over the $\poly(n)$
  % applications of $f$ clearly still gets probability failure at most
  % $2^{-\Omega(n)}$.
}

  To achieve Step 2 the filler simply applies $\alg{f}$ to $A$.
  This clearly achieves backlog 
  $$f(|A|) = f(\ceil{\delta n})$$
  with exponentially good probability.
 
  Since both Step 1 and Step 2 succeed with exponentially good
  probability, the entire process succeeds with exponentially
  good probability.

  We now analyze the running time of $\alg{f'}$.
  The initial smoothing takes time $O(M')$. Step 1 entails
  $n^{\eta}\cdot (n\delta)$ swapping-processes, each of which
  takes time $f(|B|)$. Due to flattening at the beginning of each
  application of $\alg{f}$ the running time may be increased by a
  multiplicative factor of at most $3$. Step 2 takes time
  $T(|A|)$. Adding these times we have that the running time
  $T'(n)$ of $\alg{f'}$ is
  $$T'(n) \le O(M') + 6 \delta n^{\eta+1} T(\floor{(1-\delta)n}) + T(\ceil{\delta n}).$$

  Having proved that $\alg{f'}$ achieves the desired backlog
  with the desired probability in the desired running time, the
  proof is now complete.

\end{proof}

