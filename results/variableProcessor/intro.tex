\section{Introduction}\label{sec:intro}

A fundamental challenge in processor scheduling is how to perform
load balancing, that is, how to share processors among tasks in
order to keep any one task from getting too far behind. Consider
$n$ tasks executing in time slices on $p < n$ processors. During
each time slice, a scheduler must select $p$ (distinct) tasks
that will be executed during the slice; up to one unit of work is
completed on each executed task.
During the same time slice, however, up to $p$ units of \emph{new
work} may be allocated to the tasks, with different tasks
receiving different amounts of work. The goal of a load-balancing
scheduler is to bound the backlog of the system, which is defined
to be the maximum amount of uncompleted work for any task.

As a convention, the load balancing problem is often described as
a game involving water and cups. The \defn{$p$-processor cup
game} is a multi-round game with two players, an \defn{emptier}
and a \defn{filler}, that takes place on $n$ initially empty
cups. At the beginning of each round, the filler adds up to $p$
units of water to the cups, subject to the constraint that each
cup receives at most $1$ unit of water. The emptier then selects
up to $p$ distinct cups and removes up to $1$ unit of water from
each of them. The emptier's goal is to minimize the amount of
water in the fullest cup, also known as the \defn{backlog}. In
terms of processor scheduling, the cups represent tasks, the
water represents work assigned to each task, and the emptier
represents a scheduling algorithm.

Starting with the seminal paper of Liu \cite{Liu69}, work on the $p$ processor cup game has spanned more than five decades \cite{BaruahCoPl96,GkasieniecKl17,BaruahGe95,LitmanMo11,LitmanMo05,MoirRa99,BarNi02,GuanYi12,Liu69, LiuLa73,DietzRa91, BenderFaKu19, Kuszmaul20, AdlerBeFr03, DietzSl87, LitmanMo09}. In addition to processor scheduling \cite{BaruahCoPl96,GkasieniecKl17,BaruahGe95,LitmanMo11,LitmanMo05,MoirRa99,BarNi02,GuanYi12,Liu69, LiuLa73, AdlerBeFr03, LitmanMo09, DietzRa91}, applications include network-switch buffer management~\cite{Goldwasser10,AzarLi06,RosenblumGoTa04,Gail93}, quality of service guarantees~\cite{BaruahCoPl96,AdlerBeFr03,LitmanMo09}, and data structure deamortization \cite{AmirFaId95,DietzRa91,DietzSl87,AmirFr14,Mortensen03,GoodrichPa13,FischerGa15,Kopelowitz12,BenderDaFa20}.

The game has also been studied in many different forms. Researchers have studied the game with a fixed-filling-rate constraint \cite{BaruahCoPl96,GkasieniecKl17,BaruahGe95,LitmanMo11,LitmanMo05,MoirRa99,BarNi02,GuanYi12,Liu69, LiuLa73}, with various forms of resource augmentation \cite{BenderFaKu19, Kuszmaul20, LitmanMo09, DietzRa91}, with both oblivious and adaptive adversaries \cite{AdlerBeFr03,BaruahCoPl96,Liu69,BenderFaKu19, Kuszmaul20}, with smoothed analysis \cite{Kuszmaul20, BenderFaKu19}, with a semi-clairvoyant emptier \cite{LitmanMo09}, with competitive analysis \cite{Bar-NoyFrLa02, FleischerKo04, DamaschkeZh05}, etc.

For the plain form of the $p$-processor cup game, the greedy
emptying algorithm (i.e., always empty from the fullest cups) is
known to be asymptotically optimal \cite{AdlerBeFr03,
BenderFaKu19, Kuszmaul20}, achieving backlog $O(\log n)$. The
optimal backlog for randomized emptying algorithms remains an
open question \cite{DietzRa91, BenderFaKu19, Kuszmaul20} and is
known to be between $\Omega(\log \log n)$ and $O(\log \log n +
\log p)$ \cite{Kuszmaul20}.

\paragraph{This paper: varying resources}
Although cup games have been studied in many forms, all of the
prior work on cup games shares one common assumption: the number
$p$ of processors is fixed.

In modern computing, however, computers are often shared among
multiple applications, users, and even virtual OS's. The result
is that the amount of resources (e.g., memory, processors, cache,
etc.) available to a given application may fluctuate over time.
The problem of handling cache fluctuations has received extensive
attention in recent years (see work on cache-adaptive analysis
\cite{CA1, CA2, CA3, CA4, CA5}), but the problem of handling a
varying number of processors remains largely unstudied.

This paper introduces the \defn{variable-processor cup game}, in
which the filler is allowed to \emph{change} $p$ (the total
amount of water that the filler adds, and the emptier removes,
from the cups per round) at the beginning of each round. Note
that we do not allow the resources of the filler and emptier to
vary separately. That is, as in the standard game, we take the
value of $p$ for the filler and emptier to be identical in each
round. This restriction is crucial since, if the filler is
allowed more resources than the emptier, then the filler could
trivially achieve unbounded backlog.

A priori having variable resources offers neither player a clear
advantage. When the number $p$ of processors is fixed, the greedy
emptying algorithm (i.e., always empty from the fullest cups), is
known to achieve backlog $O(\log n)$ \cite{AdlerBeFr03,
BenderFaKu19, Kuszmaul20} regardless of the value of $p$. This
seems to suggest that, when $p$ varies, the correct backlog
should remain $O(\log n)$. In fact, when we began this project,
we hoped for a straightforward reduction between the two versions
of the game.

\paragraph{Results}
We show that the variable-processor cup game is \emph{not}
equivalent to the standard $p$-processor game. By strategically
controlling the number $p$ of processors, the filler can achieve
substantially larger backlog than would otherwise be possible.

We begin by constructing filling strategies against deterministic
emptying algorithms. We show that for any positive constant
$\epsilon$, there is a filling strategy that achieves backlog
$\Omega(n^{1 - \epsilon})$ within $2^{\polylog(n)}$ rounds.
Moreover, if we allow for $n!$ rounds, then there is a filling
strategy that achieves backlog $\Omega(n)$. In contrast, for the
$p$-processor cup game with any fixed $p$, the backlog never
exceeds $O(\log n)$.

Our lower-bound construction is asymptotically optimal. By
introducing a novel set of invariants, we deduce that the greedy
emptying algorithm never lets backlog exceed $O(n)$.

A natural question is whether \emph{randomized} emptying
algorithms can do better. In particular, when the emptier is
randomized, the filler is taken to be \defn{oblivious}, meaning
that the filler cannot see what the emptier does at each step.
Thus the emptier can potentially use randomization to obfuscate
their behavior from the filler, preventing the filler from being
able to predict the heights of cups.

When studying randomized emptying strategies, we restrict
ourselves to the class of \defn{greedy-like emptying strategies}.
This means that the emptier never chooses to empty from a cup $c$
over another cup $c'$ whose fill is more than $\omega(1)$ greater
than the fill of $c$. All of the known randomized emptying
strategies for the classic $p$-processor cup game are greedy-like
\cite{BenderFaKu19, Kuszmaul20}.

Remarkably, the power of randomization does not help the emptier
very much in the variable-processor cup game. For any constant
$\epsilon > 0$, we give an oblivious filling strategy that
achieves backlog $\Omega(n^{1 - \epsilon})$ in quazi-polynomial
time (with probability $1 - 2^{-\polylog n}$), no matter what
(possibly randomized) greedy-like strategy the emptier follows. 

Our results combine to tell a surprising story. They suggest that
the problem of varying resources poses a serious theoretical
challenge for the design and analysis of load-balancing
scheduling algorithms. There are many possible avenues for future
work. Can techniques from beyond worst-case analysis (e.g.,
smoothing, resource augmentation, etc.) be used to achieve better
bounds on backlog? What about placing restrictions on the filler
(e.g., bounds on how fast $p$ can change), allowing the emptier
to be able to be semi-clairvoyant, or making stochastic
assumptions on the filler? We believe that all of these questions
warrant further attention.

% {\color{blue}
%   TODO: add bamboo citations}

% \paragraph{Outline}
% In \cref{sec:prelims} we establish the conventions and
% notations we will use to discuss the variable-processor cup game. 

% Many of the proofs in this paper are quite complicated. In 
% \cref{sec:technical_overview} we provide proof sketches with the
% main ideas of the proofs; this is helpful for understanding the
% main ideas of the paper without all the details.

% In \cref{sec:adaptive} we provide an inductive proof of a
% lower bound on backlog with an adaptive filler.
% \cref{thm:adaptivePoly} states that a filler can achieve backlog
% $\Omega(n^{1-\varepsilon})$ for any constant $\varepsilon > 0$ in
% quasi-polynomial running time. \cref{thm:factorialTimeAlg} also provides an extremal strategy
% that achieves backlog $\Omega(n)$ in incredibly long games: it
% has $O(n!)$ running time.

% In \cref{sec:upperBound} we prove a novel set of invariants maintained
% by the greedy emptier. In particular \cref{thm:invariant} establishes
% that a greedy emptier keeps the average fill of the $k$ fullest cups at most
% $2n-k$. This implies (setting $k=1$) that a greedy emptier
% prevents backlog from exceeding $O(n)$. 

% In \cref{sec:oblivious} we prove a lower bound on backlog using an oblivious filler against a greedy-like emptyier. 
% \cref{thm:obliviousPoly} states that  filler can achieve
% backlog $\Omega(n^{1-\varepsilon})$ for any constant $\varepsilon > 0$ in
% quasi-polynomial time with probability at least $1-2^{-\polylog(n)}$.













% \paragraph{Definition and Motivation}
% The \defn{cup game} is a multi-round game in which the two players, the
% \defn{filler} and the \defn{emptier}, take turns adding and removing water
% from cups. The \defn{backlog} at a state is the fill in the
% fullest cup; the emptier tries to minimize backlog while the
% filler tries to maximize backlog.
% On each round of the classic \defn{$p$-processor cup game} on $n$
% cups, the filler first distributes $p$ units of water among
% the $n$ cups with at most $1$ unit to any particular cup (without this
% restriction the filler can trivially achieve unbounded backlog by placing all
% of its fill in a single cup every round), and then the emptier 
% removes at most $1$ unit of water from each of $p$ cups.\footnote{Note that negative
% fill is not allowed, so if the emptier empties from a cup with fill below $1$
% that cup's fill becomes $0$.} The game has been studied for \defn{adaptive}
% fillers, i.e. fillers that can observe the emptier's actions, and for
% \defn{oblivious} fillers, i.e. fillers that cannot observe the emptier's actions.

% The cup game naturally arises in the study of processor-scheduling. The
% incoming water added by the filler represents work added to the system at time
% steps. At each time step after the new work comes in, each of $p$ processors
% must be allocated to a task which they will achieve $1$ unit of progress on
% before the next time step. The assignment of processors to tasks is modeled by
% the emptier deciding which cups to empty from. The backlog of the system is the
% largest amount of work left on any given task; in the cup game the
% \defn{backlog} of the cups is the fill of the fullest cup at a given state. In
% analyzing a cup game we aim to prove upper and lower bounds on backlog.

% \paragraph{Previous Work}
% The bounds on backlog are well known for the case where $p=1$, i.e. the
% \defn{single-processor cup game}.
% In the single-processor cup game an adaptive filler can achieve backlog
% $\Omega(\log n)$ and a greedy emptier never lets backlog exceed $O(\log n)$. In
% the randomized version of the single-processor cup game, i.e. when the filler
% is oblivious, which can be interpreted as a smoothed analysis of the
% deterministic version, the emptier never lets backlog exceed $O(\log \log n)$,
% and a filler can achieve backlog $\Omega(\log\log n)$.

% Recently Kuszmaul has established bounds on the case where $p>1$, i.e. the
% \defn{multi-processor cup game} \cite{wku20}. Kuszmaul showed that a greedy
% emptier never lets backlog exceed $O(\log n)$. He also proved a lower bound of
% $\Omega(\log (n-p))$ on backlog. Recently we showed a lower bound of
% $\Omega(\log n - \log (n-p))$. Combined, these lower bounds bounds imply a
% lower bound of $\Omega(\log n)$. Kuszmaul also established an upper bound of
% $O(\log\log n + \log p)$ against oblivious fillers, and a lower bound of
% $\Omega(\log\log n)$. Tight bounds on backlog against an oblivious
% filler are not yet known for large $p$.

% \paragraph{The Variable-Processor Cup Game}
% We investigate a new variant of the classic $p$-processor cup game which we call
% the \defn{variable-processor cup game}. In the variable-processor cup game the
% filler is allowed to change $p$ (the total amount of water that the filler
% adds, and the emptier removes, from the cups per round) at the beginning of
% each round. Note that we do not allow the resources of the filler and emptier
% to vary separately; just like in the classic cup game we take the resources of
% the filler and emptier to be identical.
% This restriction is crucial; if
% the filler has more resources than the emptier, then
% the filler could trivially achieve unbounded backlog, as average fill will
% increase by at least some positive constant at each round.
% Taking the resources of the players to be identical makes the game balanced,
% and hence interesting.

% The variable-processor cup game models the natural situation
% where many users are all on a server, and the number of
% processors allocated to each user is variable as other users get
% some portion of the processors.

% A priori having variable resources offers neither player a clear
% advantage: lower values of $p$ mean that the emptier is at more
% of a discretization disadvantage but also mean that the filler
% can ``anchor" fewer   cups. \footnote{A useful part of many
%   filling algorithms is maintaining an ``anchor" set of
%   ``anchored" cups. The filler always places $1$ unit of water in
%   each anchored cup. This ensures that the fill of an anchored
% cup never decreases after it is placed in the anchor set.}
% Furthermore, at any fixed value of $p$ upper bounds have been
% proven. For instance, regardless of $p$ a greedy emptier prevents
% an adaptive filler from having backlog greater than $O(\log n)$
% \cite{wku20}. Switching between different values of $p$, all of
% which the filler cannot individually use to get backlog larger
% than $O(\log n)$ is not obviously going to help the filler
% achieve larger backlog. We hoped that the variable-processor cup
% game could be simulated in the classic multi-processor cup game,
% because the extra ability given to the filler does not seem very
% strong. 

% However, we show that attempts at simulating the variable-processor cup
% game are futile because the variable-processor cup game
% is vastly different from the classic multi-processor cup game. 

% \paragraph{Outline and Results}
% In \cref{sec:prelims} we establish the conventions and
% notations we will use to discuss the variable-processor cup game. 

% Many of the proofs in this paper are quite complicated. In 
% \cref{sec:technical_overview} we provide proof sketches with the
% main ideas of the proofs; this is helpful for understanding the
% main ideas of the paper without all the details.

% In \cref{sec:adaptive} we provide an inductive proof of a
% lower bound on backlog with an adaptive filler.
% \cref{thm:adaptivePoly} states that a filler can achieve backlog
% $\Omega(n^{1-\varepsilon})$ for any constant $\varepsilon > 0$ in
% quasi-polynomial running time. \cref{thm:factorialTimeAlg} also provides an extremal strategy
% that achieves backlog $\Omega(n)$ in incredibly long games: it
% has $O(n!)$ running time.

% In \cref{sec:upperBound} we prove a novel invariant maintained
% by the greedy emptier. In particular \cref{thm:invariant} establishes
% that a greedy emptier keeps the average fill of the $k$ fullest cups at most
% $2n-k$. In particular this implies (setting $k=1$) that a greedy emptier
% prevents backlog from exceeding $O(n)$. 

% The lower bound and upper bound agree; our analysis is tight for adaptive fillers!

% In \cref{sec:oblivious} we prove a lower bound on backlog with an oblivious filler. 
% \cref{thm:obliviousPoly} states that an oblivious filler can achieve
% backlog $\Omega(n^{1-\varepsilon})$ for any constant $\varepsilon > 0$ in
% quasi-polynomial time with probability at least $1-2^{-\polylog(n)}$.
% \cref{thm:obliviousPoly} only applies to a certain class of emptiers:
% ``greedy-like emptiers". Nonetheless, this class of emptiers is very
% interesting; it contains the emptiers that are used in upper bound proofs.
% It is shocking that randomization doesn't help the emptier in this game;
% being oblivious seems like a large disadvantage for the filler!

