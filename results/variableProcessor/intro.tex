\section{Introduction}\label{sec:intro}
\paragraph{Definition and Motivation}
The \defn{cup game} is a multi-round game in which the two players, the
\defn{filler} and the \defn{emptier}, take turns adding and removing water
from cups. The \defn{backlog} at a state is the fill in the
fullest cup; the emptier tries to minimize backlog while the
filler tries to maximize backlog.
On each round of the classic \defn{$p$-processor cup game} on $n$
cups, the filler first distributes $p$ units of water among
the $n$ cups with at most $1$ unit to any particular cup (without this
restriction the filler can trivially achieve unbounded backlog by placing all
of its fill in a single cup every round), and then the emptier 
removes at most $1$ unit of water from each of $p$ cups.\footnote{Note that negative
fill is not allowed, so if the emptier empties from a cup with fill below $1$
that cup's fill becomes $0$.} The game has been studied for \defn{adaptive}
fillers, i.e. fillers that can observe the emptier's actions, and for
\defn{oblivious} fillers, i.e. fillers that cannot observe the emptier's actions.

The cup game naturally arises in the study of processor-scheduling. The
incoming water added by the filler represents work added to the system at time
steps. At each time step after the new work comes in, each of $p$ processors
must be allocated to a task which they will achieve $1$ unit of progress on
before the next time step. The assignment of processors to tasks is modeled by
the emptier deciding which cups to empty from. The backlog of the system is the
largest amount of work left on any given task; in the cup game the
\defn{backlog} of the cups is the fill of the fullest cup at a given state. In
analyzing a cup game we aim to prove upper and lower bounds on backlog.

\paragraph{Previous Work}
The bounds on backlog are well known for the case where $p=1$, i.e. the
\defn{single-processor cup game}.
In the single-processor cup game an adaptive filler can achieve backlog
$\Omega(\log n)$ and a greedy emptier never lets backlog exceed $O(\log n)$. In
the randomized version of the single-processor cup game, i.e. when the filler
is oblivious, which can be interpreted as a smoothed analysis of the
deterministic version, the emptier never lets backlog exceed $O(\log \log n)$,
and a filler can achieve backlog $\Omega(\log\log n)$.

Recently Kuszmaul has established bounds on the case where $p>1$, i.e. the
\defn{multi-processor cup game} \cite{wku20}. Kuszmaul showed that a greedy
emptier never lets backlog exceed $O(\log n)$. He also proved a lower bound of
$\Omega(\log (n-p))$ on backlog. Recently we showed a lower bound of
$\Omega(\log n - \log (n-p))$. Combined, these lower bounds bounds imply a
lower bound of $\Omega(\log n)$. Kuszmaul also established an upper bound of
$O(\log\log n + \log p)$ against oblivious fillers, and a lower bound of
$\Omega(\log\log n)$. Tight bounds on backlog against an oblivious
filler are not yet known for large $p$.

\paragraph{The Variable-Processor Cup Game}
We investigate a new variant of the classic $p$-processor cup game which we call
the \defn{variable-processor cup game}. In the variable-processor cup game the
filler is allowed to change $p$ (the total amount of water that the filler
adds, and the emptier removes, from the cups per round) at the beginning of
each round. Note that we do not allow the resources of the filler and emptier
to vary separately; just like in the classic cup game we take the resources of
the filler and emptier to be identical.
This restriction is crucial; if
the filler has more resources than the emptier, then
the filler could trivially achieve unbounded backlog, as average fill will
increase by at least some positive constant at each round.
Taking the resources of the players to be identical makes the game balanced,
and hence interesting.

The variable-processor cup game models the natural situation
where many users are all on a server, and the number of
processors allocated to each user is variable as other users get
some portion of the processors.

A priori having variable resources offers neither player a clear advantage:
lower values of $p$ mean that the emptier is at more of a discretization
disadvantage but also mean that the filler can ``anchor" fewer   
cups. \footnote{A
useful part of many filling algorithms is maintaining an ``anchor" set of
``anchored" cups. The filler always places $1$ unit of water in each anchored
cup. This ensures that the fill of an anchored cup never decreases after it is
placed in the anchor set.} Furthermore, at any fixed value of $p$ upper bounds
have been proven. For instance, regardless of $p$ a greedy emptier prevents an
adaptive filler from having backlog greater than $O(\log n)$. Switching between
different values of $p$, all of which the filler cannot individually use to get
backlog larger than $O(\log n)$ is not obviously going to help
the filler achieve larger backlog. We hoped that the
variable-processor cup game could be simulated in the classic
multi-processor cup game, because the extra ability given to the
filler does not seem very strong. 

However, we show that attempts at simulating the variable-processor cup
game are futile because the variable-processor cup game
is vastly different from the classic multi-processor cup game. 

\paragraph{Outline and Results}
In \cref{sec:prelims} we establish the conventions and
notations we will use to discuss the variable-processor cup game. 

Many of the proofs in this paper are quite complicated. In 
\cref{sec:technical_overview} we provide proof sketches with the
main ideas of the proofs; this is helpful for understanding the
main ideas of the paper without all the details.

In \cref{sec:adaptive} we provide an inductive proof of a
lower bound on backlog with an adaptive filler.
\cref{thm:adaptivePoly} states that a filler can achieve backlog
$\Omega(n^{1-\varepsilon})$ for any constant $\varepsilon > 0$ in
quasi-polynomial running time. \cref{prop:factorialTimeAlg} also provides an extremal strategy
that achieves backlog $\Omega(n)$ in incredibly long games: it
has $O(n!)$ running time.

In \cref{sec:upperBound} we prove a novel invariant maintained
by the greedy emptier. In particular \cref{thm:invariant} establishes
that a greedy emptier keeps the average fill of the $k$ fullest cups at most
$2n-k$. In particular this implies (setting $k=1$) that a greedy emptier
prevents backlog from exceeding $O(n)$. 

The lower bound and upper bound agree; our analysis is tight for adaptive fillers!

In \cref{sec:oblivious} we prove a lower bound on backlog with an oblivious filler. 
\cref{thm:obliviousPoly} states that an oblivious filler can achieve
backlog $\Omega(n^{1-\varepsilon})$ for any constant $\varepsilon > 0$ in
quasi-polynomial time with probability at least $1-2^{-\polylog(n)}$.
\cref{thm:obliviousPoly} only applies to a certain class of emptiers:
``greedy-like emptiers". Nonetheless, this class of emptiers is very
interesting; it contains the emptiers that are used in upper bound proofs.
It is shocking that randomization doesn't help the emptier in this game;
being oblivious seems like a large disadvantage for the filler!

