Next we prove the \defn{Oblivious Amplification Lemma}.

\begin{lemma}[Oblivious Amplification Lemma]
  \label{lem:obliviousAmplification} 
  Let $\delta \in (0, 1/2)$ be a constant parameter. Let $\Delta
  \le O(1)$. Let $M$ be very large. Consider a cup configuration
  in the variable-processor cup game on $n\ll M, n >
  \Omega(1/\delta^2)$ cups with average fill $\mu_0$ that is
  $R_\Delta$-flat. Let $\alg(f)$ be an oblivious filling strategy
  that either achieves mass $M$ or with probability at least
  $1-2^{-\Omega(n)}$ achieves backlog $\mu_0 + f(n)$ on such cups
  in running time $T(n)$ against a $\Delta$-greedy-like emptier.

  Consider a cup configuration in the variable-processor cup game
  on $n\ll M, n > \Omega(1/\delta^2)$ cups with average fill
  $\mu_0$ that is $R_\Delta$-flat. There exists an oblivious
  filling strategy $\alg(f')$ that either achieves mass $M$ or
  achieves backlog $f'(n)$ satisfying $$f'(n) \ge (1-\delta)^2
  f(\floor{(1-\delta)n}) + f(\ceil{\delta n}) + \mu_0$$ and
  $f'(n) \ge f(n)$, with probability at least
  $1-2^{-\Omega(n)}-1/\poly(M)$ in running time $$T'(n) \le
  M\cdot n\cdot T(\floor{(1-\delta)n}) + T(\ceil{\delta n})$$
  against a $\Delta$-greedy-like emptier.
\end{lemma}

\begin{proof}
  We use the notation from \cref{lem:obliviousManyUnknownCups},
  and from \cref{def:rep}. To summarize, we define $n_A =
  \ceil{\delta n}, n_B = \floor{(1-\delta)n}$, we refer to $A$ as
  the anchor set and $B$ as the non-anchor set, we say that the
  filler applies $\alg(f)$ to $B$ if it uses $\alg(f)$ on $B$
  while placing $1$ fill into each cup in $A$, and we say that
  $A$ is neglected during an application of $\alg(f)$ to $B$ if
  there is some round during the application where the emptier
  does not empty from all anchor cups.

  The filler defaults to using $\alg(f)$ on all the cups if 
  $$f(n) \ge (1-\delta)^2 f(n_B) + f(n_A).$$
  In this case our strategy trivially has the desired guarantees. 
  In the rest of the proof we consider the case where we cannot
  simply fall back on $\alg(f)$ to achieve the desired backlog.

  The filler's strategy is roughly as follows:\\
  \textbf{Step 1:} Use $\rep_\delta(\alg(f))$ on all the cups;
  this will get $A$ to have high average fill.\\
  \textbf{Step 2:} Flatten $A$ using $\flatalg$, and then use
  $\alg(f)$ on $A$.

  Now we analyze Step 1, and show that by appropriately choosing
  parameters it can be made to succeed.

  Note that, exactly as in the proof of
  \cref{lem:obliviousManyUnknownCups}, the emptier cannot neglect
  the anchor set more than $M$ times per donation-process, and
  the emptier cannot skip more than $M$ emptyings, without
  causing the mass of the cups to be at least $M$; we assume the
  emptier chooses not to do this.

  We choose $m\ge M^3, m = \poly(M)$ ---recall that $[m]$ is
  the set from which we choose how many times to apply $\alg(f)$
  in a donation-process. Using the ideas from the analysis in
  \cref{clm:baseChernoffBound}, we see that by a Chernoff bound
  with exponentially good probability in $n$ many more than $m/2$
  of the applications would succeed without if the emptier did
  not interfere with them. Conditional on the final application
  being a round that would be successful without the emptier
  interfering, the emptier has at best a $M/(m/2)= 1/\poly(M)$
  chance of interfering on the correct round.
  Taking a union bound, we can say that with probability at least
  $1-1/\poly(M) - 2^{-\Omega(n)}$ all applications of $\alg(f)$
  are not non-emptier-wasted and successfully achieve a cup with
  fill at least $\mu_{t_0}(B) + f(n_B)$ where $\mu_{t_0}(B)$ refers to the
  average fill of $B$ measured at the start of the application of
  $\alg(f)$.

  Let \defn{$\skips_t$} denote the number of times that the
  emptier has skipped the anchor set by step $t$. Consider how
  $\mu(B) - \skips/n_B$ changes over the course of the donation
  processes. As noted above at the end of each donation-process
  $\mu(B)$ decreases due to $B$ donating a cup with fill at least
  $\mu(B) + f(n_B)$. In particular, if $t_0$ denotes the time
  right before a cup is donated on the $i$-th donation-process
  and $t_1$ denotes the time right after a cup is donated, then
  $\mu_{t_1}(B) = \mu_{t_0}(B) - f(n_B) / (n-i)$. Now we claim that
  $\mu(B) - \skips/n_B$ is monotonically decreasing. Clearly each
  donation decreases this quantity. If the anchor set is
  neglected then $\mu(B)$ decreases. If a skip occurs, then
  $\skips/n_B$ increases by more than $\mu(B)$ can possibly
  decrease. Let $t_*$ be the time at the end of all the
  donation-processes. We have that 
  \begin{equation}
    \label{eq:harmonic1}
    \mu_{t_*}(B) - \frac{\skips_{t_*}}{n_B} \le \mu_0 - \sum_{i=1}^{n_A}\frac{f(n_B)}{n-i}.
  \end{equation}
  By conservation of mass we have 
  \begin{equation}
    \label{eq:conservationOfMass}
    n_A\cdot \mu_{t_*}(A) + n_B\cdot \mu_{t_*}(B) = \mu_0 + \skips_{t_*}.
  \end{equation}
  We can use Inequalities \eqref{eq:conservationOfMass} and
  \eqref{eq:harmonic1} to get a lower bound on $\mu_{t_*}(A)$ as
  follows:
  \begin{equation}
    \label{eq:lowerboundingAharmonic}
    \mu_{t_*}(A) = \mu_0 + \frac{n_B}{n_A}\paren{\mu_0 +
    \frac{\skips_{t_*}}{n_B} - \mu_{t_*}(B)}.
  \end{equation}
  Now we obtain a simpler form of
  Inequality~\eqref{eq:harmonic1}. Let $H_n$ denote the $n$-th
  harmonic
  number. We desire a simpler lower bound for 
  $$\sum_{i=1}^{n_A} \frac{1}{n-i} = H_{n-1}-H_{n_B-1}.$$

  We use the well known fact that 
  \begin{equation}
    \label{eq:wellKnownLogIneq}
    \frac{1}{2(n+1)} < H_n - \ln n - \gamma < \frac{1}{2n}
  \end{equation}
  where $\gamma = \Theta(1)$ denotes the Euler-Mascheroni constant.
  Of course $H_{n-1}-H_{n_B-1} \ge H_n - H_{n_B}.$ Now using
  Inequality~\eqref{eq:wellKnownLogIneq} we have
  \begin{align*}
    H_n - H_{n_B} &> \paren{\ln n + \gamma + \frac{1}{2(n+1)}} - \paren{\ln n_B + \gamma + \frac{1}{2n_B}}\\
                  &> \ln \frac{1}{1-\delta} + \frac{1}{2}\paren{\frac{n_B-n-1}{(n+1)n_B}}\\
                  &> \delta - \Theta\paren{\frac{\delta}{(1-\delta)n}}.
  \end{align*}
  Now using this lower bound on $H_n - H_{n_B}$ in
  Inequality~\eqref{eq:lowerboundingAharmonic} we have:
  \begin{align*}
    \mu_{t_*}(A) &> \mu_0 + \frac{n_B}{n_A}\paren{\delta - \Theta\paren{\frac{\delta}{(1-\delta)n}}}f(n_B)\\
                 &= \mu_0 + \frac{\floor{(1-\delta)n}}{\ceil{\delta n}}\paren{\delta - \Theta\paren{\frac{\delta}{(1-\delta)n}}}f(n_B)\\
                 &> \mu_0 + \paren{\frac{1-\delta}{\delta} - \frac{1}{\delta^2 n}}\paren{\delta - \Theta\paren{\frac{\delta}{(1-\delta)n}}}f(n_B)\\
                 &> \mu_0 + \paren{(1-\delta) - \Theta(1/(\delta n))}f(n_B).
  \end{align*}
  Thus, by choosing $n > \Omega(1/\delta^2)$ we have 
  $$\mu_{t_*}(A) > \mu_0 + (1-\delta)^2 f(n_B).$$

We have shown that in Step 1 the filler achieves average fill
$\mu_0 + (1-\delta)f(n_B)$ in $A$ (with good probability).
Now the filler flattens $A$ and uses $\alg(f)$ on $A$.
It is clear that this is possible, and succeeds with probability
at least $1-2^{-\Omega(n)}$.
This gets a cup with fill 
$$\mu_0 + (1-\delta)^2 f(n_B) + f(n_A)$$
in $A$, as desired.

Taking a union bound over the probabilities of Step 1 and Step 2
succeeding gives the desired probability. 

The running time of Step 1 is clearly $M\cdot n\cdot
T(\floor{(1-\delta)n})$ and the running time of Step 2 is clearly
$T(\ceil{\delta n})$ summing these yields the desired upper
bound on running time.

\end{proof}

