We now give a method for transforming a filling strategy for achieving
large backlog into a filling strategy for achieving high fill in
many cups, or high average fill in a set of cups (which of these
we guarantee depends on the original filling strategy). The idea
of repeating an algorithm many times is also used in the proof of
the Adaptive Amplification Lemma; the construction is slightly more
complicated in the randomized case however, and is much harder to
analyze.

\begin{definition}
  \label{def:rep}
  {\normalfont
  Let $\alg_0$ be an oblivious filling strategy, that can get
  high fill (for some definition of high) in some cup against
  greedy-like emptiers with some probability. We construct a new
  filling strategy \defn{$\rep_\delta(\alg_0)$} ($\rep$ stands
  for \enquote{repetition}) as follows:

  Say we have some configuration of $n$ cups.
  Let $n_A = \ceil{\delta n}, n_B = \floor{(1-\delta)n}$. Let $N
  \gg n$ be large, let $M=2^{\polylog(N)}$ be a chosen parameter. 
  Initialize $A$ to $\varnothing$ and $B$ to
  being all of the cups. We call $A$ the \defn{anchor set} and
  $B$ the \defn{non-anchor set}. The filler always places $1$
  unit of fill in each anchor cup on each round. The filling
  strategy consists of $n_A$ \defn{donation-processes}, which are
  procedures that result in a cup being \defn{donated} from $B$
  to $A$ (i.e. removed from $B$ and added to $A$). At the start
  of each donation-processes the filler chooses a value $m_0$
  uniformly at random from $[M]$. We say that the filler
  \defn{applies} a filling strategy $\alg$ to $B$ if the
  filler uses $\alg$ on $B$ while placing $1$ unit of fill
  in each anchor cup. During the donation-process the filler
  applies $\alg_0$ to $B$ $m_0$ times, and flattens $B$ by
  applying $\flatalg$ to $B$ for $\Theta(N^2)$ rounds before each
  application of $\alg_0$. At the end of each donation process
  the filler takes the cup given by the final application of
  $\alg_0$ (i.e. the cup that $\alg_0$ guarantees with some
  probability against a certain class of emptiers to have a
  certain high fill), and donates this cup to $A$. 

  We give pseudocode for this algorithm in \cref{alg:rep}.
 }

\begin{algorithm}
  \caption{$\rep_\delta(\alg_0)$}
  \label{alg:rep}
  \begin{algorithmic}
    \State \textbf{Input:} $\alg_0, \delta, N, M, $ set of $n$ cups
    \State \textbf{Output:} Guarantees on the sets $A, B$ (will vary based on $\alg_0$)
    \State
    \State $n_A \gets \ceil{\delta n}, n_B \gets \floor{(1-\delta) n}$
    \State $A \gets \varnothing, B \gets$ all the cups
    \State Always place $1$ fill in each cup in $A$
    \For{$i \in [n_A]$} \Comment Donation-Processes
    \State $m_0 \gets \text{random}([M])$
      \For{$j \in [m_0]$}
        \State Apply $\flatalg$ to $B$ for $\Theta(N^2)$ rounds
        \State Apply $\alg_0$ to $B$
      \EndFor
      \State Donate the cup given by $\alg_0$ from $B$ to $A$
    \EndFor
  \end{algorithmic}
\end{algorithm}

{\normalfont
We say that the emptier \defn{neglects} the anchor set on a round
if it does not empty from each anchor cup. We say that an
application of $\alg_0$ to $B$ is \defn{non-emptier-wasted} if
the emptier does not neglect the anchor set during any round of
the application of $\alg_0$.
}
\end{definition}

We use the idea of repeating an algorithm in two different contexts.
First in \cref{prop:obliviousBase} we prove a result analogous to that of
\cref{prop:adaptiveBase}: in particular, we show that we can
achieve constant fill in a known cup by using
$\rep_{\Theta(1)}(\randalg(\Theta(1))$
which achieves, by a Chernoff bound, $\Theta(n)$ unknown cups
with constant fill, and then exploiting the emptier's greedy-like
nature to achieve constant fill in a known cup.
After doing this, we prove the \defn{Oblivious Amplification
Lemma}, a result analogous to the Adaptive
Amplification Lemma: in particular, we show how to take an
algorithm for achieving some backlog, and then achieve higher
backlog by repeating the algorithm many times.
Although these results have deterministic analogues, their proofs
are different and significantly more complex than the proofs for the
deterministic cases.

In the rest of the section our aim is to achieve backlog
$N^{1-\varepsilon}$ in $N$ cups. We will use this value $N$
within all of the following proofs. Many values implicitly depend
on $N$. Note that we implicitly consider the cups to be part of a
larger game in these results. Also note that we are happy if
mass $N^2$ is achieved in the cups, because then backlog is
always at least $N$.

Before proving \cref{prop:obliviousBase} we analyze
$\rep_{\Theta(1)}(\randalg(\Theta(1)))$ in \cref{lem:obliviousManyUnknownCups}.
\begin{lemma}
  \label{lem:obliviousManyUnknownCups} Let $\Delta \le O(1)$, let
  $h \le O(1)$ with $h \ge 16+16\Delta$, let $k =
  \ceil{e^{2h+1}}$, let $\delta = \Theta(e^{-2h})$, let $n$ be at
  least a sufficiently large constant determined by $h$ and
  $\Delta$. Consider an $R_\Delta$-flat cup configuration in the
  variable-processor cup game on $n$ cups with initial average
  fill $\mu_0$.

  When applied to a $\Delta$-greedy-like emptier
  $\rep_{\delta}(\randalg(k))$ using $M=\Theta(N^2)$ either
  achieves mass at least $N^2$ in the cups, or with probability
  at least $1-2^{-\Omega(n)}$ makes an (unknown) set of
  $\Theta(n)$ cups in $A$ have fill at least $h + \mu_0$ while
  also guaranteeing that $\mu(B) \ge -h/2 + \mu_0$, where $A,B$
  are the sets defined in \cref{def:rep}. Furthermore,
  $\rep_{\delta}(\randalg(k))$ has running time $\poly(N)$.
\end{lemma}
\begin{proof}
  We use the definitions given in \cref{def:rep}.

Without loss of generality we assume that the emptier does not
neglect the anchor set more than $N^2$ in any particular donation-process;
if the emptier chooses to neglect the anchor set this
much then the anchor cups will have achieved mass $N^2$ so
\cref{lem:obliviousManyUnknownCups} is already fulfilled. 
Similarly we assume that the emptier does not choose to skip
more than $N^2$ emptyings; doing so clearly would result in mass
at least mass $N^2$ in the cups.

As in \cref{prop:obliviousTerribleProbability}, we define $d =
\sum_{i=2}^{k} 1/i$; we remark that, because harmonic numbers
grow like $x\mapsto \ln x$, it is clear that $d=\Theta(h)$. We say that an
application of $\randalg(k)$ to $D$ is \defn{lucky} if it
achieves backlog at least $\mu_S(B) - R_\Delta + d$ where $S$
denotes the state of the cups at the start of the application of
$\randalg(k)$; note that by
\cref{prop:obliviousTerribleProbability} if we condition on an
application of $\randalg(k)$ where $B$ started $R_\Delta$-flat
being non-emptier-wasted then the application has at least a
$1/k!$ chance of being lucky.

Now we prove several important bounds satisfied by $A$ and $B$.
\begin{clm}
  \label{clm:allflatteningsworkbyM}
  All applications of $\flatalg$ make $B$ be $R_\Delta$-flat and
  $B$ is always $(R_\Delta + d)$-flat.
\end{clm}
\begin{proof}
  Given that the application of $\flatalg$ immediately prior to an application
  of $\randalg(k)$ made $B$ be $R_\Delta$-flat, by
  \cref{prop:obliviousTerribleProbability} we have that $B$ will
  stay $(R_\Delta + d)$-flat during the application of $\randalg(k)$. 
  Given that the application of $\randalg(k)$ immediately prior to an
  application of $\flatalg$ resulted in $B$ being $(R_\Delta
  + d)$-flat, we have that $B$ remains $(R_\Delta + d)$-flat
  throughout the duration of the application of $\flatalg$ by
  \cref{lem:greedylikeisflat}. Given that $B$ is $(R_\Delta +
  d)$-flat before a donation occurs $B$ is clearly still $(R_\Delta +
  d)$-flat after the donation, because the only change to $B$ during
  a donation is that a cup is removed from $B$ which cannot increase
  the fill-range of $B$.
  Note that $B$ started $R_\Delta$-flat at the beginning of the
  first donation-process.
  Note that if an application of $\flatalg$ begins with $B$ being
  $(R_\Delta + d)$-flat, then by considering the flattening to
  happen in the $(|B|/2)$-processor $N^2$-extra-emptyings
  $N^2$-skip-emptyings cup game we ensure that it makes $B$ be
  $R_\Delta$-flat.
  Hence we have by induction that $B$ has always been $(R_\Delta
  + d)$-flat and that all flattening processes have made $B$ be
  $R_\Delta$-flat. 
\end{proof}

Now we aim to show that $\mu(B)$ is never very low, which we need
in order to establish that every non-emptier-wasted lucky
application of $\randalg(k)$ gets a cup with high fill.
Interestingly, in order to lower bound $\mu(B)$ we find it
convenient to first upper bound $\mu(B)$, which by greediness and
flatness of $B$ gives an upper bound on $\mu(A)$ which we then
use to get a lower bound on $\mu(B)$.

\begin{clm}
  \label{clm:muBdoesntgettoobig}
  We have always had
  $$\mu(B) \le \mu(AB) + 2.$$
\end{clm}
\begin{proof}
  There are two ways that $\mu(B)-\mu(A B)$ can increase: \\
  \textbf{Case 1:}
  The emptier could empty from $0$ cups in $B$ while emptying
  from every cup in $A$. \\
  \textbf{Case 2:}
  The filler could evict a cup with fill lower than $\mu(B)$ from
  $B$ at the end of a donation-process. \\

  Note that cases are exhaustive, in particular note that if the
  emptier skips more than $1$ emptying then $\mu(B) - \mu(AB)$
  must decrease because $|B| > |AB|/2$, as opposed to in Case 1
  where $\mu(B) - \mu(AB)$ increases.

  In Case 1, because the emptier is $\Delta$-greedy-like,
  $$\min_{a\in A} \fil(a) > \max_{b\in B} \fil(b) - \Delta.$$
  Thus $\mu(B) \le \mu(A) + \Delta$. We can use this to get an
  upper bound on $\mu(B) - \mu(AB)$. We have, 
  \begin{align*}
    \mu(B) &= \frac{\mu(AB) |AB| - \mu(A) |A|}{|B|}\\
           &\le \frac{\mu(AB) |AB| - (\mu(B) - \Delta) |A|}{|B|}.
  \end{align*}
  Rearranging terms:
  $$\mu(B) \paren{1+\frac{|A|}{|B|}} \le \frac{\mu(AB) |AB| + \Delta |A|}{|B|}.$$
  Now, because $|A| \cdot \Delta \le n_A
  \cdot \Delta < n$ (by our choice of $\delta$ to be a very small
  constant), we have 
  $$\mu(B) \frac{|AB|}{|B|}\le \frac{\mu(AB) |AB| + n}{|B|}.$$
  Isolating $\mu(B)$ we have 
  $$\mu(B) \le \mu(AB) + 1.$$

  Consider the final round on which $B$ is skipped while $A$ is
  not skipped (or consider the first round if there is no such
  round).

  From this round onwards the only increase to $\mu(B) - \mu(A
  B)$ is due to $B$ evicting cups with fill well below $\mu(B)$.
  We can upper bound the increase of $\mu(B) - \mu(AB)$ by the
  increase of $\mu(B)$ as $\mu(AB)$ is strictly increasing.

  The cup that $B$ evicts at the end of a
  donation-process has fill at least $\mu(B) - R_\Delta -
  (k-1)$, as the running time of $\randalg(k)$ is $k-1$, and
  because $B$ starts $R_\Delta$-flat by
  \cref{clm:allflatteningsworkbyM}. Evicting a cup
  with fill $\mu(B) - R_\Delta - (k -1)$ from $B$ changes
  $\mu(B)$ by $(R_\Delta + k - 1) / (|B|-1)$ where $|B|$ is the
  size of $B$ before the cup is evicted from $B$. Even if this
  happens on each of the $n_A$ donation-processes $\mu(B)$ cannot
  rise higher than $n_A (R_\Delta + k-1) / (n-n_A)$ which by
  design in choosing $n_B\gg n_A$, as was done in
  choosing $\delta = \Theta(e^{-2h})$, is at most $1$.

  Thus $\mu(B) \le \mu(AB) + 2$ is always true.

\end{proof}

Now, the upper bound on $\mu(B)-\mu(AB)$ along with the guarantee
that $B$ is flat allows us to bound the highest that a cup in $A$
could rise by greediness, which in turn upper bounds $\mu(A)$
which in turn lower bounds $\mu(B)$. 
\begin{clm}
  \label{clm:muBgreaterthanminushover2}
  We always have
  $$\mu(B) \ge -h/2 + \mu_0.$$
\end{clm}
\begin{proof}
  By \cref{clm:muBdoesntgettoobig} and \cref{clm:allflatteningsworkbyM} 
  we have that no cup in $B$ ever has fill greater than
  $u_B = \mu(A B) + 2 + R_\Delta + d$. 
  Let $u_A = u_B + \Delta + 1$. We claim that the backlog in $A$
  never exceeds $u_A$. Note that $\mu(AB), u_A, u_B$ are
  implicitly functions of the round; $\mu(AB)$ can increase from
  $\mu_0$ if the emptier skips emptyings.

  Consider how high the fill of a cup $c \in A$ could be.
  If $c$ came from $B$ then when it is donated
  to $A$ its fill is at most $u_B$; otherwise, $c$
  started with fill at most $R_\Delta$. Both of these expressions
  are less than $u_A - 1$. Now consider how
  much the fill of $c$ could increase while being in $A$. Because
  the emptier is $\Delta$-greedy-like, if a cup $c\in A$ has fill
  more than $\Delta$ higher than the backlog in $B$ then $c$ must
  be emptied from, so any cup with fill at least $u_B + \Delta =
  u_A - 1$ must be emptied from, and hence $u_A$ upper bounds the
  backlog in $A$. 

  Of course an upper bound on backlog in $A$ also serves as
  an upper bound on the average fill of $A$ as well, i.e.
  $\mu(A) \le u_A$.  Now we have
  \begin{align*}
    &\mu(B) \\
           &= -\frac{|A|}{|B|} \mu(A) + \frac{|A B|}{|B|}\mu(A B) \\
           &\ge -(\mu(AB) + 3+R_\Delta+d+\Delta) \frac{|A|}{|B|} + \frac{|AB|}{|B|}\mu(AB)\\
           &= -(3+R_\Delta+d + \Delta) \frac{|A|}{|B|} + \mu(AB)\\
           &\ge -h/2 + \mu(AB)
  \end{align*}
  where the final inequality follows because $\mu(AB) \ge 0$, and
  $|B|\gg |A|$, in particular by our choice of $\delta = \Theta(e^{-2h})$.
  Of course $\mu(AB) \ge \mu_0$ so we have
  $$\mu(B) \ge -h/2 + \mu_0.$$

\end{proof}

Now we show that at least a constant fraction of the
donation-processes succeed with exponentially good probability.
\begin{clm}
  \label{clm:baseChernoffBound}
  By choosing $M =\Theta(N^2)$ the filler can guarantee that with
  probability at least $1-2^{-\Omega(n)}$, the filler achieves
  fill at least $h+\mu_0$ in $\Theta(n)$ of the cups in $A$. 
\end{clm}
\begin{proof}
  If the emptier was not allowed to neglect the anchor set ever
  then the claim would clearly be true as each application of
  $\randalg(k)$ would simply succeed with constant
  probability, so a Chernoff bound would give that $\Theta(n)$ of
  the donation-processes donate a cup with fill at least $\mu(B)
  - R_\Delta + d \ge h + \mu_0$, where the inequality follows
  from \cref{clm:muBgreaterthanminushover2} which asserts that
  $\mu(B) \ge -h/2 + \mu_0$, and from the facts $d\ge 2h$ and $h
  \ge 16(1+\Delta)$. 

  However, the emptier is allowed to neglect
  the anchor set, and worse, the emptier can choose to neglect
  the anchor set conditional on the filler's progress during
  $\randalg(k)$! However, by applying $\randalg(k)$ a random
  number of times, chosen from $[M]$ (where $M=\Theta(N^2)$), we
  guarantee that with exponentially good
  probability in $M$ the filler succeeds many times, in particular
  $\Theta(N^2)$ times. But since the emptier cannot neglect the
  anchor set more than $N^2$ times, by appropriately large choice
  of $M$ we can make it so that the filler succeeds at least $2N^2$
  times with exponentially good probability. Then the emptier
  would have at best a $1/2$ chance of preventing the
  donation-process from giving away a cup with fill $h+\mu_0$
  whenever one such cup is achieved. We now formalize this
  reasoning.

  We can lower bound the probability of getting $\Theta(n)$ cups
  with fills all at least $h + \mu_0$ by considering an augmented
  emptier that is allowed to \defn{interfere} with $N^2$
  applications of $\randalg(k)$ per donation-process that only
  interferes with applications of $\randalg(k)$ that would
  otherwise donate a cup with fill at least $h + \mu_0$ into $A$;
  if this (augmented) emptier interferes with an application of
  $\randalg(k)$ then the application is \defn{emptier-wasted}, i.e.
  we assume no guarantees on the fill it achieved.
  The optimal strategy for such an emptier, for the goal of
  maximizing the probability that the final round in a
  donation-process is interfered with, given our filler's
  strategy of randomly choosing how many times to apply
  $\randalg(k)$ before donating a cup, is obviously to interfere
  with the first $N^2$ applications of $\randalg(k)$ that would
  have achieved a cup with fill $h+\mu_0$ without interference. 

  Let $M = 4N^2 k!$; note that as stated previously we choose $M=\Theta(N^2)$. 
  Recall that conditional on the
  emptier not interfering, each of these applications of
  $\randalg(k)$ has at least a $1/k!$ chance of getting a cup
  with fill $h$. Hence, by a Chernoff bound with exponentially
  good probability in $M$ at least $2N^2$ of the $M$ applications of
  $\randalg(k)$ have the potential to donate a cup with fill
  $h+\mu_0$ to $A$, if the emptier does not interfere. The filler
  chooses an application uniformly at random from $[M]$ on which
  to donate a cup. With probability at least $1/k!$ this is on an
  application where the filler could get a cup with fill
  $h+\mu_0$ in $A$ if the emptier does not interfere, and with
  probability at least $1/2$ the emptier does not interfere on
  this application of $\randalg(k)$, because the emptier can
  interfere on at most $N^2$ of the applications of $\randalg(k)$. 

  Against this augmented emptier whether or not
  donation-processes achieve a cup with fill $h+\mu_0$ in $A$ are
  independent events. As each happens with at least constant
  probability, by a Chernoff bound there is exponentially high
  probability that at least a constant fraction of them succeed.

  Note that we used a Chernoff bound in two distinct places: (a)
  in guaranteeing that each donation-process consists of at least
  $2N^2$ applications of $\randalg(k)$ that would donate a cup with
  fill $\mu_0 + h$ if the emptier did not interfere, and (b) in
  guaranteeing that a constant fraction of the donation-processes
  succeed given that their successes are independent and all
  happen with constant probability. The Chernoff bound in (a) is
  actually with exponentially good probability in $M \gg n$, but
  of course also holds with exponentially good probability in
  $n$. Then we can take a union bound over $\poly(n)$ events that
  all occur with exponentially good probability in $n$, which
  gives still gives exponentially good
  probability in $n$ that all of the desired events occur.

  The described augmented emptier is clearly strictly more
  powerful than the real emptier, so the result transfers over.
\end{proof}

We now analyze the running time of the filling strategy.
There are $n_A$ donation-processes. Each donation-process
consists of $\Theta(M)$ applications of $\randalg(k)$, which each take
constant time, and $\Theta(M)$
applications of $\flatalg$, which each take $\Theta(N^2)$ time.
Thus overall the algorithm takes $n_A \cdot \Theta(M) = \poly(N)$ time, as desired.
  
\end{proof}

Now, using \cref{lem:obliviousManyUnknownCups} we show in
\cref{prop:obliviousBase} that an oblivious filler can achieve
constant backlog. 
\begin{proposition}
  \label{prop:obliviousBase}
  Let $H \le O(1)$, let $\Delta \le O(1)$, let $n\ll N$ be at least a
  sufficiently large constant determined by $H$ and $\Delta$. 
  Consider an $R_\Delta$-flat cup configuration in the variable-processor cup
  game on $n$ cups with average fill $\mu_0$.
  There is an oblivious filling strategy that either
  achieves mass $N^2$ among the cups, or achieves fill at least $\mu_0 + H$
  in a chosen cup in running time $\poly(N)$ against a
  $\Delta$-greedy-like emptier with probability at least $1-2^{-\Omega(n)}.$
\end{proposition}
\begin{proof}
  The filler starts by using $\rep_\delta(\randalg(k))$ with
  parameter settings as in \cref{lem:obliviousManyUnknownCups}
  where $h = H\cdot 16(1+\Delta)$, i.e. $k = \ceil{e^{2h+1}}$,
  $\delta = \Theta(e^{-2h})$. 
  If this results in mass $N^2$ among the cups we are done; we
  assume this is not the case for the rest of the proof.
  Let the number of cups which, with exponentially good
  probability in $n$, now exist by
  \cref{lem:obliviousManyUnknownCups} with
  fill at least $h+\mu_0$ be of size $nc = \Theta(n)$.

  The filler sets $p=1$, i.e. uses a single processor. Now the
  filler exploits the emptier's greedy-like nature to to get fill
  $H$ in a chosen cup $c_0$. Specifically, for $(5/8)h$ rounds
  the filler places $1$ unit of fill into $c_0$. Because the
  emptier is $\Delta$-greedy-like it must empty from the $nc$
  cups in $A$ with fill at least $h+\mu_0$ until $c_0$ has large
  fill. Over $(5/8)h$ rounds the cups in $A$ cannot have their
  fill decrease below $(3/8)h \ge h/8 + \Delta + \mu_0$. Hence,
  any cups with fills less than $h/8+\mu_0$ must not be emptied
  from during these rounds. The fill of $c_0$ started as at least
  $-h/2+\mu_0$ as $\mu(B) \ge -h/2+\mu_0$. After $(5/8)h$ rounds
  $c_0$ has fill at least $h/8+\mu_0$, because the emptier cannot
  have emptied $c_0$ until it attained fill $h/8+\mu_0$, and if
  $c_0$ is never emptied from then it achieves fill $h/8+\mu_0$.
  Thus the filling strategy achieves backlog $h/8 +\mu_0 \ge H +
  \mu_0$ in $c_0$, a known cup, as desired.

  The running time is of course still $\poly(N)$ by
  \cref{lem:obliviousManyUnknownCups}.

\end{proof}

