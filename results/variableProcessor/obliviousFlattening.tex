\section{Oblivious Filling Strategies}\label{sec:oblivious}
In this section we prove that, surprisingly, an oblivious filler can
achieve backlog $n^{1-\varepsilon}$ against an arbitrary
``greedy-like" emptying algorithm.

We say an emptier is \defn{$\Delta$-greedy-like} if, whenever there
are two cups with fills that differ by at least $\Delta$, the emptier
never empties from the less full cup without also emptying from the
more full cup. That is, if on some round $t$, there are cups
$c_1, c_2$ with $\fil_{I_t}(c_1) > \fil_{I_t}(c_2) + \Delta$, then a
$\Delta$-greedy-like emptier doesn't empty from $c_2$ on round $t$
unless it also empties from $c_1$ on round $t$. Note that a perfectly
greedy emptier is $0$-greedy-like. We call an emptier
\defn{greedy-like} if in the cup game on $n$ cups it is
$\Delta$-greedy-like for $\Delta \le \frac{1}{128} \log\log\log
n$. The main result of this section is an oblivious filling strategy
that achieves large backlog against any (possibly randomized)
greedy-like emptier.

% With an oblivious filler we are only able to prove lower bounds
% on backlog against greedy-like emptiers; whether or not our
% results can be extended to a more general class of emptiers is an
% interesting open question. Nonetheless, greedy-like emptiers are
% of great interest because all the known randomized algorithms for
% the cup game are greedy-like \cite{BenderFaKu19, Kuszmaul20}.

As a tool in our analysis we define a new variant of the cup game: In
the $p$-processor \defn{$E$-extra-emptyings} \defn{$S$-skip-emptyings}
negative-fill cup game on $n$ cups, the filler distributes $p$ units
of water amongst the cups, but the filler may empty from $p'$ cups for
some $p' \neq p$. In particular the emptier is allowed to do $E$ extra
emptyings and is also allowed to skip $S$ emptyings over the course of
the game. Note that the emptier still cannot empty from the same cup
twice on a single round, and also that note that a
$\Delta$-greedy-like emptier must take into account extra emptyings
and skip emptyings to determine valid moves. Further, note that the
emptier is allowed to skip extra emptyings, although skipping extra
emptyings looks the same as if the extra-emptyings had simply not been
performed.  Let the \defn{regular} cup game be the $0$-extra-emptyings
$\infty$-skip-emptyings cup game: this is the cup game we usually
consider.
Allowing for some extra emptyings, and bounding the number of skip
emptyings is sometimes necessary when analyzing an algorithm that is a
subroutine of a larger algorithm however, hence it sometimes makes
sense to consider games with different values of $E,S$. Unless
explicitly stated otherwise however we are considering the regular cup
game.

The \defn{fill-range} of a set of cups at a state $S$ is $\max_c
\fil_S(c) - \min_c \fil_S(c)$. We call a cup configuration
\defn{$R$-flat} if the fill-range of the cups less than or equal to
$R$; note that in an $R$-flat cup configuration with average fill
$0$ all cups have fills in $[-R, R]$. 

For a $\Delta$-greedy-like emptier let $R_\Delta = 2(2+\Delta)$;
we now prove a key property of these emptiers: there is an
oblivious filling strategy, which we term \defn{$\flatalg$}, that
attains an $R_\Delta$-flat cup configuration against a
$\Delta$-greedy-like emptier, given cups of a known starting
fill-range.

% interesting(?) note:
% if you're willing to say "screw running time" (which we are, so
% long as it is still $2^{\polylog n}$, then we can definitely
% flatten, even if we don't know the cups are $R$-flat for some
% $R$, because the cups must be $M$-flat (i.e. 2^{\polylog(real
% n)}-flat or we're just done)

\begin{lemma}
  \label{lem:flatalg}
  Consider an $R$-flat cup configuration in the $p$-processor
  $E$-extra-emptyings $S$-skip-emptyings negative-fill cup game on $n =
  2p$ cups. There is an oblivious filling strategy
  \defn{$\flatalg$} that achieves an $R_\Delta$-flat
  configuration of cups against a $\Delta$-greedy-like emptier in
  running time $O(R+E+S)$. Furthermore,
  $\flatalg$ guarantees that the cup configuration is $R$-flat on every round.
\end{lemma}
\begin{proof}
  If $R \le R_\Delta$ the algorithm does nothing, since the
  desired fill-range is already achieved; for
  the rest of the proof we consider $R > R_\Delta$.

  The filler's strategy is to distribute fill equally amongst all
  cups at every round, placing $p/n = 1/2$ fill in each cup. 
  Let $\ell_t = \min_{c\in S_t} \fil_{S_t}(c)$, $u_t = \max_{c\in S_t} \fil_{S_t}(c)$. 

  First we show that the fill-range of the cups can only increase
  if the fill-range is very small.
  \begin{clm}
    \label{clm:fillRangeUpIFFfillRangeSmall}
    If the fill-range on round $t+1$ is larger than the
    fill-range on round $t$, then $u_{t+1} - \ell_{t+1} \le R_\Delta.$
  \end{clm}
  \begin{proof}
    First we remark that the fill of any cup changes by at most
    $1/2$ from round to round, and in particular $|u_{t+1}-u_t|
    \le 1/2$, $|\ell_{t+1} - \ell_t|\le 1/2$.
    In order for the fill-range to increase, the emptier must have
    emptied from some cup with fill in $[\ell_t, \ell_t + 1]$ without
    emptying from some cup with fill in $[u_t-1, u_t]$; if the emptier
    had not emptied from every cup with fill in $[\ell_t, \ell_t+1]$
    then we would have $\ell_{t+1} = \ell_t + 1/2$ so the
    fill-range cannot have increased, and similarly if the emptier
    had emptied from every cup with fill in $[u_t-1, u_t]$ then we
    would have $u_{t+1} = u_t - 1/2$ so again the fill-range 
    cannot have increased. Because the emptier is $\Delta$-greedy-like
    emptying from a cup with fill at most $\ell_t+1$ and not
    emptying from a cup with fill at least $u_t-1$ implies that
    $u_t-1$ and $\ell_t+1$ differ by at most $\Delta$.
    Thus, 
    $$u_{t+1} - \ell_{t+1} \le u_t +1/2 - (\ell_t -1/2)  \le \Delta + 3 \le R_\Delta.$$
  \end{proof}

  \cref{clm:fillRangeUpIFFfillRangeSmall} implies that if the
  fill-range is at most $R_\Delta$ it will stay at most
  $R_\Delta$, because fill-range cannot increase to a value
  larger than $R_\Delta$. \cref{clm:fillRangeUpIFFfillRangeSmall}
  also implies that until the fill-range is less than $R_\Delta$
  the fill-range must not increase. However the claim does not
  preclude fill-range from remaining constant for many rounds, or
  decreasing, but only by very small amounts. For this reason we
  actually do not use \cref{clm:fillRangeUpIFFfillRangeSmall} in
  the remainder of the proof; nonetheless, the fact that the
  fill-range cannot increase relative to initial fill-range
  during $\flatalg$ is an important property of $\flatalg$. In
  the rest of the proof we establish that the fill-range indeed
  must eventually be at most $R_\Delta$.

  Let $L_t$ be the set of cups $c$ with $\fil_{S_t}(c) \le \ell_t+2+\Delta$, and let
  $U_t$ be the set of cups $c$ with $\fil_{S_t}(c) \ge u_t-2-\Delta$.
  We now prove a key property of the sets $U_t$ and $L_t$: if a cup is in
  $U_t$ or $L_t$ it is also in $U_{t'}, L_{t'}$ for all $t' > t$. This
  follows immediately from \cref{clm:dontlosestuff}.
  \begin{clm}
    \label{clm:dontlosestuff}
    $U_{t} \subseteq U_{t+1}, L_t \subseteq L_{t+1}.$
  \end{clm}
  \begin{proof}
    Consider a cup $c\in U_t$.

    If $c$ is not emptied from, i.e. $\fil(c)$ has increased by
    $1/2$ from the previous round, then
    clearly $c \in U_{t+1}$, because backlog has increased by at most $1/2$, so
    $\fil(c)$ must still be within $2+\Delta$ of the backlog on round $t+1$. 

    On the other hand, if $c$ is emptied from, i.e. $\fil(c)$ has decreased by
    $1/2$, we consider two cases.\\
    \textbf{Case 1:} If $\fil_{S_t}(c) \ge u_t-\Delta -1$, then
    $\fil_{S_t}(c)$ is at least $1$ above the bottom of the
    interval defining which cups belong to $U_t$. The backlog
    increases by at most $1/2$ and the fill of $c$ decreases by $1/2$, so
    $\fil_{S_{t+1}}(c)$ is at least $1-1/2-1/2 = 0$ above the bottom of the
    interval, i.e. still in the interval. \\
    \textbf{Case 2:} On the other hand, if $\fil_{S_t}(c) <
    u_t-\Delta-1$, then every cup with fill in $[u_t-1, u_t]$
    must have been emptied from because the emptier is
    $\Delta$-greedy-like. Therefore the fullest cup
    on round $t+1$ is the same as the fullest cup on round $t$,
    because every cup with fill in $[u_t-1, u_t]$
    has had its fill decrease by $1/2$, and no cup with fill less than
    $u_t-1$ had its fill increase by more than $1/2$. Hence $u_{t+1}
    = u_t -1/2$. Because both $\fil(c)$ and the backlog
    have decreased by $1/2$, the distance between them is
    still at most $\Delta+2$, hence $c\in U_{t+1}$.\\
    The argument for why $L_t \subseteq L_{t+1}$ is symmetric.
  \end{proof}

  Now we show that under certain conditions $u_t$ decreases and
  $\ell_t$ increases.
  \begin{clm}
    \label{clm:howDoLandUchange}
    On any round $t$ where the emptier empties from at least
    $n/2$ cups, if $|U_t| \le n/2$ then $u_{t+1} = u_t - 1/2$.
    On any round $t$ where the emptier empties from at most $n/2$
    cups, if $|L_t| \le n/2$ then $\ell_{t+1} = \ell_t + 1/2$.
  \end{clm}
  \begin{proof}
    Consider a round $t$ where the emptier empties from at least
    $n/2$ cups. If there are at least $n/2$ cups outside of
    $U_t$, i.e. cups with fills in $[\ell_t, u_t-2-\Delta]$, then
    all cups with fills in $[u_t - 2, u_t]$ must be emptied from;
    if one such cup was not emptied from then by the pigeon-hole
    principle some cup outside of $U_t$ was emptied from, which
    is impossible as the emptier is $\Delta$-greedy-like. This
    clearly implies that $u_{t+1} = u_t - 1/2$: no cup with fill
    less than $u_t-2$ has gained enough fill to become the
    fullest cup, and the fullest cup from the previous round has
    lost $1/2$ unit of fill.

    By a symmetric argument, $\ell_{t+1} = \ell_{t} + 1/2$ on a
    round $t$ where the emptier empties at most $n/2$ cups and
    $|L_t| \le n/2$. 
  \end{proof}

  Now we show that eventually $L_t \cap U_t \neq \varnothing$.
  \begin{clm}
    There is a round $t_0 \le O(R + E + S)$ such that $U_{t}
    \cap L_{t} \neq \varnothing$ for all $t\ge t_0$.
  \end{clm}
  \begin{proof}
  We call a round where the emptier empties from $p' \neq p$ cups
  an \defn{unbalanced round}; we call a round that is not
  unbalanced a \defn{balanced} round. 

  Note that there are clearly at most $E+S$ unbalanced rounds.
  We now associate some unbalanced rounds with balanced rounds;
  in particular we define what it means for a balanced round to
  \defn{cancel} an unbalanced round. 
  Let $B = 2(R + \ceil{(1+1/n)(E+S)})$. For $i = 1,2,\ldots,B$ 
  (iterating in ascending order of $i$), if round $i$
  is unbalanced then we say that the first balanced round $j > i$
  that hasn't already been assigned (earlier in the sequential
  process) to cancel another unbalanced round $i' < i$, if any
  such round $j$ exists, \defn{cancels} round $i$. Note that
  cancellation is a one-to-one relation: each unbalanced round is
  cancelled by at most one balanced round and each balanced round
  cancels at most one unbalanced round. We say a balanced round
  $j$ is \defn{cancelling} if it cancels some unbalanced round $i
  < j$, and \defn{non-cancelling} if it does not cancel any
  unbalanced round.

  We claim that there is some $i \in [2(E+S)+1]$ such that among the
  rounds $[B + 2(E+S) + i-1]$ every unbalanced round has been
  cancelled by a balanced round in $[B + 2(E+S) + i-1]$, and such
  that there are $B$ non-cancelling balanced rounds. Note that there
  are at least $B + (E+S)$ balanced rounds in the first $B +
  2(E+S)$ rounds, and thus there are at least $B$ non-cancelling
  balanced rounds among the first $B+2(E+S)$ rounds, due to there
  being at most $E+S$ total unbalanced rounds. Next note that
  there must be at least $1$ non-cancelling balanced round among
  any set of $2(E+S) + 1$ rounds, because there cannot be more
  than $E+S$ unbalanced rounds, and hence there also cannot be
  more than $E+S$ cancelling rounds. Thus there is some $i \in
  [2(E+S)+1]$ such that round $B + 2(E+S) + i-1$ is balanced and
  non-cancelling. For this $i$, we have that all unbalanced
  rounds in $[B + 2(E+S) + i-1]$ are cancelled, and since $B +
  2(E+S) + i-1 \ge B + 2(E+S)$, we have that there are at least
  $B$ balanced non-cancelling rounds in $[B + 2(E+S) + i-1]$.
  These are the desired properties.

  Let $t_e$ be the first round by which there are $B = 2(R +
  \ceil{(E+S)/n})$ balanced non-cancelling rounds; we have shown
  that $t_e \le O(R+E+S)$. Note that the average fill of the cups
  cannot have decreased by more than $E/n$ from its starting
  value; similarly the average fill of the cups cannot have
  increased by more than $S/n$. Because the cups start $R$-flat,
  $u_t$ cannot have decreased by more than $R + E/n$ or else
  backlog would be less than average fill, and identically
  $\ell_t$ cannot have increased by more than $R + S/n$ or else
  anti-backlog would be larger than average fill. Now, by
  \cref{clm:howDoLandUchange} we have that for some $t \le t_e$,
  $|L_t| > n/2$: if $|L_t|\le n/2$ were always true for $t\le
  t_e$, then on every balanced round $\ell_t$ would have
  increased by $1/2$, and since $\ell_t$ increases by at most
  $1/2$ on unbalanced rounds, this implies that in total $\ell_t$
  would have increased by at least $(1/2)2(R + \ceil{(E+S)/n})$,
  which is impossible. By a symmetric argument it is impossible
  that $|U_t| \le n/2$ for all rounds. 

  Since $|U_{t+1}|\ge |U_t|$ and $|L_{t+1}| \ge |L_t|$ by
  \cref{clm:dontlosestuff}, we have that there is some round $t_0
  \le t_e$ such that for all $t \ge t_0$ we have $|U_t|> n/2$ and
  $|L_t|> n/2$. But then we have $U_t\cap L_t \neq \varnothing$, as desired.
  \end{proof}

  If there exists a cup $c \in L_t\cap U_t$, then 
  $$\fil(c) \in [u_t-2-\Delta, u_t] \cap [\ell_t, \ell_t + 2 +
  \Delta].$$ Hence we have that $$\ell_t+2+\Delta \ge
  u_t-2-\Delta.$$ Rearranging, $$u_t - \ell_t \le 2(2+\Delta) =
  R_\Delta.$$ Thus the cup configuration is $R_\Delta$-flat by
  the end of $O(R+E+S)$ rounds.

\end{proof}

% note: may as well take $\Delta' = \ceil{\Delta} \in
% \mathbb{N}$, this might be useful sometimes

Next we describe a simple oblivious filling strategy, that we
call \defn{$\randalg$}, that will be used as a subroutine in
\cref{lem:obliviousBase}; variants of this strategy
are well-known, and similar versions of it can be found in \cite{
BenderFaKu19, DietzRa91, Kuszmaul20}.
\begin{proposition}
  \label{prop:randalg}
  Consider an $R$-flat cup configuration in the regular single-processor
  $\infty$-extra-emptyings $\infty$-skip-emptyings negative-fill cup
  game on $n$ cups with initial average fill $\mu_0$.
  Let $k \in [n]$ be a parameter. Let $d = \sum_{i=2}^k 1/i$.

  There is an oblivious filling strategy \defn{$\randalg(k)$}
  with running time $k-1$ that achieves fill at least $\mu_0 -R +
  d$ in a known cup $c$ with probability at least $1/k!$ if
  we condition on the emptier not performing extra emptyings.
  $\randalg(k)$ achieves fill at most $\mu_0 + R + d$ in this cup
  (unconditionally).

  Furthermore, when applied against a $\Delta$-greedy-like emptier
  with $R = R_\Delta$, $\randalg(k)$ guarantees that
  the cup configuration is $(R + d)$-flat on every round
  (unconditionally).
\end{proposition}
\begin{proof}
  First we condition on the emptier not using extra emptying
  and show that in this case the filler has probability at least
  $1/(k-1)!$ (which we lower bound by $1/k!$ for sake of
  simplicity) of attaining a cup with fill at least $\mu_0 -R +d$.
  The filler maintains an \defn{active set}, initialized to being
  an arbitrary subset of $k$ of the cups. Every round the filler
  distributes $1$ unit of fill equally among all cups in the
  active set. Next the emptier removes $1$ unit of fill from some
  cup, or skips its emptying. Then the filler removes a random
  cup from the active set (chosen uniformly at random from the
  active set). This continues until a single cup $c$ remains in
  the active set. 

  We now bound the probability that $c$ has never been emptied
  from. Assume that on the $i$-th step of this process, i.e. when
  the size of the active set is $n-i+1$, no cups in the active
  set have ever been emptied from; consider the probability that
  after the filler removes a cup randomly from the active set
  there are still no cups in the active set that the emptier has
  emptied from. If the emptier skips its emptying on this round,
  or empties from a cup not in the active set then it is
  trivially still true that no cups in the active set have been
  emptied from. If the cup that the emptier empties from is in
  the active set then with probability $1/(k-i+1)$ it is evicted
  from the active set, in which case we still have that no
  cup in the active set has ever been emptied from. Hence with
  probability at least $1/(k-1)!$ the final cup in the
  active set, $c$, has never been emptied from. In this case, $c$
  will have gained fill $d=\sum_{i=2}^k 1/i$ as claimed. Because
  $c$ started with fill at least $-R+\mu_0$, $c$ now has fill at
  least $-R+ d+\mu_0$. 

  Now note that regardless of if the emptier uses extra emptyings
  $c$ has fill at most $\mu_0 + R + d$, as $c$ starts with fill
  at most $R$, and $c$ gains at most $1/(k-i+1)$ fill on the
  $i$-th round of this process. 

  Now we analyze this algorithm specifically for a
  $\Delta$-greedy-like emptier. 
  Let $\mathcal{A}_t$ be the event that the
  anti-backlog is smaller in $S_{t+1}$ than in $S_t$, let
  $\mathcal{B}_t$ be the event that some cup with fill equal to
  the backlog in $S_{t+1}$ was emptied from on round $t$. If
  $\mathcal{A}_t$ and $\mathcal{B}_t$ are both true on round $t$,
  then by greediness the cups are quite flat. In particular, 
  let $a$ be a cup with fill equal to the anti-backlog in state
  $S_{t+1}$ that was emptied from on round $t$, and let $b$ be
  a cup with fill equal to the backlog in state $S_{t+1}$ that
  was not emptied from on round $t$. By greediness $\fil_{I_t}(a) + \Delta
  > \fil_{I_t}(b)$. Of course $\fil_{I_t}(b) =
  \fil_{S_{t+1}}(b)$, and for $b$ to have fill equal to the backlog
  on round $t+1$, $b$ must have fill less than $1$ below 
  backlog on round $t$. Of course $\fil_{I_t}(a) \le
  \fil_{S_t}(a) + 1$, and for $a$ to have fill equal to the
  anti-backlog on round $t+1$, $a$ must have fill less than $1$
  above the anti-backlog on round $t$. Thus we have that the
  backlog and anti-backlog differ by at most $\Delta + 3 \le
  R_\Delta$ on round $t$, i.e. the cups are $R_\Delta$-flat.

  Consider a round $t_1$ where the cups are not $R_\Delta$-flat.
  Let $t_0$ be the last round that the cups were $R_\Delta$-flat.
  On all rounds $t \in (t_0, t_1)$ at least one of
  $\mathcal{A}_t$ or $\mathcal{B}_t$ must not hold. On a round
  where $\mathcal{A}_t$ does not hold, anti-backlog does not
  decrease and backlog increases by at most $1/(k-t+1)$, so fill
  range increases by at most $1/(k-t+1).$ On a round where
  $\mathcal{B}_t$ does not hold, anti-backlog decreases by at
  most $1$ and backlog decreases by at least $1-1/(k-t+1)$, as
  all cups with fill equal to the backlog in state $S_{t+1}$ were
  emptied from on round $t$, so fill-range increases by at most
  $1/(k-t+1)$. 
  Hence in total fill-range increases by at most $\sum_{i=2}^k
  1/i$ from $R$, i.e. the cups are $(R+d)$-flat on round $t_1$.

\end{proof}
