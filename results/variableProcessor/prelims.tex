\section{Preliminaries}\label{sec:prelims}
The cup game consists of a sequence of rounds. Let $S_t$ denote the state of of the cups at the start of round $t$. At the beginning of the round, the filler chooses the number
of processors $p_t$ for the round. Next, the filler distributes
$p_t$ units of water among the cups (with at most $1$ unit of
water to any particular cup). Now the game is at the
\defn{intermediate state} for round $t$, which we call state $I_t$.
Finally the emptier chooses $p_t$ cups to empty at most $1$ unit
of water from, which marks the conclusion of round $t$. The state
is then $S_{t+1}$.

If the emptier empties from a cup $c$ on round $t$ such that the fill
of $c$ is less than $1$ in state $I_t$, then $c$ now has $0$ fill (not
negative fill); we say that the emptier \defn{zeroes out} $c$ on round
$t$. Note that on any round where the emptier zeroes out a cup the
emptier has removed less total fill from the cups than the filler has
added to the cups; hence the average fill of the cups has increased.

We denote the fill of a cup $c$ at state $S$ by $\fil_S(c)$. Let
the \defn{mass} of a set of cups $X$ at state $S$ be $m_S(X) =
\sum_{c\in X} \fil_S(c)$. Denote the average fill of a set of
cups $X$ at state $S$ by $\mu_S(X)$. Note that $\mu_S(X) |X| =
m_S(X)$. Let the \defn{backlog} at state $S$ be $\max_c
\fil_S(c)$, let the \defn{anti-backlog} at state $S$ be $\min_c
\fil_S(c).$
Let the \defn{rank} of a cup at a given state be its position in
the list of the cups sorted by fill at the given state, breaking
ties arbitrarily but consistently. For example, the fullest cup
at a state has rank $1$, and the least full cup has rank $n$. Let
$[n] = \{1,2,\ldots, n\}$, let $i+[n] = \{i+1, i+2, \ldots,
i+n\}$. For a state $S$, let $S(r)$ denote the rank $r$ cup at
state $S$, and let $S(\{r_1,r_2,\ldots, r_m\})$ denote the set of
cups of ranks $r_1, r_2,\ldots r_m$.

As a tool in the analysis we define a new variant of the cup
game: the \defn{negative-fill} cup game. In the negative-fill cup
game, when the emptier empties from a cup, the cup's fill always
decreases by exactly $1$, i.e. there is no zeroing out. We refer
to the standard version of the cup game where cups can zero out
as the \defn{standard-fill} cup game when necessary for clarity.


The notion of negative fill will be useful in our lower-bound
constructions, in which we want to construct a strategy for the filler
that achieves large backlog. By analyzing a filling strategy on the
negative-fill game, we can then reason about what happens if we apply
the same filling strategy recursively to a set of cups $S$ whose
average fill $\mu$ is larger than $0$; in the recursive application,
the average fill $\mu$ acts as the ``new $0$'', and fills less than
$\mu$ act as negative fills.

Note that it is strictly easier for the filler to achieve high backlog
in the standard-fill cup game than in the negative-fill cup game;
hence a lower bound on backlog in the negative-fill cup game also
serves as a lower bound on backlog in the standard-fill cup game.  On
the other hand, during the upper bound proof we use the standard-fill
cup game: this makes it harder for the emptier to guarantee its upper
bound.

\paragraph{Other Conventions} When discussing the state of the cups
\defn{at a round \boldmath $t$}, we will take it as given that we are
referring to the starting state $S_t$ of the round. Also, when
discussing sets, we will use $XY$ as a shorthand for $X \cup
Y$. Finally, when discussing the average fill $\mu_{S_t}(X)$ of a set
of cups, we will sometimes ommit the subscript $S_t$ when the round
number is clear.
