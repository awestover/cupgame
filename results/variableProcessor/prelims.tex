\section{Preliminaries}\label{sec:prelims}
The cup game consists of a sequence of rounds. On the $t$-th
round, the state starts as $S_t$. To start, the filler chooses the number
of processors $p_t$ for the round. Next, the filler distributes
$p_t$ units of water among the cups (with at most $1$ unit of
water to any particular cup). Now the game is in an
intermediate state on round $t$, which we call state $I_t$.
Finally the emptier chooses $p_t$ cups to empty at most $1$ unit
of water from, which marks the conclusion of round $t$. The state
is then $S_{t+1}$.

Note that if the emptier empties from a cup $c$ on round $t$ with
fill at $I_t$ less than $1$, then $c$ now has $0$ fill (not
negative fill); we say that the emptier \defn{zeroes out} $c$ on
round $t$. Note that on any round where the emptier zeroes out a
cup the emptier has removed less total fill from the cups than
the filler has added to the cups; hence the average fill of the
cups has increased.

We denote the fill of a cup $c$ at state $S$ by $\fil_S(c)$. Let
the \defn{mass} of a set of cups $X$ at state $S$ be $m_S(X) =
\sum_{c\in X} \fil_S(c)$. Denote the average fill of a set of
cups $X$ at state $S$ by $\mu_S(X)$. Note that $\mu_S(X) |X| =
m_S(X)$. Let the \defn{backlog} at state $S$ be $\max_c
\fil_S(c)$, let the \defn{anti-backlog} at state $S$ be $\min_c
\fil_S(c).$
Let the \defn{rank} of a cup at a given state be its position in
a list of the cups sorted by fill at the given state, breaking
ties arbitrarily but consistently. For example, the fullest cup
at a state has rank $1$, and the least full cup has rank $n$. Let
$[n] = \{1,2,\ldots, n\}$, let $i+[n] = \{i+1, i+2, \ldots,
i+n\}$. For a state $S$, let $S(r)$ denote the rank $r$ cup at
state $S$, and let $S(\{r_1,r_2,\ldots, r_m\})$ denote the set of
cups of ranks $r_1, r_2,\ldots r_m$.

As a tool in the analysis we define a new variant of the cup
game: the \defn{negative-fill} cup game. In the negative-fill cup
game, when the emptier empties from a cup its fill always
decreases by exactly $1$, i.e. there is no zeroing out. We refer
to the standard version of the cup game where cups can zero out
as the \defn{standard-fill} cup game when necessary for clarity.
Negative fill can be interpreted as fill below average fill.
Negative fill is especially useful in our recursive lower bound
proofs in which we build on the average fill already achieved.
Note that it is strictly easier for the filler to achieve high backlog
in the standard-fill cup game than in the negative-fill cup game;
hence a lower bound on backlog in the negative-fill cup game
also serves as a lower bound on backlog in the standard-fill cup game.
On the other hand, during the upper bound
proof we use the standard-fill cup game: this makes it
harder for the emptier to guarantee its upper bound.

