\section{Technical Overview}
\label{sec:technical_overview}

In this section we provide sketches of the proofs in the paper,
omitting details.

\subsection{Adaptive Lower Bound}
In \cref{sec:adaptive} we provide filling strategies that an
adaptive filler can use to achieve backlog $\poly(n)$; in this
subsection we sketch the proofs of these results.

First we note that there is a trivial algorithm, that we call
\defn{$\trivalg$}, for achieving backlog at least $1/2$ on at
least $2$ cups in time $O(1)$.

The essential ingredient in our polynomial lower bound proof is
the \defn{Amplification Lemma} which gives a way to improve a
function for achieving a certain backlog.
\begin{lemma}
  Let $\alg(f)$ be a filling strategy that achieves backlog
  $f(n)$ on $n$ cups. There exists a filling strategy $\alg(f')$,
  the \defn{amplification} of $\alg(f)$, that achieves backlog at
  least $$f'(n) \ge (1-\delta) f(\floor{(1-\delta)n}) +
  f(\ceil{\delta n}).$$
\end{lemma}
\begin{proof}[\cref{lem:adaptiveAmplification} Proof Sketch]
The filler designates an \defn{anchor set} $A$ of size
$\ceil{\delta n}$ and a \defn{non-anchor set} $B$ of size
$\floor{(1-\delta)n}$. The filler's strategy is as follows:\\
\textbf{Step 1:} 
Make $\mu(A) \ge (1-\delta) f(|B|)$ by using $\alg(f)$ repeatedly
on $B$ while placing $1$ unit of fill into each anchor cup each
round; if the emptier does not place extra resources into $B$
this results in the filler achieving a cup with large fill in $B$
which it can swap into $A$ causing $\mu(A)$ to increase, if the
emptier does place extra resources into $B$ to prevent a cup in
$B$ from gaining high fill, then the emptier is correspondingly
placing fewer resources into $A$ causing $\mu(A)$ to increase.\\
\textbf{Step 2:} Use $\alg(f)$ once on $A$ to obtain some cup
with fill $\mu(A)+f(|A|)$.\\
Note that in order to use $\alg(f)$ on subsets of the cups the
filler will need to vary $p$.

Consider Step 1. It can be shown that by using $\alg(f)$ at most
$|A|$ times on $B$ the average fill of $A$ will have increased to
at least $(1-\delta)f(|B|)$ due either to the filler swapping
many cups with high fill into $A$, or to the emptier repeatedly
using more resources in $B$ and less in $A$ than the filler.
Step 2 trivially succeeds. Thus the filler achieves the desired
backlog:
$$(1-\delta) f(|B|) + f(|A|).$$
  
\end{proof}

We use the Amplification Lemma to give two lower bounds on
backlog: one with reasonable running time, the other with
slightly better backlog.

\begin{theorem}
  There is an adaptive filling strategy for achieving backlog
  $\Omega(n^{1-\varepsilon})$ for constant $\varepsilon \in (0,
  1/2)$ in running time $2^{O(\log^2 n)}$. 
\end{theorem}
\begin{proof}[\cref{thm:adaptivePoly} Proof Sketch]
We construct a sequence of
filling strategies with $\alg(f_{i+1})$ the amplification of
$\alg(f_i)$ using $\delta = \Theta(1)$ determined as a
function of $\varepsilon$, and $\alg(f_0)=\trivalg$.
Choosing $\delta$ appropriately, we show by induction on $i$ that
$\alg(f_{\Theta(\lg n)})$ achieves backlog
$\Omega(n^{1-\varepsilon})$ in running time $2^{O(\log^2 n)}$.
\end{proof}

\begin{theorem}
  There is an adaptive filling strategy for achieving backlog
  $\Omega(n)$ in running time $O(n!)$.
\end{theorem}
\begin{proof}[\cref{prop:factorialTimeAlg} Proof Sketch]
We construct a sequence of filling strategies with $\alg(f_{i+1})$ the
amplification of $\alg(f_i)$ using $\delta = 1/(i+1)$, and
$\alg(f_0)$ a filling strategy for achieving backlog $1$ on
$O(1)$ cups in $O(1)$ time (this is a slight modification of
$\trivalg$). We show by induction on $i$ that
$\alg(f_{\Theta(n)})$ achieves backlog $\Omega(n)$ in running
time $O(n!)$.
\end{proof}

\subsection{Upper Bound}
In \cref{sec:upperBound} we prove that a greedy emptier, i.e. an
emptier that always empties from the $p$ fullest cups, never lets
backlog exceed $O(n)$; in this subsection we sketch the proof of
this result.

The upper bound on backlog is derived by setting $k=1$ in
\cref{thm:TO_invariant}.
\begin{theorem}
  \label{thm:TO_invariant}
  The average fill of the $k$ fullest cups never exceeds $2n-k$.
\end{theorem}
\begin{proof}[\cref{thm:invariant} Proof Sketch]
 The proof is by induction on the round. Fix some
round $t$ and assume that all invariants hold on round $t$. Fix
some $k$; we aim to prove that the average fill of the $k$
fullest cups is at most $2n-k$ at the start of round $t+1$. 

Let $A$ be the cups that are among the $k$ fullest cups in $I_t$,
are emptied from, and are among the $k$ fullest cups in
$S_{t+1}$. Let $B$ be the cups that are among the $k$ fullest
cups in state $I_t$, are emptied from, and are not among the $k$
fullest cups in $S_{t+1}$. Let $C$ be the cups with ranks $|A| +
|B| + 1, \ldots, k + |B|$ in state $I_t$. The set $C$ is defined
so that the $k$ fullest cups in state $S_{t+1}$ are $AC$, since
once the cups in $B$ are emptied from, the cups in $B$ are not
among the $k$ fullest cups, so cups in $C$ take their places
among the $k$ fullest cups.

It can be shown that we may assume without loss of generality
that the rank $r$ cup at state $S_t$ is also the rank $r$ cup at
state $I_t$ for all ranks $r \in [n]$, by changing the labels of
the cups; intuitively this is true because if a cup $c$ changes
ranks from $S_t$ to $I_t$, then some other cup must have fill
very close to $c$'s fill.

We prove the invariant by considering several cases.

\noindent\textbf{Case 1}:
Some cup in $A$ zeroes out in round $t$.\\
\textbf{Analysis}:
The fill of all cups in $C$ must be at most $1$ at state $I_t$ to be
less than the fill of the cup in $A$ that zeroed out. Further,
$A$ has average fill at most $2n-(a-1)$ due to the cup with zero
fill. Combined, with some algebra, these facts imply that the
average fill in $AC$ is not too large, in particular not larger
than $2n-k$.

\noindent\textbf{Case 2}:
No cups in $A$ zero out in round $t$ and $b=0$.\\
\textbf{Analysis}:
In this case the set of cups of ranks in $[k]$ at state $S_t$ is
the same as the set of cups of ranks in $[k]$ at state $S_{t+1}$,
and these are both $A$. During round $t$ the emptier removes $a$
units of fill from the cups $A$. The filler cannot have added
more than $k$ fill to these cups, because it can add at most $1$
fill to any given cup. Also, the filler cannot have added more
than $p_t$ fill to the cups because this is the total amount of
fill that the filler is allowed to add. Hence the filler adds at
most $\min(p_t, k) = a+b=a+0=a$ fill to these cups. The emptier
thus is removing at least as much water as the filler is adding
to these cups, so the average fill has not increased, and is
still at most $2n-k$.

\noindent\textbf{Case 3}:
No cups in $A$ zero out on round $t$ and $b > 0$.\\
\textbf{Analysis}:
Consider $m_{S_{t+1}}(AC)$, which is the mass of the $k$ fullest
cups at state $S_{t+1}$. Each cup in $A$ was emptied from. The
filler adds at most $\min(k, p_t) = a+b$ fill to these cups.
Hence, 
\begin{equation}
  \label{eq:TO_bplusmAC}
m_{S_{t+1}}(AC) \le m_{S_t}(AC) + b.
\end{equation}

The key insight necessary to bound $m_{S_{t}}(AC)$ is to
notice that larger values for $m_{S_t}(A)$ correspond to smaller
values for $m_{S_t}(C)$ because the invariants are satisfied at
state $S_t$.
In particular, because
$$m_{S_t}(C) \le \frac{c}{b+c} m_{S_t}(BC) = \frac{c}{b+c}(m_{S_t}(ABC) - m_{S_t}(A)),$$
we have 
\begin{equation}
  \label{eq:monotonicTECHOVER}
  m_{S_{t}}(AC) \le \frac{c}{b+c}m_{S_t}(ABC) + \frac{b}{b+c}m_{S_t}(A).
\end{equation}
As \eqref{eq:monotonicTECHOVER} is monotonically increasing in
both $m_{S_t}(A)$ and $m_{S_t}(ABC)$ we can upper bound
\eqref{eq:monotonicTECHOVER} by substituting the extremal values
of $m_{S_t}(A)$ and $m_{S_t}(ABC)$ in, namely $|A|(2n-|A|)$ and
$|ABC|(2n-|ABC|)$.
After some algebra (or via an elegant combinatorial argument) it can be shown that 
\begin{equation}
  \label{eq:TO_goodsstsuffminuscb}
  \frac{c}{b+c}|ABC|(2n-|ABC|) + \frac{b}{b+c}|A|(2n-|A|) \le k(2n-k)-cb.
\end{equation}
Combined with \eqref{eq:TO_bplusmAC}, and the fact $b>0$,
\eqref{eq:TO_goodsstsuffminuscb} implies that the average fill of
the $k$ fullest cups in state $S_{t+1}$ is at most $2n-k$, as
desired.

We have shown the invariant holds for arbitrary $k$, so given that the
invariants all hold at state $S_t$ they also must all hold at state $S_{t+1}$.
Thus, by induction we have the invariant for all rounds $t\in\mathbb{N}$.
  
\end{proof}
\subsection{Oblivious Lower Bound}

In \cref{sec:oblivious} we provide filling strategies that an
oblivious filler can use to achieve backlog $n^{1-\varepsilon}$
for $\varepsilon \in (0, 1/2)$ constant against
\enquote{greedy-like} emptiers with probability at least
$1-2^{-\polylog(n)}$ in running time $2^{\polylog(n)}$; in this
subsection we sketch the proofs of these results. We remark that
this proof is by far our most technically difficult result;
however, interestingly, many of the ideas driving the oblivious
lower bound are similar to those driving the adaptive lower
bound. 

First we make some definitions. The \defn{fill-range} of a set of
cups at state $S$ is $\max_c \fil_S(c) - \min_c \fil_S(c)$. A cup
configuration is \defn{$R$-flat} if the fill-range is at most
$R$. An emptier is called \defn{$\Delta$-greedy-like} if whenever
there are two cups $c_1, c_2$ with $\fil_{I_t}(c_1) >
\fil_{I_t}(c_2) + \Delta$ the emptier doesn't empty from $c_2$ on
round $t$ unless it also empties from $c_1$ on round $t$. 
We call an emptier \defn{greedy-like} if it is
$\Delta$-greedy-like for $\Delta \le \frac{1}{128}\log\log\log n$
in the variable-processor cup game on $n$ cups.
For an oblivious filler we only prove lower bounds against
greedy-like emptiers; however this is a very interesting
class of emptiers because all known randomized algorithms for the
cup game are $O(1)$-greedy-like \cite{BenderFaKu19, Kuszmaul20}. We define a
new variant of the cup game: In the \defn{$E$-extra-emptyings}
\defn{$S$-skip-emptyings} cup game on $n$ cups, the filler
distributes $p$ units of water amongst the cups, and then the
emptier empties from $p$ \textit{or more, or less} cups. In
particular the emptier is allowed to do $E$ extra emptyings and
is also allowed to skip $S$ emptyings over the course of the
game. Let the \defn{regular} cup game be the $0$-extra-emptyings
$\infty$-skip-emptyings cup game: this is the regular cup game.
Allowing for some extra emptyings, and bounding the number of
skip emptyings is sometimes necessary when analyzing an algorithm
that is a subroutine of a larger algorithm however, hence it
sometimes makes sense to consider games with different values of
$E,S$. Unless explicitly stated otherwise however we are
considering the regular cup game.

Let $R_\Delta = 2(2+\Delta)$.
We now prove an important fact about $\Delta$-greedy-like emptiers.
\begin{lemma}
  \label{lem:TO_flatalg}
  There is an oblivious filling strategy \defn{$\flatalg$} that,
  given an $R$-flat configuration of cups, achieves an
  $R_\Delta$-flat configuration of cups against a
  $\Delta$-greedy-like emptier in the $p$-processor,
  $E$-extra-emptyings, $S$-skip-emptyings negative-fill cup game
  on $n=2p$ cups in running time $O(R+E+S)$. Throughout the
  duration of $\flatalg$ the cups are always $R$-flat.
\end{lemma}
\begin{proof}[\cref{lem:flatalg} Proof Sketch]
  The filler's strategy is to place $1/2$ units of fill into each
  cup on every round.
  Intuitively, if the fill-range of the cups is large then by
  greediness the emptier should be forced to empty from the
  fullest cups and not empty from the least full cups, thus
  decreasing the fill-range.
\end{proof}

Now we describe a well-known simple oblivious filling strategy
that will be used as a subroutine later.
\begin{proposition}
  \label{prop:TO_randalg} % TO stands for technical overview 
  Consider an $R$-flat configuration of cups in the
  regular single-processor negative-fill cup game on $n$ cups with
  initial average fill $\mu_0$. Let $k \in [n]$. Let $d =
  \sum_{i=2}^k 1/i.$

  There is an oblivious filling strategy \defn{$\randalg(k)$}
  that achieves backlog $\mu_0 - R + d$ with
  probability at least $1/k!$ in time $O(k)$.

  Furthermore, when applied against a $\Delta$-greedy-like
  emptier with $R = R_\Delta$, even if the emptier is allowed
  arbitrarily many extra emptyings (i.e. if the game occurs in
  the $\infty$-extra-emptyings $\infty$-skip-emptyings cup game),
  $\randalg(k)$ guarantees that the cup configuration is $(R +
  d)$-flat on every round.
\end{proposition}
\begin{proof}[\cref{prop:randalg} Proof Sketch]
  The filler maintains an \defn{active set}, initialized to being
  an arbitrary subset of $k$ of the cups. Every round the filler
  distributes $1$ unit of fill equally among all cups in the
  active set. Next the emptier removes $1$ unit of fill from some
  cup, or skips its emptying. Then the filler removes a random
  cup from the active set (chosen uniformly at random from the
  active set). This continues until a single cup $c$ remains in
  the active set. 
  If $c$ has never been emptied from, then its fill has increased
  by at least $1/k + 1/(k-1) + \cdots + 1/2 = d$
  from its starting value which was at least $\mu_0 -R$. Thus $c$
  has at least the desired fill if it has not been emptied from.
  By randomly removing cups from the active set the filler
  guarantees that, with probability at least $1/(k-1)!$, $c$ is not
  emptied from.

  Now we consider a greedy-like emptier that is allowed
  extra-emptyings. Let $\mathcal{A}_t$ be the event that the
  anti-backlog is smaller in $S_{t+1}$ than in $S_t$, let
  $\mathcal{B}_t$ be the event that some cup with fill equal to
  the backlog in $S_{t+1}$ was emptied from on round $t$. If
  $\mathcal{A}_t$ and $\mathcal{B}_t$ are both true on round $t$,
  then by greediness the cups are quite flat, in particular
  $R_\Delta$-flat. Consider a round $t_1$ where the cups are not
  $R_\Delta$-flat. Let $t_0$ be the last round that the cups were
  $R_\Delta$-flat. On all rounds $t \in (t_0, t_1)$ at least one
  of $\mathcal{A}_t$ or $\mathcal{B}_t$ must not hold. On a round
  where $\mathcal{A}_t$ does not hold, anti-backlog does not
  decrease and backlog increases by at most $1/(k-t+1)$, so fill
  range increases by at most $1/(k-t+1).$ On a round where
  $\mathcal{B}_t$ does not hold, anti-backlog decreases by at
  most $1$ and backlog decreases by at least $1-1/(k-t+1)$, as
  all cups with fill equal to the backlog in state $S_{t+1}$ were
  emptied from on round $t$, so fill-range increases by at most
  $1/(k-t+1)$. 
  Hence in total fill-range increases by at most $\sum_{i=2}^k
  1/i$ from $R$, i.e. the cups are $(R+d)$-flat on round $t_1$.

\end{proof}

We now give a method for transforming a filling strategy for
achieving large backlog into a filling strategy for achieving
high fill in many cups.
\begin{definition}
  \label{def:TO_rep}
  {\normalfont
  Let $\alg_0$ be an oblivious filling strategy, that can get
  high fill (for some definition of high) in some cup against
  greedy-like emptiers with some probability. We construct a new
  filling strategy \defn{$\rep_\delta(\alg_0)$} as follows:

  Say we have some configuration of $n\le N$ cups (recall that
  eventually we aim to get large backlog in $N$ cups).
  Let $n_A = \ceil{\delta n}, n_B = \floor{(1-\delta)n}$. Let
  $M=2^{\polylog(N)}$ be a chosen parameter. 
  Initialize $A$ to $\varnothing$ and $B$ to
  being all of the cups. We call $A$ the \defn{anchor set} and
  $B$ the \defn{non-anchor set}. The filler always places $1$
  unit of fill in each anchor cup on each round. The filling
  strategy consists of $n_A$ \defn{donation-processes}, which are
  procedures that result in a cup being \defn{donated} from $B$
  to $A$ (i.e. removed from $B$ and added to $A$). At the start
  of each donation-processes the filler chooses a value $m_0$
  uniformly at random from $[M]$. We say that the filler
  \defn{applies} a filling strategy $\alg$ to $B$ if the
  filler uses $\alg$ on $B$ while placing $1$ unit of fill
  in each anchor cup. During the donation-process the filler
  applies $\alg_0$ to $B$ $m_0$ times, and flattens $B$ by
  applying $\flatalg$ to $B$ for $\Theta(N^2)$ rounds before each
  application of $\alg_0$. At the end of each donation process
  the filler takes the cup given by the final application of
  $\alg_0$ (i.e. the cup that $\alg_0$ guarantees with some
  probability against a certain class of emptiers to have a
  certain high fill), and donates this cup to $A$. 

We say that the emptier \defn{neglects} the anchor set on a round
if it does not empty from each anchor cup. We say that an
application of $\alg_0$ to $B$ is \defn{non-emptier-wasted} if
the emptier does not neglect the anchor set during any round of
the application of $\alg_0$.
}

\end{definition}

We use $\rep$ in two distinct places: first to get constant
backlog, and second to prove the Oblivious counterpart of the
Adaptive Amplification Lemma.
For the rest of the section our goal is eventually to get backlog
$\poly(N)$ in $N$ cups for a value of $N$ that we fix now; this
value of $N$ will be used throughout all of the proofs.

First we analyze $\rep(\randalg)$.
\begin{lemma}
  \label{lem:TO_obliviousBase}
  Let $\Delta \le \frac{1}{128}\log\log\log N$, let $h =
  \frac{1}{16}\log\log\log N$, let $k=\ceil{e^{2h+1}}$, let
  $\delta = \frac{1}{2k}$, let $n =
  \log^5 N$. Consider an $R_\Delta$-flat cup configuration in
  the variable-processor cup game on $n$ cups with initial
  average fill $\mu_0$.

  Against a $\Delta$-greedy-like emptier,
  $\rep_{\delta}(\randalg(k))$ using $M = 2^{\Theta(\log^4 N)}$
  either achieves mass $N^2$ in the cups, or with probability at
  least $1-2^{-\Omega(\log^4 N)}$ makes some unknown cup in $A$
  have fill at least $h+\mu_0$ while also
  guaranteeing that $\mu(B) \ge -h/2 + \mu_0$, where $A,B$ are
  the sets defined in \cref{def:TO_rep}. The running time of
  $\rep_{\delta}(\randalg(k))$ is $2^{O(\log^4 N)}$.
\end{lemma}
\begin{proof}[\cref{lem:obliviousBase} Proof Sketch] 
  We use the definitions given in \cref{def:TO_rep}.

  Note that if the emptier neglects the anchor set $N^2$ times,
  or skips $N^2$ emptyings, then the mass of the cups will be at
  least $N^2$, so the filler is done. For the rest of the proof
  we consider the case where the emptier chooses to neglect the
  anchor set fewer than $N^2$ times, and chooses to skip fewer
  than $N^2$ emptyings.

First we show that $\mu(B)$ never sinks too low, which is one of
the guarantees of our algorithm, and necessary to ensure that we
get large fill in many cups in $A$. Let $d = \sum_{i=2}^k 1/i =
\Theta(h)$. Because extra-emptyings and skip-emptyings of $B$ are
limited, using \cref{lem:TO_flatalg} and \cref{prop:TO_randalg},
we can inductively show that $B$ is always $(R_\Delta + d)$-flat,
and that all applications of $\flatalg$ make $B$ be
$R_\Delta$-flat. Now we claim that $\mu(B) \le \mu(AB) + 2.$
There are two ways $\mu(B) - \mu(AB)$ can increase:\\
\textbf{Case 1:}
The emptier could empty from $0$ cups in $B$ while emptying
from every cup in $A$. \\
By greediness this means that $\mu(A) \ge \mu(B) - \Delta$. By
our choice of $\delta=1/(2k)$ which makes $|A|$ much
smaller than $|B|$, this implies $\mu(B) \le \mu(AB) + 1$.\\
\textbf{Case 2:}
The filler could evict a cup with fill lower than $\mu(B)$ from
$B$ at the end of a donation-process. \\ Because $B$ starts each
application of $\randalg(k)$ being $(R_\Delta + d)$-flat, and the
running time of the application is $k-1$ the cup donated from $B$
cannot have fill lower than $\mu(B) - R_\Delta - (k-1)$. But
by design in choosing $\delta = 1/(2k)$, i.e. making $n_B$
substantially larger than $|A|$, even if $B$
donated $n_A$ cups that were maximally empty the difference
$\mu(B)-\mu(AB)$ would only increase by at most $1$.\\ Combining
the analysis of Case 1 and Case 2 we have that $\mu(B) \le
\mu(AB) + 2$.

Combining the upper bound on $\mu(B)$ with the fact that $B$ is
always flat and that the emptier is greedy-like, we have that no
cup in $A$ ever has its fill exceed $u_A = \mu(AB) + 2 + R_\Delta
+ d + 1.$ Obviously $\mu(A)\le u_A$. Hence we have an upper bound
on $\mu(A)-\mu(AB)$; plugging this upper bound into the relation
$m(A) + m(B) = m(AB)$, and using the fact that $|A|/|B|$ is
sufficiently small by our choice of $\delta$, we get the
desired bound on $\mu(B)$:
$$\mu(B) \ge -h/2 + \mu(AB) \ge -h/2 + \mu_0.$$

Now we bound the probability that at least one cup in $A$ has
fill at least $\mu_0 + h$. If the emptier were not able to
neglect the anchor set, then each of the $n_A$ donation processes
would simply have at least a $1/k!$ chance of donating a cup with
fill at least $\mu(B) - R_\Delta + d$ to $A$. By our lower bound
on $\mu(B)$, and because $R_\Delta \le h/2$ by enforcing $\Delta
\le h/8$, this fill is at least $\mu_0 + h$. The probability that
at least one donation process successfully donates such a cup to
$A$ is at least 
$$1-(1-1/k!)^{n_A} \ge 1-2^{-\Omega(\log^4 N)}.$$
However, the emptier can \defn{interfere}, i.e. can neglect the
anchor set, in which case we the cup given by $\randalg(k)$ might
not have large fill. In order to mitigate interference by the
emptier, we do many, in particular $M=2^{O(\log^4 N)}$,
applications of $\randalg(k)$ and choose one application randomly
on which to actually donate a cup. By a Chernoff Bound, on any
donation process at least $N^2 2^{\log^4 N}$ applications of $\randalg(k)$
succeed. With probability at least $1-2^{-\log^4 N}$ the final
application of $\randalg(k)$ is not interfered with.
Taking a union bound, with probability at least $1-2^{-\log^4 N}$
the final application of every donation process is not interfered
with. Then using our analysis for the case where the emptier does
not interfere, and taking another union bound, we have that with
probability at least $1-2^{-\log^4 N}$ at least one
donation-process donates a cup with fill $\mu_0 + h$ to $A$.

The running time of the filling strategy is clearly $n_A O(M) (
O(N^2) + O(1)) = 2^{O(\log^4 N)},$ as each of the $n_A$
donation-processes consists of $O(M)$ applications of
$\randalg(k)$ and $O(M)$ applications of $\flatalg$.
\end{proof}

Using \cref{lem:TO_obliviousBase} we show that an
oblivious filler can achieve constant fill in a known cup.
\begin{proposition}
  \label{prop:TO_obliviousBase}
  Let $H = \frac{1}{128}\log\log\log N$, let $\Delta \le
  \frac{1}{128}\log\log\log N$, let $n = \log^5 N$. 
  Consider an $R_\Delta$-flat cup configuration in the
  variable-processor cup game on $n$ cups with average fill $\mu_0$.
  There is an oblivious filling strategy that either
  achieves mass $N^2$ among the cups, or achieves fill at least $\mu_0 + H$
  in a chosen cup in running time $2^{O(\log^4 N)}$ against a
  $\Delta$-greedy-like emptier with probability at least
  $1-2^{-\Omega(\log^4 N)}.$
\end{proposition}
\begin{proof}[\cref{prop:obliviousBase} Proof Sketch]
  The filler starts by using $\rep_\delta(\randalg(k))$ with
  parameter settings as in \cref{lem:TO_obliviousBase};
  note that $h$ from \cref{lem:TO_obliviousBase} is $8H$.

  If this results in mass $N^2$ among the cups then the filler is
  already done. Otherwise, with probability at least
  $1-2^{-\Omega(\log^4 N)}$, there is some cup in $A$ with fill
  $\mu_0 + 8H$. We assume for the rest of the proof that there is
  some cup $c_* \in A$ with $\fil(c_*) \ge \mu_0 + 8H$.

  The filler sets $p=1$, i.e. uses a single processor. Now the
  filler exploits the emptier's greedy-like nature to to get fill
  $H$ in a chosen cup $c_0 \in B$. For $5H$ rounds
  the filler places $1$ unit of fill into $c_0$. Because the
  emptier is $\Delta$-greedy-like it must empty from $c_*$ 
  while $\fil(c_*) > \fil(c_0) + \Delta$. Within $5H$ rounds
  the cups $\fil(c_*)$ cannot decrease below $3H+\mu_0 > H + \Delta + \mu_0$.
  Hence, during these $5H$ rounds, only cups with fills larger
  than $H + \mu_0$ can be emptied from by greediness. 
  The fill of $c_0$ started as at least
  $-4H+\mu_0$ as $\mu(B) \ge -h/2+\mu_0$ from
  \cref{lem:obliviousBase}. After $5H$ rounds
  $c_0$ has fill at least $H+\mu_0$, because the emptier cannot
  have emptied $c_0$ until it attained fill $H+\mu_0$, and if
  $c_0$ is never emptied from then it achieves fill $H+\mu_0$.
  Thus the filling strategy achieves backlog $H+\mu_0 \ge H$
  in $c_0$, a \emph{known} cup, as desired.

  The running time is of course still $2^{O(\log^4 N)}$ by
  \cref{lem:TO_obliviousBase}.
\end{proof}

We remark that \cref{lem:TO_obliviousBase} and
\cref{prop:TO_obliviousBase} both refer to filling strategies in
the regular cup game. In particular, we prove no guarantees on
the behavior of the algorithms in \cref{lem:TO_obliviousBase} and
\cref{prop:TO_obliviousBase} for the $E$-extra-emptyings cup game
with $E > 0$. It is somewhat surprising that we do not need to
provide guarantees such as a guarantee that the cups have some
small bounded fill-range regardless of extra-emptying as we had
to do for $\flatalg$ and $\randalg$.

Next we prove the Oblivious Amplification Lemma.
\begin{lemma}
  \label{lem:TO_obliviousAmplification} 
  Let $\delta \in (0, 1/2)$ be a constant parameter. Let $\Delta
  \le O(1)$. Consider a cup configuration
  in the variable-processor cup game on $n \le N, n >
  \Omega(1/\delta^2)$ cups with average fill $\mu_0$ that is
  $R_\Delta$-flat. Let $\alg(f)$ be an oblivious filling strategy
  that either achieves mass $N^2$ or, with failure probability at
  most $p\ge 2^{-\lg^8 N}$, achieves backlog $\mu_0 + f(n)$ on such cups
  in running time $T(n)$ against a $\Delta$-greedy-like emptier.
  Let $M = 2^{\polylog(N)}$.

  Consider a cup configuration in the variable-processor cup game
  on $n \le N, n > \Omega(1/\delta^2)$ cups with average fill
  $\mu_0$ that is $R_\Delta$-flat. There exists an oblivious
  filling strategy $\alg(f')$ that either achieves mass $N^2$ or
  with failure probability at most 
  $$p' \le np + 2^{-\lg^8 N}$$
  achieves backlog $f'(n)$ satisfying 
  $$f'(n) \ge (1-\delta)^2 f(\floor{(1-\delta)n}) + f(\ceil{\delta n}) + \mu_0$$ 
  and $f'(n) \ge f(n)$, in running time 
  $$T'(n) \le Mn\cdot T(\floor{(1-\delta)n}) + T(\ceil{\delta n})$$
  against a $\Delta$-greedy-like emptier.
\end{lemma}
\begin{proof}[\cref{lem:obliviousAmplification} Proof Sketch]
  We use the definitions and notation given in \cref{def:rep}. 

  Note that if the emptier neglects the anchor set $N^2$ times,
  or skips $N^2$ emptyings, then the mass of the cups will be at
  least $N^2$, so the filler is done. For the rest of the proof
  we consider the case where the emptier chooses to neglect
  the anchor set fewer than $N^2$ times, and choose to skip fewer
  than $N^2$ emptyings.

  The filler simply uses $\alg(f)$ on all the cups if this
  results in sufficient backlog, i.e. if
  $$f(n) \ge (1-\delta)^2 f(n_B) + f(n_A).$$
  In this case our strategy trivially has the desired guarantees. 
  In the rest of the proof we consider the case where $\alg(f)$
  does not achieve sufficient backlog.

  The filler's strategy is as follows:\\
  \textbf{Step 1:} Make $\mu(A) \ge (1-\delta)^2 f(n_B)$ by
  using $\rep_\delta(\alg(f))$ on all the cups,
  i.e. applying $\alg(f)$ repeatedly to $B$, flattening $B$ before
  each application, and then donating a cup from $B$ to $A$.\\
  \textbf{Step 2:} Flatten $A$ using $\flatalg$, and then use
  $\alg(f)$ on $A$.

  We now analyze Step 1.
  For this proof we need all donation-processes to succeed, as
  opposed to in the proof of
  \cref{lem:TO_obliviousBase} in which we only needed
  a constant fraction of the donation-processes to succeed. This
  necessitates choosing $M$ very large. In particular we choose
  $M = 2^{\log^{24} N}$ ---recall that $[M]$ is the set from
  which we randomly choose how many times to apply $\alg(f)$ in a
  donation-process. By choosing $M$ this large we cannot hope to
  guarantee that every application of $\alg(f)$ succeeds: there
  are far too many applications. On the other hand, having $M$ so
  large allows us to have a very tight concentration bound on how
  many applications of $\alg(f)$ succeed. By a Chernoff bound
  with probability at least $1-e^{-2Mp^2}$ at least $M(1-2p)$ of
  $M$ applications of $\alg(f)$ would succeed if the emptier did
  not \defn{interfere}, i.e. neglect the anchor set and do an
  extra emptying in the non-anchor set. The emptier can interfere
  with at most $N^2$ of the $M(1-2p)$ applications that would
  otherwise be successful. Let $1-q$ be the probability that a
  donation-process succeeds, i.e. the final application of
  $\alg(f)$ is not emptier-wasted and succeeds. We have $$1-q \ge
  (1-e^{-2Mp^2})\paren{\frac{M\cdot (1-2p)-N^2}{M}}.$$
  Rearranging, simplifying by loosening the bound, and using the
  assumption $p \ge 2^{-\lg^8 N}$, we can show $$q \le 2p +
  2^{-\lg^8 N}.$$

  Taking a union bound, we have that with probability at least
  $1- q \cdot n_A$ all donation-process successfully achieve a cup
  with fill at least $\mu_{S_0}(B) + f(n_B)$ where $\mu_{S_0}(B)$
  refers to the average fill of $B$ measured at the start of the
  application of $\alg(f)$; now we assume all donation-processes
  are successful, and demonstrate that this translates into the
  desired average fill in $A$.

  Let \defn{$\skips_t$} denote the number of times that the
  emptier has skipped the anchor set by round $t$. Consider how
  $\mu(B) - \skips/n_B$ changes over the course of the donation
  processes. As noted above, at the end of each donation-process
  $\mu(B)$ decreases due to $B$ donating a cup with fill at least
  $\mu(B) + f(n_B)$. In particular, if $S$ denotes the cup state
  immediately before a cup is donated on the $i$-th
  donation-process, $B_0$ denotes the set $B$ before
  the donation and $B_1$ denotes the set $B$ after the donation,
  then $\mu_{S}(B_1) = \mu_{S}(B_0) - f(n_B) / (n-i)$. Now we claim that
  $t\mapsto \mu_{S_t}(B) - \skips_t/n_B$ is monotonically decreasing. 
  Clearly donation decreases $\mu(B) - \skips/n_B$. 
  If the anchor set is neglected then $\mu(B)$ decreases, causing
  $\mu(B) - \skips/n_B$ to decrease. 
  If a skip occurs, then $\skips/n_B$ increases by more than
  $\mu(B)$ increases, causing $\mu(B)-\skips/n_B$ to decrease. 
  Let $t_*$ be the cup state at the end of all the
  donation-processes. We have that 
  \begin{equation}
    \label{eq:TO_harmonic1}
    \mu_{S_{t_*}}(B) - \frac{\skips_{t_*}}{n_B} \le \mu_0 - \sum_{i=1}^{n_A}\frac{f(n_B)}{n-i}.
  \end{equation}
  By conservation of mass we have 
  $$n_A\cdot \mu_{S_{t_*}}(A) + n_B\cdot \mu_{S_{t_*}}(B) = n\mu_0 + \skips_{t_*}.$$
  Rearranging, 
  \begin{equation}
    \label{eq:TO_lowerboundingAharmonic}
    \mu_{S_{t_*}}(A) = \mu_0 + \frac{n_B}{n_A}\paren{\mu_0 +
    \frac{\skips_{t_*}}{n_B} - \mu_{S_{t_*}}(B)}.
  \end{equation}
  Now we obtain a simpler form of
  Inequality~\eqref{eq:TO_harmonic1}. Recalling that harmonic
  numbers grow like $\ln$ we have
  $$\sum_{i=1}^{n_A} \frac{1}{n-i} \approx \ln n/(n-n_A) \approx
  \ln \frac{1}{1-\delta} > \delta.$$
  Then using Inequality~\eqref{eq:TO_harmonic1} in
  Equation~\ref{eq:TO_lowerboundingAharmonic} we essentially have
  $$\mu_{S_{t_*}}(A) \ge \mu_0 + \frac{n_B}{n_A} \delta f(n_B)
  \approx \mu_0 + (1-\delta) f(n_B).$$
  To handle the imprecisions indicated in the
  approximations above the bound becomes slightly worse:
  $$\mu_{S_{t_*}}(A) \ge \mu_0 + (1-\delta)^2 f(n_B).$$

We have shown that in Step 1 the filler achieves average fill
$\mu_0 + (1-\delta)f(n_B)$ in $A$ with failure probability at
most $q \cdot n_A$.
Now the filler flattens $A$ and uses $\alg(f)$ on $A$.
It is clear that this is possible, and succeeds with probability
at least $p$.
This gets a cup with fill 
$$\mu_0 + (1-\delta)^2 f(n_B) + f(n_A)$$
in $A$, as desired.

Taking a union bound over the probabilities of Step 1 and Step 2
succeeding gives that the entire procedure fails with probability
at most 
$$p' \le p + q \cdot n_A \le np + 2^{-\lg^8 N}.$$

The running time of Step 1 is clearly $M\cdot n\cdot
T(\floor{(1-\delta)n})$ and the running time of Step 2 is clearly
$T(\ceil{\delta n})$; summing these yields the desired upper
bound on running time.

\end{proof}

Finally we prove that an oblivious filler can achieve backlog
$N^{1-\varepsilon}$. 
\begin{theorem}
  There is an oblivious filling strategy for the
  variable-processor cup game on $N$ cups that achieves backlog
  at least $\Omega(N^{1-\varepsilon})$ for any constant $\varepsilon
  >0$ in running time $2^{\polylog(N)}$ with probability at least
  $1-2^{-\polylog(N)}$ against a $\Delta$-greedy-like emptier
  with $\Delta \le \frac{1}{128} \log\log\log N$.
\end{theorem}
\begin{proof}[\cref{thm:obliviousPoly} Proof Sketch]
  If the mass of the cups ever exceeds $N^2$ then backlog is, and
  will always be, at least $N$. For the rest of the proof we
  consider the case where mass $N^2$ is not achieved in the cups
  at any point in our process.

  We show how to achieve backlog $(N/n_b)^{1-\varepsilon}-1$ for
  some $n_b = \log^5 N$ on $N$ cups; note that this implies
  that the filler can achieve backlog
  $\Omega(N^{1-\varepsilon'})$ for any constant $\varepsilon' \in
  (0,1/2)$ on $N$ cups. Let $\delta$ be a constant, chosen as a
  function of $\varepsilon$.

  By \cref{prop:TO_obliviousBase} there is an oblivious filling
  strategy that achieves backlog $f_0(k) \ge \frac{1}{128}\log\log\log N$ on
  $k \ge n_b$ cups with probability at least $1-2^{-\log^4 N}$; we call
  this algorithm $\alg(f_0)$. Using the Oblivious Amplification
  Lemma we construct $\alg(f_{i+1})$ as the amplification of
  $\alg(f_i)$.

  One can inductively show that $f_{\Theta(\log N)}$ achieves
  backlog $(N/n_b)^{1-\varepsilon} -1$, as desired. 
  Analysis of the running time recurrence from the Oblivious Amplification
  Lemma gives that $\alg(f_{\Theta(\log N)})$ has running time
  $2^{\polylog(N)}$, while analysis of the probability recurrence
  shows that $\alg(f_{\Theta(\log N)})$ succeeds with probability
  at least $1-2^{-\polylog(N)}.$
\end{proof}

