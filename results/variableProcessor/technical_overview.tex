\section{Technical Overview}
\label{sec:technical_overview}

In this section we provide sketches of the proofs in the paper,
omitting many details.

\subsection{Adaptive Lower Bound}
In \cref{sec:adaptive} we provide filling strategies that an
adaptive filler can use to achieve backlog $\poly(n)$; in this
subsection we sketch the proofs of these results.

First we note that there is a trivial algorithm, that we call
\defn{$\trivalg$}, for achieving backlog at least $1/2$ on at
least $2$ cups in time $O(1)$.

The essential ingredient in the proof is the Amplification
Lemma, which gives a way to transform a filling strategy
$\alg(f)$ into a new filling strategy $\alg(f')$, called the
\defn{amplification} of $\alg(f)$, that achieves backlog at
least 
$$f'(n) \ge (1-\delta) f(\floor{(1-\delta)n}) + f(\ceil{\delta
n}).$$
To achieve this the filler designates an \defn{anchor set} $A$ of size
$\ceil{\delta n}$ and a \defn{non-anchor set} $B$ of size
$\floor{(1-\delta)n}$. The filler's strategy is roughly as follows:\\
\textbf{Step 1:} Get $\mu(A) \ge (1-\delta) f(|B|)$ by using $\alg(f)$ repeatedly
on $B$ to achieve cups with fill at least $f(|B|)+\mu(B)$ in $B$ and
then swapping these into $A$. The filler always places $1$ unit
of fill into each anchor cup while doing this.\\
\textbf{Step 2:} Use $\alg(f)$ once on $A$ to obtain some cup
with fill $\mu(A)+f(|A|)$.\\
Note that in order to use $\alg(f)$ on subsets of the cups the
filler will need to vary $p$.

Consider Step 1. We say that the emptier \defn{neglects} the
anchor set on a round if the emptier does not empty from all
anchor cups on that round. By neglecting the anchor set the
emptier can place more resources into the non-anchor set than the
filler, which can prevent the filler from getting a cup with fill
$\mu(B) + f(|B|)$. On the other hand, neglecting the anchor set
increases the mass of the anchor set. If the emptier neglects the
anchor set too many times, then the filler gets high fill in the
anchor set because the mass of the anchor set increases when the
anchor set is neglected. If the emptier does not neglect the
anchor set too many times, then the filler will be able to get
cups with high fill in $B$, and swap these into $A$, which also
results in the mass of the anchor set increasing. It can be shown
that after at most $|A|$ applications of $\alg(f)$ to $B$ the
anchor set will have achieved average fill at least $(1-\delta)
f(|B|)$.

Step 2 trivially succeeds.
Thus we have achieved backlog
$$(1-\delta) f(|B|) + f(|A|), $$
as desired.

Now we use the Amplification Lemma to prove two theorems.

First, say we aim to achieve backlog $\Omega(n^{1-\varepsilon})$
for constant $\epsilon \in (0, 1/2)$. We construct a sequence of
filling strategies with $\alg(f_{i+1})$ the amplification of
$\alg(f_i)$ using $\delta = \Theta(1)$ to be determined as a
function of $\varepsilon$, and $\alg(f_0)=\trivalg$.
Choosing $\delta$ appropriately, we show by induction on $i$ that
$\alg(f_{\Theta(\lg n)})$ achieves backlog
$\Omega(n^{1-\varepsilon})$ in running time $2^{O(\log^2 n)}$.

Now, say we aim to achieve backlog $\Omega(n)$.
We construct a sequence of filling strategies with $\alg(f_{i+1})$ the
amplification of $\alg(f_i)$ using $\delta = 1/(i+1)$, and
$\alg(f_0)$ a filling strategy for achieving backlog $1$ on
$O(1)$ cups in $O(1)$ time (this is a slight modification of
$\trivalg$). We show by induction on $i$ that
$\alg(f_{\Theta(n)})$ achieves backlog $\Omega(n)$ in running
time $O(n!)$.

\subsection{Upper Bound}
In \cref{sec:upperBound} we prove that a greedy emptier, i.e. an
emptier that always empties from the $p$ fullest cups, never lets
backlog exceed $O(n)$; in this subsection we sketch the proof of
this result.
This upper bound on backlog follows directly from a set of
invariants that we prove are maintained: the
average fill of the $k$ fullest cups is at most $2n-k$.

We now sketch the proof that the invariants are always
maintained. The proof is by induction on the round. Fix some
round $t$ and assume that all invariants hold on round $t$. Fix
some $k$; we aim to prove that the average fill of the $k$
fullest cups is at most $2n-k$ at the start of round $t+1$. 

Let $A$ be the cups that are among the $k$ fullest cups in $I_t$,
are emptied from, and are among the $k$ fullest cups in
$S_{t+1}$. Let $B$ be the cups that are among the $k$ fullest
cups in state $I_t$, are emptied from, and are not among the $k$
fullest cups in $S_{t+1}$. Let $C$ be the cups with ranks $|A| +
|B| + 1, \ldots, k + |B|$ in state $I_t$. The set $C$ is defined
so that the $k$ fullest cups in state $S_{t+1}$ are $AC$, since
once the cups in $B$ are emptied from, the cups in $B$ are not
among the $k$ fullest cups, so cups in $C$ take their places
among the $k$ fullest cups.

We show that we may assume without loss of generality that
$S_t(r) = I_t(r)$ for each rank $r \in [n]$, by changing the
labels of the cups; intuitively this is true because if a cup $c$
changes ranks from $S_t$ to $I_t$, then some other cup must have
fill very close to $c$'s fill.

We prove the invariant by considering several cases.

\noindent\textbf{Case 1}:
Some cup in $A$ zeroes out in round $t$.\\
\textbf{Analysis}:
The fill of all cups in $C$ must be at most $1$ at state $I_t$ to be
less than the fill of the cup in $A$ that zeroed out. Further,
$A$ has average fill at most $2n-(a-1)$ due to the cup with zero
fill. Combined, with some algebra, these facts imply that the
average fill in $AC$ is not too large, in particular not larger
than $2n-k$.

\noindent\textbf{Case 2}:
No cups in $A$ zero out in round $t$ and $b=0$.\\
\textbf{Analysis}:
In this case $S_{t+1}([k]) = S_t([k])$.
During round $t$ the emptier removes $a$ units of fill from the
cups in $S_t([k])$, specifically the cups in $A$. The filler
cannot have added more than $k$ fill to these cups, because it
can add at most $1$ fill to any given cup. Also, the filler
cannot have added more than $p_t$ fill to the cups because this
is the total amount of fill that the filler is allowed to add.
Hence the filler adds at most $\min(p_t, k) = a+b=a+0=a$ fill to
these cups.
The emptier thus is removing at least as much water as the filler
is adding to these cups, so the average fill has not increased,
and is still at most $2n-k$.

\noindent\textbf{Case 3}:
No cups in $A$ zero out on round $t$ and $b > 0$.\\
\textbf{Analysis}:
Consider $m_{S_{t+1}}(AC)$, which is the mass of the $k$ fullest
cups at state $S_{t+1}$. Each cup in $A$ was emptied from. The
filler adds at most $\min(k, p_t) = a+b$ fill to these cups.
Hence, 
$$m_{S_{t+1}}(AC) \le m_{S_t}(AC) + b.$$

The key insight necessary to bound $\mu_{S_{t+1}}(AC)$ is to
notice that larger values for $m_{S_t}(A)$ correspond to smaller
values for $m_{S_t}(C)$ because the invariants are satisfied at
state $S_t$.
In particular, because
$$m_{S_t}(C) \le \frac{c}{b+c} m_{S_t}(BC) = \frac{c}{b+c}(m_{S_t}(ABC) - m_{S_t}(A)),$$
we have 
\begin{equation}
  \label{eq:monotonicTECHOVER}
  m_{S_{t+1}}(AC) \le \frac{c}{b+c}m_{S_t}(ABC) + \frac{b}{b+c}m_{S_t}(A).
\end{equation}
As \eqref{eq:monotonicTECHOVER} is monotonically increasing in
both $m_{S_t}(A)$ and $m_{S_t}(ABC)$ we can upper bound
\eqref{eq:monotonicTECHOVER} by substituting the extremal values
of $\mu_{S_t}(A)$ and $m_{S_t}(ABC)$ in, namely $|A|(2n-|A|)$ and
$|ABC|(2n-|ABC|)$.
After some algebra (or via an elegant combinatorial argument) it can be shown that 
$$\frac{c}{b+c}|ABC|(2n-|ABC|) + \frac{b}{b+c}|A|(2n-|A|) \le k(2n-k)$$
which implies that the average fill of the $k$ fullest cups in
state $S_{t+1}$ is at most $2n-k$, as desired.

We have shown the invariant holds for arbitrary $k$, so given that the
invariants all hold at state $S_t$ they also must all hold at state $S_{t+1}$.
Thus, by induction we have the invariant for all rounds $t\in\mathbb{N}$.

\todo{
\subsection{Oblivious Lower Bound}

In \cref{sec:oblivious} we provide filling strategies that an
oblivious filler can use to achieve backlog $n^{1-\varepsilon}$
for $\varepsilon \in (0, 1/2)$ constant against
\enquote{greedy-like} emptiers with probability at least
$1-2^{-\polylog(n)}$ in running time $2^{\polylog(n)}$; in this
subsection we sketch the proofs of these results. We remark that
this proof is by far our most technically difficult result;
however, the ideas driving the oblivious lower bound are similar
to those driving the adaptive lower bound. 

We say that an emptier is \defn{$\Delta$-greedy-like} if .
We say that cups are $R$-flat if .

lemma: $\flatalg$ can flatten.
prop: $\randalg$.
Def: $\rep$. We apply flattening for $N^2$ rounds, and
apply the alg $M = 2^{\polylog(N)}$ rounds.

lemma: $\rep(\randalg)$.
prop: combine $\rep(\randalg)$ stuff.
This will be used as the base case of the argument as follows:
\begin{itemize}
  \item num cups: $\log^8 N$
  \item backlog: $\Theta(1)$ (or mass $N^2$)
  \item running time: $m=\Theta(N^2)$, so $\poly(N)$
  \item fail pr: $2^{-\Omega(\log^8 N)}$
\end{itemize}

Oblivious Amplification Lemma.
We are given $\alg(f)$ which has
\begin{itemize}
  \item num cups: $n$
  \item backlog: $f(n)$ (or mass $N^2$)
  \item running time: $T(n)$
  \item fail pr: $p$, which is assumed to be $p\ge 1/2^{\lg^8 N}$.
\end{itemize}
We get $\alg(f')$ which has
\begin{itemize}
  \item num cups: $n$
  \item backlog: $f'(n) \ge (1-\delta)f(\floor{(1-\delta)n}) + f(\ceil{\delta n})$ (or mass $N^2$)
  \item running time: use $m = 2^{\log^{24} N}$, $T'(n) \le mnT(\floor{(1-\delta)n}) + T(\ceil{\delta n})$
  \item fail pr: $p' \le 2p + 1/2^{\lg^8 N}$
\end{itemize}

Finally we recursively use the Oblivious Amplification Lemma to
get large backlog. Specifically, we let $f_{i+1}$ be the
amplification of $f_i$ using $\delta = \Theta(1)$ chosen as a
function of $\varepsilon$, and using $\randalg$ for $\alg(f_0)$.
In particular we get the following:
\begin{itemize}
  \item num cups: $N$ 
  \item backlog: $\Omega(N^{1-\varepsilon})$
  \item running time: $(2^{\log^{24} N})^{\log N} = 2^{\log^{25} N}$
  \item fail pr: the recurrence $p_{i+1} = ap_i + b, p_1 = c$
    solves to $p_k = b\Theta(a^{k-1}) + c \Theta(a^k)$. In our
    case this becomes $2^{\lg N} / 2^{\lg^8 N} + 2^{\lg
    N}/2^{\lg^8 N} = 2^{-\polylog(N)}$
\end{itemize}

QED

}

