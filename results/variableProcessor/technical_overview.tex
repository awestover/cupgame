\section{Technical Overview}
\subsection{Adaptive Lower Bound}

The essential ingredient to this proof is the Amplification
Lemma, which gives a way to transform a filling strategy
$\alg(f)$ into a new filling strategy $\alg(f')$ that achieves backlog at
least $f'(n) \ge (1-\delta) f(\floor{(1-\delta)n}) + f(\ceil{\delta n}).$
We now briefly outline how the Amplification Lemma works.
The filler designates an anchor set $A$ of size
$\ceil{\delta n}$ and a non-anchor set $B$ of size
$\floor{(1-\delta)n}$. The filler's strategy is roughly as follows:
\textbf{Step 1:} Get $\mu(A) \ge h$ by using $\alg(f)$ repeatedly
on $B$ to achieve cups with fill at least $f(n_B)$ in $B$ and
then swapping these into $A$. The filler always places $1$ unit
of fill into each anchor cup.\\
\textbf{Step 2:} Use $\alg(f)$ once on $A$ to obtain some cup
with fill $\mu(A)+f(n_A)$.\\
Note that in order to use $\alg(f)$ on subsets of the cups the
filler will need to vary $p$.
In our analysis we show that Step 1 can be made to succeed, and
in reasonable running time; Step 2 trivially can succeed.

Using the Amplification Lemma we prove two theorems regarding how
high an adaptive filler can make the backlog. First we construct
a sequence of filling strategies with $\alg(f_{i+1})$ the
amplification of $\alg(f_i)$ using $\delta = \Theta(1)$ in the
Amplification Lemma. For this sequence, we show that
$f_{\Theta(\lg n)} \ge \Omega(n^{1-\varepsilon})$ where
$\varepsilon$ is a constant that depends on $\delta$: in
particular, the smaller we make $\delta$ the smaller we can get
$\varepsilon$. We show that this filling strategy has running
time $2^{O(\lg^2 n}$, i.e. quasi-polynomial. Second, we construct
a sequence of filling strategies with $\alg(f_{i+1})$ the
amplification of $\alg(f_i)$ using $\delta = 1/(i+1)$ in the
Amplification Lemma. Analysis shows that $\alg(f_{\Theta(n)})$
strategy actually achieves backlog $\Omega(n)$, although with
running time $O(n!)$.

\subsection{Upper Bound}
We prove that a perfectly greedy emptier, i.e. an emptier that
always empties from the $p$ fullest cups, never lets backlog
exceed $O(n)$. 
This follows directly from a set of invariants that we show are
maintained, in particular that the average fill of the $k$ fullest cups
is at most $2n-k$.

We now sketch the proof of this theorem. The proof is by
induction on the rounds. Fix some round $t$ and assume that the
invariants hold on round $t$. Fix some $k$; we aim to prove that
the average fill of the $k$ fullest cups is at most $2n-k$ at the
start of round $t+1$. 

Let $A$ be the cups that are among the $k$ fullest cups in $I_t$,
are emptied from, and are among the $k$ fullest cups in
$S_{t+1}$. Let $B$ be the cups that are among the $k$ fullest
cups in state $I_t$, are emptied from, and are not among the $k$
fullest cups in $S_{t+1}$. Let $C$ be the cups with ranks $|A| +
|B| + 1, \ldots, k + |B|$ in state $I_t$. The set $C$ is defined
so that the $k$ fullest cups in state $S_{t+1}$ are $AC$, since
once the cups in $B$ are emptied from, the cups in $B$ are not
among the $k$ fullest cups, so cups in $C$ take their places
among the $k$ fullest cups.

Next, we show that we may assume without loss of
generality that $S_t(r) = I_t(r)$ for each rank $r \in [n]$, by
changing the labels of the cups.

Next we analyze several cases, and show that in each case the
desired invariant holds.

\textbf{Case 1}:
Some cup in $A$ zeroes out in round $t$.\\
\textbf{Analysis}:
The fill of all cups in $C$ must be at most $1$ at state $I_t$ to be
less than the fill of the cup in $A$ that zeroed out. Further,
$A$ has average fill at most $2n-(a-1)$ due to the cup with zero
fill. Combined, with some algebra, these facts imply that the
average fill in $AC$ is not too large, in particular not larger
than $2n-k$.

\textbf{Case 2}:
No cups in $A$ zero out in round $t$ and $b=0$.\\
\textbf{Analysis}:
In this case $S_{t+1}([k]) = S_t([k])$.
During round $t$ the emptier removes $a$ units of fill from the
cups in $S_t([k])$, specifically the cups in $A$. The filler
cannot have added more than $k$ fill to these cups, because it
can add at most $1$ fill to any given cup. Also, the filler
cannot have added more than $p_t$ fill to the cups because this
is the total amount of fill that the filler is allowed to add.
Hence the filler adds at most $\min(p_t, k) = a+b=a+0=a$ fill to
these cups.
The emptier thus is removing at least as much water as the filler
is adding to these cups, so the average fill has not increased,
and is still at most $2n-k$.

Case 3 is the most technically interesting.
\textbf{Case 3}:
No cups in $A$ zero out on round $t$ and $b > 0$.
\textbf{Analysis}:
Clearly,
$$m_{S_{t+1}}(S_{t+1}([k])) \le m_{S_t}(AC) + b.$$
The key insight necessary to bound this is to notice that larger values for
$m_{S_t}(A)$ correspond to smaller values for $m_{S_t}(C)$ because of the
invariants.
In particular, 
$$m_{S_t}(C) \le \frac{c}{b+c}(m_{S_t}(ABC) - m_{S_t}(A)).$$
But then we have 
\begin{equation}
  \label{eq:monotonicTECHOVER}
  m_{S_{t+1}}(S_{t+1}([k])) \le \frac{c}{b+c}m_{S_t}(ABC) + \frac{b}{b+c}m_{S_t}(A).
\end{equation}
As \eqref{eq:monotonicTECHOVER} is monotonically increasing in
both $m_{S_t}(A)$ and $m_{S_t}(ABC)$ we can upper bound the
expression by substituting the extremal values of $\mu_{S_t}(A)$
and $m_{S_t}(ABC)$ in, namely $2n-|A|$ and $2n-|ABC|$.
After some algebra, or via a more elegant combinatorial argument,
it can be shown that 
$$\frac{c}{b+c}|ABC|(2n-|ABC|) + \frac{b}{b+c}|A|(2n-|A|) \le
2n-k$$
which gives that the average fill of the $k$ fullest cups in
state $S_{t+1}$ is at most $2n-k$, as desired.

We have shown the invariant holds for arbitrary $k$, so given that the
invariants all hold at state $S_t$ they also must all hold at state $S_{t+1}$.
Thus, by induction we have the invariant for all rounds $t\in\mathbb{N}$.

\subsection{Oblivious Lower Bound}

This section is the most technically difficult. The ideas driving
the lower bound are similar to those driving the lower bounds
with an adaptive filler, but the oblivious nature of the filler
makes this lower bound substantially more difficult. 

First we show that flatalg can flatten.
Then we state and analyze randalg.
Then we state rep. We apply flattening for $N^2$ rounds, and
apply the alg $m = 2^{\polylog(N)}$ rounds.
Then we analyze rep(randalg).
This will be used as the base case of the argument
\begin{itemize}
  \item num cups: $\log^8 N$
  \item backlog: $\Theta(1)$ (or mass $N^2$)
  \item running time: $m=\Theta(N^2)$, so $\poly(N)$
  \item fail pr: $2^{-\Omega(\log^8 N)}$
\end{itemize}

Then we analyze rep(alg(f)), i.e. the Oblivious Amplification Lemma.
We are given $\alg(f)$ which has
\begin{itemize}
  \item num cups: $n$
  \item backlog: $f(n)$ (or mass $N^2$)
  \item running time: $T(n)$
  \item fail pr: $p$, which is assumed to be $p\ge 1/2^{\lg^8 N}$.
\end{itemize}
We get $\alg(f')$ which has
\begin{itemize}
  \item num cups: $n$
  \item backlog: $f'(n) \ge (1-\delta)f(\floor{(1-\delta)n}) + f(\ceil{\delta n})$ (or mass $N^2$)
  \item running time: use $m = 2^{\log^{24} N}$, $T'(n) \le mnT(\floor{(1-\delta)n}) + T(\ceil{\delta n})$
  \item fail pr: $p' \le 2p + 1/2^{\lg^8 N}$
\end{itemize}

Finally we recursively use the Oblivious Amplification Lemma to get large backlog.
Specifically, we let $f_{i+1}$ be the amplification of $f_i$
using $\delta = \Theta(1)$, and using $\randalg$ for $\alg(f_0)$.
In particular we get the following:
\begin{itemize}
  \item num cups: $N$ 
  \item backlog: $\Omega(N^{1-\varepsilon})$
  \item running time: $(2^{\log^{24} N})^{\log N} = 2^{\log^{25} N}$
  \item fail pr: the recurrence $p_{i+1} = ap_i + b, p_1 = c$
    solves to $p_k = b\Theta(a^{k-1}) + c \Theta(a^k)$. In our
    case this becomes $2^{\lg N} / 2^{\lg^8 N} + 2^{\lg
    N}/2^{\lg^8 N} = 2^{-\polylog(N)}$
\end{itemize}

QED

