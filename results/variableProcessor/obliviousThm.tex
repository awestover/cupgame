Finally we prove that an oblivious filler can achieve backlog
$N^{1-\varepsilon}$, just like an adaptive filler despite the
oblivious filler's disadvantage. The proof is very similar to the
proof of \cref{thm:adaptivePoly}, although significant care must
be taken in the randomized case to get the desired probabilistic
guarantees. 
\begin{theorem}
  \label{thm:obliviousPoly}
  There is an oblivious filling strategy for the
  variable-processor cup game on $N$ cups that achieves backlog
  at least $\Omega(N^{1-\varepsilon})$ for any constant $\varepsilon
  >0$ in running time $2^{\polylog(N)}$ with probability at least
  $1-2^{-\polylog(N)}$ against a $\Delta$-greedy-like emptier
  with $\Delta \le \frac{1}{128} \log\log\log N$.
\end{theorem}
\begin{proof}
  We aim to achieve backlog $(N/n_b)^{1-\varepsilon}-1$ for 
  $n_b = \Theta(\log^5 N)$ on $N$ cups.
  Let $\delta$ be a constant, chosen as a function of $\varepsilon$.

  We call the algorithm from \cref{prop:obliviousBase}
  $\alg(f_0)$. We remark that in the proof of
  \cref{thm:adaptivePoly} the size of the base case is constant,
  whereas now the size of the base case is $\polylog(N)$; the
  larger base case is necessary, as a constant-size base case
  would destroy our probability of success. The probability of
  success achieved by $\alg(f_0)$ is, by construction, sufficient
  to take a union bound over $2^{\polylog N}$ applications of
  $\alg(f_0)$. Now we construct $\alg(f_{i+1})$ as the amplification
  of $\alg(f_i)$ using \cref{lem:obliviousAmplification} and
  parameter $\delta$. Define a sequence $g_i$ as 
  $$g_i =
  \begin{cases}
    n_b\ceil{16/\delta}, & i=0\\
    \floor{ g_{i-1}/(1-\delta) }, & i\ge 1 
  \end{cases}.$$
  We claim the following regarding our construction:
  \begin{clm}
    \label{clm:fikinductionagain}
    \begin{equation}
      f_i(k) \ge (k/n_b)^{1-\varepsilon} - 1 \text{ for all } k \le g_i. \label{eqn:fikinductionAGAIN}
    \end{equation}
  \end{clm}
  \begin{proof}
  We prove \cref{clm:fikinductionagain} by induction on $i$. \\
  For $i=0$, the base case of our induction,
  \eqref{eqn:fikinductionAGAIN} is true as
  $(k/n_b)^{1-\epsilon} - 1 \le O(1)$ for $k\le g_0$, whereas
  $\alg(f_0)$ achieves backlog $\Omega(\log\log\log N) >
  \omega(1)$.\\
  Now we assume \eqref{eqn:fikinductionAGAIN} for $f_i$, and aim to
  establish \eqref{eqn:fikinductionAGAIN} for $f_{i+1}$. 
  Note that, by design of $g_i$, if $k \le g_{i+1}$ then
  $\floor{k\cdot (1-\delta)} \le g_i$. Consider any $k\in
  [g_{i+1}]$. 
  First we deal with the trivial case where $k \le g_0$. Here, 
  $$f_{i+1}(k) \ge f_i(k) \ge \cdots \ge f_0(k) \ge (k/n_b)^{1-\varepsilon} -1.$$
  Now we consider $k \ge g_0$. 
  Since $f_{i+1}$ is the amplification of $f_i$ we have by \cref{lem:obliviousAmplification} that
  $$f_{i+1}(k) \ge (1-\delta)^2 f_i(\floor{(1-\delta)k}) + f_i(\ceil{\delta k}).$$
  By our inductive hypothesis, which applies as $\ceil{\delta k}\le g_i, \floor{k\cdot (1-\delta)} \le g_i$, we have
  $$f_{i+1}(k) \ge (1-\delta)^2 (\floor{(1-\delta)k/n_b}^{1-\varepsilon}-1) + \ceil{\delta k/n_b}^{1-\varepsilon} - 1. $$
  Dropping the floor and ceiling, incurring a $-1$ for dropping the floor, we have
  $$f_{i+1}(k) \ge (1-\delta)^2 (((1-\delta)k/n_b-1)^{1-\varepsilon}-1) + (\delta k/n_b)^{1-\varepsilon} - 1.$$
  Because $(x-1)^{1-\varepsilon} \ge x^{1-\varepsilon} -1$, as $x\mapsto x^{1-\varepsilon}$ is a sub-linear
  sub-additive function, we have 
  $$f_{i+1}(k) \ge (1-\delta)^2 (((1-\delta)k/n_b)^{1-\varepsilon}-2) + (\delta k/n_b)^{1-\varepsilon}-1.$$
  Moving the $(k/n_b)^{1-\varepsilon}$ to the front we have
  $$ f_{i+1}(k) \ge (k/n_b)^{1-\varepsilon} \cdot\left((1-\delta)^{3-\varepsilon} + \delta^{1-\varepsilon} - \frac{2(1-\delta)^2}{(k/n_b)^{1-\varepsilon}} \right) -1.$$
  Because $(1-\delta)^{3-\varepsilon} \ge 1-(3-\varepsilon)\delta$, a fact called Bernoulli's Identity, we have
  $$f_{i+1}(k) \ge (k/n_b)^{1-\varepsilon} \cdot\left(1-(3-\varepsilon)\delta + \delta^{1-\varepsilon} - \frac{2(1-\delta)^2}{(k/n_b)^{1-\varepsilon}} \right)-1.$$
  Of course $-2(1-\delta)^2 > -2$, so 
  $$f_{i+1}(k) \ge (k/n_b)^{1-\varepsilon} \cdot\left(1-(3-\varepsilon)\delta + \delta^{1-\varepsilon} - 2/(k/n_b)^{1-\varepsilon} \right) -1.$$
  Because $$\frac{-2}{(k/n_b)^{1-\varepsilon}} \ge \frac{-2}{(g_0/n_b)^{1-\varepsilon}} \ge -2(\delta/16)^{1-\varepsilon} \ge -\delta^{1-\varepsilon}/2,$$
  which follows from our choice of $g_0 = \ceil{8/\delta} n_b$ and the restriction
  $\varepsilon<1/2$, we have 
  $$f_{i+1}(k) \ge (k/n_b)^{1-\varepsilon} \cdot\left(1-(3-\varepsilon)\delta + \delta^{1-\varepsilon} - \delta^{1-\varepsilon}/2 \right)-1.$$
  Finally, combining terms we have
  $$f_{i+1}(k) \ge  (k/n_b)^{1-\varepsilon} \cdot\left(1-(3-\varepsilon)\delta + \delta^{1-\varepsilon}/2\right)-1. $$
  Because $\delta^{1-\varepsilon}$ dominates $\delta$ for
  sufficiently small $\delta$, there is a choice of
  $\delta=\Theta(1)$ such that 
  $$1-(3-\varepsilon)\delta + \delta^{1-\varepsilon}/2 \ge 1.$$ 
  Taking $\delta$ to be this small we have,
  $$f_{i+1}(k) \ge (k/n_b)^{1-\varepsilon}-1,$$
  completing the proof. 
  \end{proof}

  The sequence $g_i$ larger the sequence $g_i$ from the proof of
  \cref{thm:adaptivePoly} for the same value of $\delta$: it is
  defined by the same recurrence, except that the first term is
  $n_b$ times larger. Thus we have that $g_{i_*} \ge N$ for some
  $i_* \le O(\log N)$. Hence $\alg(f_{i_*})$ achieves backlog
  $$f_{i_*}(N) \ge (N/n_b)^{1-\varepsilon}-1.$$ Let $\varepsilon'
  = 2\varepsilon$. Of course $n_b \le O(N^\varepsilon)$, so 
  $$(N/n_b)^{1-\varepsilon}-1 \ge \Omega(N^{1-\varepsilon'}).$$

  Let the running time of $f_i(N)$ be $T_i(N)$. From the
  Amplification Lemma we have following recurrence bounding $T_i(N)$:
  \begin{align*}
    T_i(n) &\le 2^{\polylog(N)} \cdot T_{i-1}(\floor{(1-\delta)n}) + T_{i-1}(\ceil{\delta n}) \\
            &\le 2^{\polylog(N)}T_{i-1}(\floor{(1-\delta)n}).
  \end{align*}
  It follows that $\alg(f_{i_*})$, recalling that $i_* \le O(\log N)$, has running time
  $$T_{i_*}(n) \le (2^{\polylog(N)})^{O(\log N)} \le 2^{\polylog(N)}$$
  as desired.

  Now we analyze the probability that the construction fails. 
  Consider the recurrence $a_{i+1} = \alpha a_i + \beta, a_0 =
  \gamma$; the recurrence bounding failure probability is a
  special case of this. Expanding, we see that the recurrence
  solves to $a_k = \Theta(\alpha^{k-1})\beta + \alpha^k \gamma$.
  In our case we have 
  $$\alpha \le N, \beta = 2^{-\log^4 N}, \gamma \le 2^{-\Omega(\log^4 N)}.$$
  Hence the recurrence solves to 
  $$p_{i_*} \le 2^{-\polylog(N)},$$
  as desired.

\end{proof}

