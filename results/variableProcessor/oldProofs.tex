%% this is an altenrative ending to the upper bound proof.
%% it's slightly less elegant / well written than the version in the paper, but equally valid
\begin{proof}
where 
\begin{equation}
  \label{eqn:redistributeA}
  &m_{S_t}(A) + \frac{c}{b+c}m_{S_t}(BC) \\
  &= \frac{c}{b+c}m_{S_t}(ABC) + \frac{b}{b+c}m_{S_t}(A).
\end{equation}
Note that the expression in \eqref{eqn:redistributeA} is monotonically
increasing in both $\mu_{S_t}(ABC)$ and $\mu_{S_t}(A)$. 
Thus, by numerically replacing both average fills with
their extremal values, $2n-|ABC|$ and $2n-|A|$, we can upper bound $m_{S_t}(AC)$ by,
\begin{equation}
\label{eq:numericallyinterpretit}
m_{S_t}(AC) \le \frac{c}{b+c}(2n-|ABC|) + \frac{b}{b+c}(2n-|A|).
\end{equation}

At this point the claim can be verified by straightforward algebra,
however this is not elegant; instead, we combinatorially interpret \eqref{eq:numericallyinterpretit}.

We define a new ``fake" state $F$, which may not represent
a valid configuration of cups (i.e. might not satisfy the invariants), where
$\mu_F(A)=2n-|A|$ and $\mu_F(ABC)=2n-|ABC|$, in particular with all the cups in $A$
having identical fill, and all the cups in $BC$ having identical fill. By construction, the right side of \eqref{eq:numericallyinterpretit} is,

\begin{align}
    &m_{F}(A) + c\mu_{F}(BC) \nonumber \\
    &= \frac{b}{b+c} m_F(A) + \frac{c}{b+c} m_F(ABC) \label{eq:Fmotivation}\\
    &= \frac{c}{b+c}(2n-|ABC|) + \frac{b}{b+c}(2n-|A|) \nonumber
\end{align}
Let us compare $F$ to the state $U$ in which every cup in $ABC$ has fill
$\mu_F(ABC) = 2n-|ABC|$. To reach $F$ from this state $U$, we must increase the fill of each cup in $A$ by some amount, and
decrease the fill of each cup in $BC$ by an amount such that the mass added to
$A$ is taken away from $BC$. To reach fill $\mu_F(A) = 2n-|A|$, the cups in $A$
must each have been increased by $|BC|$ from their previous fills of $2n-|ABC|$.
To offset an increase in $\mu_{F}(A)$ of $|BC|$, we need a corresponding
decrease in $\mu_{F}(BC)$ by $|A|$.
Thus, by comparing states $F$ and $U$, we have concluded that, 
\begin{equation}
    \mu_{F}(BC) = 2n-|ABC|-|A|
    \label{eq:BCispusheddown}
\end{equation}
Combining \eqref{eq:Fmotivation} and \eqref{eq:BCispusheddown}, we have that the right side of \eqref{eq:numericallyinterpretit} is
\begin{align*}
& m_{F}(A) + c(2n-|ABC|-|A|) \\
&= a(2n-a) + c(2n-|ABC|-a) \\
&= (a+c)(2n-a) - c(a+c+b) \\
&= (a+c)(2n-a-c) - cb.
\end{align*}
\end{proof}


% this probably belongs somewhere in the corollary proof, which will be
% substantially more complex than I originally anticipated, because Prop and
% Lemma need to take as input cups arrangements with flatness guarantees
\begin{clm}
  Without loss of generality the cup state is always $(h\sqrt{n/\lg\lg n})$-flat.
\end{clm}
\begin{proof}
In order to flatten a set of cups we must have a bound on the magnitude of the
fills of the cups. We claim that without loss of generality no cup has fill
larger in magnitude than $h\sqrt{n/\log\log n}$. If a cup has more than
$h\sqrt{n/\log\log n}$ fill we call the cup \defn{overpowered}. If there ever
is an overpowered cup then we are automatically done: the emptier has achieved
$\poly(n)$ backlog and will maintain it for at least $\poly(n)$ rounds. If a
cup ever has fill less than $-h\sqrt{n/\log\log n}$ then the absolute average
fill must be large enough such that the absolute fill of this cup is at least
$0$. Thus there is an overpowered cup. From now on we assume that there are no
overpowered cups.
\end{proof}



%% this is the
%% proof of the oblivious base case (can get log n backlog)
%% that doesn't use flattening, it uses the idea of op cups
%% unforunately, it didn't quite work
%% it's also quite messy. 
%% nevertheless is was hard to write, and I don't want to delete it. :)
\begin{proposition}
  \label{prop:obliviousBase}
  There exists an oblivious filling strategy in the variable-processor cup game
  on $n$ cups that achieves backlog $\Omega(\log n)$ against a $(T,
  \Delta)$-greedy-like emptier (where $T, \Delta \le O(1)$ are constants
  known to the filler), with constant probability.
\end{proposition}
\begin{proof}
  Let $A$, the \defn{anchor} set, be a subset of the cups chosen uniformly at
  random from all subsets of size $n/2$ of the cups, and let $B$, the
  \defn{non-anchor} set, consist of the rest of the cups ($|B| = n/2$). Let $h
  = 2 T + 4$. After attaining average positive tilt at least $h$ in a constant
  fraction of $A$, exploiting the fact that the emptier is greedy-like, the
  filler can use this positive tilt to create a known set of cups each with
  fill at least $h' = 1$. More specifically, the filling algorithm proceeds as
  follows: 
  \begin{itemize}
    \item \textbf{Step 1:} 
      Obtain large positive tilt in $A$. If at least half of the cups $c\in B$
      have $\fil(c) \ge -384h$, then we can, with constant probability, achieve
      a cup with fill $h$ in $B$ -- which we will then swap to $A$ -- by
      playing a single-processor cup game on a constant-size ($\lceil e^{385h}
      \rceil$ suffices) subset of $B$. 
      On the other hand, if at least half of the cups in $B$ have fill less than
      $-384h$, then the expectation of the positive tilt of a cup chosen
      randomly is high; this property can be exploited to get high positive
      tilt in the anchor set.

      Because the filler is oblivious it cannot know which of these states it is in; 
      thus the filler employs a hybrid of these strategies, switching between
      them with certain probabilities, which works in both cases.
  \item \textbf{Step 2:} Reduce the number of processors to a constant fraction
    $c$ of $n$, where $c$ is chosen such that there must be a set of at most $nc$ unknown cups $S$ with 
    $$\sum_{x\in S} \lfloor \tilt(x) / h \rfloor \ge nc.$$
    Using this positive tilt, the filler can raise the fill of $nc$ known cups
    to $h'$, because the emptier is greedy.
  \item \textbf{Step 3:} Recurse on the $nc$ cups that are known to have fill
    at least $h'$.
\end{itemize}
By performing $\Omega(\log n)$ levels of recursion, achieving constant backlog
$h'\ge 1/2$ at each step (relative to the average fills), the filler achieves backlog
$\Omega(\log n)$.

The strategy as stated fails if fill is extremely concentrated in a very small
number of cups; however, in this case the proposition is trivially satisfied.
In particular, we call a cup \defn{overpowered} if it contains fill at least
$h\sqrt{n}/(\log\log n)^{2/3}$. If there is ever an overpowered cup, then the
proposition is trivially satisfied, as backlog is $\Omega(\poly(n))$. Note that
the filler doesn't need to know which cup is overpowered because it will take
$\Omega(\poly(n))$ rounds for the emptier to reduce the fill below $\poly(n)$.
Hence, we can assume without loss of generality that no cup is ever
overpowered. Furthermore, if there is ever a cup with fill less than
$-h\sqrt{n}/(\log\log n)^{2/3}$ then, in order to make the absolute fill of this cup
non-negative, the absolute average fill of all the cups must be at least
$h\sqrt{n}/(\log\log n)^{2/3}$. Hence there would exist an overpowered cup. Thus we
have that without loss of generality all cups have fill bounded in magnitude by
$h\sqrt{n}/(\log\log n)^{2/3}$.

Now we detail how to achieve Step 1.
We set apart a constant fraction $\alpha n= n/1000$ of the cups in $A$, which
we call the \defn{storage-block}, or simply $C$. Let $\gamma = (\lg\lg
n)^{1/3}$; we will store between
$\gamma -\gamma/2$ and $\gamma + \gamma/2$ sets of $2\beta = \alpha n / (\gamma +
\gamma/2)$ randomly chosen cups at certain randomly chosen rounds.
For each anchor cup $c$ we will perform a procedure called a \defn{swapping-process}. 
With probability $\frac{\gamma}{n/2}$ the swapping-processes consists of what
we term a \defn{storing-operation}; in a storing-operation the filler takes a 
random subset of $\beta$ cups from $B$, and a random subset of $\beta$ cups
from $A \setminus C$, and swap these cups into the storage-block.
Note that, by a Chernoff bound, the probability that the number of
swapping-processes where we perform a storing-operation lies within $\pm
\gamma/2$ of its expectation $\gamma$ is at least $1-2e^{-n(\gamma/2)^2}$,
which is better than exponentially small in $n$, and in particular much
better than the $1-1/\polylog n$ probability of success that we need.
Thus we assume that the number of swapping-processes that consist of a
storing-operation is between $\gamma- \gamma/2$ and $\gamma+\gamma/2$.

On the other hand, with probability $1-\frac{\gamma}{n/2}$ the swapping-process
does not entail performing a storing-operation; in this case the
swapping-process is composed of a substructure, repeated many times, which we
call a \defn{round-block}, which is a set of consecutive rounds. At the beginning of
such a swapping-process we choose a round-block $j \in [n^2]$ uniformly at
random from all the round-blocks. The swapping-process proceeds for $n^2$
round-blocks; on the $j$-th round-block we swap a cup into the anchor set.

For each round-block $i \in [n^2]$, the filler selects a random subset $D_i\subset
B$ of the non-anchor cups and plays a single processor cup game on $D_i$. In this
single-processor cup game the filler employs the classic adaptive strategy for
achieving backlog $\Omega(\log |B|)$ on a set of $|B|$ cups, however modified
because it is an oblivious filler. In particular, the filler's strategy in the
single-processor cup games is to distribute water equally among an \defn{active
set} of cups, and then after the emptier removes water from some cup the filler
removes a random cup from the active set. There is at least constant
probability that this results in the active set having a single cup at the end,
with fill that has increased by at least $1/|B| + 1/(|B|-1) + \ldots + 1/1 \ge
\ln |B|$ since the start of the round-block.

On most round-blocks -- all but the $j$-th -- the filler does nothing with the
cup that it achieves in the active set at the end of the single processor cup
game. However, on the $j$-th round-block the filler swaps the winner of the
single processor cup game into the anchor set.

We consider two cases:
\begin{itemize}
  \item \textbf{Case 1:} For at least $1/2$ of the swapping-processes, at
    least $1/2$ of the cups $c \in B$ have $\fil(c) \ge -384h$.
  \item \textbf{Case 2:} For at least $1/2$ of the swapping-processes, less
    than $1/2$ of the cups $c \in B$ have $\fil(c) \ge -384h$.
\end{itemize}
We now prove that in either case we can achieve high positive tilt in $A$ with
good probability.

\begin{clm} \label{clm:reg} 
  Let $q\ge \Omega(1)$ be an appropriately small constant ($q$ is a function of
  $h\le O(1)$). In Case 1, with probability at least $1-e^{-nq^2/1024}$, we
  achieve fill at least $h$ in at least $nq/16$ of the cups in $A$ (i.e. a
  constant fraction of the cups in $A$). 
\end{clm}
\begin{proof}
  Consider a swapping-process where the filler does not perform a
  storing-operation where at least $1/2$ of the cups $c \in B$ have $\fil(c)
  \ge -96h$. Note that by assumption there are at least $n/4 - \gamma\cdot 3/2$ such rounds.
 
  Say the emptier \defn{neglects} the anchor set in a round-block if on at
  least one round of the round-block the emptier does not empty from every
  anchor cup. By playing the single-processor cup game for $n^2$ round-blocks,
  with only one round-block when we actually swap a cup into the anchor set, we
  strongly disincentive the emptier from neglecting the anchor set on more
  than a constant fraction of the round-blocks. 

  The emptier must have some function $I(i): [n^2] \to \{0,1\}$ that indicates whether or
  not they will neglect the anchor set on round-block $i$ if the filler has not
  already swapped. Note that the emptier may know when the filler performs a
  swap, so whether or not the emptier neglects a round-block $i$ depends on
  this information. However, $j$ is the only parameter of the swapping-process,
  so there is no other information that the emptier can use to decide whether
  or not to neglect a round-block. Note that $I$ could be generated randomly,
  but it still must exist. 

  If the emptier is willing to neglect the anchor set for at least $1/2$ of the
  round-blocks, i.e. $\sum_{i=1}^{n^2} I(i) \ge n^2 / 2$, then with probability
  at least $1/4$, $j \in ((3/4) n^2, n^2)$, in which case the emptier neglects
  the anchor set on at least $n^2/4$ round-blocks ($I(k)$ must be $1$ for at
  least $n^2/4$ of the first $(3/4)n^2$ round-blocks). Each time the emptier
  neglects the anchor set the mass of the anchor set increases by at least $1$.
  Thus the average fill of the anchor set will have increased by at least
  $(n^2/2)/(n/2) \ge \Omega(n)$ over the entire swapping-process in this
  case, implying that we achieve an overpowered cup and satisfy the proposition. 

  Otherwise, there is at least a $1/2$ chance that the round-block $j$, which
  is chosen uniformly at random from the round-blocks, when the filler performs
  a swap into the anchor set occurs on a round-block with $I(j)=0$, indicating
  that the emptier won't neglect the anchor set on round-block $j$. In this
  case, the round-block was a legitimate single processor cup game on $D_j$,
  the randomly chosen set of $\lceil e^{385h} \rceil$ cups on the $j$-th round.
  Then we achieve fill increase of at least $\ln \lceil e^{385h} \rceil \ge 385h$
  by the end of the round-block with probability at least $1/\lceil
  e^{385h}\rceil!$ -- the probability that we correctly guess the sequence of
  cups within the single processor cup game that the emptier empties from. 

  The probability that the random set $D_j \subset B$ contains only cups that
  are among the $n/4$ fullest cups in $B$ is $${n/2 \choose {\lceil e^{97h}
  \rceil}} / {n \choose {\lceil e^{385h}\rceil}} = O(1).$$ Note that because, by
  assumption, at least half of the cups $c \in B$ have $\fil(c) \ge -384h$, then
  the $n/4$ fullest cups in $B$ must have fill at least $-384h$. If all cups $
  c\in D_j$ have $\fil(c) \ge -384h$, then the fill of the cup in the active set
  at the end of the round-block is at least $-384h + 385h = h$, if the filler
  guesses the emptier's emptying sequence correctly.

  Say that a swapping-process where at least half of the cups $c\in B$ have
  $\fil(c) \ge -384h$ \defn{succeeds} if $D_j$ is a subset of the $n/4$ fullest
  cups in $B$, and if the filler correctly guesses the emptier's emptying
  sequence. Note that if a swapping-process succeeds, then the filler is able
  to swap a cup with fill at least $h$ into $A$. We have shown that there is a
  constant probability of a given swapping-process succeeding. Let $X_i$ be the
  binary random variable indicating whether or not the $i$-th swapping process
  where the filler does not perform a storing-operation where at least half of
  the cups $c\in B$ have $\fil(c) \ge -384h$ succeeds. Let $q \ge \Omega(1)$ be
  the probability of a swapping-process succeeding, i.e. $P(X_i=1)$. Note that
  the random variables $X_i$ are clearly independent, and identically
  distributed.

  Clearly $$\E\left[\sum_{i=1}^{n/8} X_i\right] = qn/8.$$ Note that we do not
  use all the $X_i$; we know there must be at least $n/4 - 3/2 \gamma$
  swapping-processes that do not consist of a storing-operation, but only use
  $n/8$ of the $X_i$. We make this choice because the particular constants that
  we get do not matter, and because it simplifies the analysis.
  By a Chernoff Bound (i.e. Hoeffding's Inequality applied to binary random variables),
  $$P\left(\sum_{i=1}^{n/8} X_i\le nq/16\right) \le e^{-nq^2/1024}.$$ That is, the
  probability that less than $nq/16$ of the anchor cups have fill at least $h$ is
  exponentially small in $n$, as desired.

\end{proof}

\begin{clm}
  \label{clm:xtreme}
  In Case 2, with probability at least $1- 1/\polylog(n)$, we achieve positive
  tilt $n \alpha \cdot h$ in the anchor set.
\end{clm}

\begin{proof}
  Consider a swapping-process on which the filler does a storing-operation and
  at least half of the cups $c \in B$ have $\fil(c) < -384h$. Then there is a
  set of cups with no more than $-384h \cdot n/4 = -nh\cdot 96$.

  If the storage-block $C$ already has positive tilt at least $nh\cdot 48 \ge
  n\alpha \cdot h$ then the filler has already succeeded, as no water ever
  exits the storage-block.

  Otherwise, at least one of $A$ or $B$ must have large positive tilt in order
  to offset the many cups in $B$ with fill less than $-96h$. In particular, the
  positive tilt of $A \cup B \setminus C$ must be at least $nh\cdot 96 - nh\cdot 48 =
  nh\cdot 48$. Let $D=B$ if $\tilt(B) > \tilt(A)$ and $D=A$ otherwise. Note that
  $\tilt(D) \ge nh\cdot 24$. Because there are no overpowered cups, this positive
  tilt is distributed among many cups. Note that if $X$ is a cup chosen
  uniformly randomly from $D$ then $$\E[\tilt(X)] \ge h\cdot 48.$$ Let
  $Y_1,Y_2,\ldots, Y_\beta$ be the positive tilts of a set of $\beta$ cups
  drawn from $D$, with all subsets equally likely; equivalently, consider
  $Y_1,\ldots, Y_\beta$ to be the positive tilts of $\beta$ cups sampled from
  $D$, sampling without replacement. Recall that $\beta =
  \frac{1}{2}\frac{\alpha n}{\gamma\cdot 3/2}$, $\gamma = (\lg \lg n)^{1/3}$.
  We apply Hoeffding's Inequality for samples drawn with replacement (stated in
  Corollary \ref{cor:hoeffdingwreplacement}) to $\sum_{i=1}^\beta Y_i$. Note
  that $0\le Y_i \le h\sqrt{n}/(\log\log n)^{2/3}$, as positive tilt is
  non-negative, and as there are no overpowered cups. 
  Thus we have, 
  \begin{align*}
  P\left(\frac{1}{\beta} \sum_{i=1}^{\beta} (Y_i - \E[Y_i]) \le -h\cdot 24 \right) \le\\
  \exp\left(-\frac{2\beta(h\cdot 24)^2}{(h\sqrt{n}/(\log\log n)^{2/3})^2}\right).
  \end{align*}
  Simplifying this gives,
  $$P\left(\frac{1}{\beta}\sum_{i=1}^{\beta} Y_i \le h\cdot 24\right) \le \frac{1}{\polylog(n)}.$$

  Now we will show that with probability at least $1-1/\polylog(n)$ there are
  at least $\gamma/4$ swapping-processes on which the filler does a
  storing-operation.
  Recall that at least $\gamma/2$ storing-operations happen. Choose a random
  set of $\gamma/2$ of the storing-operations that happen. Note that the
  distribution of these sets will clearly be identical to the distribution we
  would get if we simply selected a random subset of $\gamma/2$ of the
  swapping-process, or equivalently, if we had sampled $\gamma/2$ of the
  swapping-processes sampling without replacement. Let $Z_i$ be the indicator
  random variable for whether or not the $i$-th of these swapping-processes occurs on a round
  where at least half of the cups $c \in B$ have $\fil(c) < -384h$.
  Clearly $\E\left[\sum_{i=1}^{\gamma/2} Z_i\right] \ge \gamma/4.$ 
  Then a Chernoff bound gives that
  $$P\left(\sum_{i=1}^{\gamma/2} Z_i \ge \gamma/8 \right) \ge 1-e^{-2(\gamma/2) (\gamma/8)^2},$$
  and recalling that $\gamma=(\lg\lg n)^{1/3}$ this simplifies to 
  $$P\left(\sum_{i=1}^{\gamma/2} Z_i \ge \gamma/8 \right) \ge 1-\frac{1}{\polylog(n)}.$$

  Combining these, with probability at least
  $$\left(1-\frac{1}{\polylog(n)}\right)\left(1-\frac{1}{\polylog(n)}\right) =
  1-\frac{1}{\polylog(n)}$$ we achieve positive
  tilt at least $\beta h\cdot 24$ in at least $\gamma/8$ sets of $\beta$ cups. In
  total, this means that means we have positive tilt $(\beta h \cdot 24)(\gamma/8) =
  \frac{\alpha n/2}{(3/2)\gamma}\gamma h \cdot{24}{8} = n\alpha h$ in the anchor set, as desired.

\end{proof}

  By Claim \ref{clm:reg} in Case 1 the filler achieves, with probability at least
  $1-1/\polylog n$ (in fact with much better probability) fill at least $h$ in
  some constant fraction $nc$ of the cups.

  In Case 2 by Claim \ref{clm:xtreme} the filler achieves, with probability at least
  $1-1/\polylog n$, positive tilt at least $nh\alaph$ in the storage-block,
  which consists of $\alpha n$ cups. Then 
  $$\sum_{x\in C} \lfloor \tilt(x) / (h/2) \rfloor \ge \sum_{x\in C} (\tilt(x) / (h/2) -1) = n\alpha.$$

  Using the cups that have fill chunks of $h$, setting the number of processors to $1$, and
  exploiting the greedy nature of the emptier, the filler can obtain high fill
  in a set of $nc$ \emph{known} cups.

  In particular, the filler chooses a set of $nc$ cups randomly.
  With probability at least $1/2$ the average fill of $nc$ is positive by a
  Hoeffding bound and the fact that 
  OH CRAP THIS JUST BROKE. YOU COULD SMOOTH IT. BUT THATS NOT FUN.
  The filler repeatedly distributes a unit of water equally among these $nc$ cups. 
  The filler continues until the average fill of that set of cups has increased
  by $h'$. The filler uses one processor because it doesn't know how many cups
  the positive tilt is concentrated in. Then the filler recurses on the set of
  $nc$ known cups with average fill.

  Note that this transformation from a set with high positive tilt to a set of
  known cups with high positive average fill is the only part of the proof of
  Proposition \ref{prop:obliviousBase} that is specific to a greedy-like
  emptier. Against a general emptier it is not true that the emptier will
  necessarily focus on the set of cups with high positive tilt; an arbitrary
  emptier can of course foil our attempts to achieve high fill in any fixed set
  of $p$ cups, at a given setting of $p$. Extending Proposition
  \ref{prop:obliviousBase} to apply to non-greedy-like emptiers is an important
  open question.
\end{proof}




%%%  proving that the choice epsilon =1/n is valid
%%% I think I don't need this because it's obvious that it's all
%%% good in the inequality and I'm going to have to fix the
%%% domain set stuff anyways

\begin{clm}
  \label{clm:validchoices}
  There exists $\delta$ that is small enough to make $h(\delta)
  \ge 1$ and large enough to make $n\delta(1-\delta) \ge
  n_0$, when $\epsilon$ is chosen to be $4/\lg n$, or a positive constant. 

  In particular, if $\epsilon$ is chosen to be $4/\lg n$ then we will choose
  $\delta =\Theta(1/n)$, and if $\epsilon$ is chosen to be a positive
  constant then we will choose $\delta = \Theta(1)$.
\end{clm}

Note that in order for $L =1$ to make sense it must be that
$n(1-\delta)\delta n \ge n_0$, or else this term from the sum would be
contributing essentially $0$ backlog to the sum.
If $L=1$ and if $h(\delta)\ge 1$, then $f'(k_0) \ge c k_0^{1-\epsilon}$, meaning
we have constructed from $f$ a new function $f'$ that satisfies the
inequality $f'(k) \ge ck^{1-\epsilon}$ for $k\le g/(1-\delta)$, as opposed to
only for $k \le g$ as in the case of $f$. 
\footnote{Note that although $f'(k) \ge ck^{1-\epsilon}$ holds for at least
  as many $k$ as $f(k) \ge c k^{1-\epsilon}$, it does not necessarily hold
  for strictly more; in particular, if $\lfloor g/(1-\delta) \rfloor = g$
  then the inequality on $f'$ holds for no more $k$ than the inequality on
  $f$, as $f$ and $f'$ are functions on $\mathbb{N}$. In general we have to
  be somewhat careful about the fact that there are an integer number of cups
  throughout this proof (this issue was deferred from earlier proofs to be
dealt with here).} 
Thus by repeatedly amplifying a function, we should be able to arbitrarily
extend the region where the function satisfies the desired inequality, which
will allow us to attain the desired backlog.
We now prove Claim \ref{clm:validchoices}.
\begin{proof}
  First we show that making $h(\delta) > 1$ is possible.
Consider the Taylor series for $(1-\delta)^{2-\epsilon}$ about $\delta = 0$:
$$(1-\delta)^{2-\epsilon} = 1 - (2-\epsilon)\delta + O(\delta^2).$$

So, to find a $\delta$ where $h(\delta) \ge 1$ it suffices -- note that we
choose to neglect the $\delta^2$ term as it does not strengthen the lower
bound substantially -- to find a $\delta$ with
$$(1-(2-\epsilon)\delta)(1+\delta^{1-\epsilon}) \ge 1.$$
Rearranging we have 
$$\delta^{1-\epsilon} \ge (2-\epsilon)\delta + (2-\epsilon)\delta^{2-\epsilon}.$$
This clearly is true for sufficiently small $\delta$, as
$\delta^{1-\epsilon}$ will be much greater than $\delta$ or
$\delta^{2-\epsilon}$.
However it will be beneficial to have a more explicit criterion for possible
choices of $\delta$ in terms of $\epsilon$. To get this, we enforce a 
stronger inequality on $\delta^{1-\epsilon}$ by overestimating
$\delta^{2-\epsilon}$ as $\delta$. 
Then, $\delta$ satisfying
\begin{equation}
  \label{eqn:deltaUpperIneq}
  \delta \le \frac{1}{(2(2-\epsilon))^{1/\epsilon}}
\end{equation}
will make $h(\delta) \ge 1.$

In addition to the constraint that $\delta$ must be small enough such that
$h(\delta) \ge 1$, the only other constraint on $\delta$ is that $\delta$
must be large enough that the sum from the Amplification Lemma can have at least
two terms, i.e. such that $L \ge 1$. We need $L\ge 1$ because otherwise the
Amplification Lemma doesn't give a larger function.
That is, we want
$$n\delta(1-\delta) \ge n_0. $$
Recall that we choose $\delta < 1/2$, so $1-\delta > 1/2$. Thus to make
$\delta$ sufficiently big it suffices to chose $\delta$ with 
\begin{equation}
  \label{eqn:deltaLowerBound}
  \delta \ge 2n_0/n.
\end{equation}
Any choice of $\delta$ that is sufficiently large to make $L \ge 1$ and
simultaneously small enough to make $h(\delta) \ge 1$ is a valid choice of
$\delta$. That is, $\delta$ is valid if and only if it satisfies
\begin{equation}
  \label{eqn:deltainequality}
  \frac{2n_0}{n} \le \delta \le  \frac{1}{(2(2-\epsilon))^{1/\epsilon}}.
\end{equation}
To achieve the desired backlog of $\Omega(n)$ we can use $\epsilon =
\gamma/\lg n$ for appropriate constant $\gamma$, as $$n^{1-\gamma/\lg n} =
n/2^\gamma = \Omega(n).$$
We show that there is a valid choice of $\gamma$ such that the
following inequality is satisfied:
\begin{equation}
  \label{eqn:thatinequality}
 2n_0/n \le \frac{1}{(2(2-\gamma/\lg n))^{(1/\gamma)\lg n}}.
\end{equation}
Note that 
$$(2(2-\gamma/\lg n))^{(1/\gamma)\lg n} \le 4^{(1/\gamma)\lg n} \le n^{2/\gamma}.$$
Thus, clearly by choosing e.g. $\gamma = 4$ we have the desired inequality.
Inequality \ref{eqn:thatinequality} implies that there is a valid choice of
$\delta$ when we chose $\epsilon = \gamma / \lg n$. When proving that we can
achieve backlog $\Omega(n)$ we use $\epsilon = 4 / \lg n$, and $\delta =
O(1/n)$ satisfying Inequality \ref{eqn:deltainequality} for our choice
of $\epsilon$. When proving that we can achieve backlog
$\Omega(n^{1-\epsilon})$ for constant $\epsilon > 0$ we choose $\delta > 0$ to be
a constant satisfying Inequality \ref{eqn:deltaUpperIneq}, and $\delta$,
being constant, trivially satisfies Inequality \ref{eqn:deltaLowerBound}.
  
\end{proof}

