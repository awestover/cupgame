\section{Oblivious Filler Lower Bound}\label{sec:oblivious}
In this section we prove that, surprisingly, an oblivious filler
can achieve backlog $n^{1-\varepsilon}$, although only against a
certain class of ``greedy-like" emptiers.

The \defn{fill-range} of a set of cups at a state $S$ is $\max_c
\fil_S(c) - \min_c \fil_S(c)$. We call a cup configuration
\defn{$R$-flat} if the fill-range of the cups less than or equal to
$R$; note that in an $R$-flat cup configuration with average fill
$0$ all cups have fills in $[-R, R]$. We say an emptier is
\defn{$\Delta$-greedy-like} if, whenever there are two cups with
fills that differ by at least $\Delta$, the emptier never empties
from the less full cup without also emptying from the more full
cup. That is, if on some round $t$, there are cups $c_1, c_2$
with $\fil_{I_t}(c_1) > \fil_{I_t}(c_2) + \Delta$, then a
$\Delta$-greedy-like emptier doesn't empty from $c_2$ on round
$t$ unless it also empties from $c_1$ on round $t$. Note that a
perfectly greedy emptier is $0$-greedy-like. We call an emptier
\defn{greedy-like} if it is $\Delta$-greedy-like for $\Delta \le
O(1)$. 

With an oblivious filler we are only able to prove lower bounds
on backlog against greedy-like emptiers; whether or not our
results can be extended to a more general class of emptiers is an
interesting open question. Nonetheless, greedy-like emptiers are
of great interest because all the known randomized algorithms for
the cup game are greedy-like \cite{mbe19, wku20}.

As a tool in our analysis we define a new variant of the cup
game: In the $p$-processor \defn{$E$-extra-emptyings}
\defn{$S$-skip-emptyings} negative-fill cup game on $n$ cups, the filler
distributes $p$ units of water amongst the cups, and then the
emptier empties from $p$ \textit{or more, or less} cups. In
particular the emptier is allowed to do $E$ extra
emptyings---we say that the emptier does an extra emptying if
it empties from a cup beyond the $p$ cups it typically is
allowed to empty from---and is also allowed to skip $S$
emptyings---we say that the emptier skips an emptying if it
doesn't do an emptying---over the course of the
game. Note that the emptier still cannot empty from the same cup
twice on a single round, and also that note
that a $\Delta$-greedy-like emptier must take into account extra
emptyings and skip emptyings to determine valid moves. Further,
note that the emptier is allowed to skip extra emptyings,
although skipping extra emptyings looks the same as if the
extra-emptyings had simply not been performed.
It may seem strange that we are limiting the number of times that
the emptier can skip-emptyings; in the regular cup game the
emptier is allowed to skip as many times as it wants. However,
this will turn out to be a useful idea.

For a $\Delta$-greedy-like emptier let $R_\Delta = 2(2+\Delta)$;
we now prove a key property of these emptiers: there is an
oblivious filling strategy, which we term \defn{$\flatalg$}, that
attains an $R_\Delta$-flat cup configuration against a
$\Delta$-greedy-like emptier, given cups of a known starting
fill-range.

% interesting(?) note:
% if you're willing to say "screw running time" (which we are, so
% long as it is still $2^{\polylog n}$, then we can definitely
% flatten, even if we don't know the cups are $R$-flat for some
% $R$, because the cups must be $M$-flat (i.e. 2^{\polylog(real
% n)}-flat or we're just done)

\begin{lemma}
  \label{lem:greedylikeisflat}
  Consider an $R$-flat cup configuration in the $p$-processor
  $E$-extra-emptyings $S$-skip-emptyings negative-fill cup game on $n =
  2p$ cups. There is an oblivious filling strategy
  \defn{$\flatalg$} that achieves an $R_\Delta$-flat
  configuration of cups against a $\Delta$-greedy-like emptier in
  running time $2(R + \ceil{(1+1/n)(E + S)})$. Furthermore,
  $\flatalg$ guarantees that the cup configuration is $R$-flat on every round.
\end{lemma}
\begin{proof}
  If $R \le R_\Delta$ the algorithm does nothing, since the
  desired fill-range is already achieved; for
  the rest of the proof we consider $R > R_\Delta$.

  The filler's strategy is to distribute fill equally amongst all
  cups at every round, placing $p/n = 1/2$ fill in each cup. 
  Let $\ell_t = \min_{c\in S_t} \fil_{S_t}(c)$, $u_t = \max_{c\in S_t} \fil_{S_t}(c)$. 

  First we show that the fill-range of the cups can only increase
  if the fill-range is very small.
  \begin{clm}
    \label{clm:fillRangeUpIFFfillRangeSmall}
    If $u_{t+1}-\ell_{t+1} > u_t - \ell_t$ then 
    $$u_{t+1} - \ell_{t+1} \le R_\Delta.$$
  \end{clm}
  \begin{proof}
    First we remark that the fill of any cup changes by at most
    $1/2$ from round to round, and in particular $|u_{t+1}-u_t|
    \le 1/2$, $|\ell_{t+1} - \ell_t|\le 1/2$.
  In order for the fill-range to increase, the emptier must have
  emptied from some cup with fill in $[\ell_t, \ell_t + 1]$ without
  emptying from some cup with fill in $[u_t-1, u_t]$; if the emptier
  had not emptied from every cup with fill in $[\ell_t, \ell_t+1]$
  then we would have $\ell_{t+1} = \ell_t + 1/2$ so the
  fill-range could not have increased, and similarly if the emptier
  had emptied from every cup with fill in $[u_t-1, u_t]$ then we
  would have $u_{t+1} = u_t - 1/2$ so again the fill-range could
  not have increased. Because the emptier is $\Delta$-greedy-like
  emptying from a cup with fill at most $\ell_t+1$ and not
  emptying from a cup with fill at least $u_t-1$ implies that
  $u_t-1$ and $\ell_t+1$ differ by at most $\Delta$.
  Thus, 
  $$u_{t+1} - \ell_{t+1} \le u_t +1/2 - (\ell_t -1/2)  \le \Delta + 3 \le R_\Delta.$$
  \end{proof}
  Because by
  \cref{clm:fillRangeUpIFFfillRangeSmall}
  whenever the fill-range of the cups increases it increases to a
  value bounded above by $R_\Delta \le R$, we have by induction
  that the fill-range of the cups never exceeds $R$, i.e. the
  cups are always $R$-flat. While
  \cref{clm:fillRangeUpIFFfillRangeSmall}
  does imply that the fill-range must decrease until the
  fill-range is at most $R_\Delta$, and once the fill-range is at
  most $R_\Delta$ it is always at most $R_\Delta$,
  \cref{clm:fillRangeUpIFFfillRangeSmall}
  does not preclude the possibility that the fill-range doesn't
  change for many rounds, or decreases by a very small amount.
  For this reason we actually do not use
  \cref{clm:fillRangeUpIFFfillRangeSmall}
  in the remainder of the proof; we proved this result because
  the fact that fill-range does not increase during $\flatalg$ is
  an important property of $\flatalg$. In the rest of the proof
  we establish that the fill-range indeed must eventually be at
  most $R_\Delta$.

  Let $L_t$ be the set of cups $c$ with $\fil_{S_t}(c) \le \ell_t+2+\Delta$, and let
  $U_t$ be the set of cups $c$ with $\fil_{S_t}(c) \ge u_t-2-\Delta$.

  Now we prove a key property of the sets $U_t$ and $L_t$: if a cup is in
  $U_t$ or $L_t$ it is also in $U_{t'}, L_{t'}$ for all $t' > t$. This
  follows immediately from \cref{clm:dontlosestuff}.
  \begin{clm}
    \label{clm:dontlosestuff}
    $$U_{t} \subseteq U_{t+1},\quad L_t \subseteq L_{t+1}.$$
  \end{clm}
  \begin{proof}
    Consider a cup $c\in U_t$.

    If $c$ is not emptied from, i.e. $\fil(c)$ has increased by
    $1/2$ from the previous round, then
    clearly $c \in U_{t+1}$, because backlog has increased by at most $1/2$, so
    $\fil(c)$ must still be within $2+\Delta$ of the backlog on round $t+1$. 

    On the other hand, if $c$ is emptied from, i.e. $\fil(c)$ has decreased by
    $1/2$, we consider two cases.\\
    \textbf{Case 1:} If $\fil_{S_t}(c) \ge u_t-\Delta -1$, then
    $\fil_{S_t}(c)$ is at least $1$ above the bottom of the
    interval defining which cups belong to $U_t$. The backlog
    increases by at most $1/2$ and the fill of $c$ decreases by $1/2$, so
    $\fil_{S_{t+1}}(c)$ is at least $1-1/2-1/2 = 0$ above the bottom of the
    interval, i.e. still in the interval. \\
    \textbf{Case 2:} On the other hand, if $\fil_{S_t}(c) <
    u_t-\Delta-1$, then every cup with fill in $[u_t-1, u_t]$
    must have been emptied from because the emptier is
    $\Delta$-greedy-like. Therefore the fullest cup
    on round $t+1$ is the same as the fullest cup on round $t$,
    because every cup with fill in $[u_t-1, u_t]$
    has had its fill decrease by $1/2$, and no cup with fill less than
    $u_t-1$ had its fill increase by more than $1/2$. Hence $u_{t+1}
    = u_t -1/2$. Because both $\fil(c)$ and the backlog
    have decreased by $1/2$, the distance between them is
    still at most $\Delta+2$, hence $c\in U_{t+1}$.\\
    The argument for why $L_t \subseteq L_{t+1}$ is symmetric.
  \end{proof}

  Now we show that under certain conditions $u_t$ decreases and
  $\ell_t$ increases.
  \begin{clm}
    \label{clm:howDoLandUchange}
    On any round $t$ where the emptier empties from at least
    $n/2$ cups, if $|U_t| \le n/2$ then $u_{t+1} = u_t - 1/2$.
    On any round $t$ where the emptier empties from at most $n/2$
    cups, if $|L_t| \le n/2$ then $\ell_{t+1} = \ell_t + 1/2$.
  \end{clm}
  \begin{proof}
    Consider a round $t$ where the emptier empties from at least
    $n/2$ cups. If there are at least $n/2$ cups outside of
    $U_t$, i.e. cups with fills in $[\ell_t, u_t-2-\Delta]$, then
    all cups with fills in $[u_t - 2, u_t]$ must be emptied from;
    if one such cup was not emptied from then by the pigeon-hole
    principle some cup outside of $U_t$ was emptied from, which
    is impossible as the emptier is $\Delta$-greedy-like. This
    clearly implies that $u_{t+1} = u_t - 1/2$: no cup with fill
    less than $u_t-2$ has gained enough fill to become the
    fullest cup, and the fullest cup from the previous round has
    lost $1/2$ unit of fill.

    By a symmetric argument $\ell_{t+1} = \ell_{t} + 1/2$ if the
    emptier empties at most $n/2$ cups on a round $t$ where
    $|L_t| \le n/2$. 
  \end{proof}

  Now we show that eventually $L_t \cap U_t \neq \varnothing$.
  \begin{clm}
    There is a round $t_0 \le 2(R + \ceil{(1+1/n)(E + S)})$ such that $U_{t}
    \cap L_{t} \neq \varnothing$ for all $t\ge t_0$.
  \end{clm}
  \begin{proof}
  We call a round where the emptier doesn't use $p=n/2$
  resources, i.e. a round where the number of skipped emptyings
  and the number of extra emptyings are not equal, an
  \defn{unbalanced round}; we call a round that is not unbalanced a
  \defn{balanced} round. 

  Note that there are clearly at most $E+S$ unbalanced rounds.
  We now associate some unbalanced rounds with balanced rounds;
  in particular we define what it means for a balanced round to
  \defn{cancel} an unbalanced round. We define cancellation by a
  sequential process. For $i = 1,2,\ldots,
  2(R+\ceil{(1+1/n)(E+S)})$ (iterating in ascending order of $i$), if round $i$
  is unbalanced then we say that the first balanced round $j > i$
  that hasn't already been assigned (earlier in the sequential
  process) to cancel another unbalanced round $i' < i$, if any
  such round $j$ exists, \defn{cancels} round $i$. Note that
  cancellation is a one-to-one relation: each unbalanced round is
  cancelled by at most one balanced round and each balanced round
  cancels at most one unbalanced round.

  Consider rounds of the form $2(R + \ceil{(E+S)/n}) + (E+S) + i$
  for $i \in [E+S+1]-1$. We claim that there is some such $i$
  such that among rounds $[2(R + \ceil{(E+S)/n}) + (E+S) + i]$
  every unbalanced round has been cancelled, and such that there
  are $2(R+\ceil{(E+S)/n})$ balanced rounds not cancelling other
  rounds. Assume for contradiction that such an $i$ does not
  exist. Note that there are at least $2(R + \ceil{(E+S)/n})$
  balanced rounds in the first $2(R + \ceil{(E+S)/n}) + (S+E)$
  rounds. Thus every balanced round $2R+(E+S)+\ceil{(E+S)/n} +
  i-1$ for $i \in [E+S+1]$ is necessarily a cancelling round, or
  else there would be a round by which there are no uncancelled
  unbalanced rounds. Hence by round $2(R + \ceil{(E+S)/n}) + 2(E +
  S)$, there must have been $E+S$ cancelled rounds, so on round
  $2(R + \ceil{(E+S)/n}) + 2(E+S)$ all unbalanced rounds are
  cancelled, which leaves $2(R + \ceil{(E+S)/n})$ balanced rounds
  that are not cancelling any rounds, as desired.

  Let $t_e$ be the first round by which there are $2(R +
  \ceil{(E+S)/n})$ balanced non-cancelling rounds. Note that the
  average fill of the cups cannot have decreased by more than
  $E/n$ from its starting value; similarly the average fill of
  the cups cannot have increased by more than $S/n$. Because the
  cups start $R$-flat, we have that $u_t$ cannot have decreased
  by more than $R + E/n$ or else $u_t$ would necessarily be below
  the average fill, and identically $\ell_t$ cannot have
  increased by more than $R + S/n$ or else it would be above the
  average fill. Now, by \cref{clm:howDoLandUchange} we have that
  eventually $|L_t| > n/2$: if $|L_t|\le n/2$ were always true,
  then on every balanced round $\ell_t$ would have increased by
  $1/2$, and since $\ell_t$ increases by at most $1/2$ on
  unbalanced rounds, this implies that in total $\ell_t$ would
  have increased by at least $(1/2)2(R + \ceil{(E+S)/n})$, which
  is impossible. By a symmetric argument it is impossible that
  $|U_t| \le n/2$ for all rounds. 

  Since $|U_{t+1}|\ge |U_t|$ and $|L_{t+1}| \ge |L_t|$ by
  \cref{clm:dontlosestuff}, we have that there is some round $t_0
  \in [2(R + \ceil{(1+1/n)(E+S)})]$ such that for all $t \ge t_0$
  we have $|U_t|> n/2$ and $|L_t|> n/2$. But then we
  have $U_t\cap L_t \neq \varnothing$, as desired.
  \end{proof}

  If there exists a cup $c \in L_t\cap U_t$, then 
  $$\fil(c) \in [u_t-2-\Delta, u_t] \cap [\ell_t, \ell_t + 2 +
  \Delta].$$ Hence we have that $$\ell_t+2+\Delta \ge
  u_t-2-\Delta.$$ Rearranging, $$u_t - \ell_t \le 2(2+\Delta) =
  R_\Delta.$$ Thus the cup configuration is $R_\Delta$-flat by
  the end of this flattening process.

\end{proof}

% note: may as well take $\Delta' = \ceil{\Delta} \in
% \mathbb{N}$, this might be useful sometimes

Next we describe a simple oblivious filling strategy, that we
call \defn{$\randalg$}, that will be used as a subroutine in
\cref{lem:obliviousManyUnknownCups}; variants of this strategy
are well-known, and similar versions of it can be found in \cite{
mbe19, mbe15, die91, wku20}.
\begin{proposition}
  \label{prop:obliviousTerribleProbability}
  Consider an $R$-flat cup configuration in the single-processor
  $\infty$-extra-emptyings $\infty$-skip-emptyings negative-fill cup
  game on $n$ cups with initial average fill $\mu_0$.
  Let $k \in [n]$ be a parameter. Let $d = \sum_{i=2}^k 1/i$.

  There is an oblivious filling strategy \defn{$\randalg(k)$}
  with running time $k-1$ that achieves fill at least $\mu_0 -R +
  d$ in a known cup $c$ with probability at least $1/k!$ if
  we condition on the emptier not performing extra emptyings.
  $\randalg(k)$ achieves fill at most $\mu_0 + R + d$ in this cup
  (unconditionally).

  Furthermore, when applied against a $\Delta$-greedy-like emptier
  with $R = R_\Delta$, $\randalg(k)$ guarantees that
  the cup configuration is $(R + d)$-flat on every round
  (unconditionally).
\end{proposition}
\begin{proof}
  First we condition on the emptier does not using extra emptying
  and show that in this case the filler has probability at least
  $1/(k-1)!$ (which we lower bound by $1/k!$ for sake of
  simplicity) of attaining a cup with fill at least $\mu_0 -R +d$.
  The filler maintains an \defn{active set}, initialized to being
  an arbitrary subset of $k$ of the cups. Every round the filler
  distributes $1$ unit of fill equally among all cups in the
  active set. Next the emptier removes $1$ unit of fill from some
  cup, or skips its emptying. Then the filler removes a random
  cup from the active set (chosen uniformly at random from the
  active set). This continues until a single cup $c$ remains in
  the active set. 

  We now bound the probability that $c$ has never been emptied
  from. Assume that on the $i$-th step of this process, i.e. when
  the size of the active set is $n-i+1$, no cups in the active
  set have ever been emptied from; consider the probability that
  after the filler removes a cup randomly from the active set
  there are still no cups in the active set that the emptier has
  emptied from. If the emptier skips its emptying on this round,
  or empties from a cup not in the active set then it is
  trivially still true that no cups in the active set have been
  emptied from. If the cup that the emptier empties from is in
  the active set then with probability $1/(k-i+1)$ it is evicted
  from the active set, in which case we still have that no
  cup in the active set has ever been emptied from. Hence with
  probability at least $1/(k-1)!$ the final cup in the
  active set, $c$, has never been emptied from. In this case, $c$
  will have gained fill $d=\sum_{i=2}^k 1/i$ as claimed. Because
  $c$ started with fill at least $-R+\mu_0$, $c$ now has fill at
  least $-R+ d+\mu_0$. 

  Now note that regardless of if the emptier uses extra emptyings
  $c$ has fill at most $\mu_0 + R + d$ , as $c$ starts with fill
  at most $R$, and $c$ gains at most $1/(k-i+1)$ fill on the
  $i$-th round of this process. 

  Now we analyze this algorithm specifically for a
  $\Delta$-greedy-like emptier. Consider a round $t$ on which
  $\min_c \fil_{S_{t+1}}(c) < \min_c \fil_{S_{t}}(c)$, which
  implies that a cup $c_1$ with $\fil_{S_t}(c_1) < \min_c
  \fil_{S_t}(c) +1$ was emptied from on round $t$, and also on
  which a cup $c_0$ that has $\fil_{S_{t+1}}(c_0) = \max_{c}
  \fil_{S_{t+1}}(c)$ was not emptied from on round $t$. Because
  the emptier is $\Delta$-greedy-like this
  implies that $\fil_{I_t}(c_0) - \fil_{I_t}(c_1) \le \Delta$
  and then $\max_c \fil_{S_{t+1}}(c) - \min_c
  \fil_{S_{t+1}}(c) \le \Delta + 2$, i.e. the cups are $(\Delta +
  2)$-flat. 

  Consider some round $t_1$ on which the cups are not $(\Delta +
  2)$-flat; let $t_0$ be the last round on which the cups were
  $R$-flat (note that if the cups are $(\Delta+2)$-flat they are
  also $R$-flat as $\Delta + 2 < R$). Consider how the fill-range
  of the cups changes during the set of rounds
  $t$ with $t_0 < t \le t_1$. On any such round $t$ either $\min_c
  \fil_{S_{t+1}}(c) \ge \min_c \fil_{S_{t}}(c)$ in which case the
  fill-range increases by at most $1/(k-t+1)$ where $k-t+1$ is the
  size of the active set on round $t$, or all cups on round $t+1$
  with fill equal to the backlog were emptied from, meaning that
  backlog decreased by at least $1-1/(k-t+1)$. In either case the
  fill-range increases by at most $1/(k-t+1)$. Thus in total the
  fill-range is at most $R + d$. That is, the cups are
  $(R+d)$-flat on round $t_1$, as desired.

\end{proof}
