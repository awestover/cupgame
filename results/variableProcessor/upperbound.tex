\section{Upper Bound}\label{sec:upperBound}

In this section we analyze the \defn{greedy emptier}, which always empties
from the $p$ fullest cups. We prove in \cref{cor:upperbound} that the
greedy emptier prevents backlog from exceeding $O(n)$. 

In order to analyze the greedy emptier, we establish a system of invariants
that hold at every step of the game. 

Let $\mu_S(X)$ and $m_S(X)$ denote the average fill and the mass, respectively,
of a set of cups $X$ at state $S$ (where $S=S_t$ or $S=I_t$).\footnote{Note that
in the lower bound proofs (i.e. \cref{sec:adaptive} and 
\cref{sec:oblivious}) when we use the notation $m$ (for mass) and $\mu$ (for
average fill), we omit the subscript indicating the state at which the
properties are measured. In those proofs the state is implicitly clear.
However, in this section it is necessary to make the state $S$ explicit in
the notation.} We will use concatenation of sets to
denote unions, i.e. $AB = A\cup B$. 

The main result of the section is the following theorem.  
\begin{theorem}
  \label{thm:invariant}
  In the variable-processor cup game on $n$ cups, the greedy emptier maintains, at every step $t$,
  the invariants
  \begin{equation}
    \label{eq:invariants}
      \mu_{S_t}(S_t([k])) \le 2n-k
  \end{equation}
  for all  $k \in [n]$.
\end{theorem}

By applying \cref{thm:invariant} to the case of $k = 1$, we arrive at a bound on backlog:
\begin{corollary}
  In the variable-processor cup game on $n$ cups, the greedy
  emptying strategy never lets backlog exceed $O(n)$.
  \label{cor:upperbound}
\end{corollary}

\begin{proof}[Proof of \cref{thm:invariant}]
We prove the invariants by induction on $t$.
The invariants hold trivially for $t=1$ (the base case for the inductive proof): 
the cups start empty so $\mu_{S_1}(S_1([k])) = 0 \le 2n-k$ for all $k \in [n]$.

Fix a round $t \ge 1$, and any $k \in [n]$. We assume the invariants for all
values of $k' \in[n]$ for state $S_t$ (we will only explicitly use two of the
invariants for each $k$, but the invariants that we need depend on the
choice of $p_t$ by the filler) and show that
the invariant on the $k$ fullest cups holds on round $t+1$, i.e. that
$$\mu_{S_{t+1}}(S_{t+1}([k])) \le 2n-k.$$

Note that because the emptier is greedy it always empties from the cups
$I_t([p_t])$. Let $A$, with $a=|A|$, be $A = I_t([\min(k, p_t)]) \cap
S_{t+1}([k])$; $A$ consists of the cups that are among the $k$ fullest cups in
$I_t$, are emptied from, and are among the $k$ fullest cups in $S_{t+1}$. Let
$B$, with $b=|B|$, be $I_t([\min(k, p_t)]) \setminus A$; $B$ consists of the
cups that are among the $k$ fullest cups in state $I_t$, are emptied from, and
are not among the $k$ fullest cups in $S_{t+1}$. Let $C = I_t(a+b+[k-a])$, with
$c=k-a = |C|$; $C$ consists of the cups with ranks $a + b + 1, \ldots, k + b$
in state $I_t$. The set $C$ is defined so that $S_{t+1}([k]) = AC$, since once
the cups in $B$ are emptied from, the cups in $B$ are not among the $k$ fullest
cups, so cups in $C$ take their places among the $k$ fullest cups.

Note that $k-a \ge 0$ as $a+b \le k$, and also $|ABC| = k+b \le n$, because by
definition the $b$ cups in $B$ must not be among the $k$ fullest cups in state
$S_{t+1}$ so there are at least $k+b$ cups. 
Note that $a + b = \min(k, p_t)$. We also have that $A = I_t([a])$ and $B =
I_t(a+[b])$, as every cup in $A$ must have higher fill than all cups in $B$ in
order to remain above the cups in $B$ after $1$ unit of water is removed from
all cups in $AB$.

We now establish the following claim, which we call the \defn{interchangeability of cups}:
\begin{clm}
  \label{clm:interchangable}
  There exists a cup state $S_t'$ such that: (a) $S_t'$ satisfies the
  invariants \eqref{eq:invariants}, (b) $S_t'(r) = I_t(r)$ for all ranks
  $r\in[n]$, and (c) the filler can legally place water into cups in order to
  transform $S_t'$ into $I_t$. 
\end{clm}
\begin{proof}
  Fix $r \in [n]$. We will show that $S_t$ can be transformed into a state
  $S_t^r$ by relabelling only cups with ranks in $[r]$ such that (a) $S_t^r$
  satisfies the invariants \eqref{eq:invariants}, (b) $S_t^r([r]) = I_t([r])$
  and (c) the filler can legally place water into cups in order to transform
  $S_t^r$ into $I_t$.

Say there are cups $x, y$ with $x\in S_t([r]) \setminus I_t([r]), y \in
 I_t([r])\setminus S_t([r])$. Let the fills of cups $x,y$ at state $S_t$
 be $f_x, f_y$; note that 
 \begin{equation}
     f_x > f_y.
     \label{eq:fxfy}
 \end{equation} Let the amount of fill that the filler
 adds to these cups be $\Delta_x, \Delta_y \in [0,1]$; note that 
 \begin{equation}
 f_x +\Delta_x <f_y + \Delta_y.
 \label{eq:fxdxfydy}
 \end{equation}
 
Define a new state $S_t'$ where cup $x$ has fill $f_y$ and cup $y$ has fill
$f_x$. Note that the filler can transform state $S_t'$ into state $I_t$ by
placing water into cups as before, except changing the amount of water placed
into cups $x$ and $y$ to be  $f_x-f_y+\Delta_x$ and $f_y-f_x + \Delta_y$,
respectively.

In order to verify that the transformation from $S_t'$ to $I_t$ is a valid step
for the filler, one must check three conditions. First, the amount of water
placed by the filler is unchanged: this is because $(f_x-f_y + \Delta_x) +
(f_y-f_x+\Delta_y) = \Delta_x + \Delta_y$. Second, the fills placed in cups $x$
and $y$ are at most $1$: this is because $f_x-f_y+\Delta_x<\Delta_y \le 1$ (by
\eqref{eq:fxdxfydy}) and $f_y-f_x + \Delta_x < \Delta_x \le 1$ (by
\eqref{eq:fxfy}). Third, the fills placed in cups $x$ and $y$ are non-negative:
this is because $f_x-f_y + \Delta_x > \Delta_x \ge 0$ (by \eqref{eq:fxfy})
and $f_y-f_x+\Delta_y > \Delta_x \ge 0$ (by
\eqref{eq:fxdxfydy}). 

We can repeatedly apply this process to swap each cup in $I_t([r])\setminus
S_t([r])$ into being in $S_t'([r])$. At
the end of this process we will have some state $S_t^r$ for which
$S_t^r([r]) = I_t([r])$. Note that $S_t^r$ is simply a relabeling of $S_t$,
hence it must satisfy the same invariants \eqref{eq:invariants} satisfied by
$S_t$. Further, $S_t^r$ can be transformed into $I_t$ by a valid filling step.

Now we repeatedly apply this process, in descending order of ranks. 
In particular, we have the following process: create a sequence of states by
starting with $S_t^{n-1}$, and to get to state $S_t^{r}$ from state $S_t^{r+1}$
apply the process described above. 
Note that $S_t^{n-1}$ satisfies $S_t^{n-1}([n-1]) = I_t([n-1])$ and thus also
$S_t^{n-1}(n) = I_t(n)$.
If $S_t^{r+1}$ satisfies $S_t^{r+1}(r') = I_t(r')$ for all $r'>r+1$ then
$S_t^r$ satisfies $S_t^r(r') = I_t(r')$ for all $r > r$, because the transition
from $S_t^{r+1}$ to $S_t^r$ has not changed the labels of any cups with ranks
in $(r+1, n]$, but the transition does enforce $S_t^r([r]) = I_t([r])$, and
consequently $S_t^r(r+1) = I_t(r+1)$.
We continue with the sequential process until arriving at state $S_t^1$ in
which we have $S_t^1(r) = I_t(r)$ for all $r$.
Throughout the process each $S_t^r$ has satisfied the invariants
\eqref{eq:invariants}, so $S_t^1$ satisfies the invariants
\eqref{eq:invariants}. Further, throughout the process from each $S_t^r$ it is
possible to legally place water into cups in order to transform $S_t^r$ into
$I_t$.

Hence $S_t^1$ satisfies all the properties desired, and the proof of 
\cref{clm:interchangable} is complete.

\end{proof}

\cref{clm:interchangable} tells us that we may assume without loss of
generality that $S_t(r) = I_t(r)$ for each rank $r \in [n]$. We will make
this assumption for the rest of the proof. 

In order to complete the proof of the theorem, we break it into three cases. 

\begin{clm}
  If some cup in $A$ zeroes out in round $t$, then the invariant
  $\mu_{S_{t+1}}(S_{t+1}([k])) \le 2n-k$ holds.
\end{clm}
\begin{proof}
  Say a cup in $A$ zeroes out in step $t$. 
  Of course
  $$m_{S_{t+1}}(I_t([a-1])) \le (a-1)(2n-(a-1))$$
  because the $a-1$ fullest cups must have satisfied the invariant (with $k = a - 1$) on round
  $t$. Moreover, because $\fil_{S_{t+1}}(I_{t+1}(a)) = 0$
  $$m_{S_{t+1}}(I_t([a])) = m_{S_{t+1}}(I_t([a-1])).$$
  Combining the above equations, we get that
  $$m_{S_{t+1}}(A) \le (a-1)(2n-(a-1)).$$
  Furthermore, the fill of all cups in $C$ must be at most $1$ at state $I_t$ to be
  less than the fill of the cup in $A$ that zeroed out. Thus, 
  \begin{align*}
      m_{S_{t+1}}(S_{t+1}([k])) & = m_{S_{t + 1}}(AC)\\ 
                                & \le (a-1)(2n-(a-1))+k-a\\
                                &= a(2n-a) +a -2n+a-1 + k -a\\
                                &= a(2n-a) + (k-n) + (a-n) -1\\
                                &< a(2n-a)
  \end{align*}
  as desired. As $k$ increases from $1$ to $n$, $k(2n-k)$ strictly increases (it is a
  quadratic in $k$ that achieves its maximum value at $k=n$).
  Thus $a(2n-a) \le k(2n-k)$ because $a\le k$.
  Therefore,
  $$m_{S_{t+1}}(S_{t+1}([k])) \le k(2n-k).$$
\end{proof}

\begin{clm}
  If no cups in $A$ zero out in round $t$ and $b=0$, then the invariant
  $\mu_{S_{t+1}}(S_{t+1}([k])) \le 2n-k$ holds.
\end{clm}
\begin{proof}
If $b=0$, then $S_{t+1}([k]) = S_t([k])$. 
During round $t$ the emptier removes $a$ units of fill from the cups in $S_t([k])$,
specifically the cups in $A$. The filler cannot have added more than $k$ fill
to these cups, because it can add at most $1$ fill to any given cup. Also, the
filler cannot have added more than $p_t$ fill to the cups because this is the
total amount of fill that the filler is allowed to add. Hence the filler adds
at most $\min(p_t, k) = a+b=a+0=a$ fill to these cups.
Thus the invariant holds:
$$m_{S_{t+1}}(S_{t+1}([k])) \le m_{S_t}(S_t([k]))+a-a \le k(2n-k).$$
\end{proof}

The remaining case, in which no cups in $A$ zero out and $b > 0$ is the most technically interesting.
\begin{clm}
  If no cups in $A$ zero out on round $t$ and $b > 0$, then the invariant
  $\mu_{S_{t+1}}(S_{t+1}([k])) \le 2n-k$ holds.
\end{clm}
\begin{proof}
Because $b>0$ and $a+b \le k$ we have that $a
< k$, and $c = k-a > 0$. Recall that $S_{t+1}([k]) = AC$, so the mass of the
$k$ fullest cups at $S_{t+1}$ is the mass of $AC$ at $S_t$ plus any water added
to cups in $AC$ by the filler, minus any water removed from cups in $AC$ by the
emptier. The emptier removes exactly $a$ units of water from $AC$.
The filler adds no more than $p_t$ units of water to $AC$ (because the filler
adds at most $p_t$ total units of water per round) and the filler also
adds no more than $k = |AC|$ units of water to $AC$ (because the filler adds
at most $1$ unit of water to each of the $k$ cups in $AC$).
Thus, the filler adds no more than $a+b = \min(p_t, k)$ units of water to $AC$.
Combining these observations we have:
\begin{equation}
m_{S_{t+1}}(S_{t+1}([k])) \le m_{S_t}(AC) + b.
\label{eq:emptiereptiessomestufffillerfillssomestuff}
\end{equation}

% This is easy to bound if $m_{S_t}(C) \le m_{S_t}(BC) - b$, because 
% $$m_{S_t}(A) + m_{S_t}(BC)  = m_{S_t}(ABC) \le m_{S_t}([k])$$
% which would imply the invariant for $S_{t+1}$, $k$.
% If $\mu_{S_t}(C)$ is not significantly less than $\mu_{S_t}(BC)$ we have more difficulty.
The key insight necessary to bound this is to notice that larger values for
$m_{S_t}(A)$ correspond to smaller values for $m_{S_t}(C)$ because of the
invariants; the higher fill in $A$ \defn{pushes down} the fill that $C$ can
have. By capturing the pushing-down relationship combinatorially we will achieve the desired inequality.

We can upper bound $m_{S_t}(C)$ by 
\begin{align*}
m_{S_t}(C) & \le \frac{c}{b+c}m_{S_t}(BC) \\
&= \frac{c}{b+c}(m_{S_t}(ABC) - m_{S_t}(A))
\end{align*}
 because
$\mu_{S_t}(C) \le \mu_{S_t}(B)$ without loss of generality by the
interchangeability of cups.
Thus we have 
\begin{align}
  m_{S_t}(AC) &\le m_{S_t}(A) + \frac{c}{b+c}m_{S_t}(BC)\label{eqn:BCdiscounted}\\
  &= \frac{c}{b+c}m_{S_t}(ABC) + \frac{b}{b+c}m_{S_t}(A)\label{eqn:redistributeA}.
\end{align}

Note that the expression in \eqref{eqn:redistributeA} is monotonically
increasing in both $\mu_{S_t}(ABC)$ and $\mu_{S_t}(A)$. Thus, by numerically
replacing both average fills with their extremal values, $2n-|ABC|$ and
$2n-|A|$. At this point the claim can be verified by straightforward (but quite
messy) algebra (and by combining
\eqref{eq:emptiereptiessomestufffillerfillssomestuff} with
\eqref{eqn:redistributeA}). We instead give a more intuitive argument, in which
we examine the right side of \eqref{eqn:BCdiscounted} combinatorially.

 Consider a new configuration of fills $F$ achieved by starting with state
 $S_t$, and moving water from $BC$ into $A$ until $\mu_{F}(A) = 2n-|A|$.
 \footnote{Note that whether or not $F$ satisfies the invariants is
 irrelevant.} This transformation increases (strictly increases if and only if
 we move a non-zero amount of water) the right side of
 \eqref{eqn:BCdiscounted}. In particular, if mass $\Delta \ge 0$ fill is moved
 from $BC$ to $A$, then the right side of \eqref{eqn:BCdiscounted} increases by
 $\frac{b}{b+c} \Delta \ge 0$. Note that the fact that moving water from $BC$
 into $A$ increases the right side of \eqref{eqn:BCdiscounted} formally
 captures the way the system of invariants being proven forces a tradeoff
 between the fill in $A$ and the fill in $BC$---that is, higher fill in $A$
 pushes down the fill that $BC$ (and consequently $C$) can have.

  Since $\mu_F(A)$ is above $\mu_{F}(ABC)$, the greater than average fill of
  $A$ must be counter-balanced by the lower than average fill of $BC$. In
  particular we must have
  $$(\mu_F(A) - \mu_F(ABC))|A| = (\mu_F(ABC) -\mu_F(BC))|BC|.$$
  Note that 
  \begin{align*}
  & \mu_F(A) -\mu_F(ABC) \\
  &= (2n-|A|) - \mu_F(ABC) \\
  &\ge (2n-|A|) - (2n-|ABC|) \\
  &= |BC|.    
  \end{align*}
  Hence we must have 
  $$\mu_F(ABC) - \mu_F(BC) \ge |A|.$$
  Thus 
  \begin{equation}
      \mu_F(BC) \le \mu_F(ABC) - |A| \le 2n-|ABC| -|A|.
      \label{eq:BCispusheddown}
  \end{equation}
  Combing \eqref{eqn:BCdiscounted} with the fact that the transformation from
  $S_t$ to $F$ only increases the right side of \eqref{eqn:BCdiscounted}, along
  with \eqref{eq:BCispusheddown}, we have the following bound:
  \begin{align}
    m_{S_t}(AC)
  &\le m_{F}(A) + c\mu_{F}(BC) \nonumber \\
  &\le a(2n-a) + c(2n-|ABC|-a) \nonumber \\
  &\le (a+c)(2n-a) - c(a+c+b) \nonumber \\
  &\le (a+c)(2n-a-c) - cb. \label{eq:eqnwithcb}
  \end{align}
  
By \eqref{eq:emptiereptiessomestufffillerfillssomestuff} and \eqref{eq:eqnwithcb}, we have that
\begin{align*}
    m_{S_{t+1}}(S_{t + 1}([k])) & \le m_{S_t}(AC) + b \\
                                & \le (a+c)(2n-a-c) - cb + b \\
                                & = k(2n-k) - cb + b \\
                                & \le k(2n-k),
\end{align*}
where the final inequality uses the fact that $c \ge 1$. This completes the proof of the claim. 
  
\end{proof}

We have shown the invariant holds for arbitrary $k$, so given that the
invariants all hold at state $S_t$ they also must all hold at state $S_{t+1}$.
Thus, by induction we have the invariant for all rounds $t\in\mathbb{N}$.
\end{proof}

