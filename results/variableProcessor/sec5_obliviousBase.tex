\todo{emptier turn skipping RIP}
\todo{integer stuff}

\section{Oblivious Filler Lower Bound}\label{sec:oblivious}
In this section we prove that, surprisingly, an oblivious filler
can achieve backlog $n^{1-\varepsilon}$, although only against a
certain class of ``greedy-like" emptiers.

We call a cup configuration \defn{$M$-flat} if the difference
between the fill of the fullest cup and the fill of the least
full is at most $M$; note that in an $M$-flat cup configuration
with average fill $0$ all cups have fills in $[-M, M]$. We say an
emptier is \defn{$\Delta$-greedy-like} if, whenever there are two
cups with fills that differ by at least $\Delta$, the emptier
never empties from the less full cup without also emptying from
the more full cup. That is, if on some round $t$, there are cups
$c_1, c_2$ with $\fil_{I_t}(c_1) > \fil_{I_t}(c_2) + \Delta$,
then a $\Delta$-greedy-like emptier doesn't empty from $c_2$ on
round $t$ unless it also empties from $c_1$ on round $t$. Note
that a perfectly greedy emptier is $0$-greedy-like. We call an
emptier \defn{greedy-like} if it is $\Delta$-greedy-like for
$\Delta \le O(1)$. 

With an oblivious filler we are only able to prove lower bounds
on backlog against greedy-like emptiers; whether or not our
results can be extended to a more general class of emptiers is an
interesting open question. Nonetheless, greedy-like emptiers are
of great interest because all the known randomized algorithm for
the cup game are greedy-like \cite{mbe19, wku20}.

As a tool in our analysis we define a new variant of the cup
game: In the $p$-processor \defn{$E$-extra-empties} negative-fill cup
game on $n$ cups, the filler distributes $p$ units of water
amongst the cups, and then the emptier empties from $p$
\textit{or more} cups. In particular the emptier is
allowed to empty $E$ extra cups over the course of the game. Note
that the emptier still cannot empty from the same cup twice on a
single round.

We now prove a crucial property of greedy-like emptiers: 
\begin{lemma}
  \label{lem:greedylikeisflat}
  Let $R_\Delta = 2(2+\Delta)$.
  Consider an $M$-flat cup configuration in the $p$-processor
  $E$-extra-empties negative-fill cup game on $n = 2p$ cups. An
  oblivious filler can achieve a $R_\Delta$-flat configuration
  of cups against a $\Delta$-greedy-like emptier in running time
  $2(M+E)$. Furthermore, throughout this process the cup configuration
  is $M$-flat on every round.
\end{lemma}
\begin{proof}
  If $M \le R_\Delta$ the algorithm does nothing, since the
  desired flatness is already achieved; for
  the rest of the proof we consider $M > R_\Delta$.

  The filler's strategy is to distribute fill equally amongst all
  cups at every round, placing $p/n = 1/2$ fill in each cup. 

  Let $\ell_t = \min_{c\in S_t} \fil_{S_t}(c)$, $u_t = \max_{c\in S_t} \fil_{S_t}(c)$. 
  Let $L_t$ be the set of cups $c$ with $\fil_{S_t}(c) \le l_t+2+\Delta$, and let
  $U_t$ be the set of cups $c$ with $\fil_{S_t}(c) \ge u_t-2-\Delta$.

  Now we prove a key property of the sets $U_t$ and $L_t$: once a cup is in
  $U_t$ or $L_t$ it is always in $U_{t'}, L_{t'}$ for all $t' > t$. This
  follows immediately from \cref{clm:dontlosestuff}.
  \begin{clm}
    \label{clm:dontlosestuff}
    $$U_{t} \subseteq U_{t+1},\quad L_t \subseteq L_{t+1}.$$
  \end{clm}
  \begin{proof}
    Consider a cup $c\in U_t$.

    If $c$ is not emptied from, i.e. $\fil(c)$ has increased by
    $1/2$ from the previous round, then
    clearly $c \in U_{t+1}$, because backlog has increased by at most $1/2$, so
    $\fil(c)$ must still be within $2+\Delta$ of the backlog on round $t+1$. 

    On the other hand, if $c$ is emptied from, i.e. $\fil(c)$ has decreased by
    $1/2$, we consider two cases.\\
    \textbf{Case 1:} If $\fil_{S_t}(c) \ge u_t-\Delta -1$, then
    $\fil_{S_t}(c)$ is at least $1$ above the bottom of the
    interval defining which cups belong to $U_t$. The backlog
    increases by at most $1/2$ and the fill of $c$ decreases by $1/2$, so
    $\fil_{S_{t+1}}(c)$ is at least $1-1/2-1/2 = 0$ above the bottom of the
    interval, i.e. still in the interval. \\
    \textbf{Case 2:} On the other hand, if $\fil_{S_t}(c) <
    u_t-\Delta-1$, then every cup with fill in $[u_t-1, u_t]$
    must have been emptied from because the emptier is
    $\Delta$-greedy-like. Therefore the fullest cup
    on round $t+1$ is the same as the fullest cup on round $t$,
    because every cup with fill in $[u_t-1, u_t]$
    has had its fill decrease by $1/2$, and no cup with fill less than
    $u_t-1$ had its fill increase by more than $1/2$. Hence $u_{t+1}
    = u_t -1/2$. Because both $\fil(c)$ and the backlog
    have decreased by $1/2$, the distance between them is
    still at most $\Delta+2$, hence $c\in U_{t+1}$.\\
    The argument for why $L_t \subseteq L_{t+1}$ is symmetric.
  \end{proof}

  Now that we have shown that $L_t$ and $U_t$ never lose cups, we will show
  that they each eventually gain more than $n/2$ cups.

  \begin{clm}
    \label{clm:howDoLandUchange}
    On any round $t$, if $|U_t| \le n/2$ then $u_{t+1} = u_t - 1/2$.
    On any round $t$ where the emptier doesn't use extra resources, if
    $|L_t| \le n/2$ then $\ell_{t+1} = \ell_t+ 1/2$.
  \end{clm}
  \begin{proof}
    If there are at least $n/2$ cups outside of $U_t$, i.e. cups
    with fills in $[\ell_t, u_t-2-\Delta]$, then all cups with
    fills in $[u_t - 2, u_t]$ must be emptied from; if one such cup
    was not emptied from then by the pigeon-hole principle some
    cup outside of $U_t$ was emptied from, which is impossible as
    the emptier is $\Delta$-greedy-like.
    This clearly implies that $u_{t+1} = u_t - 1/2$: no cup with fill
    less than $u_t-2$ has gained enough fill to become the fullest
    cup, and the fullest cup from the previous rounds has lost $1/2$
    units of fill.

    Now consider a round where the emptier doesn't use extra
    resources and where $|L_t| \le n/2$. 
    Clearly no cup with fill in $[\ell_t, \ell_t+2]$ can be
    emptied from; if one such cup were emptied from, then by the
    pigeon-hole principle some cup outside of $L_t$ was not
    emptied from, which is impossible as the emptier is
    $\Delta$-greedy-like. Hence we have $\ell_{t+1} =
    \ell_{t} + 1/2$.

    We remark however that we cannot guarantee that $\ell_{t+1} =
    \ell_t + 1/2$ on all rounds where $|L_t| \le n/2$, because the
    emptier could do extra emptying; in the most extreme case the
    emptier could empty from every cup in which case we
    would have $\ell_{t+1} = \ell_t - 1/2$. 
  \end{proof}
  
  We call a round where the emptier uses extra resources an
  \defn{emptier-extra-resource} round. At most $E$ of the
  $2(M+E)$ total rounds are emptier-extra-resource rounds. When
  the emptier uses extra resources it can potentially hurt the
  filler's efforts to achieve a flat configuration of cups, in
  particular by making $\ell_{t+1} < \ell_t$.
  However, the affect of emptier-extra-resource rounds is
  countered by rounds where the emptier does not use extra
  resources. In particular, we now define what it means for a
  non-emptier-extra-resource round $j$ to cancel an
  emptier-extra-resource round $i < j$. For $i = 1,2,\ldots,
  2(M+E)$, if round $i$ is an emptier-extra-resource round then
  the first non-emptier-extra-resource round $j > i$ that has not
  already cancelled some emptier-extra-resource round $i' < i$ in
  this sequential labelling process, provided such a round
  exists, is said to \defn{cancel} round $i$. Each
  emptier-extra-resource round is cancelled by at most one later
  round, some emptier-extra-resource rounds may not be cancelled
  at all.

  Consider rounds of the form $2M + i$ for $i \in [2E+1]-1$. We
  claim there is some $i$ such that there are $2M$
  non-emptier-extra-resource rounds among rounds $[2M + i]$ that
  are not cancelling other rounds. Assume for contradiction that
  this is not so. Then every non-emptier-extra-resource round $2M + i$
  is necessarily a cancelling round. Hence by round $2(M + E)$,
  there must have been $E$ cancelled tasks, so on round $2(M+E)$
  all emptier-extra-resource rounds are cancelled.

  Let $t_e$ be some round by which there are $2M$
  non-emptier-extra-resource, non-cancelling rounds. The value of
  $u_t$ cannot have shrunk by more than $M$ because the
  configuration started $M$-flat. Hence by
  \cref{clm:howDoLandUchange} there is some round $t_u \in [t_e]$
  such that $|U_t|\ge n/2$. Identically, there is some round
  $t_\ell \in [t_e]$ such that $|L_t| \ge n/2$. Since by
  \cref{clm:dontlosestuff} $|U_{t+1}|\ge |U_t|$ and $|L_{t+1}|
  \ge |L_t|$, we have that there is some round $t_0 =\max(t_u,
  t_\ell)$ on which both $|U_{t_0}|$ and $|L_{t_0}|$ exceed
  $n/2$. Then $U_{t_0} \cap L_{t_0} \neq \varnothing$.
  Furthermore, the sets must intersect for all $t_0 \le t \le
  [2(M+E)]$. In order for the sets to intersect it must be that
  the intervals $[u_t-2-\Delta, u_t]$ and $[\ell_t,
  \ell_t+2+\Delta]$ intersect. Hence we have that
  $$\ell_t+2+\Delta \ge u_t-2-\Delta.$$ 
  Or, rearranging, 
  $$u_t - \ell_t \le 2(2+\Delta) = R_\Delta.$$ 
  Thus the cup configuration is $R_\Delta$-flat.

  Now we establish that throughout this flattening process the cup
  configuration is always $M$-flat. Consider a round where
  $u_{t+1} - \ell_{t+1} > u_t - \ell_t$. For this to happen the
  emptier must have used less than $n-p$ extra emptying, or else
  the fill of every cup would simply decrease by $1/2$ which
  wouldn't affect the difference $u_{t+1} - \ell_{t+1}$.
  In order for the difference $u_t - \ell_t$ to 
  increase either a) some cup with fill in $[\ell_t, \ell_t + 1/2]$ was
  emptied from and some cup with fill in $[u_t-1, u_t]$ was not
  emptied from, or b) some cup with fill in $[u_t-1/2, u_t]$ was
  not emptied from and some cup with fill in $[\ell_t, \ell_t+1]$ was
  emptied from. In either case, because the emptier is
  $\Delta$-greedy-like, such an action implies
  $$u_{t+1} - \ell_{t+1} \le u_t +1/2 - (\ell_t -1/2)  \le \Delta + 5/2 \le M.$$
  Since the cup configuration starts $M$-flat, and after any
  round where the distance $u_t-\ell_t$ increases it increases to
  a value at most $M$, we have that the cups are always $M$-flat.
\end{proof}

% note: may as well take $\Delta' = \ceil{\Delta} \in
% \mathbb{N}$, this might be useful sometimes

Next we describe a simple oblivious filling strategy that will be used as a
subroutine in \cref{lem:obliviousManyUnknownCups}; this strategy is very
well-known, and similar versions of it can be found in
\cite{ mbe19, mbe15, die91, wku20}.
\begin{proposition}
  \label{prop:obliviousTerribleProbability}
  Consider an $R$-flat cup configuration in the negative-fill
  single-processor cup game on $n$ cups with average fill $0$.
  Let $d = \sum_{i=2}^n 1/i$.

  There is an oblivious filling strategy that achieves fill at
  least $-R + d$ in a randomly chosen cup with probability at
  least $1/n!$. This filling strategy guarantees that the chosen
  cup has fill at most $R + d$, and has running time $n-1$.
  Further, when applied against a $\Delta$-greedy-like emptier
  with $R = R_\Delta$, this filling strategy guarantees that
  the cups always remain $(R + d)$-flat.
\end{proposition}
\begin{proof}
  The filler maintains an \defn{active set}, initialized to being
  all of the cups. Every round the filler distributes $1$ unit of
  fill equally among all cups in the active set. Then the emptier
  removes $1$ unit of fill from some cup. Finally, the filler
  removes a cup uniformly at random from the active set. This
  continues until a single cup $c$ remains in the active set. 

  We now bound the probability that $c$ has never been emptied
  from. On the $i$-th step of this process, i.e. when the size of
  the active set is $n-i+1$, consider the cup $c'$ that the
  emptier empties. If $c'$ is in the active-set, then with
  probability at least $1/(n-i+1)$ the filler removes it from the
  active set. If $c'$ is not in the active set, then it is
  irrelevant. Hence with probability at least $1/n!$ the final
  cup in the active set, $c$, has never been emptied from. In
  this case, $c$ will have gained fill $d=\sum_{i=2}^n 1/i$ as
  claimed. Because $c$ started with fill at least $-R$, $c$ now
  has fill at least $-R+ d$. 

  Further, $c$ has fill at most $R + d$, as $c$ starts with fill
  at most $R$, and $c$ gains at most $1/(n-i+1)$ fill on the
  $i$-th round of this process.

  Now we analyze this algorithm specifically for a
  $\Delta$-greedy-like emptier. Consider a round on which the
  minimum fill of the cups becomes lower, i.e. where the emptier
  empties from some cup $c$ with fill less than $1$ above the
  lowest fill. On such a round, because the emptier is
  $\Delta$-greedy-like, the backlog is no more than $\Delta$
  greater than $\fil(c)$. Hence the cups are $(\Delta+1)$-flat on
  such a round. If the cups are always $(\Delta+1)$-flat then we
  are done. Otherwise, consider the round after the last round on
  which the cups are $(\Delta + 1)$-flat, (or the first round of
  the process if the cups are never $(\Delta+1)$-flat). From this
  round on, the emptier cannot empty a cup with fill within $1$
  of the fill of the lowest cup, hence the fill of the lowest cup
  cannot decrease. Consider how much the backlog could increase.
  The backlog could increase by $d$ from whatever value it starts
  at. The backlog starts as at most $\max(\Delta+1, R) = R$, and
  hence throughout the process the cups remain $(R+d)$-flat, as
  desired.

\end{proof}

Now we are equipped to prove \cref{lem:obliviousManyUnknownCups},
which shows that we can force a constant fraction of the cups to
have high fill; using \cref{lem:obliviousManyUnknownCups} and
exploiting the greedy-like nature of the emptier we can get a
known cup with high fill (we show this in
\cref{lem:obliviousBase}).
\begin{lemma}
  \label{lem:obliviousManyUnknownCups}
  Let $\Delta \le O(1)$, let $h \le O(1)$ with $h \ge
  16+16\Delta$, let $n$ be at least a sufficiently large constant
  determined by $h$ and $\Delta$, and let $M \le \poly(n)$.
  Consider an $M$-flat cup configuration in the negative-fill
  variable-processor cup game on $n$ cups with average fill $0$.
  Let $A, B, A'$ be disjoint constant-fraction-size subsets of
  the cups with $|A| = \Theta(n)$ sufficiently small and with
  $|A| + |B| + |A'| = n$. These sets will change over the course
  of the filler's strategy, but $|A|$ will remain fixed and $|A|
  \ll |A|$ will always hold.

  There is an oblivious filling strategy that makes an unknown
  set of $\Theta(n)$ cups in $A$ have fill at least $h$ with
  probability at least $1-2^{-\Omega(n)}$ in running time
  $\poly(n)$ against a $\Delta$-greedy-like emptier.
  The filling strategy also guarantees that $\mu(B) \ge -h/2$.
\end{lemma}
\begin{proof}
  We refer to $A$ as the \defn{anchor} set, $B$ as the
  \defn{non-anchor} set, and $A'$ as the \defn{garbage} set.
  Throughout the proof the filler uses $p=|A|+1$.
  The filler initializes the sets as $A' = \varnothing$, and $B$
  is all the cups besides the cups in $A$.
  The set $A$ is chosen to satisfy 
  \begin{equation}
    \label{eq:chooseBmuchbiggerthanA}
    |A| \le (n - 2|A|) / (2e^{2h+1} + 1).
  \end{equation}

We denote by \randalg the oblivious filling
strategy given by \cref{prop:obliviousTerribleProbability}. 
We denote by \flatalg the oblivious filling
strategy given by \cref{lem:greedylikeisflat}.
We say that the filler \defn{applies} a filling strategy
\genericalg to a set of cups $D \subseteq B$ if the filler uses
\genericalg on $D$ while placing $1$ unit of fill in each anchor cup. 

We now describe the filler's strategy.

The filler starts by flattening the cups, i.e. using \flatalg on
$A\cup B$ for $2M$ rounds. After this, the filling strategy
always places $1$ unit of water in each anchor cup. The filler
performs a series of $|A|$ \defn{swapping-processes}, one per
anchor cup, which are procedures that the filler uses to get a
new cup---which will sometimes have high fill---in $A$. On each
swapping-process the filler applies \randalg many times to
arbitrarily chosen constant-size sets $D \subset B$ with $|D| =
\ceil{e^{2h+1}}$. The number of times that the filler applies
\randalg is chosen at the start of the swapping-process, chosen
uniformly at random from $[m]$ ($m = \poly(n)$ to be
specified). At the end of the swapping-process, the filler does a
``swap": the filler takes the cup given by \randalg in $B$ and
puts it in $A$, and the filler takes the cup in $A$ associated
with the current swapping-process and moves it into $A'$.
Before each application of \randalg the filler flattens $B$ by
applying \flatalg to $B$ for $\poly(n)$ rounds (exactly how many
rounds will be specified later in the proof). 

We remark that this construction is similar to the construction
in \cref{lem:adaptiveAmplification}, but has a major difference
that substantially complicates the analysis: in the adaptive
lower bound construction the filler halts after achieving the
desired average fill in the anchor set, whereas the oblivious
filler cannot halt but rather must rely on the emptier's
greediness to guarantee that each application of \randalg has
constant probability of generating a cup with high fill.

We proceed to analyze our algorithm.

First note that the initial flattening of $A\cup B$ succeeds
because the emptier is not allowed to do any extra emptying on
$A\cup B$ so by setting $p=n/2$ the filler makes the flattening
happen in the $0$-extra-empties, $(n/2)$-processor cup game,
which by \cref{lem:greedylikeisflat} gets an $R_\Delta$-flat cup
configuration in running time $2M$.

We say that a property of the cups has \defn{always} held if the
property has held since the start of the first swapping-process;
i.e. from now on we only consider rounds after the initial
flattening of $A\cup B$.

We say that the emptier \defn{neglects} the anchor set on a round
if it does not empty from each anchor cup. We say that an
application of \randalg to $D\subset B$ is \defn{successful} if
the emptier does not neglect the anchor set during any round of
the application of \randalg. We define $d = \sum_{i=2}^{|D|} 1/i$ (recall
that $|D| = \ceil{e^{2h+1}}$). We say that an application of
\randalg to $D$ is \defn{lucky} if it achieves backlog at
least $\mu(B) - R_\Delta + d$; note that by 
\cref{prop:obliviousTerribleProbability} any successful
application of \randalg where $B$ started $R_\Delta$-flat has
at least a $1/|D|!$ chance of being lucky. 

Now we prove several important bounds on fills of cups in $A$ and $B$.
\begin{clm}
  \label{clm:oblivBaseIntenseInduction1}
  If all applications of \flatalg so far have made
  $B$ be $R_\Delta$-flat, then $B$ has always been $(R_\Delta +
  d)$-flat and $\mu(B)$ has always been at most $1$.
\end{clm}
\begin{proof}
  Given that the application of \flatalg immediately prior to an application
  of \randalg made $B$ be $R_\Delta$-flat, by
  \cref{prop:obliviousTerribleProbability} we have that $B$ will
  stay $(R_\Delta + d)$-flat during the application of \randalg. 
  Given that the application of \randalg immediately prior to an
  application of \flatalg resulted in $B$ being $(R_\Delta
  + d)$-flat, we have that $B$ remains $(R_\Delta + d)$-flat
  throughout the duration of the application of \flatalg by
  \cref{lem:greedylikeisflat}. Given that $B$ is $(R_\Delta +
  d)$-flat before a swap occurs $B$ is clearly still $(R_\Delta +
  d)$-flat after the swap, because the only change to $B$ during
  a swap is that a cup is removed from $B$ which cannot increase
  the backlog in $B$ or decrease the fill of the least full cup
  in $B$. Note that $B$ started $R_\Delta$-flat before the first
  application of \flatalg because $A\cup B$ was flattened.
  Hence we have by induction that $B$ has always been $(R_\Delta
  + d)$-flat. 
  % start, flatalg, randalg, flatalg, randalg, ..., flatalg
  % randalg, swap, flatalg, randalg, flatalg, randalg, ..., swap,
  % ..., swap, ..., right now

  Now consider how high $\mu(B)$ could rise.
  The only time when $\mu(B)$ rises is at the end of a
  swapping-process. The cup that $B$ evicts at the end of a
  swapping-process has fill at least $\mu(B) - R_\Delta - (|D|-1)$,
  as the running time of \randalg is $|D|-1$, and $B$ started
  $R_\Delta$-flat by assumption. The highest that $\mu(B)$ can
  rise is clearly if a cup with fill as far below $\mu(B)$ as possible is
  evicted at every swapping-process. Evicting a cup with fill
  $\mu(B) - R_\Delta - (|D| -1)$ from $B$ changes $\mu(B)$ by 
  $(R_\Delta + |D| - 1) / (|B|-1)$ where $|B|$
  is the size of $B$ before the cup is evicted from $B$. 
  Even if this happens on each of the $|A|$ swapping processes
  $\mu(B)$ cannot rise higher than $|A| (R_\Delta + |D|-1) /
  (n-2|A|)$ which by design in choosing $|B| \gg |A|$,
  as was done in \eqref{eq:chooseBmuchbiggerthanA}, is
  at most $1$.

\end{proof}

Let $u_A = 1 + (R_\Delta + d) + \Delta + 1$, $\ell_A = -|B|
-u_A \cdot (|A| -1)$.
\begin{clm}
  \label{clm:oblivBaseIntenseInduction2}
  If $B$ has always been $(R_\Delta + d)$-flat and $\mu(B)$ has
  never exceeded $1$, then the fills of cups in $A$ have always
  been in $[\ell_A, u_A]$ and we have always had $\mu(B) > -h/2$.
\end{clm}
\begin{proof}
  First consider how high the fill of a cup $c\in A$ could be.
  Let $u_B = 1 + (R_\Delta + d)$; note that the assumptions on
  $B$ guarantee that $u_B$ has always been an upper bound for the
  backlog in $B$. If $c$ came from $B$ then when it is swapped
  into $A$ its fill is at most $u_B < u_A$. Otherwise, $c$
  started with fill at most $R_\Delta < u_A$. Now consider how
  much the fill of $c$ could increase while being in $A$. Because
  the emptier is $\Delta$-greedy-like, if a cup $c\in A$ has fill
  more than $\Delta$ higher than the backlog in $B$ then $c$ must
  be emptied from, so any cup with fill at least $u_B + \Delta =
  u_A - 1$ must be emptied from, and hence $u_A$ upper bounds the
  backlog in $A$. 

  Of course an upper bound on backlog in $A$ also serves as
  an upper bound on the average fill of $A$ as well, i.e. $\mu(A)
  \le u_A$. Then, because $A\cup B$ has average fill $0$, we have that 
  \begin{equation}
    \label{eq:muBisPrettyFineBecauseItsHUGE}
    \mu(B) = -\frac{|A|}{|B|} \mu(A) \ge -u_A \frac{|A|}{|B|}.
  \end{equation}
  Note that $|B| \gg |A|$ so
  \eqref{eq:muBisPrettyFineBecauseItsHUGE} is not very negative.
  In particular, by \eqref{eq:chooseBmuchbiggerthanA}
  \eqref{eq:muBisPrettyFineBecauseItsHUGE} can be loosened to $\mu(B) \ge -h/2$.

  Because $\mu(B) \le 1$ we have $\mu(A) \ge -|B|/|A|$. The mass
  of a subset of $|A|-1$ of the cups is at most $(|A|-1)u_A$, so
  we can lower bound the fill of any particular cup $c \in A$ by 
  $$\fil(c) \ge -|B| -u_A \cdot (|A| -1).$$
\end{proof}

Let $r = |A|(\ell_A + u_A)$; note that $r = \Theta(n^2) =
\poly(n)$.
\begin{clm}
  \label{clm:oblivBaseIntenseInduction3}
  If at the start of an application of \flatalg $B$ is $(R_\Delta
  + d)$-flat, $\mu(B) \le 1$, and all cups in $A$ have fills in
  $[\ell_A, u_A]$ then by applying \flatalg for $2((R_\Delta + d)
  + r)$ rounds, the filler guarantees that $B$ will be
  $R_\Delta$-flat at the end of the application of \flatalg. 
\end{clm}
\begin{proof}
  If all cups in $A$ start the application with fill at
  least $\ell_A$ and the emptier uses $r+1$ extra
  empties during the application of \flatalg, then by the
  pigeon-hole principle some cup in $A$
  will have fill higher than $u_A$ by the end of the application,
  as no cup in $A$ loses water during the application of \flatalg.
  
  \cref{lem:greedylikeisflat} says that $B$ will remain
  $(R_\Delta + d)$-flat throughout the application of \flatalg,
  and $\mu(B)$ obviously cannot rise during the application of
  \flatalg. Hence throughout the entire of the process it will still be the
  case that $\mu(B) \le 1$ and that $B$ is $(R_\Delta + d)$-flat.
  Then \cref{clm:oblivBaseIntenseInduction2} says that the
  backlog in $A$ cannot have grown larger than $u_A$. Hence it is
  impossible for the emptier to have used $r+1$ extra empties.

  Thus, we can consider the application of \flatalg as happening
  in the $(|B|/2)$-processor $r$-extra-empties cup game. By
  \cref{lem:greedylikeisflat} we thus have that the cup
  configuration at the end of the application of \flatalg will be
  $R_\Delta$-flat by applying \flatalg for $2((R_\Delta + d) +
  r)$ rounds.
\end{proof}

Now we combine \cref{clm:oblivBaseIntenseInduction1}
\cref{clm:oblivBaseIntenseInduction2} and
\cref{clm:oblivBaseIntenseInduction3} to get the following:
\begin{clm}
  \label{clm:flatteningAlwaysWorks}
  All applications of \flatalg make $B$ be $R_\Delta$-flat.
 \end{clm}
\begin{proof}
  This follows by induction on the flattening processes. Assume
  that all previous flattening processes have made $B$ be
  $R_\Delta$-flat. Then by
  \cref{clm:oblivBaseIntenseInduction1} we have that $\mu(B) \le
  1$ has always held, and that $B$ has always been $(R_\Delta +
  d)$-flat. Thus by \cref{clm:oblivBaseIntenseInduction2} the
  fills of cups in $A$ have always been in $[\ell_A, u_A]$. Thus
  by \cref{clm:oblivBaseIntenseInduction3} the next flattening
  successfully makes $B$ be $R_\Delta$-flat.

  Note that the first application of \flatalg makes $B$ be
  $R_\Delta$-flat because $A\cup B$ is flattened. 
  Hence by induction all applications of \flatalg make $B$ be
  $R_\Delta$-flat.
\end{proof}

\todo{this proof feels sketchy}
Now we show that this guarantees that with constant probability
the final application of \randalg a swapping-process is both
lucky and successful. 
\begin{clm}
  There exists choice of $m = \poly(n)$ such that with at least
  constant probability the final application of \randalg on any
  fixed swapping-process is both lucky and successful.
\end{clm}
\begin{proof}
  Fix some swapping-process. By \cref{clm:flatteningAlwaysWorks}
  we have that the fill of each cup in $A$ starts the
  swapping-process with fill at least $\ell_A$, and never exceeds
  $u_A$ throughout the course of the swapping-process. This
  places an upper bound of $r$ on the number of applications of
  \randalg on which $A$ can be neglected. 

  The filler chooses $m = 4|D|! r$. By a Chernoff bound, there is
  exponentially high probability that of $4|D|! r$ applications
  of \randalg at least $2r$ are lucky.
  The emptier can choose at most $r$ of these to neglect, so
  there is at least a $1/2$ chance that the randomly chosen final
  application of \randalg is successful, conditioning on it
  lucky. The final application is lucky with probability
  $1/|D|!$. 
  Hence overall this choice of $m$ makes the final round lucky
  and successful with constant probability $1/(2|D|!)$.
\end{proof}

\begin{clm}
With probability at least $1-2^{-\Omega(n)}$, the filler achieves fill
at least $h$ in at least $\Theta(n)$ of the cups in $A$. 
\end{clm}
\begin{proof}
  By \cref{prop:obliviousTerribleProbability} using \randalg on
  $|D| = \lceil e^{2h+1} \rceil$ gives the filler a cup that
  has fill at least $\mu(B) + d-R_\Delta$ with probability
  $1/|D|!$. By \cref{clm:muBdoesntSinkTooLow} we have that
  $\mu(B) \ge -h/2$, so with probability $1/|D|!$ this generated
  cup has fill at least $h$ if it wasn't neglected.

  By \cref{clm:emptierCantNeglectAnchorTooMuch} there is a choice
  of $c_\Delta$ large enough that the probability of this cup
  having been neglected is at most $1/2$. In particular, we
  choose $c_\Delta = 4r|D|!$. By applying \randalg $|A|\cdot
  c_\Delta$ times we have by a Chernoff bound that with
  exponentially good probability in $|A|\cdot c_\Delta =
  \Theta(n)$ there are at least $2|A|r$ applications where the
  filler would succeed if the emptier doesn't neglect the anchor
  set. As shown, the emptier cannot
  neglect the anchor set more than $|A|r$ times. Hence, there
  is at least a $(1/2)/|D|!$ chance that on the $j$-th
  application of \randalg the emptier doesn't neglect the anchor
  set and the filler correctly guesses the emptier's emptying
  sequence. Thus, overall, there is at least a constant
  probability of achieving fill $h$ in a cup in $A$.

  Say that a swapping-process is \defn{victorious} if the filler
  is able to swap a cup with fill at least $h$ into $A$. The
  events that swapping-processes are victorious are independent
  events; each happens with constant probability. Hence by a
  Chernoff bound with exponentially good probability in $n$ at
  least a constant fraction of them succeed, as desired.

\end{proof}

We now briefly analyze the running time of the filling strategy.
There are $|A|$ swapping-processes. Each swapping-process
consists of $|A|\cdot c_\Delta$ applications of \randalg, and
the flattening procedure before each application. 
Clearly this all takes $\poly(n)$ time, as desired.
  
\end{proof}

Finally, using \cref{lem:obliviousManyUnknownCups} we can show in
\cref{lem:obliviousBase} that an oblivious filler can achieve
constant backlog. We remark that \cref{lem:obliviousBase} plays a
similar role in the proof of the lower bound on backlog as
\cref{prop:adaptiveBase} does in the adaptive case, but is vastly
more complicated to prove (in particular,
\cref{prop:adaptiveBase} is trivial, whereas we have already
proved several lemmas and propositions as preperation for the
proof of \cref{lem:obliviousBase}).
\begin{lemma}
  \label{lem:obliviousBase}
  Let $H \le O(1)$, let $\Delta \le O(1)$, let $n$ be at
  least a sufficiently large constant determined by $H$ and
  $\Delta$, and let $M \le \poly(n)$. 
  Consider an $M$-flat cup configuration in the negative-fill variable-processor cup
  game on $n$ cups with average fill $0$.
  Given this configuration, an oblivious filler can achieve fill $H$
  in a chosen cup in running time $\poly(n)$ against a
  $\Delta$-greedy-like emptier with probability at least $1-2^{-\Omega(n)}.$
\end{lemma}
\begin{proof}
  The filler starts by performing the procedure detailed in
  \ref{lem:obliviousManyUnknownCups}, using $h = H\cdot
  16(1+\Delta)$. Let the number of cups which must now exist with
  fill $h$ be of size $nc = \Theta(n)$.

  The filler reduces the number of processors to $p=nc$. 
  Now the filler exploits the filler's greedy-like nature to
  to get fill $H$ in a set $S\subset B$ of $nc$ chosen cups.

  The filler places $1$ unit of fill into each cup in $S$.
  Because the emptier is greedy-like it must focus on the $nc$
  cups in $A$ with fill at least $h$ until the cups in $S$ have
  sufficiently high fill. In particular, $(5/8)h$ rounds suffice.
  Over $(5/8)h$ rounds the $nc$ high cups in $A$ cannot have
  their fill decrease below $(3/8)h \ge h/8 + \Delta$. Hence, any
  cups with fills less than $h/8$ must not be emptied from during
  these rounds. The fills of the cups in $S$ must start
  as at least $-h/2$. After $(5/8)h$ rounds the fills of the cups
  in $S$ are at least $h/8$, because throughout this process the
  emptier cannot have emptied from them until they got fill at
  least $h/8$, and if they are never emptied from then they
  achieve fill $h/8$.

  Thus the filling strategy achieves backlog $h/8 \ge H$ in some
  known cup (in fact in all cups in $S$, but a single cup
  suffices), as desired.

\end{proof}

