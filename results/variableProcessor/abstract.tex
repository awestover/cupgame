\maketitle
\abstract{ 
  In the \defn{cup game} two players, the \defn{filler} and the
  \defn{emptier}, take turns adding and removing water from a set
  of cups, subject to certain constraints. In the classic
  $p$-processor cup game the filler distributes $p$ units of
  water among the $n$ cups, giving at most $1$ unit of water to
  any particular cup, and the emptier chooses $p$ cups to remove
  at most $1$ unit of water from. Analysis of the cup game is
  important for applications in processor scheduling, buffer
  management in networks, quality of service guarantees, and
  deamortization.

  We consider a new variant of the classic $p$-processor cup game
  in which the resources of the emptier and filler are variable:
  in our variant of the game, the \defn{variable-processor cup
  game}, the filler is allowed to \emph{change $p$} at the
  beginning of each round. Surprisingly, we found that the
  variable-processor cup game is fundamentally different from the
  classic fixed-resources version of the game.

  We give an adaptive filling strategy that achieves backlog
  $\Omega(n^{1-\varepsilon})$ for any constant $\varepsilon >0$
  of our choice in running time $2^{O(\log^2 n)}$. This is
  enormous compared to in the $p$-processor cup game where an
  emptier can prevent backlog from exceeding $O(\log n)$. We also
  present an adaptive filling strategy that is able to
  achieve backlog $\Omega(n)$ in running time $O(n!)$.

  We demonstrate, using a novel set of invariants, that a greedy
  emptier never lets backlog exceed $O(n)$; this
  matches our lower bound, so our analysis is tight.

  We also give an oblivious filling strategy that achieves
  backlog $\Omega(n^{1-\varepsilon})$ for $\varepsilon>0$
  constant of our choice in time $2^{\polylog n}$ against any
  ``greedy-like" emptier with probability at least
  $1-2^{-\polylog(n)}$. Being oblivious, i.e. not being able to
  observe the game state, seems a large disadvantage, but in the
  variable-processor cup game the lower bound is the same as in
  the adaptive case!
}
\thispagestyle{fancy} % page number to bottom right

