\maketitle \abstract{

The problem of scheduling tasks on $p$ processors so that no
task ever gets too far behind is often described as a game
with cups and water. In the $p$-processor cup game on $n$ cups,
there are two players, a filler and an emptier, that take turns
adding and removing water from a set of $n$ cups. In each turn,
the filler adds $p$ units of water to the cups, placing at most
$1$ unit of water in each cup, and then the emptier selects $p$
cups to remove up to $1$ unit of water from. The emptier's goal
is to minimize the backlog, which is the height of the fullest
cup.

The $p$-processor cup game has been studied in many different
settings, dating back to the late 1960's. All of the past work
shares one common assumption: that $p$ is fixed. This paper
initiates the study of what happens when the number of available
processors $p$ varies over time, resulting in what we call the
\emph{variable-processor cup game}.

Remarkably, the optimal bounds for the variable-processor cup
game differ dramatically from its classical counterpart. Whereas
the $p$-processor cup has optimal backlog $\Theta(\log n)$, the
variable-processor game has optimal backlog $\Theta(n)$.
Moreover, there is an efficient filling strategy that yields
backlog $\Omega(n^{1 - \epsilon})$ in quazi-polynomial time against any
deterministic emptying strategy.

We additionally show that straightforward uses of randomization
cannot be used to help the emptier. In particular, for any
positive constant $\Delta$, and any $\Delta$-greedy-like randomized
emptying algorithm $\mathcal{A}$, there is a filling strategy
that achieves backlog $\Omega(n^{1 - \epsilon})$ against $\mathcal{A}$ in
quazi-polynomial time.

\thispagestyle{fancy} % page number to bottom right

