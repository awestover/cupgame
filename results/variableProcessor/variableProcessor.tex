\documentclass[twocolumn]{article}[10pt]
\usepackage[left=1in, right=1in, top=1in, bottom=1in]{geometry}
\usepackage[subtle]{savetrees}

\usepackage{amsthm}
\usepackage{amssymb}
\usepackage{amsmath}
\usepackage{mathtools}
\usepackage{hyperref}
\usepackage{xcolor}
\usepackage{xspace}

\newcommand{\defn}[1]{{\textit{\textbf{\boldmath #1}}}\xspace}
\renewcommand{\paragraph}[1]{\vspace{0.09in}\noindent{\bf \boldmath #1.}} 
\DeclareMathOperator{\E}{\mathbb{E}}
\DeclareMathOperator{\Var}{\text{Var}}
\DeclareMathOperator{\img}{Im}
\DeclareMathOperator{\polylog}{\text{polylog}}
\DeclareMathOperator{\poly}{\text{poly}}
\DeclareMathOperator{\st}{\text{ such that }}
\DeclareMathOperator{\tilt}{\text{tilt}}
\DeclareMathOperator{\fil}{\text{fill}}
\newcommand{\norm}[1]{\left\lVert#1\right\rVert}

\newcommand{\contr}[0]{\[ \Rightarrow\!\Leftarrow \]}
\newcommand{\defeq}{\vcentcolon=}
\newcommand{\eqdef}{=\vcentcolon}

\newtheorem{fact}{Fact}
\newtheorem{definition}{Definition}
\newtheorem{remark}{Remark}
\newtheorem{proposition}{Proposition}
\newtheorem{clm}{Claim}
\newtheorem{lemma}{Lemma}
\newtheorem{corollary}{Corollary}
\newtheorem{theorem}{Theorem}
\newtheorem{conjecture}{Conjecture}

\usepackage{authblk}
\usepackage{fancyhdr}
\pagestyle{fancy}
\fancyhead{}
\fancyfoot{}
\fancyfoot[R]{\thepage}
\renewcommand{\headrulewidth}{0pt}

\title{Variable-Processor Cup Games}
\date{\vspace{-5ex}}

\author[1]{\small William Kuszmaul\thanks{Supported by a Hertz fellowship and a NSF GRFP fellowship}}
\author[2]{\small Alek Westover\thanks{Supported by MIT PRIMES}}

\affil[ ]{\footnotesize MIT\textsuperscript{1}, MIT PRIMES\textsuperscript{2}}
\affil[ ]{\textit{kuszmaul@mit.edu, alek.westover@gmail.com}}


\begin{document}
\maketitle
\abstract{ 
  In a \defn{cup game} a filler and an emptier take turns adding and removing water
  from cups, subject to certain constraints. In one version of the cup game,
  the $p$-processor cup game, the filler distributes $p$ units of water among
  the $n$ cups with at most $1$ unit of water to any particular cup, and the
  emptier chooses $p$ cups to remove at most one unit of water from. Analysis
  of the cup game is important for applications in processor scheduling, buffer
  management in networks, quality of service guarantees, and deamortization.

  We investigate a new variant of the classic multi-processor cup game, which
  we call the \defn{variable-processor cup game}, in which the resources of the
  emptier and filler are variable. In particular, in the variable processor cup
  game the filler is allowed to change $p$ at the beginning of each round. 
  Although the modification to allow variable resources seems small, we will
  show that it drastically alters the outcome of the game.

  We construct a filling strategy that an adaptive filler can use to get
  backlog $\Omega(n)$ in running time $2^{O(n)}$. We also construct a filling
  strategy that an adaptive filler can use to get backlog
  $\Omega(n^{1-\epsilon})$ for any constant $\epsilon > 0$ in running time
  $2^{O(\log ^2 n)}$. Not only is this bound shockingly large, but the steep
  trade off-curve between running-time and backlog is very surprising: the time
  required goes up from quasi-polynomial time to exponential time!

  Furthermore, we demonstrate that this lower bound on backlog is tight: 
  using a novel set of invariants we prove that a greedy emptier never lets
  backlog exceed $O(n)$.

  We also investigate bounds on an oblivious filler. We show, using
  concentration bounds for random variables (Hoeffding's Inequality), that --
  surprisingly -- an oblivious filler can achieve essentially the same lower
  bounds as an adaptive filler. However, oblivious lower bounds are only
  supposed to take $O(\poly(n))$ time, so we apply the strategies from the
  adaptive case to a subset of the cups. Doing so, an oblivious
  filler can achieve backlog $2^{\Omega(\sqrt{\log n})}$ in running time
  $O(\poly(n))$ with constant probability against any ``greedy-like" emptier.
}

\section{Introduction}\label{sec:intro}
\paragraph{Definition and Motivation}
The \defn{cup game} is a multi-round game in which the two players -- the
\defn{filler} and the \defn{emptier} -- take turns adding and removing water
from cups. On each round of the classic \defn{$p$-processor cup game} on $n$
cups, the filler first distributes $p$ units of water among
the $n$ cups with at most $1$ unit to any particular cup (without this
restriction the filler can trivially achieve unbounded backlog by placing all
of its fill in a single cup every round), and then the emptier 
removes $1$ unit of water from each of $p$ cups.

The cup game naturally arises in the study of processor-scheduling.
The incoming water added by the filler represents work added to the system at
time steps. At each time step after the new work comes in each of $p$
processors must be allocated to a task, which they will achieve $1$ unit of
progress on before the next time step. The assignment of processors to tasks is
modeled by the emptier deciding which cups to empty from. The backlog of
the system is the largest amount of work left on any given task. To model this,
in the cup game, the \defn{backlog} of the cups is defined to be the fill of
the fullest cup at a given state. 
It is important to know bounds on how large backlog can get.

\paragraph{Previous Work}
The bounds on backlog are well known for the case where $p=1$, i.e. the
\defn{single-processor cup game}.
In the single-processor cup game an adaptive filler can achieve backlog
$\Omega(\log n)$ and a greedy emptier never lets backlog exceed $O(\log n)$.
The bounds are much better against an oblivious filler.
In the randomized version of the single-processor cup game, which can be
interpreted as a smoothed analysis of the deterministic version, the emptier
never lets backlog exceed $O(\log \log n)$, and a filler can achieve backlog
$\Omega(\log\log n)$.

Recently Kuszmaul has achieved bounds on the case where $p>1$, i.e. the
\defn{multi-processor cup game} \cite{wku20}. Kuszmaul showed that in the 
$p$-processor cup game on $n$ cups a greedy emptier never lets backlog exceed
$O(\log n)$. He also proved a lower bound of $\Omega(\log (n-p))$. Recently we
showed a lower bound of $\Omega(\log n - \log (n-p))$. Combined these bounds
imply a lower bound of $\Omega(\log n)$.
Kuszmaul also established an upper bound of $O(\log\log n + \log p)$ against
oblivious fillers, and a lower bound of $\Omega(\log\log n)$ (note that tight
bounds on backlog against an oblivious filler are not yet known for large $p$).

\paragraph{Our Variant}
We investigate a variant of the vanilla multi-processor cup game which we call
the \defn{variable-processor cup game}. In the variable-processor cup game the
filler is allowed to change $p$ (the total amount of water that the filler
adds, and the emptier removes, from the cups per round) at the beginning of
each round. Note that we do not allow the resources of the filler and emptier
to vary separately; just like in the classic cup game we take the resources of
the filler and emptier to be identical.
Although this restriction may seem artificial, it is crucial; if
the filler has more resources than the emptier, then
the filler could trivially achieve unbounded backlog, as average fill will
increase by at least some positive constant at each round.
Taking the resources of the players to be identical makes the game balanced,
and hence interesting.

A priori having variable resources offers neither player a clear advantage:
lower values of $p$ mean that the emptier is at more of a discretization
disadvantage but also mean that the filler can ``anchor" fewer cups\footnote{A
useful part of many filling algorithms is maintaining an ``anchor" set of
``anchored" cups. The filler always places $1$ unit of water in each anchored
cup. This ensures that the fill of an anchored cup never decreases after it is
placed in the anchor set.}. We hoped that the variable-processor cup game could
be simulated in the vanilla multi-processor cup game, because the extra
ability given to the filler does not seem very strong. 

% We invented the new version of the cup game arose as we tried to get a bound of
% $\Omega(\log p)$ backlog in the multi-processor game against an oblivious
% filler, which would combine with previous results to give us a lower bound that
% matches our upper bound: $O(\log\log n + \log p)$ \cite{wku20}. In Proposition
% \ref{prop:obliviousBase} we prove that there is an oblivious filling strategy
% in the variable-processor cup game on $n$ cups that achieve backlog
% $\Omega(\log n)$ as desired. \footnote{Note that we have $\Omega(\log n)$ in
%   this proposition instead of $\Omega(\log p)$ because the filler can increase
%   the number of processors, so it increases the number of processors to $n-1$
%   to start. A nearly identical construction could be used to show that backlog
%   $\Omega(\log p_{\max})$ can be achieved, where the number of processors
%   starts at $p_{\max}$ and the filler does not ever increase the number of
%   processors. However, using $p_{\max} = n$ is natural in the
% variable-processor cup game, so we do not consider the game with the
% restriction that the filler can not increase the number of processors above
% some $p_{\max} < n$.}
However, we show that attempts at simulating the variable-processor cup
game are futile because the variable-processor cup game
is--surprisingly--vastly different from the multi-processor cup game. 

\paragraph{Outline and Results}
In Section \ref{sec:prelims} we establish the conventions and notations we will
use to discuss the variable-processor cup game. 

In Section \ref{sec:adaptive} we provide an inductive
proof of a lower bound on backlog in Corollary \ref{cor:adaptivePoly}.
The base case of the argument is a direct consequence of Proposition
\ref{prop:adaptiveBase}, and the inductive step follows from the ``Adaptive
Amplification Lemma" (Lemma \ref{lem:adaptiveAmplification}). Corollary
\ref{cor:adaptivePoly} gives a lower bound of $\Omega(n)$ on backlog. Corollary
\ref{cor:adaptivePoly} provides two algorithms: one algorithm with running time
$2^{O(n)}$ that achieves backlog $\Omega(n)$, and another with running time
$2^{O(\log^2 n)}$ that achieves backlog $\Omega(n^{1-\epsilon})$ for any
constant $\epsilon > 0$.

In Section \ref{sec:adaptiveUpperBound} we prove a novel invariant: the average
fill of the $k$ fullest cups is at most $n-k$. In particular this implies
(setting $k=1$) that backlog is $O(n)$. Thus, our analysis is tight.

Section \ref{sec:oblivious} has similar macro-structure to Section
\ref{sec:adaptive}: We lower bound backlog in Corollary
\ref{cor:obliviousPoly}, using Proposition \ref{prop:obliviousBase} as the base
case of the inductive argument and the ``Oblivious Amplification Lemma" (Lemma
\ref{lem:obliviousAmplification}) to facilitate the inductive step. Corollary
\ref{cor:obliviousPoly} gives the lower bound $2^{\Omega(\sqrt{\log n})}$ on
backlog, against a ``greedy-like" emptier. In particular the corollary asserts
that we can achieve this backlog in time $O(\poly(n))$. Note that the
restriction on runtime of the filler is the main way in which this bound
differs from the adaptive case.

\section{Preliminaries}\label{sec:prelims}
The cup game consists of a sequence of rounds. On the $t$-th round the state
starts as $S_t$. The filler chooses the number of processors $p_t$ for the round. 
Then the filler distributes $p_t$ units of water among the cups (with at most
$1$ unit of water to any particular cup). After this, the game is in an intermediate
state, which we call state $I_t$. Then the emptier chooses
$p_t$ cups to empty $1$ unit of water from. This concludes the round; the state of the game is now $S_{t+1}$.

Denote the fill of a cup $c$ by $\fil(c)$. Let the \defn{positive tilt} of a cup $c$ be
$\tilt(c) = \max(0, \fil(c))$, and let the positive tilt of a set $X$ of
cups be $\sum_{c\in X} \tilt(c)$. Let the \defn{mass} of a set of cups $X$
be $m(X) = \sum_{c\in X} \fil(c)$. Denote the average fill of a set of cups $X$
by $\mu(X)$. Note that $\mu(X) |X| = m(X)$.

Let the \defn{rank} of a cup at a given state be its position in a list of the
cups sorted by fill at the given state, breaking ties arbitrarily but
consistently. For example, the fullest cup at a state has rank $1$, and the
least full cup has rank $n$.

We adopt the convention of allowing for negative fill: i.e. regardless of the
fills of the cups the emptier always empties exactly $1$ unit of water from
$p$ cups. \footnote{
  Allowing for cups to have negative fill makes the game strictly harder for
  the filler, as it means that none of the emptier's emptying is ever wasted by
  cups "zeroing-out". In variants of the cup game without negative fill, when
  the emptier empties from a cup with fill less than $1$ the cups fill goes to
  $0$. This however implies that the total mass of water removed by the emptier
  is less than $p$. Our filling strategy does not rely on cups zeroing out to
  achieve large backlog however, so the lower bounds hold regardless of our
  choice of allowing or not allowing negative fill. 

  On the other hand, the proof of the upper bound does depend on the fact that
  cups never zero out. In particular, throughout the proof of the upper bound
  we call the average fill of the cups $0$ and say that it never changes. We
  chose to allow negative fill (which has no physical analog in work
  scheduling) because it makes the problem more elegant.
}

\section{Adaptive Filler Lower Bound}\label{sec:adaptive}
\begin{proposition}
\label{prop:adaptiveBase}
  There exists an adaptive filling strategy for the variable-processor cup game
  on $n$ cups that achieves backlog at least $\frac{1}{4}\ln (n/2)$, where fill
  is relative to the average fill of the cups, with negative fill allowed.
\end{proposition}
\begin{proof}
  Let $h = \frac{1}{4}\ln (n/2)$ be the desired fill. Once a cup with fill at
  least $h$ is achieved the filler stops, the process completed.  
  Let $A$ consist of the $n/2$ fullest cups, and $B$ consist of the rest of the
  cups (at any given state, so $A, B$ are implicitly functions of the round
  $t$).

  If the process is not yet complete, that is $\fil(c) < h$ for all cups $c$,
  then $\tilt(A\cup B) < h\cdot n$. Assume for
  sake of contradiction that there are more than $n/2$ cups $i$ with $\fil(c)
  \le -2h$. The mass of those cups would be at most $-hn$, but there isn't
  enough positive tilt to oppose this, a contradiction. Hence there are at most
  $n/2$ cups $c$ with $\fil(c) \le -2h$. 

  We set the number of processors equal to $1$ and play a single processor cup
  game on $n/2$ cups that have fill at least $-2h$ (which must exist) for $n/2
  -1$ steps. We initialize our \defn{active set} --the set of cups that we
  place fill in-- to be $A$. Note that that $\fil(c) \ge -2h$ for all cups $c\in
  A$, as $A$ consists of the $n/2$ fullest cups. We will remove $1$ cup from
  the active set at each step.
  At each step the filler distributes water equally among the cups in its
  active set. Then, the emptier will choose some cup to empty from. If this cup
  is in the active set the filler removes it from the active set. Otherwise, the
  filler chooses an arbitrary cup to remove from the active set.

  After $n/2-1$ steps, the active set will consist of a single cup. This cup's
  fill has increased by $1/(n/2) + 1/(n/2 - 1) + \cdots + 1/2 + 1/1
  \ge \ln n/2 = 4h$. Thus such a cup has fill at least $2h$ now, so the
  proposition is satisfied.
\end{proof}

\begin{lemma}[The Adaptive Amplification Lemma]\label{lem:adaptiveAmplification}
  Let $f$ be an adaptive filling strategy that achieves backlog $f(n)$ in the
  variable-processor cup game on $n$ cups (relative to average fill, with
  negative fill allowed).
  Let $n_0 \in \mathbb{N}$ be a constant such that we can achieve backlog $1$ on $n_0$ cups.
  Let $\delta\in(0,1)$ be a parameter, and let $L\in\mathbb{N}$ be a constant
  such that $n_0 \le (1-\delta)\delta^L n \le n_0/\delta$.

  Then, there exists an adaptive filling strategy that achieves backlog $f'(n)$ satisfying
  $$f'(n) \ge (1-\delta)\sum_{\ell= 0}^{L} f((1-\delta)\delta^\ell n)$$
  and $f'(n) \ge 1$, in the variable processor cup game on $n\ge n_0$ cups.
\end{lemma}
\begin{proof}
  The basic idea of this analysis is as follows:
  \begin{enumerate}
    \item Using $f$ repeatedly, achieve average fill at least $(1-\delta)
      f(n(1-\delta))$ in a set of $n\delta$ cups. 
    \item Reduce the number of processors to $n\delta$.
    \item Recurse on the $n\delta$ cups with high average fill.
  \end{enumerate}

  Let $A$, the \defn{anchor set}, be initialized to consist of the $n\delta$
  fullest cups, and let $B$ the \defn{non-anchor set} be initialized to consist
  of the rest of the cups (so $|B| = (1-\delta)n$).
  Let $n_\ell = n\delta^{\ell-1}$, $h_\ell = (1-\delta)f(n_\ell(1-\delta))$;
  the filler will achieve a set of at least $n_\ell \delta$ cups with average
  fill at least $h_\ell$ on the $\ell$-th
  level of recursion. On the $\ell$-th level of recursion $|A| = \delta\cdot
  n_\ell, |B| = (1-\delta)\cdot n_\ell$.

  We now elaborate on how to achieve Step 1.
  Our filling strategy always places $1$ unit of water in each anchor cup. This
  ensures that average fill in the anchor set is non-decreasing.

  On the $\ell$-th level of recursion the filler uses the following
  \defn{process} to achieve the desired average fill in $A$: repeatedly apply
  $f$ to $B$, and then take the cup generated by $f$ within $B$ to have large
  backlog and swap it with a cup in $A$; repeat until $A$ has the desired
  average fill. Note that $$\mu(A) \cdot |A| +\mu(B)\cdot |B| = 0,$$ so
  $$\mu(A) = - \mu(B) \cdot (1-\delta)/ \delta.$$

  Thus, if at any point in the process $B$ has average fill lower than $-h_\ell
  \cdot \delta/(1-\delta)$, then $A$ has average fill at least $h_\ell$, so the
  process is finished. So long as $B$ has average fill at least $-h_\ell\cdot
  \delta/(1-\delta)$ we will apply $f$ to $B$.
  
  It is somewhat complicated to apply $f$ to $B$ however, because we need to
  guarantee that in the steps that the algorithm takes while applying $f$ the
  emptier always empties the same amount of water from $B$ as the filler fills
  $B$ with. This might not be the case if the emptier does not empty from each
  anchor cup at each step. Say that the emptier \defn{neglects} the anchor set
  on an application of $f$ if there is some step during the application of $f$
  in which the emptier does not empty from some anchor cup.

  We will apply $f$ to $B$ at most $h_\ell n_\ell\delta + 1$ times, and at the
  end of an application of $f$ we only swap the generated cup into $A$ if the
  emptier has not neglected the anchor set during the application of $f$.

  Note that each time the emptier neglects the anchor set the mass of the
  anchor set increases by $1$. If the emptier neglects the anchor set $h_\ell
  n_\ell\delta + 1$ times, then the average fill in the anchor set increases by
  more than $h_\ell$, so the desired average fill is achieved in the anchor set.

  Otherwise, there must have been an application of $f$ for which the emptier
  did not neglect the anchor set. We only swap a cup into the anchor set if
  this is the case. In this case we achieve fill 
  $$-h_\ell \cdot \delta/(1-\delta) + f(n_\ell (1-\delta)) = (1-\delta)f(n_\ell
  (1-\delta)) = h_\ell$$
  in a non-anchor cup, and swap it with the smallest cup in the anchor set.

  We achieve average fill $h_\ell$ in the anchor set for $L$ levels of
  recursion. Summing $h_\ell$ for $0\le \ell \le L$ yields the desired result.

  Note that as $n\ge n_0$ we can always simply use Proposition
  \ref{prop:adaptiveBase} to achieve backlog $1$. We will revert to this option
  if it gives larger fill than we get by repeatedly applying $f$.
\end{proof}

\begin{corollary}
  \label{cor:adaptivePoly}
  There are adaptive filling strategies for the variable-processor cup game on
  $n$ cups that achieve backlog $\Omega(n)$ in time $2^{O(n)}$ and backlog
  $\Omega(n^{1-\epsilon})$ for any constant $\epsilon > 0$ in time $2^{O(\log^2 n)}$.
\end{corollary}
\begin{proof}$ $\\
  Fix $\epsilon \in (0,1/2)$, and let $c, \delta$ be parameters, with $c\in
  (0,1), 0 < \delta \ll 1/2$ -- these will depend on $\epsilon, n$.
  Say that we aim to achieve backlog at least $cn^{1-\epsilon}$.
  Observe that if we apply the Amplification Lemma to the function satisfying
  $f(k) \ge ck^{1-\epsilon}$ for $k \le g$ then for any $k_0$ with
  $k_0(1-\delta)\le g$ (which enforces $k_0 \le g/ (1-\delta)$) we have the
  following:
  \begin{align*}
  f'(k_0)\ge&\\
  &(1-\delta)\sum_{\ell=0}^L c (((1-\delta)\delta^\ell)k_0)^{1-\epsilon}\\
  &= ck_0^{1-\epsilon} (1-\delta)^{2-\epsilon} \sum_{\ell=0}^L (\delta^\ell)^{1-\epsilon},
  \end{align*}
  where $L$ is the greatest integer such that $(1-\delta)\delta^Ln \ge n_0$
  where $n_0$ is a constant such that we can achieve backlog $1$ on $n_0$ cups
  (this definition is identical to the definition in the statement of Lemma
  \ref{lem:adaptiveAmplification}).
  Note that as $\delta$ will be very small, $\sum_{\ell=0}^L
  (\delta^L)^{1-\epsilon}$ is very well approximated by
  $1+\delta^{1-\epsilon}$, so we will not loose much by relaxing our lower
  bound on $f'(k_0)$ to only use the first $2$ terms of the sum. Then we have 
  $$f'(k_0) \ge ck_0^{1-\epsilon}(1-\delta)^{2-\epsilon}(1+\delta^{1-\epsilon}).$$
  Let 
  $$h(\delta) = (1-\delta)^{2-\epsilon}(1+\delta^{1-\epsilon}).$$
  We prove the following claim:

  \begin{clm}
    \label{clm:validchoices}
    There exists an appropriate choice of $\delta$ that is small enough such
    that $h(\delta) >1$ and large enough such that $L \ge 1$, when $\epsilon$
    is chosen to be $4/\lg n$, or a positive constant. In particular, if
    $\epsilon$ is chosen to be $4/\lg n$ then we will choose $\delta \le
    O(1/n)$, and if $\epsilon$ is chosen to be a positive constant then we will
    choose $\delta \le O(1)$.
  \end{clm}

  Note that if $h(\delta)\ge 1$, then $f'(k_0) \ge c k_0^{1-\epsilon}$, meaning
  we have constructed from $f$ a new function $f'$ that satisfies the
  inequality $f'(k) \ge ck^{1-\epsilon}$ for $k\le g/(1-\delta)$, as opposed to
  only for $k \le g$ as in the case of $f$. 
  \footnote{Note that although $f'(k) \ge ck^{1-\epsilon}$ holds for at least as many $k$
    as $f(k) \ge c k^{1-\epsilon}$, it does not necessarily hold for strictly
    more; in particular, if $\lfloor g/(1-\delta) \rfloor = g$ then the inequality on
    $f'$ holds for no more $k$ than the inequality on $f$, as $f$ and $f'$ are
    functions on $\mathbb{N}$. We have to be somewhat careful about the fact that
    there are an integer number of cups.} 
  Then by repeatedly amplifying a function, we should be able to arbitrarily
  extend the support, which would help us attain the desired backlog.

  We now prove Claim \ref{clm:validchoices}.
  \begin{proof}
  Consider the Taylor series for $(1-\delta)^{2-\epsilon}$ about $\delta = 0$:
  $$(1-\delta)^{2-\epsilon} = 1 - (2-\epsilon)\delta + O(\delta^2).$$
 
  So, to find a $\delta$ where $h(\delta) \ge 1$ it suffices -- note that, again, we choose
  to neglect the $\delta^2$ term as it does not help us substantially because
  it is so small -- to find a $\delta$ with 
  $$(1-(2-\epsilon)\delta)(1+\delta^{1-\epsilon}) \ge 1.$$
  Rearranging we have 
  $$\delta^{1-\epsilon} \ge (2-\epsilon)\delta + (2-\epsilon)\delta^{2-\epsilon}.$$
  This clearly is true for sufficiently small $\delta$, as
  $\delta^{1-\epsilon}$ will be much greater than $\delta$ or
  $\delta^{2-\epsilon}$.
  However it will be beneficial to have a more explicit criterion for possible
  choices of $\delta$ in terms of $\epsilon$. To get this, we enforce a much
  stronger inequality on $\delta^{1-\epsilon}$ by vastly overestimating
  $\delta^{2-\epsilon}$ as $\delta$. Surprisingly even with this overestimate
  we are still able to get the desired value of $\epsilon$ to work, as we will demonstrate later.
  We have,
  \begin{equation}
    \label{eqn:deltaUpperIneq}
    \delta \le \frac{1}{(2(2-\epsilon))^{1/\epsilon}}. 
  \end{equation}

  In addition to the constraint that $\delta$ must be small enough such that
  $h(\delta) \ge 1$, the only other constraint on $\delta$ is that $\delta$
  must be large enough that the sum from the Amplification Lemma has at least
  two terms, i.e. such that $L \ge 1$.
  The condition $L \ge 1$ enforces 
  $$\delta(1-\delta)n \ge n_0. $$
  Recall that we choose $\delta < 1/2$, so $1-\delta > 1/2$. Thus to make
  $\delta$ sufficiently big it suffices to chose $\delta$ with 
  \begin{equation}
    \label{eqn:deltaLowerBound}
    \delta \ge 2n_0/n.
  \end{equation}
  Any choice of $\delta$ that is sufficiently large to make $L \ge 1$ and
  simultaneously small enough to make $h(\delta) \ge 1$ is a valid choice of
  $\delta$. That is, $\delta$ is valid if and only if it satisfies
  \begin{equation}
    \label{eqn:deltainequality}
       2n_0/n \le \delta \le  \frac{1}{(2(2-\epsilon))^{1/\epsilon}}.
  \end{equation}
  To achieve the desired backlog of $\Omega(n)$ we can use $\epsilon =
  \gamma/\lg n$ for appropriate constant $\gamma$, as $$n^{1-\gamma/\lg n} =
  n/2^\gamma = \Omega(n).$$
  We show that there is a valid choice of $\gamma$ such that the following inequality is satisfied:
  \begin{equation}
    \label{eqn:thatinequality}
   2n_0/n \le \frac{1}{(2(2-\gamma/\lg n))^{(1/\gamma)\lg n}}.
  \end{equation}
  Note that 
  $$(2(2-\gamma/\lg n))^{(1/\gamma)\lg n} \le 4^{(1/\gamma)\lg n} \le n^{2/\gamma}$$
  Clearly by choosing e.g. $\gamma = 4$ we have the desired inequality.
  Inequality \ref{eqn:thatinequality} implies that there is a valid choice of
  $\delta$ when we chose $\epsilon = \gamma / \lg n$. When proving that we can
  achieve backlog $\Omega(n)$ we use $\epsilon = 4 / \lg n$, and $\delta =
  O(1/n)$ satisfying Inequality \ref{eqn:deltainequality}, based on our choice
  of $\epsilon$. When proving that we can achieve backlog
  $\Omega(n^{1-\epsilon})$ for constant $\epsilon > 0$ we choose $\delta$ to be
  a constant satisfying Inequality \ref{eqn:deltaUpperIneq}, and $\delta$, being constant, is
  trivially not too small, hence satisfies Inequality \ref{eqn:deltaLowerBound}.
    
  \end{proof}

  Now we proceed to show that with the appropriate values of $\delta, \epsilon$ we
  can achieve a filling strategy that achieves backlog $cn^{1-\epsilon}$ on $n$ cups.
  First we present a simple existential argument which asserts that a strategy
  that achieves the desired backlog exists. Then we provide two constructive
  arguments: one achieving backlog $\Omega(n)$ in running time $2^{O(n)}$,
  the other achieving backlog $\Omega(n^{1-\epsilon})$ for constant $\epsilon>
  0$ in running time $2^{O(\log^2 n)}$. Both constructive arguments rely on
  repeated application of the Amplification Lemma.
  
  \paragraph{Existential Argument}
  Let $\epsilon > 0$ be constant. By Claim \ref{clm:validchoices}, there is a
  valid constant setting of $\delta$; let $\delta \ll 1/2$ be appropriate
  constant. Let $f^*$ be the supremum over all filling strategies of the
  backlog achievable on $n$ cups. Then $f^*$ must be greater than or equal to
  the amplification of $f^*$. Assume for contradiction that there is some least
  $n_*$ such that 
  $$\begin{cases}
    f^*(k)< ck^{1-\epsilon}, & k > n_*\\
    f^*(k)> ck^{1-\epsilon}, & k < n_*
  \end{cases} $$
  Note that $n_*(1-\delta)\delta \ge n_0$ by appropriate choice of constant
  $c$, and Proposition \ref{prop:adaptiveBase}, which states that we can get
  backlog $O(\log n_*)$ on $n_*$ cups\footnote{Note: this is where it is
  crucial that $\epsilon, \delta$ are constants.}.
  Because $f^*$ satisfies the Amplification Lemma we have:
\begin{align*}
  f^*(n_*) & \\
           &\ge (1-\delta)\sum_{\ell=0}^L f^*((1-\delta)\delta^\ell n_*) \\
           &\ge cn_*^{1-\epsilon} h(\delta)\\
           &\ge cn_*^{1-\epsilon}
\end{align*}
which is a contradiction. Hence $f^*$ achieves backlog $cn^{1-\epsilon}$ for all $n$.

\paragraph{Constructive Argument achieving backlog $\Omega(n^{1-\epsilon})$ (for constant $\epsilon > 0$) in time $2^{O(\log^2 n)}$}
It is desirable to have an algorithm for achieving this backlog with bounded
running time; we now modify the existential argument to make it constructive, 
which yields an algorithm for achieving backlog $cn^{1-\epsilon}$ on $n$ cups 
in finite running time. We again use constant $\epsilon > 0$ and appropriate constant $\delta$.

  We start with the algorithm given by Proposition \ref{prop:adaptiveBase} for achieving backlog
  $$f_0(k) = 
  \begin{cases} 
    \lg k, & k\geq 1, \\
    0 & \text{else.}
  \end{cases}$$
  Then we construct an algorithm that achieves better backlog using the
  Amplification Lemma (Lemma \ref{lem:adaptiveAmplification}):
  we construct $f_{i+1}$ as the amplification of $f_{i}$. 

  Define a sequence $g_i \in \mathbb{N}^\infty$ with 
  $$ g_i = \begin{cases}
    \lceil 1/\delta \rceil \gg 1,  & i = 0,\\
    \lceil g_{i-1}/(1-\delta)\rceil -1 & i  \ge 1
  \end{cases} $$
  Note that is, $g_{i+1}$ is the greatest integer strictly less than $g_i/(1-\delta)$.
  Note that $ (1/\delta) / (1-\delta) > (1+\delta)/\delta = 1/\delta + 1.$
  Thus $g_1 = 1+ g_0$, and in general, $g_{i+1} > g_i$, because the difference
  $g_{i+1}-g_i$ can only grow as $i$ grows.

  We claim the following regarding this construction:
  \begin{equation}
    \label{clm:fikinduction}
    f_i(k) \ge ck^{1-\epsilon} \text{ for all } k < g_i. \tag{*}
  \end{equation}
  We shall prove Claim \ref{clm:fikinduction} by induction on $i$.

  Claim \ref{clm:fikinduction} is true in the base case of $f_0$ by taking $c$
  sufficiently small, in particular small enough that $f_0(k) \ge
  ck^{1-\epsilon}$ holds for $k < g_0$.\footnote{Note: this is where it is
    crucial that $\epsilon, \delta$ are constants.}
  As our inductive hypothesis we assume Claim \ref{clm:fikinduction} for $f_i$;
  we aim to show that Claim \ref{clm:fikinduction} holds for $f_{i+1}$. Note
  the key property of $g_i$, that $g_{i+1}\cdot(1-\delta) < g_i$. Also note
  that (without loss of generality) the $f_i$ are monotonically increasing
  functions: given more cups we can always achieve higher fill than with fewer
  cups. Thus we have, for any $k<g_{i+1}$,
  \begin{align*}
    f_{i+1}(k) &\\
    &\ge (1-\delta)\sum_{\ell=0}^L f_i((1-\delta)\delta^\ell k)\\
    &\ge ck^{1-\epsilon}h(\delta)\\
    &\ge ck^{1-\epsilon},
  \end{align*}
  as desired. 

  Note that $g_{i+1} \ge g_i + 1$ so by continuing this process we eventually
  reach some $f_{i_*}$ such that $f_{i_*}(n) \ge cn^{1-\epsilon}$.
  Note that $i_* \le n$.
  Let the running time $f_{i_*}(n)$ be $T(n)$.
  Note that $f_{i_*}(n)$ must call $f_{i_*-1}(n(1-\delta)\delta^\ell)$ as many
  as $n(1-\delta)\delta^\ell$ times, for all $0 \le \ell\le L$. However, we
  only use the terms of the sum where $\ell=0,1$, so we could use a modified
  version of the Amplification Lemma in which we truncate the sum.
  This we have the following (loose) recurrence bounding $T(n)$:
  $$T(n) \le \delta n \cdot T(n(1-\delta)) + T(\delta n).$$
  We can upper bound this by
  $$n^{\frac{\log n}{\log (1/(1-\delta))}}.$$
  Continuing for $O(\log n)$ levels of recursion should be sufficient to
  achieve the desired backlog. This gives running time
  $$T(n) \le ((1+\delta) n)^{O(\log n)} \le 2^{O(\lg^2 n)}$$
  as desired.
  {\color{red} ok, technically I'm ignoring integer problems in saying lets
  only do $O(\lg n)$ levels of recursion, it'll be enough. I should prove it.
But for $\delta$ constant it seems pretty obvious.}

  \paragraph{Constructive Argument for backlog $\Omega(n)$ in time $2^{O(n)}$}
  We describe a simple filling strategy that gives the desired backlog. Let
  $n_0 \le O(1)$ be a constant such that we can achieve backlog $1$ on $n_0$
  cups, and note that this is possible by Proposition \ref{prop:adaptiveBase}.
  We construct a function that achieves large backlog on $n$ cups.
  To achieve large backlog on $n$ cups we first recursively apply our function to
  $(1-\delta)n$ cups repeatedly (for each of the $\delta n$ cups that we are
  attempting to get high fill in), as described in the proof of the
  Amplification Lemma, and transfer over the cups that we get. Then we achieve backlog $1$ on
  the $\delta n$ cups whose average fill has been increased. The backlog we
  achieve satisfies the following recurrence:
  $$f(n) \ge \begin{cases}
    (1-\delta)f((1-\delta)n) + 1, & \text{if } n\delta(1-\delta) > n_0\\
    0, \text{ else.}
  \end{cases}$$
  Let $(1-\delta)^c = \delta$, let $\delta^2 n < n_0 < (1-\delta)^{2c-1} n$ by our choice of $\delta = O(1/n)$.
    We can get backlog 
    $$\sum_{i=1}^c (1-\delta)^i. $$
    To see this, consider a binary tree representing our algorithm. At every
    branch we both proceed to recurse on a $1-\delta$ fraction of the cups, and
    achieve backlog $1$ on a $\delta$ fraction of the cups.

    The sum evaluates to 
    $$\frac{(1-\delta)^2}{\delta}$$
    which, if we chose $\delta = 1/n$, becomes $\Omega(n)$.

    The running time satisfies the recurrence 
    $$T(n) = \delta n T((1-\delta)n) + O(1)$$
    because to achieve backlog $f(n)$ we must achieve backlog
    $f((1-\delta)n)$ $\delta n$ times, and then achieve backlog $1$ on the
    remaining cups. Solving this recurrence yields that the running time is
    $$\frac{(\delta n)^c - 1}{\delta n - 1}.$$
    Recalling that $\delta = O(1/n)$ this becomes 
    $$2^{O(n)}.$$


    % Solving this inequality yields
    % $$\epsilon \ge \frac{2 \ln(n) - W(\frac{1}{2} n^2 \ln n)}{\ln n}$$
    % where $W$ is the Lambert-W function, i.e. the inverse of the function $z\mapsto ze^z$.
    % Note that as $n\to \infty$ smaller and smaller values of $\epsilon$ are permissible.

    % Note that the Lambert-W function satisfies $W(x) = \ln x -\ln\ln x + o(1)$.
    % Using this, we can get a very loose lower bound of $\Omega(n^{1-2/\ln n})$,
    % although of course the expression with the $W$ in it is a better lower bound.


\end{proof}

\section{Adaptive Filler Upper Bound}\label{sec:adaptiveUpperBound}
Let $\mu_S(X)$ and $m_S(X)$ denote the average fill and mass of a set of cups $X$
at state $S$ (e.g. $S=S_t$ or $S=I_t$).\footnote{Note that in the lower bound
  proofs (e.g. Section \ref{sec:adaptive}) when we used the notation $m$ (for
  mass) and $\mu$ (for average fill), we omitted the subscript indicating the
  state at which the properties were measured. In those proofs it was
sufficiently clear to leave the state implicit. However, in this section the
state is crucial, and needs to be explicit in the notation.}
Let $S_t(\{r_1, \ldots, r_m\})$ and $I_t(\{r_1,\ldots, r_m\})$ denote the cups
of ranks $r_1, r_2, \ldots, r_m$ at states $S_t$ and $I_t$ respectively.
Let $[n] = \{1,2,\ldots, n\}$, let $i+[n] = \{i+1, i+2, \ldots, i+n\}$. We will
use concatenation of sets to denote unions, i.e. $AB = A\cup B$.\\
We establish the following Lemma:

\begin{lemma}
  The greedy emptier maintains the invariant $$\mu_{S_t}(S_t([k])) \le n-k
  \text{ for all } t\ge 1, k \in [n].$$ In particular, for $k=1$, this
  says that the greedy emptier never lets backlog exceed $O(n)$.
\end{lemma}
\begin{proof}
First note that the invariant is trivial when $k=n$, as the average fill of the
set of all cups is by definition $0$.

We will prove the invariant by induction on $t$.
The invariant holds trivially for $t=1$ (the base case for the inductive proof): 
the cups start empty so $\mu_{S_1}(S_1([k])) = 0 \le n-k$.

Fix a round $t \ge 1$, and any $k \in [n-1]$. We assume invariant for all
values of $k\in[n]$ for state $S_t$ (we will only explicitly use two of the
invariants for each $k$, but the invariants that we need depend on the
choice of $p_t$ by the filler, so we actually need all of them) and show that
the invariant on the $k$ fullest cups holds on round $t+1$, i.e. that
$$\mu_{S_{t+1}}(S_{t+1}([k])) \le n-k.$$

Note that because the emptier is greedy it always empties from the cups $I_t([p_t])$.
Let $A$, with $a=|A|$, be $A = I_t([\min(k, p_t)]) \cap S_{t+1}([k])$; $A$
consists of cups that are among the $k$ fullest cups in $I_t$, are emptied from, and
are among the $k$ fullest cups in $S_{t+1}$.
Let $B$, with $b=|B|$, be $I_t([\min(k, p_t)]) \setminus A$; $B$ consists of
the cups that are among the $k$ fullest cups in state $I_t$, are emptied from,
and aren't among the $k$ fullest cups in $S_{t+1}$. 
Let $C = I_t(a+b+[k-a])$, with $c=k-a = |C|$ (Note that $k-a\ge 0$ as $a+b \le k$). 

Note that $a+ b = \min(k, p_t)$. We also have that $A = I_t([a])$ and $B =
I_t(a+[b])$, as every cup in $A$ must have higher fill than all cups in $B$ in
order to remain above the cups in $B$ after $1$ unit of water is removed from
all cups in $AB$.
Further, note that $S_{t+1}([k]) = AC$ because, once the cups in $B$
are emptied from, the cups in $B$ are not among the $k$ fullest cups, so the
cups in $C$ take their places among the $k$ fullest cups.\\
With these definitions made, we proceed to prove the Lemma.

First we prove that without loss of generality $S_t([a+b]) = I_t([a+b])$; we
call this fact the \defn{interchangeability of cups}.
\begin{proof}
  Say there are cups $x, y$ with $x\in S_t([a+b]) \setminus I_t([a+b]), y \in
  I_t([a+b])\setminus S_t([a+b])$. Let the fills of cups $x,y$ at state $S_t$
  be $f_x, f_y$; note that $f_x > f_y$. Let the amount of fill that the emptier
  adds to these cups be $\Delta_x, \Delta_y \le 1$; note that $f_x +\Delta_x <
  f_y + \Delta_y$.

Define a new state $S_t'$ where cup $x$ has fill $f_y$ and cup $y$ has fill $f_x$. 
Let the amount of water that the filler places in these cups from the new state be
$f_x-f_y+\Delta_x$ and $f_y-f_x + \Delta_y$ for cups $x,y$ respectively.
This is valid as both fill amounts are at most $1$: $f_x-f_y+\Delta_x<\Delta_y
\le 1$ and $f_y-f_x + \Delta_x < \Delta_x \le 1$.

We can repeatedly apply this process to swap each cup in $I_t([a+b])\setminus
S_t([a+b])$ into being one of the $a+b$ fullest cups in the new state $S_t'$.
At the end of this process we will have some ``fake" state $S_t^f$. Note that
$S_t^f$ must satisfy the invariants if $S_t$ satisfied the invariants, because
our process can be thought of as just relabelling the cups; in particular
$\fil(S_t^f(r)) = \fil(S_t(r))$ for all ranks $r \in [n]$.

It is without loss of generality that we start in state $S_t^f$ because from
state $I_t$ we could equally well have come from state $S_t$ or state $S_t^f$.
Thus we consider state $I_t$ to have come from state $S_t^f$.
\end{proof}

Now we proceed with the proof of the Lemma.\\
First we consider the case $b=0$. If $b=0$, then $S_{t+1}([k]) = S_t([k])$. 
The emptier has removed $a$ units of fill from the cups in $S_t([k])$
(specifically the cups in $A$), and the filler has distributed at most $a$
units of among the cups in $S_t([k])$. Thus the invariant holds:
$$m_{S_{t+1}}(S_{t+1}([k])) \le m_{S_t}(S_t([k]))+a-a \le k(n-k).$$

Now consider the case $b\neq 0$. Because $b>0$, and $a+b \le k$ we have that $a
< k$, and $c = k-a > 0$. Recall that $S_{t+1}([k]) = AC$, so the mass of the
$k$ fullest cups at $S_{t+1}$ is the mass of $AC$ at $S_t$ plus any water added
to cups in $AC$ by the filler, minus any water removed from cups in $AC$ by the
emptier. The emptier removes exactly $a$ units of water from $AC$.
The filler adds no more than $p_t$ units of water from $AC$ (because the filler
adds at most $p_t$ total units of water per round) and the filler also
adds no more than $k = |AC|$ units of water from $AC$ (because the filler adds
at most $1$ unit of water to each of the $k$ cups in $AC$).
Thus, the filler adds no more than $a+b = \min(p_t, k)$ units of water to $AC$.
Combining these observations we have:
$$m_{S_{t+1}}(S_{t+1}([k])) \le m_{S_t}(A) + m_{S_t}(C) + b.$$

% This is easy to bound if $m_{S_t}(C) \le m_{S_t}(BC) - b$, because 
% $$m_{S_t}(A) + m_{S_t}(BC)  = m_{S_t}(ABC) \le m_{S_t}([k])$$
% which would imply the invariant for $S_{t+1}$, $k$.
% If $\mu_{S_t}(C)$ is not significantly less than $\mu_{S_t}(BC)$ we have more difficulty.
The key insight necessary to bound this is to notice that larger values for
$m_{S_t}(A)$ correspond to smaller values for $m_{S_t}(C)$ because of the
invariants; the higher fill in $A$ \defn{pushes down} the fill that $C$ can
have. By quantifying exactly how much higher fill in $A$ pushes down fill in
$C$ we can achieve the desired inequality.
We can upper bound $m_{S_t}(C)$ by 
$$\frac{c}{b+c}m_{S_t}(BC) = (m_{S_t}(ABC) - m_{S_t}(A))\frac{c}{b+c}$$ because
$\mu_{S_t}(C) \le \mu_{S_t}(B)$ without loss of generality by the
interchangeability of cups.
Thus we have 
\begin{equation}
  \label{eqn:BCdiscounted}
m_{S_t}(AC) \le m_{S_t}(A) + \frac{c}{b+c}m_{S_t}(BC)
\end{equation}
{\color{red}
where 
\begin{equation}
  \label{eqn:redistributeA}
\begin{split}
  &m_{S_t}(A) + \frac{c}{b+c}m_{S_t}(BC) \\
  &= \frac{c}{b+c}m_{S_t}(ABC) + \frac{b}{b+c}m_{S_t}(A).
\end{split}
\end{equation}
Note that the expression in Equation \ref{eqn:redistributeA} is monotonically
increasing in both $\mu_{S_t}(ABC)$ and $\mu_{S_t}(A)$. 
Thus, by numerically replacing both average fills with
their extremal values ($n-|ABC|, n-|A|$) we upper bound $m_{S_t}(A) + m_{S_t}(C)$.
At this point the inequality can be verified by straightforward algebra,
however this is not elegant; instead, we combinatorially interpret the sum.

We define a new ``fake" state $F$, which may not represent
a valid configuration of cups (i.e. might not satisfy the invariants), where
$\mu_F(A)=n-|A|$ and $\mu_F(ABC)=n-|ABC|$, in particular with all the cups in $A$
having identical fill, and all the cups in $BC$ having identical fill.
We can think of $F$ as having come from a state where every cup has fill
$\mu_F(ABC) = n-|ABC|$. To reach $F$ from this state where every cup has
identical fill we must increase the fill of each cup in $A$ by some amount, and
decrease the fill of each cup in $BC$ by an amount such that the mass added to
$A$ is taken away from $BC$. To reach fill $\mu_F(A) = n-|A|$, the cups in $A$
must have been increased by $|BC|$ from their previous fill of $n-|ABC|$.
To equalize an increase in $\mu_{F}(A)$ of $|BC|$, we need a corresponding
decrease in $\mu_{F}(BC)$ by $|A|$.
That is, $$\mu_{F}(BC) = n-|ABC|-|A|.$$
Thus we have the following bound:
\begin{align*}
  m_{S_t}(A) + m_{S_t}(C)& \\
&\le m_{F}(A) + c\mu_{F}(BC) \tag{*}\\
&\le a(n-a) + c(n-|ABC|-a) \\
&\le (a+c)(n-a) - c(a+c+b) \\
&\le (a+c)(n-a-c) - cb,
\end{align*}
where (*) follows from Equation \ref{eqn:redistributeA}.
}

{\color{blue}
  Consider a new configuration of fills $F$ achieved by starting with state
  $S_t$, and moving water from $BC$ into $A$ until $\mu_{F}(A) = n-|A|$.
  \footnote{Note that whether or not $F$ satisfies the invariants is irrelevant.}
  This transformation increases (strictly increases if and only if we move a
  non-zero amount of water) the mass in $AC$ because water in $BC$
  counts less towards mass in $AC$ than water in $A$ by Inequality
  \ref{eqn:BCdiscounted}. In particular, if mass $\Delta \ge 0$ fill is moved from
  $BC$ to $A$, then the mass of $AC$ increases by $\frac{b}{b+c} \Delta \ge 0$.

  Since $\mu_F(A)$ is above $\mu_{F}(ABC)$, the greater than average fill of
  $A$ must be counter-balanced by the lower than average fill of $BC$. In
  particular we must have
  $$(\mu_F(A) - \mu_F(ABC))|A| = (\mu_F(ABC) -\mu_F(BC))|BC|.$$
  Note that $$\mu_F(A) -\mu_F(ABC) \ge (n-|A|) - (n-|ABC|) = |BC|.$$
  Hence we must have 
  $$\mu_F(ABC) - \mu_F(BC) \ge |A|.$$
  Thus 
  $$\mu_F(BC) \le \mu_F(ABC) - |A| \le n-|ABC| -|A|.$$

  Thus we have the following bound:
  \begin{align*}
    m_{S_t}(A) + m_{S_t}(C)& \\
  &\le m_{F}(A) + c\mu_{F}(BC)\\
  &\le a(n-a) + c(n-|ABC|-a) \\
  &\le (a+c)(n-a) - c(a+c+b) \\
  &\le (a+c)(n-a-c) - cb.
  \end{align*}
}

Recall that we were considering $b> 0$, and since $b>0$ we have that $c = k-a
\ge b > 0$, i.e. $c \ge 1$. 
Hence we have 
$$m_{S_t}(A) + m_{S_t}(C) \le k(n-k) -b$$
So 
$$m_{S_t}(A) + m_{S_t}(C)+b \le k(n-k).$$
As shown previously the left hand side of the above expression is an upper
bound for $m_{S_{t+1}}([k])$.
Hence the invariant holds.

The proof was for arbitrary $k$, so given that the invariants all hold at state
$S_t$ they also must all hold at state $S_{t+1}$.
Thus, by induction we have the invariant for all rounds $t\in\mathbb{N}$.
\end{proof}


\section{Oblivious Filler Lower Bound}\label{sec:oblivious}
An important theorem that we use throughout our analysis is Hoeffding's Inequality:
\begin{theorem}[Hoeffding's Inequality]
  Let $X_i$ for $i=1,2,\ldots, k$ be independent bounded random variables with
  $X_i \in [a,b]$ for all $i$. Then,
  $$P\left(\Big|\frac{1}{k} \sum_{i=1}^k (X_i - \E[X_i])\Big|\ge t\right) \le
  2\exp\left(-\frac{2kt^2}{(b-a)^2}\right) $$
\end{theorem}
Let $S$ be a finite population, let $X_i$ for $i=1,2\ldots, k$ be chosen
uniformly at random from $S \setminus \{X_1,\ldots, X_{i-1}\}$, and let $Y_i$
for $i=1,2,\ldots, k$ be chosen uniformly at random from $S$.
Note that $\{X_1,\ldots, X_k\}$ represents a sample of $S$ chosen without
replacement, whereas $\{Y_1,\ldots, Y_k\}$ represents a sample with
replacement. Note that as the $Y_i$ are independent random variables
Hoeffding's Inequality provides a bound on the probability of $\sum_{i=1}^k
Y_i$ deviating from its mean by more than $t$.

The same bound can be given on the probability of $\sum_{i=1}^k X_i$ deviating
significantly from its mean, because the probability of $\sum_{i=1}^k X_i$
deviating from it's expectation by more than $t$ is at most the probability of
$\sum_{i=1}^k Y_i$ deviating from it's mean by $t$.
Formally we can write this as 
\begin{corollary}
  \label{cor:hoeffdingwreplacement}
  Let $S$ be a finite set with $\min(S) \ge a, \max(S) \le b$, and let $X_i$ for $i=1,2\ldots, k$ be chosen
uniformly at random from $S \setminus \{X_1,\ldots, X_{i-1}\}$.
Then 
  $$P\left(\Big|\frac{1}{k} \sum_{i=1}^k (X_i - \E[X_i])\Big|\ge t\right) \le
  2\exp\left(-\frac{2kt^2}{(b-a)^2}\right) $$
\end{corollary}

Hoeffding proved Corollary \ref{cor:hoeffdingwreplacement} in his seminal work
\cite{who62} (the result follows from his Theorem 4, combined with Hoeffding's
Inequality for independent random variables).
The intuition behind Corollary \ref{cor:hoeffdingwreplacement} is that samples
drawn without replacement should be more tightly concentrated around the mean
than samples drawn with replacement.

Call an emptying strategy $(T, \Delta)$\defn{-greedy-like} if is satisfies the
following property: for any cup $c$, if $\fil(c) + \Delta > T$, and there are
at least $p$ cups containing fill greater than $\fil(c) + \Delta$, the emptier
does not empty from cup $c$. Combinatorially the quantity $T$ is the threshold
above which the filler ``notices" cups, and $\pm\Delta$ is the tolerance in cup
fills within which the emptier is allowed to not be greedy.

Of particular interest is the greedy emptier, which is $(0,0)$-greedy-like, nad
the smoothed greedy emptier, which is $(1, 0)$-greedy-like.

\begin{proposition}
  \label{prop:obliviousBase}
  There exists an oblivious filling strategy in the variable-processor cup game
  on $n$ cups that achieves backlog $\Omega(\log n)$ against a $(T,
  \Delta)$-greedy-like emptier (where $T, \Delta \le O(1)$ are constants
  known to the filler), with probability at least $1-1/\polylog(n)$.
\end{proposition}
\begin{proof}
  Let $A$, the \defn{anchor} set, be a subset of the cups chosen uniformly at
  random from all subsets of size $n/2$ of the cups, and let $B$, the
  \defn{non-anchor} set, consist of the rest of the cups ($|B| = n/2$). Let $h
  = 2 T + g $ where $g$ is a sufficiently large constant. At each level of our
  recursive procedure we will achieve fill $h$ in some fraction of the cups in
  $A$, and because the emptier is $(T, \Delta)$-greedy-like, we can turn this
  into a known set of cups with fill at least $h' = T + (g+\Delta)/2$. Our
  strategy to achieve backlog $\Omega(\log n)$ overall is roughly as follows:
  \begin{itemize}
    \item \textbf{Step 1:} 
      Obtain large positive tilt in $A$. To do this, we first play a single
      processor cup game on a randomly chosen constant-size subset of $B$.
      Afterwords we do one of the following:
      \begin{itemize}
        \item Take a random cup from $B$ to swap into $A$. This helps gets high
          positive tilt in $A$ if randomly chosen cups have high expected
          positive tilt.
        \item Take the ``winning" cup from the single processor cup game, i.e.
          the cup that has had its fill increase by at least $h$ with constant
          probability, and swap it into $A$. This helps get high positive tilt
          if the subset that we play a single processor cup game on has average
          fill that is not very negative.
      \end{itemize}
  \item \textbf{Step 2:} Reduce the number of processors to a constant fraction
    $c$ of $n$, and raise the fill of $nc$ cups to $h'$. This step relies on
    the emptier being greedy.
  \item \textbf{Step 3:} Recurse on the $nc$ cups that are known to have fill
    at least $h'$.
\end{itemize}
We can perform $\Omega(\log n)$ levels of recursion, achieving constant backlog
at each step (relative to the average fills); doing so yields backlog
$\Omega(\log n)$.

Our strategy fails if fill is extremely concentrated in a very small number of
cups; however, in this case the proposition is trivially satisfied. In
particular, we call a cup \defn{overpowered} if it contains fill at least
$\sqrt{\frac{nh}{\log\log n}}$. If there is ever an overpowered cup, then the
proposition is trivially satisfied, as we have $\Omega(\poly(n))$ backlog. Note
that we don't need to know which cup is overpowered because it will take
$\Omega(\poly(n))$ rounds for the emptier to reduce the fill below $\poly(n)$.
Hence, we can assume without loss of generality that no cup is ever
overpowered. 

Now we detail how to achieve Step 1. For each anchor cup $c$ we will perform a
procedure called a \defn{swapping-process}. A swapping-process is composed of
the substructure of a \defn{round-block}, which is a set of rounds.
At the beginning of a swapping-process we choose an index $j \in [n^2]$. The
swapping-process proceeds for $n^2$ round-blocks; on the $j$-th round-block we
swap a cup into the anchor set.

On each of the $n^2$ round-blocks, the filler selects a random subset $C\subset
B$ of the non-anchor cups and plays a single processor cup game on $C$. In this
single-processor cup game the filler employs the classic adaptive strategy for
achieving backlog $\Omega(\log |B|)$ on a set of $|B|$ cups, however modified
because it is an oblivious filler. In particular, the filler's strategy in the
single-processor cup games is to distribute water equally among an \defn{active
set} of cups, and then after the emptier removes water from some cup the filler
removes a random cup from the active set. There is at least constant
probability that this results in the active set having a single cup at the end,
with fill that has increased by at least $1/|B| + 1/(|B|-1) + \ldots + 1/1 \ge
\ln |B|$ since the start of the round-block.

On most round-blocks (all but the $j$-th), the filler does nothing with the cup
that it achieves in the active set at the end of the single processor cup game.
However, on round-block $j$ with $1/2$ probability the filler swaps the winner
of the single processor cup game into the anchor set, and with $1/2$
probability the filler swaps a random cup from $B$ into the anchor set.

We consider two cases:
\begin{itemize}
  \item \textbf{Case 1:} For at least $1/2$ of the swapping-processes, at
    least $1/2$ of the cups $c \in B$ have $\fil(c) \ge -h$.
  \item \textbf{Case 2:} For at least $1/2$ of the swapping-processes, less
    than $1/2$ of the cups $c \in B$ have $\fil(c) \ge -h$.
\end{itemize}
We now prove that in either case we can achieve high positive tilt in $A$ with
good probability.

\begin{clm} \label{clm:reg} 
  Let $q\ge \Omega(1)$ be an appropriately small constant ($q$ is a function of
  $h\le O(1)$). In Case 1, with probability at least $1-e^{-n(q/2)^2}$, we
  achieve fill at least $h$ in at least $qn/4$ of the cups in $A$ -- i.e. a
  constant fraction of the cups in $A$. In particular, this implies that we
  achieve positive tilt $nhq/4$ in $A$.
\end{clm}
\begin{proof}
  Consider a swapping-process where at least $1/2$ of the cups $c \in B$
  have $\fil(c) \ge -h$.

  Say the emptier \defn{neglects} the anchor set in a round-block if on at
  least one round of the round-block the emptier does not empty from every
  anchor cup. By playing the single-processor cup game for $n^2$ round-blocks,
  with only one round-block when we actually swap a cup into the anchor set, we
  strongly disincentive the emptier from neglecting the anchor set on more
  than a constant fraction of the round-blocks. 

  The emptier must have some binary function, $I(i)$ that indicates whether or
  not they will neglect the anchor set on round-block $i$ if the filler has not
  already swapped. Note that the emptier will know when the filler perform a
  swap, so whether or not the emptier neglects a round-block $i$ depends on
  this information. However, $j$ is the only parameter of the swapping-process,
  so there is no other information that the emptier can use to decide whether
  or not to neglect a round-block, because on any round-block when we simply
  redistribute water amongst the non-anchor cups we effectively have not
  changed anything about the game state. 

  If the emptier is willing to neglect the anchor set for at least $1/2$ of the
  round-blocks, i.e. $\sum_{i=1}^{n^2} I(i) \ge n^2 / 2$, then with probability
  at least $1/4$, $j \in ((3/4) n^2, n^2)$, in which case the emptier neglects
  the anchor set on at least $n^2/4$ round-blocks ($I(k)$ must be $1$ for at
  least $n^2/4$ of the first $(3/4)n^2$ round-blocks). Each time the emptier
  neglects the anchor set the mass of the anchor set increases by at least $1$.
  Thus the average fill of the anchor set will have increased by at least
  $(n^2/2)/(n/2) \ge \Omega(n)$ over the entire swapping-process in this
  case, implying that we achieve the desired backlog. 

  Otherwise, there is at least a $1/2$ chance that the round-block $j$, which
  is chosen uniformly at random from the round-blocks, when the filler performs
  a swap into the anchor set occurs on a round-block with $I(j)=0$, indicating
  that the emptier won't neglect the anchor set on round-block $j$. In this
  case, the round-block was a legitimate single processor cup game on $C_j$,
  the randomly chosen set of $e^{2h}$ cups on the $j$-th round. Then we achieve
  fill increase $\ge 2h$ by the end of the round-block with probability at
  least $1/e^{2h}!$ -- the probability that we correctly guess the sequence of
  cups within the single processor cup game that the emptier empties from. 

  The probability that the random set $C_j \subset B$ contains only cups that
  are among the $n/4$ fullest cups in $B$ is $${n/2 \choose e^{2h}} / {n
  \choose e^{2h}} = O(1).$$ Note that because, by assumption, at least half of
  the cups $c \in B$ have $\fil(c) \ge -h$, then the $n/4$ fullest cups in $B$
  must have fill at least $-h$. If all cups $ c\in C_j$ have $\fil(c) \ge -h$,
  then the fill of the cup in the active set at the end of the round-block is
  at least $-h + 2h = h$, if the filler guesses the emptier's emptying sequence
  correctly.

  Say that a swapping-process where at least half of the cups $c\in B$ have
  $\fil(c) \ge -h$ \defn{succeeds} if $C_j$ is a subset of the $n/4$ fullest cups
  in $B$, and if the filler correctly guesses the emptier's emptying sequence.
  Note that if a swapping-process succeeds, then the filler is able to swap a cup
  with fill at least $h$ into $A$. We have shown that there is a constant
  probability of a given swapping-process succeeding. Let $X_i$ be the binary
  random variable indicating whether or not the $i$-th swapping process where at
  least half of the cups $c\in B$ have $\fil(c) \ge -h$ succeeds. Let $q \ge
  \Omega(1)$ be the probability of a swapping-process succeeding, i.e.
  $P(X_i=1)$. Note that the random variables $X_i$ are clearly independent, and
  identically distributed.

  Clearly $$\E\left[\sum_{i=1}^{n/2} X_i\right] = qn/2.$$ Then by a Chernoff
  Bound (i.e. Hoeffding's Inequality applied to Binary Random Variables),
  $$P\left(\sum_{i=1}^{n/2} X_i\le nq/4\right) \le e^{-n(q/2)^2}.$$ That is, the
  probability that less than $nq/4$ of the anchor cups have fill at least $h$ is
  exponentially small in $n$, as desired.

\end{proof}

\begin{clm}
  \label{clm:xtreme}
  In Case 2, with probability at least $1- 1/\polylog(n)$, we achieve positive
  tilt $hn/8$ in the anchor set.
\end{clm}

\begin{proof}
  Consider a swapping-process where less than $1/2$ of the cups $c\in B$
  have $\fil(c) \ge -h$.

  % RIP this totally doesn't take into account that B might start with neg fill.

  %! oh crap and fill is sinking! make sure it doesn't sink too much!!!
  We assume for simplicity that the average fill of $B$ is $0$. In reality this
  is not the case, but by a Hoeffding bound and the fact that overpowered cups
  don't exist, the fill is really tightly concentrated around $0$, so this is
  almost without loss of generality.

  Let $Y_i$ be the positive tilt of a randomly chosen cup at the start of the
  $i$-th swapping process. {\color{red}Note that the $Y_i$ are not
  really independent, but it is probably ok}. Note that $0\le Y_i \le hn/\lg\lg
  n$.

  We have
  $$\E[\tilt(X)] = \frac{1}{2}\E[|\fil(X)|] \ge h$$
  because negative tilt is at least $nh/2$ and positive tilt must oppose this.
  
  Now we have, by Hoeffding's inequality, that 
  $$P\left(\Big|\frac{1}{n/2} \sum_{i=1}^{n/2} (Y_i - \E[Y_i])\Big|\ge h/2
  \right) \le$$
  $$2\exp\left(-\frac{n(h/2)^2}{(\sqrt{hn/\lg\lg n})^2}\right) $$
  $$P\left(\frac{1}{n/2}\sum_{i=1}^{n/2} Y_i \le h/4\right) \le 1/\polylog(n) $$

\end{proof}
  By Claim \ref{clm:reg} and Claim \ref{clm:xtreme}, in both Case 1 and Case 2,
  we achieve, with probability at least $1-1/\polylog n$,
  positive tilt at least $nc$ in the anchor set for some constant $c$, which is
  a function of $h$. Using the positive tilt, setting the number of processors
  to $1$, the filler can obtain high fill in a set of $nc$ \emph{known} cups.
  In particular, the filler chooses a set of $nc$ cups randomly, which by
  Hoeffding's Inequality will have average fill close to $0$, and then the
  filler distributes a unit of water equally among these cups. The filler
  continues until the average fill of that set of cups has increased by $h'$.
  The filler uses one processor because it doesn't know how many cups the
  positive tilt is concentrated in. Then the filler recurses on the set of $nc$ 
  known cups with average fill.

  Note that this transformation from a set with high positive tilt to a set of
  known cups with high positive average fill is the only part of the proof of
  Proposition \ref{prop:obliviousBase} that is specific to a greedy-like
  emptier. Against a general emptier it is not true that the emptier will
  necessarily focus on the set of cups with high positive tilt; an arbitrary
  emptier can of course foil our attempts to achieve high fill in any fixed set
  of $p$ cups, at a given setting of $p$. Extending Proposition
  \ref{prop:obliviousBase} to apply to non-greedy-like emptiers is an important
  open question.
\end{proof}

\begin{lemma}[The Oblivious Amplification Lemma]
  \label{lem:obliviousAmplification}
  Given an oblivious filling strategy for achieving backlog $f(n)$ in the
  variable-processor cup game on $n$ cups that succeeds with probability at
  least $1/2$, there exists a strategy for achieving backlog 
  $$f'(n) \ge \frac{1}{32}(f(n/2) + f(n/4) + f(n/8) + \cdots) $$ that succeeds
  with constant probability.
\end{lemma}
\begin{proof}
  We essentially perform the same proof as Proposition
  \ref{prop:obliviousBase}, but some new issues arise, which we proceed to
  highlight and address. 

\begin{clm}
  Let a cup be \defn{verysad} if it has fill $< -nh/\lg\lg n$.
  WLOG there are no verysad cups. 
\end{clm}
\begin{proof}
  First note that because WLOG there are no overpowered cups, there fewer than $n/2$ verysad cups.

  Consider 2 cases:
  \begin{itemize}
    \item If the mass of the verysad cups is less than $nh/8$ then we can
      ignore them and accept a $-h/8$ penalty to the average fill.
    \item On the other hand, if the mass of the verysad cups is greater than
      $nh/8$, then by the end the average fill of everything else is already
      $h/8$ which is also basically as desired.
  \end{itemize}
\end{proof}

\begin{clm}
  WLOG $A,B$ have average fill $\ge -h/8$.
  In particular, we can construct a subset of $n/2$
  cups with average fill $\ge -h/8$ with high probability in $n$. 
\end{clm}
\begin{proof}

  Recall the definition of an overpowered cup as a cup with fill $\ge nh / \lg \lg n$,
  and the fact that WLOG there are no overpowered cups.
  So, If we randomly pick $B$ then this means that we are pretty good. 
  Formalizing this, let $X_i$ be the fill of the $n/2$-th randomly chosen cup
  for $B$. Unfortunately these are not quite independent events.

  Lets say we pick $2n$ things from $n$ things with replacement. Claim: with
  exponentially good probability we have $n/2$ distinct things. 
  Proof: Chernoff bound. Let $X_i$ be indicator variable for cup $i$ (whether
  it was chosen or not). Probability that $X_i$ was chosen: $1-((n-1)/n)^n
  \approx 1-1/e > 1/2$ for large $n$. 
  Then by a Chernoff Bound we have that $\sum_i X_i$ is tightly concentrated
  around its mean, which is larger than $n$. In particular, with probability
  exponentially close to $1$ in $n$ we have that at least $n/2$ cups were chosen.

    Initial solution: no overpowered cups WLOG, so if we pick them randomly star holds
    by Hoeffding's. (kinda, bc stuff isnt really independent, can probably swap
    with replacement to fix this tho)
  
\end{proof}
\begin{clm}
  What if $C$ needs to be big because we need big backlog? 
\end{clm}
\begin{proof}
 this isn't a problem because the base case is the only case that needs to
 explicitly deal with positive and negative fill.
\end{proof}
These concerns resolved, the exact same argument as in Proposition
\ref{prop:obliviousBase} gives the desired result.

\end{proof}

\begin{corollary}
  \label{cor:obliviousPoly}
  There is an oblivious filling strategy for the variable-processor cup game on
  $n$ cups that achieves backlog $2^{\Omega(\sqrt{\log n})}$ in running time
  $O(n)$
\end{corollary}
\begin{proof}
  We must reduce want to reduce $\log^2 n$ to $\log n$ to achieve the
  appropriate running-time, so we reduce $n$ to $n' = 2^{\sqrt{\log n}}$. This
  detail taken care of we apply exactly the same recursive construction of
  $f_{\theta(\log n)}$ as in Corollary \ref{cor:adaptivePoly}, but using
  repeated application of the Oblivious Amplification Lemma rather than the
  Adaptive Amplification Lemma, which yields the disclaimer that the backlog is
  only achieved with constant probability.
  So we achieve backlog $\Omega(2^{\log n'})$ in running time $O(2^{\log^2
  n'})$. By design, expressing this in terms of $n$ we have running time $O(n)$
  (randomized lower bounds are not supposed to take longer than $\poly(n)$
  time), and as a consequence we get backlog $\Omega(2^{\sqrt{\log n}})$.
\end{proof}

\section{Conclusion}
Many important open questions remain open.

\bibliographystyle{plain}
\bibliography{paper}
\end{document}
