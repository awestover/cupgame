We now give a method for transforming a filling strategy for achieving
large backlog into a filling strategy for achieving high fill in
many cups, or high average fill in a set of cups (which of these
we guarantee depends on the original filling strategy). The idea
of repeating an algorithm many times is also used in the proof of
the Adaptive Amplification Lemma; the construction is slightly more
complicated in the randomized case however, and is much harder to
analyze.

\begin{definition}
  \label{def:rep}
  {\normalfont
  Let $\alg_0$ be an oblivious filling strategy, that can get
  high fill in some cup against greedy-like emptiers with some
  probability. We construct a new filling strategy
  $\rep_\delta(\alg_0)$, which we call the
  $\delta$-\defn{repetition} of $\alg_0$, as follows:

  Let $n_A = \ceil{\delta n}, n_B = \floor{(1-\delta)n}$. Let $M
  \gg n$ be large, let $m=\poly(M)$ be a chosen parameter. 
  Initialize $A$ to $\varnothing$ and $B$ to
  being all of the cups. We call $A$ the \defn{anchor set} and
  $B$ the \defn{non-anchor set}. The filler always places $1$
  unit of fill in each anchor cup on each round. The filling
  strategy consists of $n_A$ \defn{donation-processes}, which are
  procedures that result in a cup being \defn{donated} from $B$
  to $A$ (i.e. removed from $B$ and added to $A$). At the start
  of each donation-processes the filler chooses a value $m_0$
  uniformly at random from $[m]$. We say that the filler
  \defn{applies} a filling strategy $\alg$ to $B$ if the
  filler uses $\alg$ on $B$ while placing $1$ unit of fill
  in each anchor cup. During the donation-process the filler
  applies $\alg_0$ to $B$ $m_0$ times, and flattens $B$ by
  applying $\flatalg$ to $B$ for $\Theta(M)$ rounds before each
  application of $\alg_0$. At the end of each donation process
  the filler takes the cup given by the final application of
  $\alg_0$ (i.e. the cup that $\alg_0$ guarantees with some
  probability against a certain class of emptiers to have a
  certain high fill), and donates this cup to $A$. 

 A pseudocode description of this algorithm can be found in
 \cref{alg:rep}.
 }

\begin{algorithm}
  \caption{rep}
  \label{alg:rep}
  \begin{algorithmic}
    \State \textbf{Input:} $\alg_0, \delta, M, m, $ set of $n$ cups
    \State \textbf{Output:} Guarantees on the sets $A, B$ (will vary based on $\alg_0$)
    \State
    \State $n_A \gets \ceil{\delta n}, n_B \gets \floor{(1-\delta) n}$
    \State $A \gets \varnothing, B \gets$ all the cups
    \For{$i \in [n_A]$} \Comment Donation-processes
    \State $m_0 \gets \text{random}([m])$
      \For{$j \in [m_0]$}
        \State Apply $\flatalg$ to $B$ for $\Theta(M)$ rounds
        \State Apply $\alg_0$ to $B$
      \EndFor
      \State Donate the cup given by $\alg_0$ from $B$ to $A$
    \EndFor
  \end{algorithmic}
\end{algorithm}
\end{definition}

We use the idea of repeating an algorithm in two different contexts.
First in \cref{prop:obliviousBase} we prove a result analogous to that of
\cref{prop:adaptiveBase}: in particular, we show that we can
achieve constant fill in a known cup by using
$\rep_{\Theta(1)}(\randalg(\Theta(1))$
which achieves, by a Chernoff bound, $\Theta(n)$ unknown cups
with constant fill, and then exploiting the emptier's greedy-like
nature to achieve constant fill in a known cup.
After doing this, we prove the \defn{Oblivious Amplification
Lemma}, a result analogous to the Adaptive
Amplification Lemma: in particular, we show how to take an
algorithm for achieving some backlog, and then achieve higher
backlog by repeating the algorithm many times.
Although these results have deterministic analogues, their proofs
are different and significantly more complex than the proofs for the
deterministic cases.

Before proving \cref{prop:obliviousBase} we analyze
$\rep_{\Theta(1)}(\randalg(\Theta(1)))$ in \cref{lem:obliviousManyUnknownCups}.
\begin{lemma}
  \label{lem:obliviousManyUnknownCups}
  Let $\Delta \le O(1)$, let $h \le O(1)$ with $h \ge
  16+16\Delta$, let $k = \ceil{e^{2h+1}}$, let $\delta =
  \Theta(e^{-2h})$, let $n$ be at least a sufficiently large
  constant determined by $h$ and $\Delta$. Let $M\gg n$ be very
  large. Consider an $R_\Delta$-flat cup configuration in the
  variable-processor cup game on $n$ cups with initial average
  fill $\mu_0$.

  When applied to a $\Delta$-greedy-like emptier
  $\rep_{\delta}(\randalg(k))$ either achieves mass at least $M$ in
  the cups, or with probability at least $1-2^{-\Omega(n)}$ makes an
  (unknown) set of $\Theta(n)$ cups in $A$ have fill at least $h + \mu_0$
  while also guaranteeing that $\mu(B) \ge -h/2 + \mu_0$,
  where $A,B$ are the sets defined in \cref{def:rep}.
  This strategy has running time $\poly(M)$.
\end{lemma}
\begin{proof}
  We use all definitions given in the \cref{def:rep}.

Without loss of generality we assume that the emptier does not
neglect the anchor set more than $M$ in a particular donation
process; if the emptier chooses to neglect the anchor set this
much then the anchor cups will have achieved mass $M$ so
\cref{lem:obliviousManyUnknownCups} is already fulfilled. 
Similarly we assume that the emptier does not choose to skip
more than $M$ emptyings.

We say that the emptier \defn{neglects} the anchor set on a round
if it does not empty from each anchor cup. We say that an
application of $\randalg(k)$ to $B$ is \defn{non-emptier-wasted} if
the emptier does not neglect the anchor set during any round of
the application of $\randalg(k)$. We define $d = \sum_{i=2}^{k} 1/i =
\Theta(h)$. We say that an application of $\randalg(k)$ to $D$ is
\defn{lucky} if it achieves backlog at least $\mu(B) - R_\Delta +
d$ where $\mu(B)$ is measured at the start of the application of
$\randalg(k)$; note that by \cref{prop:obliviousTerribleProbability}
if we condition on an application of $\randalg(k)$ where $B$ started
$R_\Delta$-flat being non-emptier-wasted then the application has
at least a $1/k!$ chance of being lucky.

Now we prove several important bounds on fills of cups in $A$ and $B$.
\begin{clm}
  \label{clm:allflatteningsworkbyM}
  All applications of $\flatalg$ make $B$ be $R_\Delta$-flat and
  $B$ is always $(R_\Delta + d)$-flat.
\end{clm}
\begin{proof}
  Given that the application of $\flatalg$ immediately prior to an application
  of $\randalg(k)$ made $B$ be $R_\Delta$-flat, by
  \cref{prop:obliviousTerribleProbability} we have that $B$ will
  stay $(R_\Delta + d)$-flat during the application of $\randalg(k)$. 
  Given that the application of $\randalg(k)$ immediately prior to an
  application of $\flatalg$ resulted in $B$ being $(R_\Delta
  + d)$-flat, we have that $B$ remains $(R_\Delta + d)$-flat
  throughout the duration of the application of $\flatalg$ by
  \cref{lem:greedylikeisflat}. Given that $B$ is $(R_\Delta +
  d)$-flat before a donation occurs $B$ is clearly still $(R_\Delta +
  d)$-flat after the donation, because the only change to $B$ during
  a donation is that a cup is removed from $B$ which cannot increase
  the fill-range of $B$.
  Note that $B$ started $R_\Delta$-flat at the beginning of the
  first donation-process.
  Note that if an application of $\flatalg$ begins with $B$ being
  $(R_\Delta + d)$-flat, then by considering the flattening to
  happen in the $(|B|/2)$-processor $M$-extra-emptyings
  $M$-skip-emptyings cup game we ensure that it makes $B$ be
  $R_\Delta$-flat.
  Hence we have by induction that $B$ has always been $(R_\Delta
  + d)$-flat and that all flattening processes have made $B$ be
  $R_\Delta$-flat. 
\end{proof}

Now we aim to show that $\mu(B)$ is never very low, which we need
in order to establish that every non-emptier-wasted lucky
application of $\randalg(k)$ gets a cup with high fill. Interestingly
in order to lower bound $\mu(B)$ we first must upper bound
$\mu(B)$, which by greediness and flatness of $B$ gives an upper
bound on $\mu(A)$ which we use to get a lower bound on $\mu(B)$.

\begin{clm}
  \label{clm:muBdoesntgettoobig}
  We have always had
  $$\mu(B) \le 2 + \mu(A B).$$
\end{clm}
\begin{proof}
  There are two ways that $\mu(B)-\mu(A B)$ can increase: \\
  \textbf{Case 1:}
  The emptier could empty from $0$ cups in $B$ while emptying
  from every cup in $A$. \\
  \textbf{Case 2:}
  The filler could evict a cup with fill lower than $\mu(B)$ from
  $B$ at the end of a donation-process. \\

  Note that cases are exhaustive, in particular note that if the
  emptier skips more than $1$ emptying then $\mu(B) - \mu(AB)$
  must decrease because $|B|\approx |AB|$, in particular
  by our choice of $\delta = \Theta(e^{-2h})$, as opposed to in Case 1
  where $\mu(B) - \mu(AB)$ increases.

  In Case 1, because the emptier is $\Delta$-greedy-like,
  $$\min_{a\in A} \fil(a) > \max_{b\in B} \fil(b) - \Delta.$$
  Thus $\mu(B) \le \mu(A) + \Delta$. As $|B| \gg |A|$, in
  particular by our choice of $\delta = \Theta(e^{-2h})$, this can be
  loosened to $\mu(B) \le 1 + \mu(A B)$.

  Consider the final round on which $B$ is skipped while $A$ is
  not skipped (or consider the first round if there is no such
  round).

  From this round onwards the only increase to $\mu(B) - \mu(A
  B)$ is due to $B$ evicting cups with fill well below $\mu(B)$.
  We can upper bound the increase of $\mu(B) - \mu(A B)$ by the
  increase of $\mu(B)$ as $\mu(A B)$ is strictly increasing.

  The cup that $B$ evicts at the end of a
  donation-process has fill at least $\mu(B) - R_\Delta -
  (k-1)$, as the running time of $\randalg(k)$ is $k-1$, and
  because $B$ starts $R_\Delta$-flat by
  \cref{clm:allflatteningsworkbyM}. Evicting a cup
  with fill $\mu(B) - R_\Delta - (k -1)$ from $B$ changes
  $\mu(B)$ by $(R_\Delta + k - 1) / (|B|-1)$ where $|B|$ is the
  size of $B$ before the cup is evicted from $B$. Even if this
  happens on each of the $n_A$ donation-processes $\mu(B)$ cannot
  rise higher than $n_A (R_\Delta + k-1) / (n-n_A)$ which by
  design in choosing $n_B\gg n_A$, as was done in
  choosing $\delta = \Theta(e^{-2h})$, is at most $1$.

  Thus $\mu(B) \le 2 + \mu(A B)$ is always true.

\end{proof}

The upper bound on $\mu(B)$ along with the guarantee that $B$ is
flat allows us to bound the highest that a cup in $A$ could rise
by greediness, which in turn upper bounds $\mu(A)$ which in turn
lower bounds $\mu(B)$. In particular we have
\begin{clm}
  \label{clm:muBgreaterthanminushover2}
  We always have
  $$\mu(B) \ge -h/2 + \mu_0.$$
\end{clm}
\begin{proof}
  By \cref{clm:muBdoesntgettoobig} and \cref{clm:allflatteningsworkbyM} 
  we have that no cup in $B$ ever has fill greater than
  $u_B = \mu(A B) + 2 + R_\Delta + d$. 
  Let $u_A = u_B + \Delta + 1$. We claim that the backlog in $A$
  never exceeds $u_A$. Note that $\mu(AB), u_A, u_B$ are
  implicitly functions of the round; $\mu(AB)$ can increase from
  $\mu_0$ if the emptier skips emptyings.

  Consider how high the fill of a cup $c \in A$ could be.
  If $c$ came from $B$ then when it is donated
  to $A$ its fill is at most $u_B < u_A$. Otherwise, $c$
  started with fill at most $R_\Delta < u_A$. Now consider how
  much the fill of $c$ could increase while being in $A$. Because
  the emptier is $\Delta$-greedy-like, if a cup $c\in A$ has fill
  more than $\Delta$ higher than the backlog in $B$ then $c$ must
  be emptied from, so any cup with fill at least $u_B + \Delta =
  u_A - 1$ must be emptied from, and hence $u_A$ upper bounds the
  backlog in $A$. 

  Of course an upper bound on backlog in $A$ also serves as
  an upper bound on the average fill of $A$ as well, i.e.
  $\mu(A) \le u_A$. 
  Rearranging the expression 
  $$|B|\mu(B) + |A|\mu(A) = |AB|\mu(AB)$$
  we have
  \begin{align*}
    &\mu(B) \\
           &= -\frac{|A|}{|B|} \mu(A) + \frac{|A B|}{|B|}\mu(A B) \\
           &\ge -(\mu(AB) + 3+R_\Delta+d+\Delta) \frac{|A|}{|B|} + \frac{|AB|}{|B|}\mu(AB)\\
           &= -(3+R_\Delta+d + \Delta) \frac{|A|}{|B|} + \mu(AB)\\
           &\ge -h/2 + \mu(AB)
  \end{align*}
  where the final inequality follows because $\mu(AB) \ge 0$, and
  $|B|\gg |A|$, in particular by our choice of $\delta = \Theta(e^{-2h})$.
  Of course $\mu(AB) \ge \mu_0$ so we have
  $$\mu(B) \ge -h/2 + \mu_0.$$

\end{proof}

Now we show that at least a constant fraction of the
donation-processes succeed with exponentially good probability.
\begin{clm}
  \label{clm:baseChernoffBound}
  There exists choice of $m =\Theta(M)$ such that with
  probability at least $1-2^{-\Omega(n)}$, the filler achieves
  fill at least $h+\mu_0$ in $\Theta(n)$ of the cups in $A$. 
\end{clm}
\begin{proof}
  If the emptier was not allowed to neglect the anchor set ever
  then the claim would clearly be true as each application of
  $\randalg(k)$ would unconditionally succeed with constant
  probability, so a Chernoff bound would give that $\Theta(n)$ of
  the donation-processes donate a cup with fill at least $\mu(B)
  - R_\Delta + d \ge h + \mu_0$, where the inequality follows
  from \cref{clm:muBgreaterthanminushover2} which asserts that
  $\mu(B) \ge -h/2 + \mu_0$, and from the facts $d\ge 2h$ and $h
  \ge 16(1+\Delta)$. However, the emptier is allowed to neglect
  the anchor set, and worse, the emptier can choose to neglect
  the anchor set conditional on the filler's progress during
  $\randalg(k)$. However, by applying $\randalg(k)$ a random
  number of times, chosen from $[m]$ (where $m=\Theta(M)$ which
  is quite large), we guarantee that with exponentially good
  probability the filler succeeds many times, in particular
  $\Theta(M)$ times. But since the emptier cannot neglect the
  anchor set more than $M$ times, by appropriately large choice
  of $m$ we can make it so that the filler succeeds at least $2M$
  times with exponentially good probability. Then the emptier
  would have at best a $1/2$ chance of preventing the
  donation-process from giving away a cup with fill $h+\mu_0$
  whenever one such cup is achieved. We now formalize this
  reasoning.

  We can lower bound the probability of getting $\Theta(n)$ cups
  with fills all at least $h + \mu_0$ by considering an augmented emptier
  that is allowed to interfere with $M$ applications of $\randalg(k)$
  per donation-process that only interferes with applications of
  $\randalg(k)$ that would otherwise donate a cup with fill at
  least $h + \mu_0$ into $A$. 
  The optimal strategy for such an emptier, given our filler's
  strategy of randomly choosing how many times to apply
  $\randalg(k)$ before donating a cup, is to simply interfere with the first $M$ applications of
  $\randalg(k)$ that would have achieved a cup
  with fill $h+\mu_0$ without interference. 
  Let $m = 4M k! = \Theta(M)$. Recall that conditional on
  the emptier not interfering, each of these applications of
  $\randalg(k)$ has at least a $1/k!$ chance of getting a cup with
  fill $h$. Hence, by a Chernoff bound with exponentially good
  probability at least $2M$ of $m$ applications of $\randalg(k)$ have
  the potential to donate a cup with fill $h+\mu_0$ to $A$, if the
  emptier does not interfere. The filler chooses an application
  uniformly at random from $[m]$ on which to donate a cup.
  With probability at least $1/k!$ this is on an
  application where the filler could get a cup with fill
  $h+\mu_0$ in $A$ if the emptier does not interfere, and with probability at
  least $1/2$ the emptier does not interfere on this application
  of $\randalg(k)$, because the emptier can interfere on at most $M$
  of the applications of $\randalg(k)$. 

  Against this augmented emptier whether or not 
  donation-processes achieve a cup with fill $h+\mu_0$ in $A$ are
  independent events. As each happens with at least constant
  probability, by a Chernoff bound there is exponentially high
  probability that at least a constant fraction of them succeed.

  Note that we used the Chernoff bound $\Theta(n)$; by a union
  bound there is exponentially good probability that all of the
  desired events occur.

\end{proof}

We now analyze the running time of the filling strategy.
There are $n_A$ donation-processes. Each donation-process
consists of $\Theta(M)$ applications of $\randalg(k)$, which each take
constant time, and $\Theta(M)$
applications of $\flatalg$, which each take $\poly(M)$ time.
Thus overall the algorithm takes $\poly(M)$ time, as desired.
  
\end{proof}

Now, using \cref{lem:obliviousManyUnknownCups} we show in
\cref{prop:obliviousBase} that an oblivious filler can achieve
constant backlog. 
\begin{proposition}
  \label{prop:obliviousBase}
  Let $H \le O(1)$, let $\Delta \le O(1)$, let $n$ be at
  least a sufficiently large constant determined by $H$ and
  $\Delta$. 
  Let $M \gg n$ be very large.
  Consider an $R_\Delta$-flat cup configuration in the variable-processor cup
  game on $n$ cups with average fill $\mu_0$.
  There is an oblivious filling strategy that either
  achieves mass $M$ among the cups, or achieves fill at least $\mu_0 + H$
  in a chosen cup in running time $\poly(M)$ against a
  $\Delta$-greedy-like emptier with probability at least $1-2^{-\Omega(n)}.$
\end{proposition}
\begin{proof}
  The filler starts by using $\rep_\delta(\randalg(k))$ with
  parameter settings as in \cref{lem:obliviousManyUnknownCups}
  where $h = H\cdot 16(1+\Delta)$, i.e. $k = \ceil{e^{2h+1}}$,
  $\delta = \Theta(e^{-2h})$. 
  Let the number of cups which must now exist by
  \cref{lem:obliviousManyUnknownCups} with
  fill at least $h+\mu_0$ be of size $nc = \Theta(n)$.

  The filler sets $p=1$, i.e. uses a single processor. Now the
  filler exploits the emptier's greedy-like nature to to get fill
  $H$ in a chosen cup $c_0$. Specifically, for $(5/8)h$ rounds
  the filler places $1$ unit of fill into $c_0$. Because the
  emptier is $\Delta$-greedy-like it must empty from the $nc$
  cups in $A$ with fill at least $h+\mu_0$ until $c_0$ has large
  fill. Over $(5/8)h$ rounds the cups in $A$ cannot have their
  fill decrease below $(3/8)h \ge h/8 + \Delta + \mu_0$. Hence,
  any cups with fills less than $h/8+\mu_0$ must not be emptied
  from during these rounds. The fill of $c_0$ started as at least
  $-h/2+\mu_0$ as $\mu(B) \ge -h/2+\mu_0$. After $(5/8)h$ rounds
  $c_0$ has fill at least $h/8+\mu_0$, because the emptier cannot
  have emptied $c_0$ until it attained fill $h/8+\mu_0$, and if
  $c_0$ is never emptied from then it achieves fill $h/8+\mu_0$.
  Thus the filling strategy achieves backlog $h/8 +\mu_0 \ge H +
  \mu_0$ in $c_0$, a known cup, as desired.

\end{proof}

Next we prove the \defn{Oblivious Amplification Lemma}.

\begin{lemma}[Oblivious Amplification Lemma]
  \label{lem:obliviousAmplification} 
  Let $\delta \in (0, 1/2)$ be a parameter. Let $\Delta \le
  O(1)$. Let $M$ be very large. Consider a cup configuration in
  the variable-processor cup game on $n\ll M$ cups with average fill
  $\mu_0$ that is $R_\Delta$-flat. Let $\alg(f)$ be an oblivious
  filling strategy that either achieves mass $M$ or achieves
  backlog $\mu_0 + f(n)$ on such cups with probability at least
  $1-2^{-\Omega(n)}$ in running time $T(n)$ against a
  $\Delta$-greedy-like emptier.

  Consider a cup configuration in the variable-processor cup game
  on $n\ll M$ cups with average fill $\mu_0$ that is $R_\Delta$-flat.
  There exists an oblivious filling strategy $\alg(f')$ that
  either achieves mass $M$ or achieves backlog $f'(n)$ satisfying
  $$f'(n) \ge (1-\delta) f(\floor{(1-\delta)n}) + f(\ceil{\delta n}) + \mu_0$$
  and $f'(n) \ge f(n)$, with probability at least
  $1-2^{-\Omega(n)}-1/\poly(M)$ in running time 
  $$T'(n) \le M\cdot n\cdot T(\floor{(1-\delta)n}) + T(\ceil{\delta n})$$ 
  against a $\Delta$-greedy-like emptier.
\end{lemma}

\begin{proof}
  We use the notation from \cref{lem:obliviousManyUnknownCups},
  and from \cref{def:rep}. To summarize, we define $n_A =
  \ceil{\delta n}, n_B = \floor{(1-\delta)n}$, we refer to $A$ as
  the anchor set and $B$ as the non-anchor set, we say that the
  filler applies $\alg(f)$ to $B$ if it uses $\alg(f)$ on $B$
  while placing $1$ fill into each cup in $A$, and we say that
  $A$ is neglected during an application of $\alg(f)$ to $B$ if
  there is some round during the application where the emptier
  does not empty from all anchor cups.

  The filler defaults to using $\alg(f)$ on all the cups if 
  $$f(n) \ge (1-\delta) f(n_B) + f(n_A).$$
  In this case our strategy trivially has the desired guarantees. 
  In the rest of the proof we consider the case where we cannot
  simply fall back on $\alg(f)$ to achieve the desired backlog.

  The filler's strategy is roughly as follows:\\
  \textbf{Step 1:} Use $\rep_\delta(\alg(f))$ on all the cups;
  this will get $A$ to have high average fill.\\
  \textbf{Step 2:} Flatten $A$ using $\flatalg$, and then use
  $\alg(f)$ on $A$.

  Now we analyze Step 1, and show that by appropriately choosing
  parameters it can be made to succeed.

  Note that, exactly as in the proof of
  \cref{lem:obliviousManyUnknownCups}, the emptier cannot neglect
  the anchor set more than $M$ times per donation-process, and
  the emptier cannot skip more than $M$ emptyings, without
  causing the mass of the cups to be at least $M$; we assume the
  emptier chooses not to do this.

  We choose $m\ge M^3, m = \poly(M)$ ---recall that $[m]$ is
  the set from which we choose how many times to apply $\alg(f)$
  in a donation-process. Using the ideas from the analysis in
  \cref{clm:baseChernoffBound}, we see that by a Chernoff bound
  with exponentially good probability in $n$ many more than $m/2$
  of the applications would succeed without if the emptier did
  not interfere with them. Conditional on the final application
  being a round that would be successful without the emptier
  interfering, the emptier has at best a $M/(m/2)= 1/\poly(M)$
  chance of interfering on the correct round.
  Taking a union bound, we can say that with probability at least
  $1-1/\poly(M) - 2^{-\Omega(n)}$ all applications of $\alg(f)$
  are not non-emptier-wasted and successfully achieve a cup with
  fill at least $\mu_{t_0}(B) + f(n_B)$ where $\mu_{t_0}(B)$ refers to the
  average fill of $B$ measured at the start of the application of
  $\alg(f)$.

  Let \defn{$\skips_t$} denote the number of times that the
  emptier has skipped the anchor set by step $t$. Consider how
  $\mu(B) - \skips/n_B$ changes over the course of the donation
  processes. As noted above at the end of each donation-process
  $\mu(B)$ decreases due to $B$ donating a cup with fill at least
  $\mu(B) + f(n_B)$. In particular, if $t_0$ denotes the time
  right before a cup is donated on the $i$-th donation-process
  and $t_1$ denotes the time right after a cup is donated, then
  $\mu_{t_1}(B) = \mu_{t_0}(B) - f(n_B) / (n-i)$. Now we claim that
  $\mu(B) - \skips/n_B$ is monotonically decreasing. Clearly each
  donation decreases this quantity. If the anchor set is
  neglected then $\mu(B)$ decreases. If a skip occurs, then
  $\skips/n_B$ increases by more than $\mu(B)$ can possibly
  decrease. Let $t_*$ be the time at the end of all the
  donation-processes. We have that 
  \begin{equation}
    \label{eq:harmonic1}
  \mu_{t_*}(B) - \skips_{t_*}/n_B \le \mu_0 -
  \sum_{i=1}^{n_A}\frac{f(n_B)}{n-i}.
  \end{equation}
  By conservation of mass we have 
  \begin{equation}
    \label{eq:conservationOfMass}
    n_A\mu_{t_*}(A) + n_B \mu_{t_*}(B) = \mu_0 + \skips.
  \end{equation}
  We can use Inequalities \eqref{eq:conservationOfMass} and
  \eqref{eq:harmonic1} to get a lower bound on $\mu_{t_*}(A)$ as
  follows:
  \begin{equation}
    \label{eq:lowerboundingAharmonic}
    \mu_{t_*}(A) = \mu_0 + \frac{n_B}{n_A}\paren{\mu_0 +
    \frac{\skips_{t_*}}{n_B} - \mu_{t_*}(B)}.
  \end{equation}
  Now we obtain a simpler form of
  Inequality~\eqref{eq:harmonic1}. Let $H_n$ denote the $n$-th
  harmonic
  number. We desire a simpler lower bound for 
  $$\sum_{i=1}^{n_A} \frac{1}{n-i} = H_{n-1}-H_{n_B-1}.$$

  We use the well known fact that 
  \begin{equation}
    \label{eq:wellKnownLogIneq}
    \frac{1}{2(n+1} < H_n - \ln n - \gamma < \frac{1}{2n}
  \end{equation}
  where $\gamma = \Theta(1)$ denotes the Euler-Macaroni constant.
  Of course $H_{n-1}-H_{n_B-1} \ge H_n - H_{n_B}.$ Now using
  Inequality~\eqref{eq:wellKnownLogIneq} we have
  \begin{align*}
    H_n - H_{n_B} &> \paren{\ln n + \gamma + \frac{1}{2(n+1)}} - \paren{\ln n_B + \gamma + \frac{1}{2n_B}}
                  &> \ln \frac{1}{1-\delta} + \frac{1}{2}\paren{\frac{n_B-n-1}{(n+1)n_B}}
                  &\approx \delta - \frac{\delta}{2n(1-\delta)}.
  \end{align*}
  Now using this lower bound on $H_n - H_{n_B}$ in
  Inequality~\eqref{eq:lowerboundingAharmonic} we have:
  \begin{align*}
    \mu_{t_*}(A) &> \mu_0 + \frac{n_B}{n_A}\paren{stuff}.
                 &> \mu_0 + (1-\delta)^2 f(n_B)
  \end{align*}
  where the final inequality follows from choosing $n$
  sufficiently large.


  \todo{\textbf{take into account skip-emptyings and
  extra-emptyings}
  For a moment we ignore the fact that the emptier can skip
  emptyings and neglect the anchor set, or equivalently we
  consider the case where $\mu(AB)=\mu_0$ always holds. Then, if
  $\mu(B)$ is always at least $\mu_0 - \delta f(n_B)$ we have
  that each of these swapped cups has fill at least $\mu_0 +
  (1-\delta)f(n_B)$, as desired; on the other hand, if $\mu(B)$
  ever sinks below $\mu_0 -\delta f(n_B)$ then, because $\mu(B)$
  is monotonically decreasing, in the end we will have $$\mu(A)
  \ge \mu_0 + \frac{n_B}{n_A} \delta f(n_B) = \mu_0 +
  (1-\delta)f(n_B).$$ Now we extend the argument to deal with
  skip-emptyings and extra-emptyings.


  intuitively, extra-emptyings can only happen when $\mu(B)
  \approx \mu(A)$ and skip-emptyings just help us out.

  $\rep'$ is like rep, but you also flatten $A$ at the start of
  each donation-process

  if you're extra-emptying then 

}
  \todo{fix thm}

We have shown that in Step 1 the filler achieves average fill
$\mu_0 + (1-\delta)f(n_B)$ in $A$ (with good probability).
Now the filler flattens $A$ and uses $\alg(f)$ on $A$.
It is clear that this is possible, and succeeds with probability
at least $1-2^{-\Omega(n)}$.
This gets a cup with fill 
$$\mu_0 + (1-\delta)f(n_B) + f(n_A)$$
in $A$, as desired.

Taking a union bound over the probabilities of Step 1 and Step 2
succeeding gives the desired probability. 

The running time of Step 1 is clearly $M\cdot n\cdot
T(\floor{(1-\delta)n})$ and the running time of Step 2 is clearly
$T(\ceil{\delta n})$ summing these yields the desired upper
bound on running time.

\end{proof}

