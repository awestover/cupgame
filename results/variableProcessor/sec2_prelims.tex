\section{Preliminaries}\label{sec:prelims}
The cup game consists of a sequence of rounds. On the $t$-th round, the state
starts as $S_t$. The filler chooses the number of processors $p_t$ for the
round. Then the filler distributes $p_t$ units of water among the cups (with at
most $1$ unit of water to any particular cup). After this, the game is in an
intermediate state, which we call state $I_t$. Then the emptier chooses $p_t$
cups to empty at most $1$ unit of water from. Note that if the fill of a cup
that the emptier empties from is less than $1$ the emptier reduces the fill of
this cup to $0$ by emptying from it; we say that the emptier \defn{zeroes out}
a cup at round $t$ if the emptier empties, on round $t$, from a cup with fill
at state $I_t$ that is less than $1$. Note that on any round where the emptier
zeroes out a cup the emptier has removed less fill than the filler has added;
hence the average fill will increase. This concludes the round; the state of
the game is now $S_{t+1}$.

Denote the fill of a cup $c$ by $\fil(c)$. 
Let the \defn{mass} of a set of cups $X$ be $m(X) = \sum_{c\in X} \fil(c)$. 
Denote the average fill of a set of cups $X$ by $\mu(X)$. Note that $\mu(X) |X| = m(X)$.

Let the \defn{rank} of a cup at a given state be its position in a list of the
cups sorted by fill at the given state, breaking ties arbitrarily but
consistently. For example, the fullest cup at a state has rank $1$, and the
least full cup has rank $n$. Let $[n] = \{1,2,\ldots, n\}$, let
$i+[n] = \{i+1, i+2, \ldots, i+n\}$.

Many of our lower bound proofs will adopt the convention of
allowing for negative fill. We call this the \defn{negative-fill
cup game}. Specifically, in the negative-fill cup game, when the
emptier empties from a cup its fill always decreases by exactly
$1$: there is no zeroing out. Negative-fill can be interpreted as
fill below some average fill. Measuring fill like this is
important however, as our lower bound results are used
recursively, building on the average fill already achieved. Note
that it is strictly easier for the filler to achieve high backlog
when cups can zero out, because then some of the emptier's
resources are wasted. On the other hand, during the upper bound
proof we show that a greedy emptier maintains the desired
invariants even if cups zero out. This is crucial as the game is
harder for the emptier when cups can zero out.
Some results are proved for the variable-processor negative-fill
cup game, and some results are proved for the single-processor
negative-fill cup game.

