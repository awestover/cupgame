\section{Oblivious Filler Lower Bound}\label{sec:oblivious}
In this section we prove that, surprisingly, an oblivious filler
can achieve backlog $n^{1-\varepsilon}$, although only against a
certain class of ``greedy-like" emptiers.

The \defn{fill-range} of a set of cups at a state $S$ is $\max_c
\fil_S(c) - \min_c \fil_S(c)$. We call a cup configuration
\defn{$R$-flat} if the fill-range of the cups less than or equal to
$R$; note that in an $R$-flat cup configuration with average fill
$0$ all cups have fills in $[-R, R]$. We say an emptier is
\defn{$\Delta$-greedy-like} if, whenever there are two cups with
fills that differ by at least $\Delta$, the emptier never empties
from the less full cup without also emptying from the more full
cup. That is, if on some round $t$, there are cups $c_1, c_2$
with $\fil_{I_t}(c_1) > \fil_{I_t}(c_2) + \Delta$, then a
$\Delta$-greedy-like emptier doesn't empty from $c_2$ on round
$t$ unless it also empties from $c_1$ on round $t$. Note that a
perfectly greedy emptier is $0$-greedy-like. We call an emptier
\defn{greedy-like} if it is $\Delta$-greedy-like for $\Delta \le
O(1)$. 

With an oblivious filler we are only able to prove lower bounds
on backlog against greedy-like emptiers; whether or not our
results can be extended to a more general class of emptiers is an
interesting open question. Nonetheless, greedy-like emptiers are
of great interest because all the known randomized algorithms for
the cup game are greedy-like \cite{mbe19, wku20}.

As a tool in our analysis we define a new variant of the cup
game: In the $p$-processor \defn{$E$-extra-emptyings}
\defn{$S$-skip-emptyings} negative-fill cup game on $n$ cups, the filler
distributes $p$ units of water amongst the cups, and then the
emptier empties from $p$ \textit{or more, or less} cups. In
particular the emptier is allowed to do $E$ extra
emptyings---we say that the emptier does an extra emptying if
it empties from a cup beyond the $p$ cups it typically is
allowed to empty from---and is also allowed to skip $S$
emptyings---we say that the emptier skips an emptying if it
doesn't do an emptying---over the course of the
game. Note that the emptier still cannot empty from the same cup
twice on a single round. Also note that the emptier is allowed to
skip extra emptyings in addition to regular emptyings. Also note
that a $\Delta$-greedy-like emptier must take into account extra
emptyings and skip emptyings to determine valid moves. 

For a $\Delta$-greedy-like emptier let $R_\Delta = 2(2+\Delta)$;
we now prove a key property of these emptiers: an oblivious
filler can attain an $R_\Delta$-flat cup configuration against a
$\Delta$-greedy-like emptier, given cups of a known starting
fill-range.

\begin{lemma}
  \label{lem:greedylikeisflat}
  Consider an $R$-flat cup configuration in the $p$-processor
  $E$-extra-emptyings $S$-skip-emptyings negative-fill cup game on $n =
  2p$ cups. There is an oblivious filling strategy
  \defn{\flatalg} that achieves an $R_\Delta$-flat
  configuration of cups against a $\Delta$-greedy-like emptier in
  running time $2(R + \ceil{(1+1/n)(E + S)})$. Furthermore,
  \flatalg guarantees that the cup configuration is $R$-flat on every round.
\end{lemma}
\begin{proof}
  If $R \le R_\Delta$ the algorithm does nothing, since the
  desired fill-range is already achieved; for
  the rest of the proof we consider $R > R_\Delta$.

  The filler's strategy is to distribute fill equally amongst all
  cups at every round, placing $p/n = 1/2$ fill in each cup. 
  Let $\ell_t = \min_{c\in S_t} \fil_{S_t}(c)$, $u_t = \max_{c\in S_t} \fil_{S_t}(c)$. 

  First we show that the fill-range of the cups can only increase
  if the fill-range is very small.
  \begin{clm}
    \label{clm:uandlgrowclosertogetherunlesstheyreveryclosealready}
    If $u_{t+1}-\ell_{t+1} > u_t - \ell_t$ then 
    $$u_{t+1} - \ell_{t+1} \le R.$$
  \end{clm}
  \begin{proof}
    First we remark that the fill of any cup changes by at most
    $1/2$ from round to round, and in particular $|u_{t+1}-u_t|
    \le 1/2$, $|\ell_{t+1} - \ell_t|\le 1/2$.
  In order for the fill-range to increase the emptier must have
  emptied from some cup with fill in $[\ell_t, \ell_t + 1]$ without
  emptying from some cup with fill in $[u_t-1, u_t]$; if the emptier
  had not emptied from every cup with fill in $[\ell_t, \ell_t+1]$
  then we would have $\ell_{t+1} = \ell_t + 1/2$ so the
  fill-range cannot have increased, and similarly if the emptier
  had emptied from every cup with fill in $[u_t-1, u_t]$ then we
  would have $u_{t+1} = u_t - 1/2$ so again the fill-range could
  not have increased. Because the emptier is $\Delta$-greedy-like
  emptying from a cup with fill at most $\ell_t+1$ and not
  emptying from a cup with fill at least $u_t-1$ implies that
  $u_t-1$ and $\ell_t+1$ differ by at most $\Delta$.
  Thus, 
  $$u_{t+1} - \ell_{t+1} \le u_t +1/2 - (\ell_t -1/2)  \le \Delta + 3 \le R.$$
  \end{proof}
  Because by \cref{clm:uandlgrowclosertogetherunlesstheyreveryclosealready}
  whenever the fill-range of the cups increases it increases to a
  value at most $R$, we have by induction that the fill-range of
  the cups never exceeds $R$, i.e. the cups are always $R$-flat.

  Let $L_t$ be the set of cups $c$ with $\fil_{S_t}(c) \le \ell_t+2+\Delta$, and let
  $U_t$ be the set of cups $c$ with $\fil_{S_t}(c) \ge u_t-2-\Delta$.

  Now we prove a key property of the sets $U_t$ and $L_t$: if a cup is in
  $U_t$ or $L_t$ it is also in $U_{t'}, L_{t'}$ for all $t' > t$. This
  follows immediately from \cref{clm:dontlosestuff}.
  \begin{clm}
    \label{clm:dontlosestuff}
    $$U_{t} \subseteq U_{t+1},\quad L_t \subseteq L_{t+1}.$$
  \end{clm}
  \begin{proof}
    Consider a cup $c\in U_t$.

    If $c$ is not emptied from, i.e. $\fil(c)$ has increased by
    $1/2$ from the previous round, then
    clearly $c \in U_{t+1}$, because backlog has increased by at most $1/2$, so
    $\fil(c)$ must still be within $2+\Delta$ of the backlog on round $t+1$. 

    On the other hand, if $c$ is emptied from, i.e. $\fil(c)$ has decreased by
    $1/2$, we consider two cases.\\
    \textbf{Case 1:} If $\fil_{S_t}(c) \ge u_t-\Delta -1$, then
    $\fil_{S_t}(c)$ is at least $1$ above the bottom of the
    interval defining which cups belong to $U_t$. The backlog
    increases by at most $1/2$ and the fill of $c$ decreases by $1/2$, so
    $\fil_{S_{t+1}}(c)$ is at least $1-1/2-1/2 = 0$ above the bottom of the
    interval, i.e. still in the interval. \\
    \textbf{Case 2:} On the other hand, if $\fil_{S_t}(c) <
    u_t-\Delta-1$, then every cup with fill in $[u_t-1, u_t]$
    must have been emptied from because the emptier is
    $\Delta$-greedy-like. Therefore the fullest cup
    on round $t+1$ is the same as the fullest cup on round $t$,
    because every cup with fill in $[u_t-1, u_t]$
    has had its fill decrease by $1/2$, and no cup with fill less than
    $u_t-1$ had its fill increase by more than $1/2$. Hence $u_{t+1}
    = u_t -1/2$. Because both $\fil(c)$ and the backlog
    have decreased by $1/2$, the distance between them is
    still at most $\Delta+2$, hence $c\in U_{t+1}$.\\
    The argument for why $L_t \subseteq L_{t+1}$ is symmetric.
  \end{proof}

  Now we show that under certain conditions $u_t$ decreases and
  $\ell_t$ increases.
  \begin{clm}
    \label{clm:howDoLandUchange}
    On any round $t$ where the emptier empties from at least
    $n/2$ cups, if $|U_t| \le n/2$ then $u_{t+1} = u_t - 1/2$.
    On any round $t$ where the emptier empties from at most $n/2$
    cups, if $|L_t| \le n/2$ then $\ell_{t+1} = \ell_t + 1/2$.
  \end{clm}
  \begin{proof}
    Consider a round $t$ where the emptier empties from at least
    $n/2$ cups. If there are at least $n/2$ cups outside of
    $U_t$, i.e. cups with fills in $[\ell_t, u_t-2-\Delta]$, then
    all cups with fills in $[u_t - 2, u_t]$ must be emptied from;
    if one such cup was not emptied from then by the pigeon-hole
    principle some cup outside of $U_t$ was emptied from, which
    is impossible as the emptier is $\Delta$-greedy-like. This
    clearly implies that $u_{t+1} = u_t - 1/2$: no cup with fill
    less than $u_t-2$ has gained enough fill to become the
    fullest cup, and the fullest cup from the previous round has
    lost $1/2$ unit of fill.

    By a symmetric argument $\ell_{t+1} = \ell_{t} + 1/2$ if the
    emptier empties at most $n/2$ cups on a round $t$ where
    $|L_t| \le n/2$. 
  \end{proof}

  Now we show that eventually $L_t \cap U_t \neq \varnothing$.
  \begin{clm}
    There is a round $t_0 \le 2(R + \ceil{(1+1/n)(E + S)})$ such that $U_{t}
    \cap L_{t} \neq \varnothing$ for all $t\ge t_0$.
  \end{clm}
  \begin{proof}
  We call a round where the emptier doesn't use $p=n/2$
  resources, i.e. a round where the number of skipped emptyings
  and the number of extra emptyings are not equal, an
  \defn{unbalanced round}; we call a round that is not unbalanced a
  \defn{balanced} round. 

  Note that there are clearly at most $E+S$ unbalanced rounds.
  We now associate some unbalanced rounds with balanced rounds;
  in particular we define what it means for a balanced round to
  \defn{cancel} an unbalanced round. We define cancellation by a
  sequential process. For $i = 1,2,\ldots,
  2(R+\ceil{(1+1/n)(E+S)})$ (iterating in ascending order of $i$), if round $i$
  is unbalanced then we say that the first balanced round $j > i$
  that hasn't already been assigned (earlier in the sequential
  process) to cancel another unbalanced round $i' < i$, if any
  such round $j$ exists, \defn{cancels} round $i$. Note that
  cancellation is a one-to-one relation: each unbalanced round is
  cancelled by at most one balanced round and each balanced round
  cancels at most one unbalanced round.

  Consider rounds of the form $2(R + \ceil{(E+S)/n}) + (E+S) + i$
  for $i \in [E+S+1]-1$. We claim that there is some such $i$
  such that among rounds $[2(R + \ceil{(E+S)/n}) + (E+S) + i]$
  every unbalanced round has been cancelled, and such that there
  are $2(R+\ceil{(E+S)/n})$ balanced rounds not cancelling other
  rounds. Assume for contradiction that such an $i$ does not
  exist. Note that there are at least $2(R + \ceil{(E+S)/n})$
  balanced rounds in the first $2(R + \ceil{(E+S)/n}) + (S+E)$
  rounds. Thus every balanced round $2R+(E+S)+\ceil{(E+S)/n} +
  i-1$ for $i \in [E+S+1]$ is necessarily a cancelling round, or
  else there would be a round by which there are no uncancelled
  unbalanced rounds. Hence by round $2(R + \ceil{(E+S)/n}) + 2(E +
  S)$, there must have been $E+S$ cancelled rounds, so on round
  $2(R + \ceil{(E+S)/n}) + 2(E+S)$ all unbalanced rounds are
  cancelled, which leaves $2(R + \ceil{(E+S)/n})$ balanced rounds
  that are not cancelling any rounds, as desired.

  Let $t_e$ be the first round by which there are $2(R +
  \ceil{(E+S)/n})$ balanced non-cancelling rounds. Note that the
  average fill of the cups cannot have decreased by more than
  $E/n$ from its starting value; similarly the average fill of
  the cups cannot have increased by more than $S/n$. Because the
  cups start $R$-flat, we have that $u_t$ cannot have decreased
  by more than $R + E/n$ or else $u_t$ would necessarily be below
  the average fill, and identically $\ell_t$ cannot have
  increased by more than $R + S/n$ or else it would be above the
  average fill. Now, by \cref{clm:howDoLandUchange} we have that
  eventually $|L_t| > n/2$: If $|L_t|\le n/2$ were always true,
  then on every balanced round $\ell_t$ would have increased by
  $1/2$, and since $\ell_t$ increases by at most $1/2$ on
  unbalanced rounds, this implies that in total $\ell_t$ would
  have increased by at least $(1/2)2(R + \ceil{(E+S)/n})$, which
  is impossible. By a symmetric argument it is impossible that
  $|U_t| \le n/2$ for all rounds. 

  Since $|U_{t+1}|\ge |U_t|$ and $|L_{t+1}| \ge |L_t|$ by
  \cref{clm:dontlosestuff}, we have that there is some round $t_0
  \in [2(R + \ceil{(1+1/n)(E+S)})]$ such that for all $t \ge t_0$
  we have $|U_t|> n/2$ and $|L_t|> n/2$. But then we
  have $U_t\cap L_t \neq \varnothing$, as desired.
  \end{proof}

  If there exists a cup $c \in L_t\cap U_t$, then 
  $$\fil(c) \in [u_t-2-\Delta, u_t] \cap [\ell_t, \ell_t + 2 +
  \Delta].$$ Hence we have that $$\ell_t+2+\Delta \ge
  u_t-2-\Delta.$$ Rearranging, $$u_t - \ell_t \le 2(2+\Delta) =
  R_\Delta.$$ Thus the cup configuration is $R_\Delta$-flat by
  the end of this flattening process.

\end{proof}

% note: may as well take $\Delta' = \ceil{\Delta} \in
% \mathbb{N}$, this might be useful sometimes

Next we describe a simple oblivious filling strategy that will be used as a
subroutine in \cref{lem:obliviousManyUnknownCups}; this strategy is very
well-known, and similar versions of it can be found in
\cite{ mbe19, mbe15, die91, wku20}.
\begin{proposition}
  \label{prop:obliviousTerribleProbability}
  Consider an $R$-flat cup configuration in the single-processor
  $E$-extra-emptyings $S$-skip-emptyings negative-fill cup
  game on $n$ cups with initial average fill $\mu_0$.
  Let $d = \sum_{i=2}^n 1/i$.

  There is an oblivious filling strategy \defn{\randalg} with running
  time $n-1$ that achieves a cup with fill at most $\mu_0 + R +
  d$; if we condition on the emptier not performing extra
  emptying then \randalg achieves fill at least $\mu_0 -R + d$ in
  a known cup $c$ with probability at least $1/(n-1)!$.

  Furthermore, when applied against a $\Delta$-greedy-like emptier
  with $R = R_\Delta$, \randalg guarantees that
  the cup configuration is $(R + d)$-flat on every round.
\end{proposition}
\begin{proof}
  First we condition on the emptier does not using extra emptying
  and show that in this case the filler has probability at least
  $1/(n-1)!$ of attaining a cup with fill at least $\mu_0 -R +d$.
  The filler maintains an \defn{active set}, initialized to being
  all of the cups. Every round the filler distributes $1$ unit of
  fill equally among all cups in the active set. Next the emptier
  removes $1$ unit of fill from some cup, or skips its emptying.
  Then the filler removes a random cup from the active set
  (chosen uniformly at random from the active set). This
  continues until a single cup $c$ remains in the active set. 

  We now bound the probability that $c$ has never been emptied
  from. Assume that on the $i$-th step of this process, i.e. when
  the size of the active set is $n-i+1$, no cups in the active
  set have ever been emptied from; consider the probability that
  after the filler removes a cup randomly from the active set
  there are still no cups in the active set that the emptier has
  emptied from. If the emptier skips its emptying on this round,
  or empties from a cup not in the active set then it is
  trivially still true that no cups in the active set have been
  emptied from. If the cup that the emptier empties from is in
  the active set then with probability $1/(n-i+1)$ it is evicted
  from the active set, in which case we still have that no
  cup in the active set has ever been emptied from. Hence with
  probability at least $1/(n-1)!$ the final cup in the
  active set, $c$, has never been emptied from. In this case, $c$
  will have gained fill $d=\sum_{i=2}^n 1/i$ as claimed. Because
  $c$ started with fill at least $-R+\mu_0$, $c$ now has fill at
  least $-R+ d+\mu_0$. 

  Now note that regardless of if the emptier uses extra emptyings
  $c$ has fill at most $\mu_0 + R + d$ , as $c$ starts with fill
  at most $R$, and $c$ gains at most $1/(n-i+1)$ fill on the
  $i$-th round of this process. 

  Now we analyze this algorithm specifically for a
  $\Delta$-greedy-like emptier. Consider a round $t$ on which
  $\min_c \fil_{S_{t+1}}(c) < \min_c \fil_{S_{t}}(c)$, and where a cup
  $c_0$ that has $\fil_{S_{t+1}}(c_0) = \max_{c}
  \fil_{S_{t+1}}(c)$ was not emptied from on round $t$. Because
  the emptier is $\Delta$-greedy-like this
  implies that $\fil_{I_t}(c_0) - \min_c \fil_{I_t}(c) \le \Delta
  + 1$ and then $\max_c \fil_{S_{t+1}}(c) - \min_c
  \fil_{S_{t+1}}(c) \le \Delta + 2$, i.e. the cups are $(\Delta +
  2)$-flat. 

  Consider some round $t_1$ on which the cups are not $(\Delta +
  2)$-flat; let $t_0$ be the last round on which the cups where
  $R$-flat (note that if the cups are $(\Delta+2)$-flat they are
  also $R$-flat as $\Delta + 2 < R$). Consider how the fill-range
  of the cups changes during the set of rounds
  $t$ with $t_0 < t \le t_1$. On any such round $t$ either $\min_c
  \fil_{S_{t+1}}(c) \ge \min_c \fil_{S_{t}}(c)$ in which case the
  fill-range increases by at most $1/(n-t+1)$ where $n-t+1$ is the
  size of the active set on round $t$, or all cups on round $t+1$
  with fill equal to the backlog were emptied from, meaning that
  backlog decreased by at least $1-1/(n-t+1)$. In either case the
  fill-range increases by at most $1/(n-t+1)$. Thus in total the
  fill-range is at most $R + d$. That is, the cups are
  $(R+d)$-flat on round $t_1$, as desired.

\end{proof}

Now we show that we can force a constant fraction of the cups to
have high fill; using \cref{lem:obliviousManyUnknownCups} and
exploiting the greedy-like nature of the emptier we can get a
known cup with high fill (we show this in
\cref{prop:obliviousBase}).
\begin{lemma}
  \label{lem:obliviousManyUnknownCups}
  Let $\Delta \le O(1)$, let $h \le O(1)$ with $h \ge
  16+16\Delta$, let $n$ be at least a sufficiently large constant
  determined by $h$ and $\Delta$, and let $R \le \poly(n)$.
  Consider an $R$-flat cup configuration in the
  variable-processor cup game on $n$ cups.
  Let $A, B$ be disjoint subsets of the cups with $|AB| = n$.
  Over the course of the algorithm $B$ will give some cups to
  $A$, but $|A|$ will always satisfy $|A| \ll |B|$, and $|A|$
  will eventually be $\Theta(n)$.
  Let $M\gg n$ be very large.

  There is an oblivious filling strategy that either achieves
  mass at least $M$ in the cups, or makes an unknown
  set of $\Theta(n)$ cups in $A$ have fill at least $h$ with
  probability at least $1-2^{-\Omega(n)}$ in running time
  $\poly(M)$ against a $\Delta$-greedy-like emptier while also
  guaranteeing that $\mu(B) \ge -h/2$.
\end{lemma}
\begin{proof}
  We refer to $A$ as the \defn{anchor} set, and $B$ as the
  \defn{non-anchor} set. Let $n_A = \Theta(n)$ be small enough to
  satisfy
  \begin{equation}
    \label{eq:chooseBmuchbiggerthanA}
    n_A \le (n - n_A) / (2e^{2h+1} + 1).
  \end{equation}
  The filler initializes $A$ to $\varnothing$, and $B$ to be all
  of the cups. Over the course of the algorithm $B$ will give
  away $n_A$ cups to $A$. Note that $|B| \ge n-n_A \gg n_A \ge |A|$.

We denote by \randalg the oblivious filling
strategy given by \cref{prop:obliviousTerribleProbability}. 
We denote by \flatalg the oblivious filling
strategy given by \cref{lem:greedylikeisflat}.
We say that the filler \defn{applies} a filling strategy
\genericalg to a set of cups $D \subseteq B$ if the filler uses
\genericalg on $D$ while placing $1$ unit of fill in each anchor cup. 

We now describe the filler's strategy.

The filler starts by flattening the cups, i.e. using \flatalg on
all of the cups for $\poly(M)$ rounds (setting $p=n/2$). After
this the filling strategy always places $1$ unit of water in to
each anchor cup on every round. The filler performs a series of
$n_A$ \defn{swapping-processes}, one per anchor cup, which are
procedures that the filler uses to get a new cup---which will
sometimes have high fill---in the anchor set. On each
swapping-process the filler applies \randalg many times to
arbitrarily chosen constant-size sets $D \subset B$ with $|D| =
\ceil{e^{2h+1}}$. The number of times that the filler applies
\randalg is chosen at the start of the swapping-process, chosen
uniformly at random from $[m]$ ($m = \poly(M)$ to be specified).
At the end of the swapping-process, the filler does a \defn{swap}:
the filler takes the cup given by \randalg in $B$ and moves it
into $A$. After performing a swap the filler must increase $p$ by
$1$ so that $p=|A| + 1$. Before each application of \randalg the filler flattens
$B$ by applying \flatalg to $B$ for $\poly(M)$ rounds. 

We remark that this construction is similar to the construction
in \cref{lem:adaptiveAmplification}, but has a major difference
that substantially complicates the analysis: in the adaptive
lower bound construction the filler halts after achieving the
desired average fill in the anchor set, whereas the oblivious
filler cannot halt but rather must rely on the emptier's
greediness to guarantee that each application of \randalg has
constant probability of generating a cup with high fill.

We proceed to analyze our algorithm.

Note that if the emptier skips more than $M$ emptyings, or
neglects the anchor set more than $M$ times without decreasing
the fills of any anchor cups in between these times, then the filler has
achieved mass $M$ in the cups, in which case the filler has
fulfilled the statement of \cref{lem:obliviousManyUnknownCups}. 
Throughout the remainder of the analysis we consider the case of an
emptier that chooses to not skip more than $M$ times, and chooses
to not neglect the anchor set more than $M$ times without
decreasing the fill of anchor cups in between these times.

First note that the initial flattening makes the cups
$R_\Delta$-flat by \cref{lem:greedylikeisflat}. In particular,
note that the flattening happens in the $(n/2)$-processor
$0$-extra-emptyings $M$-skip-emptyings variable-processor cup
game on $n$ cups.

We say that a property of the cups has \defn{always} held if the
property has held since the start of the first swapping-process;
i.e. from now on we only consider rounds after the initial
flattening.

We say that the emptier \defn{neglects} the anchor set on a round
if it does not empty from each anchor cup. We say that an
application of \randalg to $D\subset B$ is
\defn{non-emptier-wasted} if the emptier does not neglect the
anchor set during any round of the application of \randalg. We
define $d = \sum_{i=2}^{|D|} 1/i$ (recall that $|D| =
\ceil{e^{2h+1}}$). We say that an application of \randalg to $D$
is \defn{lucky} if it achieves backlog at least $\mu(B) -
R_\Delta + d$; note that by
\cref{prop:obliviousTerribleProbability} if we condition on an
application of \randalg where $B$ started $R_\Delta$-flat being
non-emptier-wasted then the application has at least a $1/|D|!$
chance of being lucky.

Now we prove several important bounds on fills of cups in $A$ and $B$.
\begin{clm}
  \label{clm:allflatteningsworkbyM}
  All applications of \flatalg make $B$ be $R_\Delta$-flat and
  $B$ is always $(R_\Delta + d)$-flat.
\end{clm}
\begin{proof}
  Given that the application of \flatalg immediately prior to an application
  of \randalg made $B$ be $R_\Delta$-flat, by
  \cref{prop:obliviousTerribleProbability} we have that $B$ will
  stay $(R_\Delta + d)$-flat during the application of \randalg. 
  Given that the application of \randalg immediately prior to an
  application of \flatalg resulted in $B$ being $(R_\Delta
  + d)$-flat, we have that $B$ remains $(R_\Delta + d)$-flat
  throughout the duration of the application of \flatalg by
  \cref{lem:greedylikeisflat}. Given that $B$ is $(R_\Delta +
  d)$-flat before a swap occurs $B$ is clearly still $(R_\Delta +
  d)$-flat after the swap, because the only change to $B$ during
  a swap is that a cup is removed from $B$ which cannot increase
  the fill-range of $B$.
  Note that $B$ started $R_\Delta$-flat before the first
  application of \flatalg because all the cups was flattened.
  Note that if an application of \flatalg begins with $B$ being
  $(R_\Delta + d)$-flat, then by considering the flattening to
  happen in the $(|B|/2)$-processor $M$-extra-emptyings
  $M$-skip-emptyings cup game we ensure that it makes $B$ be
  $R_\Delta$-flat, because the emptier cannot skip more than $M$
  emptyings and also cannot do more than $M$ extra emptyings or
  the mass would be at least $M$.
  Hence we have by induction that $B$ has always been $(R_\Delta
  + d)$-flat and that all flattening processes have made $B$ be
  $R_\Delta$-flat. 
  % start, flatalg, randalg, flatalg, randalg, ..., flatalg
  % randalg, swap, flatalg, randalg, flatalg, randalg, ..., swap,
  % ..., swap, ..., right now
\end{proof}

Now we aim to show that $\mu(B)$ is never too low, which we need
in order to establish that every non-emptier-wasted lucky
application of \randalg gets a cup with high fill. Interestingly
in order to lower bound $\mu(B)$ we first must upper bound
$\mu(B)$, which by greediness and flatness of $B$ gives an upper
bound on $\mu(A)$ which we use to get a lower bound on $\mu(B)$.

\begin{clm}
  \label{clm:muBdoesntgettoobig}
  We have always had
  $$\mu(B) \le 2 + \mu(A B).$$
\end{clm}
\begin{proof}
  There are two ways that $\mu(B)-\mu(A B)$ can increase: \\
  \textbf{Case 1:}
  The emptier could empty from $0$ cups in $B$ while emptying
  from every cup in $A$. \\
  \textbf{Case 2:}
  The filler could evict a cup with fill lower than $\mu(B)$ from
  $B$ at the end of a swapping-process. \\

  Note that cases are exhaustive, in particular note that if the
  emptier skips more than $1$ emptying then $\mu(B) - \mu(AB)$
  must decrease because $|A|\approx |AB|$, in particular
  \eqref{eq:chooseBmuchbiggerthanA}, as opposed to in Case 1
  where $\mu(B) - \mu(AB)$ increases.

  In Case 1, because the emptier is $\Delta$-greedy-like,
  $$\min_{a\in A} \fil(a) > \max_{b\in B} \fil(b) - \Delta.$$
  Thus $\mu(B) \le \mu(A) + \Delta$. As $|B| \gg |A|$, in
  particular by \eqref{eq:chooseBmuchbiggerthanA}, this can be
  loosened to $\mu(B) \le 1 + \mu(A B)$.

  Consider the final round on which $B$ is skipped while $A$ is
  not skipped (or consider the first round if there is no such
  round).

  From this round onwards the only increase to $\mu(B) - \mu(A
  B)$ is due to $B$ evicting cups with fill well below $\mu(B)$.
  We can upper bound the increase of $\mu(B) - \mu(A B)$ by the
  increase of $\mu(B)$ as $\mu(A B)$ is strictly increasing.

  The cup that $B$ evicts at the end of a
  swapping-process has fill at least $\mu(B) - R_\Delta -
  (|D|-1)$, as the running time of \randalg is $|D|-1$, and
  because $B$ starts $R_\Delta$-flat by
  \cref{clm:allflatteningsworkbyM}. Evicting a cup
  with fill $\mu(B) - R_\Delta - (|D| -1)$ from $B$ changes
  $\mu(B)$ by $(R_\Delta + |D| - 1) / (|B|-1)$ where $|B|$ is the
  size of $B$ before the cup is evicted from $B$. Even if this
  happens on each of the $n_A$ swapping processes $\mu(B)$ cannot
  rise higher than $n_A (R_\Delta + |D|-1) / (n-n_A)$ which by
  design in choosing $|B| \gg |A|$, as was done in
  \eqref{eq:chooseBmuchbiggerthanA}, is at most $1$.

  Thus it always is the case that $\mu(B) \le 2 + \mu(A B).$

\end{proof}

The upper bound on $\mu(B)$ along with the guarantee that $B$ is
flat allows us to bound the highest that a cup in $A$ could rise
by greediness, which in turn upper bounds $\mu(A)$ which in turn
lower bounds $\mu(B)$. In particular we have
\begin{clm}
  \label{clm:muBgreaterthanminushover2}
  We always have
  $$\mu(B) \ge -h/2.$$
\end{clm}
\begin{proof}
  By \cref{clm:muBdoesntgettoobig} and \cref{clm:allflatteningsworkbyM} 
  we have that no cup in $B$ ever has fill greater than
  $u_B = \mu(A B) + 2 + R_\Delta + d$. 

  Let $u_A = u_B + \Delta + 1$. We claim that the backlog in $A$
  never exceeds $u_A$.

  Consider how high the fill of a cup $c \in A$ could be.
  If $c$ came from $B$ then when it is swapped
  into $A$ its fill is at most $u_B < u_A$. Otherwise, $c$
  started with fill at most $R_\Delta < u_A$. Now consider how
  much the fill of $c$ could increase while being in $A$. Because
  the emptier is $\Delta$-greedy-like, if a cup $c\in A$ has fill
  more than $\Delta$ higher than the backlog in $B$ then $c$ must
  be emptied from, so any cup with fill at least $u_B + \Delta =
  u_A - 1$ must be emptied from, and hence $u_A$ upper bounds the
  backlog in $A$. 

  Of course an upper bound on backlog in $A$ also serves as
  an upper bound on the average fill of $A$ as well, i.e.
  $\mu(A) \le u_A$. 
  Rearranging the expression 
  $$|B|\mu(B) + |A|\mu(A) = |AB|\mu(AB)$$
  we have
  \begin{align*}
    \mu(B) &\\
           &= -\frac{|A|}{|B|} \mu(A) + \frac{|A B|}{|B|}\mu(A B) \\
           &\ge -(\mu(AB) + 3+R_\Delta+d+\Delta) \frac{|A|}{|B|} + \frac{|AB|}{|B|}\mu(AB)\\
           &= -(3+R_\Delta+d + \Delta) \frac{|A|}{|B|} + \mu(AB)\\
           &\ge -h/2
  \end{align*}
  where the final inequality follows because $\mu(AB) \ge 0$, and
  $|B|\gg |A|$, in particular by \eqref{eq:chooseBmuchbiggerthanA}.

\end{proof}

Now we show that this guarantees that with constant probability
the final application of \randalg on a swapping-process is both
lucky and successful. 
\begin{clm}
  \label{clm:existsMsuchthatSwapHasConstantPrOfSuccess}
  There exists choice of $m = \poly(M)$ such that with at least
  constant probability the final application of \randalg on any
  fixed swapping-process is both lucky and successful.
\end{clm}
\begin{proof}
  Fix some swapping-process.
  The emptier can neglect the anchor set a limited number of
  times in this swapping-process, in particular at most $M$
  times. Hence, the emptier must have some policy for deciding
  whether or not to neglect the $i$-th application of \randalg.
  The emptier's policy can of course depend on whether or not the
  application of \randalg looks like it is going to be lucky.
  The emptier must come up with some function $I: [m] \to
  \{0,1\}$ where $I(i)=1$ means that the emptier will thwart the
  $i$-th application of \randalg if it looks like it is going to
  be lucky, and if the emptier has not already neglected the
  anchor set $M$ times.

  The filler chooses $m = 4 M |D|!$. By a Chernoff bound, there is
  exponentially high probability that of $4 M |D|!$ applications
  of \randalg at least $2M$ look like they are going to be lucky. 
  The emptier can choose at most $M$ of these on which to neglect
  the anchor set and thwart the application of \randalg. Thus,
  conditioning on the final round looking like it is going to be
  lucky, there is at least a $1/2$ chance that it is successful.
  The final application looks like it is going to be lucky with
  probability $1/|D|!$. 
  Hence our choice of $m$ makes the final round lucky
  and successful with constant probability at least $1/(2|D|!)$.
\end{proof}

\begin{clm}
With probability at least $1-2^{-\Omega(n)}$, the filler achieves fill
at least $h$ in at least $\Theta(n)$ of the cups in $A$. 
\end{clm}
\begin{proof}
  By \cref{clm:existsMsuchthatSwapHasConstantPrOfSuccess} each
  swapping-process has at least constant probability of swapping
  a cup with fill at least $\mu(B) + d - R_\Delta$ into $A$. The
  events that the swapping-processes swap such a cup into $A$ are
  independent, so by a Chernoff bound there is exponentially high
  probability that at least a constant fraction of them succeed.
  By \cref{clm:muBgreaterthanminushover2} $\mu(B) \ge -h/2$.
  Recalling that $d\ge 2h$ and $h \ge 16(1+\Delta)$, we have that
  such a swapped cup has fill at least $h$, as desired.
\end{proof}

We now analyze the running time of the filling strategy.
There are $|A|$ swapping-processes. Each swapping-process
consists of $\poly(M)$ applications of \randalg, which each take
constant time, and $\poly(M)$
applications of \flatalg, which each take $\poly(M)$ time.
Thus overall the algorithm takes $\poly(M)$ time, as desired.
  
\end{proof}

Finally, using \cref{lem:obliviousManyUnknownCups} we can show in
\cref{prop:obliviousBase} that an oblivious filler can achieve
constant backlog. We remark that \cref{prop:obliviousBase} plays a
similar role in the proof of the lower bound on backlog as
\cref{prop:adaptiveBase} does in the adaptive case, but is vastly
more complicated to prove (in particular,
\cref{prop:adaptiveBase} is trivial, whereas we have already
proved several lemmas and propositions as preparation for the
proof of \cref{prop:obliviousBase}).
\begin{proposition}
  \label{prop:obliviousBase}
  Let $H \le O(1)$, let $\Delta \le O(1)$, let $n$ be at
  least a sufficiently large constant determined by $H$ and
  $\Delta$, and let $R \le \poly(n)$. 
  Let $M \gg n$ be very large.
  Consider an $R$-flat cup configuration in the negative-fill variable-processor cup
  game on $n$ cups with average fill $0$.
  Given this configuration, an oblivious filler can either
  achieve mass $M$ among the cups, or achieve fill $H$
  in a chosen cup in running time $\poly(M)$ against a
  $\Delta$-greedy-like emptier with probability at least $1-2^{-\Omega(n)}.$
\end{proposition}
\begin{proof}
  The filler starts by performing the procedure detailed in
  \cref{lem:obliviousManyUnknownCups}, using $h = H\cdot
  16(1+\Delta)$. Let the number of cups which must now exist with
  fill $h$ be of size $nc = \Theta(n)$.

  The filler reduces the number of processors to $p=nc$. 
  Now the filler exploits the filler's greedy-like nature to
  to get fill $H$ in a set $S\subset B$ of $nc$ chosen cups.

  The filler places $1$ unit of fill into each cup in $S$.
  Because the emptier is greedy-like it must focus on the $nc$
  cups in $A$ with fill at least $h$ until the cups in $S$ have
  sufficiently high fill. In particular, $(5/8)h$ rounds suffice.
  Over $(5/8)h$ rounds the $nc$ high cups in $A$ cannot have
  their fill decrease below $(3/8)h \ge h/8 + \Delta$. Hence, any
  cups with fills less than $h/8$ must not be emptied from during
  these rounds. The fills of the cups in $S$ must start as at
  least $-h/2$ as $\mu(B) \ge -h/2$. After $(5/8)h$ rounds the
  fills of the cups in $S$ are at least $h/8$, because throughout
  this process the emptier cannot have emptied from them until
  they got fill at least $h/8$, and if they are never emptied
  from then they achieve fill $h/8$.

  Thus the filling strategy achieves backlog $h/8 \ge H$ in some
  known cup (in fact in all cups in $S$, but a single cup
  suffices), as desired.

\end{proof}

