We now give a method for transforming a filling strategy for achieving
large backlog into a filling strategy for achieving high fill in
many cups, or high average fill in a set of cups (which of these
we guarantee depends on the original filling strategy). The idea
of repeating an algorithm many times is also used in the proof of
the Adaptive Amplification Lemma; the construction is slightly more
complicated in the randomized case however, and is much harder to
analyze.

\begin{definition}
  \label{def:rep}
  {\normalfont
  Let $\alg_0$ be an oblivious filling strategy, that can get
  high fill (for some definition of high) in some cup against
  greedy-like emptiers with some probability. We construct a new
  filling strategy \defn{$\rep_\delta(\alg_0)$} ($\rep$ stands
  for \enquote{repetition}) as follows:

  Say we have some configuration of $n\le N$ cups (recall that
  eventaully we aim to get large backlog in $N$ cups).
  Let $n_A = \ceil{\delta n}, n_B = \floor{(1-\delta)n}$. Let
  $M=2^{\polylog(N)}$ be a chosen parameter. 
  Initialize $A$ to $\varnothing$ and $B$ to
  being all of the cups. We call $A$ the \defn{anchor set} and
  $B$ the \defn{non-anchor set}. The filler always places $1$
  unit of fill in each anchor cup on each round. The filling
  strategy consists of $n_A$ \defn{donation-processes}, which are
  procedures that result in a cup being \defn{donated} from $B$
  to $A$ (i.e. removed from $B$ and added to $A$). At the start
  of each donation-processes the filler chooses a value $m_0$
  uniformly at random from $[M]$. We say that the filler
  \defn{applies} a filling strategy $\alg$ to $B$ if the
  filler uses $\alg$ on $B$ while placing $1$ unit of fill
  in each anchor cup. During the donation-process the filler
  applies $\alg_0$ to $B$ $m_0$ times, and flattens $B$ by
  applying $\flatalg$ to $B$ for $\Theta(N^2)$ rounds before each
  application of $\alg_0$. At the end of each donation process
  the filler takes the cup given by the final application of
  $\alg_0$ (i.e. the cup that $\alg_0$ guarantees with some
  probability against a certain class of emptiers to have a
  certain high fill), and donates this cup to $A$. 

  We give pseudocode for this algorithm in \cref{alg:rep}.
 }

\begin{algorithm}
  \caption{$\rep_\delta(\alg_0)$}
  \label{alg:rep}
  \begin{algorithmic}
    \State \textbf{Input:} $\alg_0, \delta, N, M, $ set of $n$ cups
    \State \textbf{Output:} Guarantees on the sets $A, B$ (will vary based on $\alg_0$)
    \State
    \State $n_A \gets \ceil{\delta n}, n_B \gets \floor{(1-\delta) n}$
    \State $A \gets \varnothing, B \gets$ all the cups
    \State Always place $1$ fill in each cup in $A$
    \For{$i \in [n_A]$} \Comment Donation-Processes
    \State $m_0 \gets \text{random}([M])$
      \For{$j \in [m_0]$}
        \State Apply $\flatalg$ to $B$ for $\Theta(N^2)$ rounds
        \State Apply $\alg_0$ to $B$
      \EndFor
      \State Donate the cup given by $\alg_0$ from $B$ to $A$
    \EndFor
  \end{algorithmic}
\end{algorithm}

{\normalfont
We say that the emptier \defn{neglects} the anchor set on a round
if it does not empty from each anchor cup. We say that an
application of $\alg_0$ to $B$ is \defn{non-emptier-wasted} if
the emptier does not neglect the anchor set during any round of
the application of $\alg_0$.
}
\end{definition}

We use the idea of repeating an algorithm in two different contexts.
First in \cref{prop:obliviousBase} we prove a result analogous to that of
\cref{prop:adaptiveBase}: in particular, we show that we can
achieve constant fill in a known cup by using
$\rep_{\Theta(1)}(\randalg(\Theta(1))$
which achieves, by a Chernoff bound, $\Theta(n)$ unknown cups
with constant fill, and then exploiting the emptier's greedy-like
nature to achieve constant fill in a known cup.
After doing this, we prove the \defn{Oblivious Amplification
Lemma}, a result analogous to the Adaptive
Amplification Lemma: in particular, we show how to take an
algorithm for achieving some backlog, and then achieve higher
backlog by repeating the algorithm many times.
Although these results have deterministic analogues, their proofs
are different and significantly more complex than the proofs for the
deterministic cases.

In the rest of the section our aim is to achieve backlog
$N^{1-\varepsilon}$ in $N$ cups. We will use this value $N$
within all of the following proofs. Many values implicitly depend
on $N$. Note that we implicitly consider the cups to be part of a
larger game in these results. Also note that we are happy if
mass $N^2$ is achieved in the cups, because then backlog is
always at least $N$.

Before proving \cref{prop:obliviousBase} we analyze
$\rep_{\Theta(1)}(\randalg(\Theta(1)))$ in
\cref{lem:obliviousBase}.
\begin{lemma}
  \label{lem:obliviousBase}
  Let $\Delta \le \frac{1}{128}\log\log\log N$, let $h =
  \frac{1}{16}\log\log\log N$, let $k=\ceil{e^{2h+1}}$, let
  $\delta = \frac{1}{2k}$, let $n = \log^5 N$. Consider an
  $R_\Delta$-flat cup configuration in the variable-processor cup
  game on $n$ cups with initial average fill $\mu_0$.

  Against a $\Delta$-greedy-like emptier,
  $\rep_{\delta}(\randalg(k))$ using $M = 2^{\Theta(\log^4 N)}$
  either achieves mass $N^2$ in the cups, or with probability at
  least $1-2^{-\Omega(\log^4 N)}$ makes an unknown cup in $A$
  have fill at least $h+\mu_0$ while also
  guaranteeing that $\mu(B) \ge -h/2 + \mu_0$, where $A,B$ are
  the sets defined in \cref{def:rep}. The running time of
  $\rep_{\delta}(\randalg(k))$ is $2^{O(\log^4 N)}$.
\end{lemma}
\begin{proof}
  We use the definitions given in \cref{def:rep}.

  Note that if the emptier neglects the anchor set $N^2$ times,
  or skips $N^2$ emptyings, then the mass of the cups will be at
  least $N^2$, so the filler is done. For the rest of the proof
  we consider the case where the emptier chooses to neglect
  the anchor set fewer than $N^2$ times, and chooses to skip fewer
  than $N^2$ emptyings.

As in \cref{prop:randalg}, we define $d =
\sum_{i=2}^{k} 1/i$; we remark that, because harmonic numbers
grow like $x\mapsto \ln x$, it is clear that $d=\Theta(h)$. We say that an
application of $\randalg(k)$ to $B$ is \defn{lucky} if it
achieves backlog at least $\mu_S(B) - R_\Delta + d$ where $S$
denotes the state of the cups at the start of the application of
$\randalg(k)$; note that by
\cref{prop:randalg} if we condition on an
application of $\randalg(k)$ where $B$ started $R_\Delta$-flat
being non-emptier-wasted then the application has at least a
$1/k!$ chance of being lucky.

Now we prove several important bounds satisfied by $A$ and $B$.
\begin{clm}
  \label{clm:allflatteningsworkbyM}
  All applications of $\flatalg$ make $B$ be $R_\Delta$-flat and
  $B$ is always $(R_\Delta + d)$-flat.
\end{clm}
\begin{proof}
  Given that the application of $\flatalg$ immediately prior to an application
  of $\randalg(k)$ made $B$ be $R_\Delta$-flat, by
  \cref{prop:randalg} we have that $B$ will
  stay $(R_\Delta + d)$-flat during the application of $\randalg(k)$. 
  Given that the application of $\randalg(k)$ immediately prior to an
  application of $\flatalg$ resulted in $B$ being $(R_\Delta
  + d)$-flat, we have that $B$ remains $(R_\Delta + d)$-flat
  throughout the duration of the application of $\flatalg$ by
  \cref{lem:flatalg}. Given that $B$ is $(R_\Delta +
  d)$-flat before a donation occurs $B$ is clearly still $(R_\Delta +
  d)$-flat after the donation, because the only change to $B$ during
  a donation is that a cup is removed from $B$ which cannot increase
  the fill-range of $B$.
  Note that $B$ started $R_\Delta$-flat at the beginning of the
  first donation-process.
  Note that if an application of $\flatalg$ begins with $B$ being
  $(R_\Delta + d)$-flat, then by considering the flattening to
  happen in the $(|B|/2)$-processor $N^2$-extra-emptyings
  $N^2$-skip-emptyings cup game we ensure that it makes $B$ be
  $R_\Delta$-flat.
  Hence we have by induction that $B$ has always been $(R_\Delta
  + d)$-flat and that all flattening processes have made $B$ be
  $R_\Delta$-flat. 
\end{proof}

Now we aim to show that $\mu(B)$ is never very low, which we need
in order to establish that every non-emptier-wasted lucky
application of $\randalg(k)$ gets a cup with high fill.
Interestingly, in order to lower bound $\mu(B)$ we find it
convenient to first upper bound $\mu(B)$, which by greediness and
flatness of $B$ gives an upper bound on $\mu(A)$ which we then
use to get a lower bound on $\mu(B)$.

\begin{clm}
  \label{clm:muBdoesntgettoobig}
  We have always had
  $$\mu(B) \le \mu(AB) + 2.$$
\end{clm}
\begin{proof}
  There are two ways that $\mu(B)-\mu(A B)$ can increase: \\
  \textbf{Case 1:}
  The emptier could empty from $0$ cups in $B$ while emptying
  from every cup in $A$. \\
  \textbf{Case 2:}
  The filler could evict a cup with fill lower than $\mu(B)$ from
  $B$ at the end of a donation-process. \\

  Note that cases are exhaustive, in particular note that if the
  emptier skips more than $1$ emptying then $\mu(B) - \mu(AB)$
  must decrease because $|B| > |AB|/2$, as opposed to in Case 1
  where $\mu(B) - \mu(AB)$ increases.

  In Case 1, because the emptier is $\Delta$-greedy-like,
  $$\min_{a\in A} \fil(a) > \max_{b\in B} \fil(b) - \Delta.$$
  Thus $\mu(B) \le \mu(A) + \Delta$. We can use this to get an
  upper bound on $\mu(B) - \mu(AB)$. We have, 
  \begin{align*}
    \mu(B) &= \frac{\mu(AB) |AB| - \mu(A) |A|}{|B|}\\
           &\le \frac{\mu(AB) |AB| - (\mu(B) - \Delta) |A|}{|B|}.
  \end{align*}
  Rearranging terms:
  $$\mu(B) \paren{1+\frac{|A|}{|B|}} \le \frac{\mu(AB) |AB| + \Delta |A|}{|B|}.$$
  Now, because $|A| \cdot \Delta \le n_A
  \cdot \Delta < n$ (by choosing $\delta$ very small), we have 
  $$\mu(B) \frac{|AB|}{|B|}\le \frac{\mu(AB) |AB| + n}{|B|}.$$
  Isolating $\mu(B)$ we have 
  $$\mu(B) \le \mu(AB) + 1.$$

  Consider the final round on which $B$ is skipped while $A$ is
  not skipped (or consider the first round if there is no such
  round).

  From this round onwards the only increase to $\mu(B) - \mu(A
  B)$ is due to $B$ evicting cups with fill well below $\mu(B)$.
  We can upper bound the increase of $\mu(B) - \mu(AB)$ by the
  increase of $\mu(B)$ as $\mu(AB)$ is strictly increasing.

  The cup that $B$ evicts at the end of a
  donation-process has fill at least $\mu(B) - R_\Delta -
  (k-1)$, as the running time of $\randalg(k)$ is $k-1$, and
  because $B$ starts $R_\Delta$-flat by
  \cref{clm:allflatteningsworkbyM}. Evicting a cup
  with fill $\mu(B) - R_\Delta - (k -1)$ from $B$ changes
  $\mu(B)$ by $(R_\Delta + k - 1) / (|B|-1)$ where $|B|$ is the
  size of $B$ before the cup is evicted from $B$. Even if this
  happens on each of the $n_A$ donation-processes $\mu(B)$ cannot
  rise higher than $n_A (R_\Delta + k-1) / (n-n_A)$ which by
  design in choosing $\delta = 1/(2k)$ is at most $1$.

  Thus $\mu(B) \le \mu(AB) + 2$ is always true.

\end{proof}

Now, the upper bound on $\mu(B)-\mu(AB)$ along with the guarantee
that $B$ is flat allows us to bound the highest that a cup in $A$
could rise by greediness, which in turn upper bounds $\mu(A)$
which in turn lower bounds $\mu(B)$. 
\begin{clm}
  \label{clm:muBgreaterthanminushover2}
  We always have
  $$\mu(B) \ge -h/2 + \mu_0.$$
\end{clm}
\begin{proof}
  By \cref{clm:muBdoesntgettoobig} and \cref{clm:allflatteningsworkbyM} 
  we have that no cup in $B$ ever has fill greater than
  $u_B = \mu(A B) + 2 + R_\Delta + d$. 
  Let $u_A = u_B + \Delta + 1$. We claim that the backlog in $A$
  never exceeds $u_A$. Note that $\mu(AB), u_A, u_B$ are
  implicitly functions of the round; $\mu(AB)$ can increase from
  $\mu_0$ if the emptier skips emptyings.

  Consider how high the fill of a cup $c \in A$ could be.
  If $c$ came from $B$ then when it is donated
  to $A$ its fill is at most $u_B$; otherwise, $c$
  started with fill at most $R_\Delta$. Both of these expressions
  are less than $u_A - 1$. Now consider how
  much the fill of $c$ could increase while being in $A$. Because
  the emptier is $\Delta$-greedy-like, if a cup $c\in A$ has fill
  more than $\Delta$ higher than the backlog in $B$ then $c$ must
  be emptied from, so any cup with fill at least $u_B + \Delta =
  u_A - 1$ must be emptied from, and hence $u_A$ upper bounds the
  backlog in $A$. 

  Of course an upper bound on backlog in $A$ also serves as
  an upper bound on the average fill of $A$ as well, i.e.
  $\mu(A) \le u_A$.  Now we have
  \begin{align*}
    &\mu(B) \\
           &= -\frac{|A|}{|B|} \mu(A) + \frac{|A B|}{|B|}\mu(A B) \\
           &\ge -(\mu(AB) + 3+R_\Delta+d+\Delta) \frac{|A|}{|B|} + \frac{|AB|}{|B|}\mu(AB)\\
           &= -(3+R_\Delta+d + \Delta) \frac{|A|}{|B|} + \mu(AB)\\
           &\ge -h/2 + \mu(AB)
  \end{align*}
  where the final inequality follows because $\mu(AB) \ge 0$, and
  because $|A|/|B|$ is sufficiently small by our choice of $\delta = 1/(2k)$.
  Of course $\mu(AB) \ge \mu_0$ so we have
  $$\mu(B) \ge -h/2 + \mu_0.$$

\end{proof}

Now we show that at least a constant fraction of the
donation-processes succeed with exponentially good probability.
\begin{clm}
  \label{clm:baseChernoffBound}
  By choosing $M = 2^{\Theta(\log^4 N)}$ the filler can guarantee that with
  probability at least $1-2^{-\Omega(\log^4 N)}$, the filler achieves
  fill $h+\mu_0$ in some cup in $A$. 
\end{clm}
\begin{proof}
Now we bound the probability that at least one cup in $A$ has
fill at least $\mu_0 + h$. If the emptier does not
\defn{interfere} with an application of $\randalg(k)$, i.e. the
emptier does not ever neglect the anchor set and put more resources
into $B$ than the filler does during the application, then the
application of $\randalg(k)$ achieves a cup with fill at least
$\mu(B) - R_\Delta + d$. By our lower bound
on $\mu(B)$ from \cref{clm:muBgreaterthanminushover2}, namely
$\mu(B) \ge -h/2 + \mu_0$, and because $d\ge 2h$, and $R_\Delta
\le h/2$ due to $\Delta \le h/8$, we have that a successful
donation process donates a cup with fill at least $\mu_0 + h$ to
$A$. If on each of the $n_A$ donation processes the emptier does
not interfere with the final application of $\randalg(k)$, then 
the probability that at least one donation process successfully
donates a cup with fill at least $\mu_0 + h$ to $A$ is at least 
$$1-(1-1/k!)^{n_A} \ge 1-e^{- n\delta/k!}.$$
Now we aim to show that $e^{-n \delta/k!} \le 2^{-\Omega(\log^4 N)}$.
This is equivalent to showing $n\delta/k! \ge \Omega(\log^4 N)$.
Using $k^k$ as an upper bound for $k!$, and recalling that
$n=\log^5 N$, we have that it suffices to show that 
$$k^k \le n\delta / \Omega(\log^4 N) \le \delta O(\log N).$$
Recalling that $\delta = 1/(2k)$ this is equivalent to showing
that
$$k^{k+1} \le O(\log N).$$
Now recalling that 
$$k = \Theta(e^{2h}) = \Theta(e^{\frac{1}{8}\log\log\log N}) =
\Theta((\log\log N)^{1/4}),$$ 
we get 
$$k^{k+1} \le 2^{O((\log\log\log N)(\log\log N)^{1/4})} \le
O(2^{\log\log N}) = O(\log N).$$
Hence we have our desired bound on the probability of getting the
desired backlog in a cup, conditional on the emptier not
interfering with the final application $\randalg(k)$ on each
donation process.

  However, the emptier is allowed to interfere, and if the
  emptier interferes with an application of $\randalg(k)$ then
  the application might not get large fill. In fact, the emptier
  can even interfere conditional on the filler's progress during
  $\randalg(k)$; in particular it could choose to only interfere
  with applications of $\randalg(k)$ that look like they might
  succeed! However, by applying $\randalg(k)$ a random
  number of times in each donation process, chosen from $[M]$,
  where $M = 2N^2 k! 2^{\log^4 N}$, by a Chernoff Bound there are
  at least $N^2 2^{\log^4 N}$ successful applications of
  $\randalg(k)$ in the donation process with incredibly good
  probability: probability at least $1-2^{-2^{\lg^4 N}}.$
  But since the emptier cannot neglect the anchor set more than
  $N^2$ times, the emptier has at best a $2^{-\log^4 N}$ chance
  of interfering with the final application of $\randalg(k)$. 
  Taking a union bound over all donation processes, we have that
  with probability at least $1-2^{\Omega(\log^4 N)}$ the emptier
  does not interfere with the final application of $\randalg(k)$
  on any donation processes. We previously found that the
  probability of achieving the desired backlog in $A$ conditional
  on the final application of $\randalg(k)$ in each donation
  process not being interfered with; multiplying these
  probabilities gives that with probability at least
  $1-2^{-\Omega(\log^4 N)}$ we achieve a cup in $A$ with fill
  at least $\mu_0 + h$.

\end{proof}

We now analyze the running time of the filling strategy. There
are $n_A$ donation-processes. Each donation-process consists of
$O(M)$ applications of $\randalg(k)$, which each take time
$O(1)$, and $O(M)$ applications of $\flatalg$, which each take
$\Theta(N^2)$ time. Thus overall the algorithm takes time $$n_A
\cdot O(M) (O(1) + O(N^2)) = 2^{O(\log^4 N)},$$ as desired.
  
\end{proof}

Now, using \cref{lem:obliviousBase} we show in
\cref{prop:obliviousBase} that an oblivious filler can achieve
constant fill in a known cup. 
\begin{proposition}
  \label{prop:obliviousBase}
  Let $H = \frac{1}{128}\log\log\log N$, let $\Delta \le
  \frac{1}{128}\log\log\log N$, let $n = \log^5 N$. 
  Consider an $R_\Delta$-flat cup configuration in the
  variable-processor cup game on $n$ cups with average fill $\mu_0$.
  There is an oblivious filling strategy that either
  achieves mass $N^2$ among the cups, or achieves fill at least $\mu_0 + H$
  in a chosen cup in running time $2^{O(\log^4 N)}$ against a
  $\Delta$-greedy-like emptier with probability at least
  $1-2^{-\Omega(\log^4 N)}.$
\end{proposition}
\begin{proof}
  The filler starts by using $\rep_\delta(\randalg(k))$ with
  parameter settings as in \cref{lem:obliviousBase};
  note that $h$ from \cref{lem:obliviousBase} is $8H$.

  If this results in mass $N^2$ among the cups then the filler is
  already done. Otherwise, with probability at least
  $1-2^{-\Omega(\log^4 N)}$, there is some cup in $A$ with fill
  $\mu_0 + 8H$. We assume for the rest of the proof that there is
  some cup $c_* \in A$ with $\fil(c_*) \ge \mu_0 + 8H$.

  The filler sets $p=1$, i.e. uses a single processor. Now the
  filler exploits the emptier's greedy-like nature to to get fill
  $H$ in a chosen cup $c_0 \in B$. For $5H$ rounds
  the filler places $1$ unit of fill into $c_0$. Because the
  emptier is $\Delta$-greedy-like it must empty from $c_*$ 
  while $\fil(c_*) > \fil(c_0) + \Delta$. Within $5H$ rounds
  the cups $\fil(c_*)$ cannot decrease below $3H+\mu_0 > H + \Delta + \mu_0$.
  Hence, during these $5H$ rounds, only cups with fills larger
  than $H + \mu_0$ can be emptied from by greediness. 
  The fill of $c_0$ started as at least
  $-4H+\mu_0$ as $\mu(B) \ge -h/2+\mu_0$ from
  \cref{lem:obliviousBase}. After $5H$ rounds
  $c_0$ has fill at least $H+\mu_0$, because the emptier cannot
  have emptied $c_0$ until it attained fill $H+\mu_0$, and if
  $c_0$ is never emptied from then it achieves fill $H+\mu_0$.
  Thus the filling strategy achieves backlog $H+\mu_0 \ge H$
  in $c_0$, a \emph{known} cup, as desired.

  The running time is of course still $2^{O(\log^4 N)}$ by
  \cref{lem:obliviousBase}.
\end{proof}


We remark that \cref{lem:obliviousBase} and
\cref{prop:obliviousBase} both refer to filling strategies in
the regular cup game. In particular, we prove no guarantees on
the behavior of the algorithms in \cref{lem:obliviousBase} and
\cref{prop:obliviousBase} for the $E$-extra-emptyings cup game
with $E > 0$. It is somewhat surprising that we do not need to
provide guarantees such as a guarantee that the cups have some
small bounded fill-range regardless of extra-emptying as we had
to do for $\flatalg$ and $\randalg$.

