\documentclass[twocolumn]{article}[10pt]
\usepackage[left=1in, right=1in, top=1in, bottom=1in]{geometry}
\usepackage[subtle]{savetrees}

\usepackage{amsthm}
\usepackage{amssymb}
\usepackage{amsmath}
\usepackage{mathtools}
\usepackage{hyperref}
\usepackage{xcolor}
\usepackage{xspace}

\newcommand{\defn}[1]{{\textit{\textbf{\boldmath #1}}}\xspace}
\renewcommand{\paragraph}[1]{\vspace{0.09in}\noindent{\bf \boldmath #1.}} 
\DeclareMathOperator{\E}{\mathbb{E}}
\DeclareMathOperator{\Var}{\text{Var}}
\DeclareMathOperator{\img}{Im}
\DeclareMathOperator{\polylog}{\text{polylog}}
\DeclareMathOperator{\poly}{\text{poly}}
\DeclareMathOperator{\st}{\text{ such that }}
\DeclareMathOperator{\tilt}{\text{tilt}}
\DeclareMathOperator{\fil}{\text{fill}}
\newcommand{\norm}[1]{\left\lVert#1\right\rVert}

\newcommand{\contr}[0]{\[ \Rightarrow\!\Leftarrow \]}
\newcommand{\defeq}{\vcentcolon=}
\newcommand{\eqdef}{=\vcentcolon}

\newtheorem{fact}{Fact}
\newtheorem{definition}{Definition}
\newtheorem{remark}{Remark}
\newtheorem{proposition}{Proposition}
\newtheorem{clm}{Claim}
\newtheorem{lemma}{Lemma}
\newtheorem{corollary}{Corollary}
\newtheorem{theorem}{Theorem}
\newtheorem{conjecture}{Conjecture}

\usepackage{authblk}
\usepackage{fancyhdr}
\pagestyle{fancy}
\fancyhead{}
\fancyfoot{}
\fancyfoot[R]{\thepage}
\renewcommand{\headrulewidth}{0pt}

\title{Oblivious Lower Bound via Flattening}
\date{\vspace{-5ex}}

\author[1]{\small William Kuszmaul\thanks{Supported by a Hertz fellowship and a NSF GRFP fellowship}}
\author[2]{\small Alek Westover\thanks{Supported by MIT PRIMES}}

\affil[ ]{\footnotesize MIT\textsuperscript{1}, MIT PRIMES\textsuperscript{2}}
\affil[ ]{\textit{kuszmaul@mit.edu, alek.westover@gmail.com}}

\begin{document}
\maketitle
\section{Oblivious Lower Bound}
An important theorem that we use throughout our analysis is Hoeffding's Inequality:
\begin{theorem}[Hoeffding's Inequality]
  Let $X_i$ for $i=1,2,\ldots, k$ be independent bounded random variables with
  $X_i \in [a,b]$ for all $i$. Then,
  $$P\left(\Big|\frac{1}{k} \sum_{i=1}^k (X_i - \E[X_i])\Big|\ge t\right) \le
  2\exp\left(-\frac{2kt^2}{(b-a)^2}\right) $$
\end{theorem}
Let $S$ be a finite population, let $X_i$ for $i=1,2\ldots, k$ be chosen
uniformly at random from $S \setminus \{X_1,\ldots, X_{i-1}\}$, and let $Y_i$
for $i=1,2,\ldots, k$ be chosen uniformly at random from $S$.
Note that $\{X_1,\ldots, X_k\}$ represents a sample of $S$ chosen without
replacement, whereas $\{Y_1,\ldots, Y_k\}$ represents a sample with
replacement. Note that as the $Y_i$ are independent random variables
Hoeffding's Inequality provides a bound on the probability of $\sum_{i=1}^k
Y_i$ deviating from its mean by more than $t$.

The same bound can be given on the probability of $\sum_{i=1}^k X_i$ deviating
significantly from its mean, because the probability of $\sum_{i=1}^k X_i$
deviating from it's expectation by more than $t$ is at most the probability of
$\sum_{i=1}^k Y_i$ deviating from it's mean by $t$.
Formally we can write this as 
\begin{corollary}
  \label{cor:hoeffdingwreplacement}
  Let $S$ be a finite set with $\min(S) \ge a, \max(S) \le b$, and let $X_i$
  for $i=1,2\ldots, k$ be chosen uniformly at random from $S \setminus
  \{X_1,\ldots, X_{i-1}\}$.
Then 
  $$P\left(\Big|\frac{1}{k} \sum_{i=1}^k (X_i - \E[X_i])\Big|\ge t\right) \le
  2\exp\left(-\frac{2kt^2}{(b-a)^2}\right) $$
\end{corollary}

Hoeffding proved Corollary \ref{cor:hoeffdingwreplacement} in his seminal work
\cite{who62} (the result follows from his Theorem 4, combined with Hoeffding's
Inequality for independent random variables).
The intuition behind Corollary \ref{cor:hoeffdingwreplacement} is that samples
drawn without replacement should be more tightly concentrated around the mean
than samples drawn with replacement.

Another important, yet very trivial, corollary of Hoeffding's Inequality is the
Chernoff Bound (i.e. Hoeffding's Inequality applied to binary random
variables):
\begin{corollary}
  \label{cor:chernoffbound}
  Let $X_i$ for $i=1,2,\ldots, k$ be independent identically distributed binary
  random variables (i.e. $X_i\in \{0,1\}$). Then 

  $$P\left(\Big|\frac{1}{k} \sum_{i=1}^k (X_i - \E[X_i])\Big|\ge t\right) \le
  2\exp\left(-2kt^2 \right) $$
\end{corollary}

We proceed with our analysis of oblivious lower bounds.

Call a cup configuration $T$\defn{-flat} if the fill of every cup is in the
interval $[-T, T]$.

We call an emptier $\Delta$\defn{-greedy-like} if, when there are two cups
$c_1, c_2$ with fills satisfying $\fil(c_1) > \fil(c_2) + \Delta$ the emptier
never empties from $c_2$ without emptying from $c_1$ on the same round.
Intuitively, a $\Delta$-greedy-like emptier has a $\pm \Delta$ range where it
is allowed to ``not be greedy". Note that a perfectly greedy emptier is
$0$-greedy-like. We call an emptier \defn{greedy-like} if it is
$\Delta$-greedy-like for $\Delta \le O(1)$.

In the randomized setting we are only able to prove lower bounds for backlog
against greedy-like emptiers; whether or not our results can be extended to a
more general class of emptiers is an interesting open question. 

We now prove a crucial property of greedy-like emptiers: that they are \defn{flattenable}, i.e.:
\begin{proposition}
  \label{prop:greedylikeisflat}
  Given a cup configuration that is $T$-flat, an oblivious filler can, in
  running time $2T$, achieve a $2(2+\Delta)$-flat configuration of cups against
  a $\Delta$-greedy-like emptier. 
\end{proposition}

\begin{proof}
  The filler sets $p=n/2$ and distributes fill equally amongst
  all cups at every round, in particular placing $1/2$ units of water in each cup.
  Let $\ell_t = \min_{c\in S_t} \fil_{S_t}(c)$, $u_t=\max_{c\in S_t} \fil_{S_t}(c)$. Let
  $L_t$ be the set of cups $c$ with $\fil_{S_t}(c) \le l_t+2+\Delta$, and let
  $U_t$ be the set of cups $c$ with $\fil_{S_t}(c) \ge u_t-2-\Delta$.

  There are two ways to think of $U_t$.
  First we can consider $U_t$ as capturing cups in the union of intervals of length $1$,
  $\Delta$, and $1$. Note the key property that if a cup with fill in
  $[u_t-\Delta-2, u_t-\Delta-1]$ is emptied from, then all cups with fills in 
  $[u_t-1, u_t]$ must be emptied from, because the emptier is $\Delta$-greedy-like.
  On the other hand, we can consider $U_t$ as capturing cups with fill in the union of $[u_t-2, u_t]$ and
  $[u_t-\Delta-2, u_t-2]$. This is useful as the interval of width $\Delta$
  serves as a ``buffer". In particular, if there are more than $n/2$ cups
  outside of $U_t$ then all cups in $[u_t-2, u_t]$ must be emptied from because
  the emptier is $\Delta$-greedy-like. $L_t$ is of course completely symmetric to $U_t$.

  First we prove a key property of the sets $U_t$ and $L_t$: once a cup is in
  $U_t$ or $L_t$ it is always in $U_{t'}, L_{t'}$ for all $t' > t$. This
  follows immediately from the following claim:
  \begin{clm}
    \label{clm:dontlosestuff}
    $$U_{t} \subseteq U_{t+1}, L_t \subseteq L_{t+1}.$$
  \end{clm}
  \begin{proof}
    Consider a cup $c\in U_t$.

    If $c$ is not emptied from, i.e. $\fil(c)$ has increased by $1/2$, then
    clearly $c \in U_{t+1}$, because backlog has increased by at most $1/2$, so
    the fill of $c$ must still be within $2+\Delta$ of the backlog on round $t+1$. 

    On the other hand, if $c$ is emptied from, i.e. $\fil(c)$ has decreased by
    $1/2$, we consider two cases.
    \begin{itemize}
      \item If $\fil_{S_t}(c) \ge u_t-\Delta -1$, then, as $u_{t+1} \le u_t+1/2$, 
        $$\fil_{S_{t+1}}(c) \ge u_t-\Delta-1 - 1/2\ge u_{t+1}-\Delta-2.$$
      \item On the other hand, if $\fil_{S_t}(c) < u_t-\Delta-1$, then every cup
        with fill in $[u_t-1, u_t]$ must have been emptied
        from. The fullest cup at round $t+1$ is the same as the fullest cup on
        round $t$, because the fills of all cups with fill in
        $[u_t-1, u_t]$ have decreased by $1/2$, and no cup with fill less than
        $u_t-1$ had fill increase by more than $1/2$. Hence $u_{t+1} = u_t -1/2$.
        Because both the fill of $c$ and the backlog have decreased by the same
        amount, the distance between them is still at most $\Delta+2$, hence
        $c\in U_{t+1}$.
    \end{itemize}

    The argument for $L_t \subseteq L_{t+1}$ is essentially identical.
  \end{proof}

  Now that we have shown that $L_t$ and $U_t$ never lose cups, we will show
  that they eventually gain a substantial number of cups. 

  \begin{clm}
    \label{clm:smallthenbigger}
    As long as $|U_t| \le n/2$ we have $u_{t+1} = u_t -1/2$. Identically, as
    long as $|L_t| \le n/2$ we have $\ell_{t+1} = \ell_t+ 1/2$.
  \end{clm}
  \begin{proof}
    If there are more than $n/2$ cups outside of $U_t$ then there must be some
    cup with fill less than $u_t-\Delta-2$ that is emptied from. Because the
    emptier is $\Delta$-greedy-like this means that the emptier must empty from
    every cup with fill at least $u_t-2$. Thus $u_{t+1} = u_t -1/2$: no cup
    with fill less than $u_t-2$ could have become the fullest cup, and the
    previous fullest cup has lost $1/2$ units of fill. 

    The proof is identical for $L_t$.
  \end{proof}

  By Claim \ref{clm:smallthenbigger} we see that both $|U_t|$ and $|L_t|$ must
  eventually exceed $n/2$ at some times $t_u, t_\ell \le 2T$, by the assumption
  that the initial configuration is $T$-flat. Since by Claim
  \ref{clm:dontlosestuff} $|U_{t+1}|\ge |U_t|$ and $|L_{t+1}| \ge |L_t|$ we
  have that there is some round $t_0 =\max(t_u, t_\ell) \le 2T$ on which both
  $|U_{t_0}|$ and $|L_{t_0}|$ exceed $n/2$. Then $U_{t_0} \cap L_{t_0} \neq
  \varnothing$. Furthermore, the sets must intersect for all $t_0 \le t \le 2T$. 
  In order for the sets to intersect it must be that the intervals
  $[u_t-2-\Delta, u_t]$ and $[\ell_t, \ell_t+2+\Delta]$ intersect. Hence we have that 
  $$\ell_t+2+\Delta \ge u_t-2-\Delta.$$ Since $u_t \ge 0$ and $\ell_t \le 0$
  this implies that all cups have fill in $[-2(2+\Delta), 2(2+\Delta)]$.

\end{proof}

Given a $\Delta$-greedy-like filler, let $R_\Delta = 2(2+\Delta).$ By
Proposition \ref{prop:greedylikeisflat}, if a filler is given a $T$-flat
configuration of cups they can achieve a $R_\Delta$-flat configuration of cups.

Now we are equipped to prove the following proposition:
\begin{proposition}
  \label{prop:obliviousBase}
  There exists an oblivious filling strategy for the variable-processor cup
  game on $n$ cups -- where $n \ge \Omega(1)$ is sufficiently large -- that,
  given a $T$-flat configuration of cups with $T \le \poly(n)$, can achieve
  backlog $\Omega(\log n)$ in running time $\poly(n)$ against a
  $\Delta$-greedy-like emptier where $\Delta \le O(1)$ is a constant known to
  the filler, with constant probability.
\end{proposition}
\begin{proof}
  The filler starts by flattening all the cups, using the
  flattening procedure detailed in Proposition
  \ref{prop:greedylikeisflat}. 

  Let $A$, the \defn{anchor} set, be a subset of the cups chosen uniformly at
  random from all subsets of size $n/2$ of the cups, and let $B$, the
  \defn{non-anchor} set, consist of the rest of the cups ($|B| = n/2$). Let $h
  = 8\Delta + 8$, and let $h' = 2$. Note that the average fill of $A$ and $B$
  both must start as at least $-R_\Delta$ due to the flattening.

  The filler's strategy is roughly as follows: 
  \begin{itemize}
    \item \textbf{Step 1:} Make a constant fraction of the cups in $A$ have
      fill at least $h$ by playing single processor cup games on constant-size
      subsets of $B$ and then swapping the cup within $B$ that has high fill,
      with constant probability, into $A$. By a Chernoff bound this makes a
      constant fraction of $A$, say $nc$ cups, have fill at least $h$ with
      exponentially good probability. Between single-processor cup games the
      filler flattens $B$.
    \item \textbf{Step 2:} Reduce the number of processors to $nc$, and raise
      the fill of $nc$ \emph{known} cups to fill $h'$. The emptier must first
      empty from the cups with fill $h$ before emptying from the cups that
      the filler is attempting to get fill $h'$ in.
    \item \textbf{Step 3:} Recurse on the $nc$ cups that are known to have fill
      at least $h'$.
  \end{itemize}

By performing $\Omega(\log n)$ levels of recursion, increasing the fill by a
constant amount at each level of recursion, the filler achieves backlog
$\Omega(\log n)$. Say the probability of Step 1 succeeding is at least
$1-e^{-nk}$. Then the probability that any of $(1/2)\log_{1/c} n$ levels of
recursion fail is bounded above by (by the union bound)
$$e^{-nk} + e^{-nck} + e^{-nc^2 k} + \cdots + e^{-nc^{\log_{1/c} \sqrt{n}} k} $$
which is bounded above by 
$$O\left(\frac{\log n}{e^{k\sqrt{n}}}\right)$$
which, for sufficiently large $n$ can clearly be made constant.
Hence the probability that every level of recursion succeeds is at least constant.
Hence, once we show that Step 1 succeeds with the desired probability and that
Step 2 is possible, we have that the entire process successfully achieves
backlog $\Omega(\log n)$ with constant probability.

We now describe how to achieve Step 1.

The filler performs a series of \defn{swapping-process}, which are procedures
that the filler uses to get a new cup in $A$. A swapping-process is composed of
a substructure, repeated many times, which we call a \defn{round-block}; a
round-block is a set of rounds. A swapping-process will consists of $n\cdot
c_\Delta$ round-blocks ($c_\Delta \le O(1)$ a function of $\Delta$ to be
specified); at the beginning of each swapping-process the filler chooses a
round-blocks $j$ uniformly at random from $[n\cdot c_\Delta]$. 

For each round-block $i\in [n\cdot c_\Delta]$, the filler selects a random subset
$D_i\subset B$ of the non-anchor cups and plays a single processor cup game on
$D_i$. In this single-processor cup game the filler essentially employs the
classic adaptive strategy for achieving backlog $\Omega(\log |B|)$ on a set of
$|B|$ cups, with slight modifications for the fact that it is oblivious. In
particular, the filler will only achieve this fill with constant probability.
While doing this, the filler always places $1$ unit of fill in each cup in the
anchor set. Note that the filler sets $p=n/2+1$.

At the end of each round-block the filler applies the flattening procedure to
flatten the non-anchor set. Note that this will not affect the running-time beyond a
multiplicative factor (of, say, $3$). 

On most round-blocks -- all but the $j$-th -- the filler does nothing with the
cup that it achieves with constant probability in its single processor cup
game. However, on the $j$-th round-block the filler swaps the ``winner" of the
single processor cup game into the anchor set (with constant probability there
is a winner).

\begin{clm} \label{clm:reg} 
  With probability at least $1-e^{-O(n)}$, the filler achieves fill
  at least $h$ in at least $nc = O(n)$ of the cups in $A$. 
\end{clm}
\begin{proof}
  Consider a particular swapping-process. Let $j$, the round-block on which the
  filler will perform the swap, be chosen uniformly randomly from $[n\cdot
  c_\Delta]$ ($c_\Delta$ to be determined).
 
  Say the emptier \defn{neglects} the anchor set during a round-block if on at
  least one round of the round-block the emptier does not empty from every cup
  in the anchor set. By playing the single-processor cup game for many
  round-blocks with only one round-block when the filler actually swaps a cup
  into the anchor set, the filler prevents the emptier from neglecting the
  anchor set too often.

  The fill of any cup in the anchor set can clearly never exceed
  $R_\Delta+\Delta$ because $B$ is $R_\Delta$-flat at the start of each
  round-block (a cup with fill this high would necessarily be emptied from).
  Let $\mu_\Delta = 2R_\Delta+\Delta$; the emptier can neglect the anchor set
  no more than $(n/2)\mu_\Delta$ times. 
  % might need to mess with the size of A and B a bit to make this more true...
  Furthermore, the average fill of $B$ is thus always at least $-\mu_\Delta$.
  As $B$ is $R_\Delta$-flat this also means that the fills of cups in $B$ at
  the start of each round-block are at least $-\mu_\Delta-R_\Delta$.

  On each round-block the filler chooses a random subset $D_i \subset B$ of
  $\lceil e^{2h} \rceil$ cups. If the emptier does not neglect the anchor set
  on round-block $i$ then the filler plays a legitimate single-processor cup
  game on $n$ cups. The filler maintains an \defn{active-set} of cups, which is
  a subset of $D_i$ initialized to $D_i$. On each round of the round-block the
  filler distributes $1$ unit of fill equally among all cups in the active set.
  Then the emptier removes fill from some cup in $B$. The filler chooses a
  random cup to remove from the active set. The probability that the cup the
  emptier emptied from is not in the active set after a random cup is removed
  from the active set by the filler is at least constant. By the end of the
  round-block the active-set will consist of a single cup. With constant
  probability, in particular probability at least $$q_0 = 1/\lceil e^{2h}
  \rceil!,$$ this cup has gained fill at least $\ln \lceil e^{2h} \rceil \ge
  2h$. Recalling that the cups fill started as at least $-\mu_\Delta-R_\Delta$,
  we have that this cup now has fill at least $2h-\mu_\Delta-R_\Delta$; by
  design in choosing $h$ this quantity is at least $h$. 

  Now we shall choose $c_\Delta$, choosing it large enough such that with
  constant probability there is some round-block on which the emptier doesn't
  neglect the anchor set on which the filler succeeds.

  We choose $$c_\Delta = 2\frac{1}{q_0}\mu_\Delta.$$ By having $n\cdot
  c_\Delta$ round-blocks, we make it so that there should be at least
  $n\mu_\Delta$ round-blocks on which the filler correctly guesses the
  emptier's emptying sequence. Formally this is due to a Chernoff bound: the
  expectation of the number of rounds when the filler correctly guesses the
  emptier's emptying sequence is at least $2n\mu_\Delta$, and the probability
  that it deviates from its expectation by more than $n\mu_\Delta$ is hence
  exponentially small in $n$. As shown before, the emptier cannot neglect the
  anchor set more than $(n/2)\mu_\Delta$ times. The filler correctly guesses
  the emptiers emptying sequence on the $j$-th round-block. Conditioned on this
  event, the $j$ is chosen uniformly randomly from all the round-blocks on
  which the filler correctly guesses the emptiers emptying sequence. Since the
  emptier can neglect the anchor set on at most half of these round-blocks
  there is at least a $1/2$ chance that $j$ is chosen on a round-block where
  the filler does not neglect the anchor set. Thus, overall, there is at least a
  constant probability of 

  Say that a swapping-process \defn{succeeds} if the filler is able to swap a
  cup with fill at least $h$ into $A$. We have shown that there is a constant
  probability of a given swapping-process succeeding. Let $X_i$ be the binary
  random variable indicating whether or not the $i$-th swapping process
  succeeds. Let $q \ge \Omega(1)$ be the probability of a swapping-process
  succeeding, i.e. $P(X_i=1)$. Note that the random variables $X_i$ are clearly
  independent, and identically distributed.

  Clearly $$\E\left[\sum_{i=1}^{n/4} X_i\right] = qn/4.$$ 
  By a Chernoff Bound (i.e. Hoeffding's Inequality applied to binary random variables),
  $$P\left(\sum_{i=1}^{n/4} X_i\le nq/8\right) \le e^{-nq^2/128}.$$ That is, the
  probability that less than $nq/8$ of the anchor cups have fill at least $h$ is
  exponentially small in $n$, as desired.

\end{proof}

Hence Step 1 is possible.

Step 2 is easily achieved by setting $p=nc$ and uniformly distributing the
fillers fill among a chosen set of $nc$ cups. The greedy nature of the emptier
will force it to focus on the cups which must exist in $A$ with large positive
fill until the new cups have sufficiently high fill. In particular, the fills
of the cups in $nc$ must be at least $-\mu_\Delta-\Delta \ge -h$. After
removing from the very full cups for $h+2$ rounds the fills of these new cups
are clearly at least $2$.

\end{proof}

\begin{lemma}[The Oblivious Amplification Lemma]
   \label{lem:obliviousAmplification}
  Let $f$ be an oblivious filling strategy that achieves backlog $f(n)$ in the
  variable-processor cup game on $n$ cups with constant probability (relative
  to average fill, with negative fill allowed). Let $\delta \in (0,1)$ be a
  parameter. Then, there exists an adaptive filling strategy that, with
  constant probability, either achieves backlog $$f'(n) \ge
  (1-\delta)\Big(f((1-\delta)n) + f((1-\delta)\delta n)\Big)$$ or achieves
  backlog $\Omega(\poly(n))$ in the variable processor cup game on $n$
  cups.
\end{lemma}
\begin{proof}
  
\end{proof}

\begin{corollary}
  \label{cor:obliviousPoly}
  There is an oblivious filling strategy for the variable-processor cup game on
  $n$ cups that achieves backlog at least $2^{\Omega(\sqrt{\log n})}$ in
  running time $O(n)$ with constant probability.
\end{corollary}
\begin{proof}
  Given the Oblivious Amplification Lemma we could try to apply the same
  strategy as outlined in the proof of Corollary \ref{cor:adaptivePoly} 
  to achieve backlog $\Omega(n^{1-\epsilon})$ for constant $\epsilon >0$ in time $2^{O(\log^2 n)}$.
  Because of our definition of an overpowered cup as a cup with fill at least
  $\tilde{\Omega}(\sqrt{n})$, we can't get quite as close to linear as an
  adaptive filer could. However, the more pressing problem is that of running
  time: randomized algorithms are traditionally supposed to have polynomial-running time.
  By artificially reducing $n$, i.e. ignoring some portion of the cups, we can
  get an algorithm that achieves high backlog, but in polynomial time.
  In particular, we want to choose a subset of $n'$ of the cups to focus on,
  where $2^{O(\log^2 n')} = O(n)$. An appropriate choice is $n' = 2^{\sqrt{\log n}}$.

  With $n'$ chosen, we apply the exact same strategy as given in the paragraph
  in the proof of Corollary on the $2^{O(\log^2 n)}$-time construction for
  achieving backlog $\Omega(n^{1-\epsilon})$ for constant $\epsilon >0$, but using
  repeated application of the Oblivious Amplification Lemma rather than the
  Adaptive Amplification Lemma, which yields the disclaimer that the backlog is
  only achieved with constant probability.
  Thus, we achieve backlog $\Omega(n')$ in running time $2^{O(\log^2
  n')}$. By design, expressing this in terms of $n$ the filler achieves 
  backlog $2^{\Omega(\sqrt{\log n})}$ in running time $O(n)$.

  For completeness we -- briefly (as they are nearly identical) -- present the
  ideas from Proposition \ref{prop:constructive_nepsil} in the randomized
  context.

  Fix constant $\epsilon > 0$ and choose appropriate constant $\delta$, as
  mandated by Claim \ref{clm:validchoices}. Choose constant $c$, according to
  an inequality to be specified later. We aim to achieve backlog
  $c(n')^{1-\epsilon}$.
  As before, we define a sequence of functions. First,
  $$f_0(k) = 
  \begin{cases} 
    \lg k, & k\geq 1, \\
    0 & \text{else.}
  \end{cases}$$
  Then, we define $f_i$ as the amplification of $f_{i-1}$ for $i \ge 1$, where
  amplification is as defined in the Oblivious Amplification Lemma. 
  However, this time the filling strategy $f_i$ achieves backlog $f_i(k)$ on
  $k$ cups only with constant probability.
  As before we define a sequence $g_i$ as 
  $$ g_i = \begin{cases}
    \lceil 1/\delta \rceil \gg 1,  & i = 0,\\
    \lceil g_{i-1}/(1-\delta)\rceil -1 & i  \ge 1.
  \end{cases} $$
  Claim \ref{clm:fikinduction}, which states that 
  $$f_i(k) \ge ck^{1-\epsilon} \text{ for all } k < g_i,$$
  holds with no modifications required.

  Again, because $g_i$ is increasing, we achieve the desired backlog
  $c(n')^{1-\epsilon}$ in finite time. In particular, applying identical
  arguments to those in Corollary \ref{cor:adaptivePoly}, we find that the
  running time is $2^{O(\log^2 n')}$.

  As stated earlier, by design of $n'$, this means we get backlog
  $2^{\Omega(\sqrt{\log n})}$ in time $O(n)$.

\end{proof}



% this probably belongs somewhere in the corollary proof, which will be substantially more complex than I orignally anticipated, becasue Prop and Lemma need to take as input cups arrangements with flattness guarantees
\begin{clm}
  Without loss of generality the cup state is always $(h\sqrt{n/\lg\lg n})$-flat.
\end{clm}
\begin{proof}
In order to flatten a set of cups we must have a bound on the magnitude of the
fills of the cups. We claim that without loss of generality no cup has fill
larger in magnitude than $h\sqrt{n/\log\log n}$. If a cup has more than
$h\sqrt{n/\log\log n}$ fill we call the cup \defn{overpowered}. If there ever
is an overpowered cup then we are automatically done: the emptier has achieved
$\poly(n)$ backlog and will maintain it for at least $\poly(n)$ rounds. If a
cup ever has fill less than $-h\sqrt{n/\log\log n}$ then the absolute average
fill must be large enough such that the absolute fill of this cup is at least
$0$. Thus there is an overpowered cup. From now on we assume that there are no
overpowered cups.
\end{proof}

\end{document}
