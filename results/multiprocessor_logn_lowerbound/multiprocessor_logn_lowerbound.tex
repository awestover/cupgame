\documentclass[twocolumn,11pt]{article}
% \usepackage[subtle]{savetrees}
\usepackage[left=1in, right=1in, top=1in, bottom=1in]{geometry}

\usepackage{amsthm}
\usepackage{amssymb}
\usepackage{amsmath}
\usepackage{mathtools}
\usepackage{hyperref}

\newtheorem{fact}{Fact}
\newtheorem{definition}{Definition}
\newtheorem{remark}{Remark}
\newtheorem{proposition}{Proposition}
\newtheorem{lemma}{Lemma}
\newtheorem{clm}{Claim}
\newtheorem{corollary}{Corollary}
\newtheorem{theorem}{Theorem}

\author{William Kuszmaul\textsuperscript{1}, Alek
Westover\textsuperscript{1}\\ 
\texttt{kuszmaul@mit.edu}, \texttt{alek.westover@gmail.com} }
\title{An Adaptive Filling Strategy for backlog
$\Omega(\log \frac{p}{n-p})$ \\ in the $p$-processor cup game on $n$ cups}
\date{\textsuperscript{1}Massachusetts Institute of Technology}

\begin{document}
\maketitle
\begin{abstract}
  The $p$-processors cup game on $n$ tasks is an extensively
  studied model for the problem of scheduling tasks so that no
  task ever gets too far behind. The game consists of an emptier
  and a filler adding and removing at most $p$ units of water
  from a set of $n$ cups each turn according to certain
  constraints. The emptier aims to minimize the backlog, or fill
  in the largest.

  For $p=1$ it is relatively straightforward to show that
  $\Theta(\log n)$ is tight for backlog: a greedy strategy
  achieves backlog $O(\log n)$ and there is a lower bound of
  $\Omega(\log n)$.
  Recently, in a much more complicated analysis
  Kuszmaul\cite{ku20}
  established that a greedy emptier achieves backlog $O(\log n)$
  even for $p>1$. Kuszmaul also showed a $\Omega(\log (n-p))$
  lower bound.

  It was conjectured that for large $p$ $\Theta(\log (n-p))$ is
  tight. However, in this work we present a filling strategy for
  achieving backlog $\Omega(\log \frac{p}{n-p})$ for $p> n-\sqrt{n}$.
  Combined with Kuszmaul's work, this implies that $\Theta(\log
  n)$ is the optimal backlog for the multi-processor cup game.

  % The problem of scheduling tasks on $p$ processors so that no
  % task ever gets too far behind is often described as a game with
  % cups and water. In the $p$-processor cup game on $n$ cups,
  % there are two players, a filler and an emptier, that take turns
  % adding and removing water from a set of $n$ cups. In each turn,
  % the filler adds $p$ units of water to the cups, placing at most
  % $1$ unit of water in each cup, and then the emptier selects $p$
  % cups to remove up to $1$ unit of water from; note that if a cup
  % with fill less than $1$ is emptied from, its fill becomes $0$
  % (not negative), and the average fill of the cups increases. The
  % emptier's goal is to minimize the backlog, which is the height
  % of the fullest cup.

\end{abstract}

\section{Introduction}
The $p$-processor cup game on $n$ cups is a multi-round game with
two players, the \textbf{filler} and the \textbf{emptier}, that
take turns adding and removing water from the cups. In
particular, on each round the filler distributes $p$ units of
fill arbitrarily amongst the cups, subject only to the
restriction that the filler cannot place more than $1$ unit of
fill into any given cup. Next, the emptier chooses $p$ cups and
removes at most $1$ unit of water from each of these. The
\textbf{mass} of the cups is the sum the fills of all of the
cups. The mass will increase when the emptier removes water from
a cup with less than $1$ unit of fill, because cups must have
non-negative fills. The \textbf{backlog} is the height of the
fullest cup; the emptier aims to minimize backlog while the
filler aims to maximize backlog.

Kuszmaul\cite{ku20} showed that in the $p$-processor cup game on $n$ cups
the greedy emptying strategy---i.e. empty from the $p$ fullest
cups---never lets backlog exceed $O(\log n)$. Kuszmaul also
provided a filling strategy that achieves backlog $\Omega(\log
(n-p))$. For small $p$, e.g. $p \le n/2$, the upper-bound and
lower-bound match, but for large $p$ the optimal backlog has
remained an open question. It was conjectured that the $O(\log
n)$ upper bound could be reduced to $O(\log(n-p))$ even for large
$p$; however, we show that this is not the case.  

We remark that we consider only \textbf{adaptive} fillers, i.e. fillers
that can observe the state of the cups. The emptier can achieve
better bounds on backlog by using randomization, in which case
the emptier is not allowed to view the state of the cups, and is
called \textbf{oblivious}.

\section{Lower Bound on Backlog}
\begin{theorem}
  There is a filling strategy for the $p$-processor cup game on
  $n$ cups that achieves backlog $\Omega(\log n)$ against any emptier.
\end{theorem}
\begin{proof}
Kuszmaul's construction shows that there is a filling strategy
that achieves backlog $\Omega(\log (n-p))$.  If $p \le n -
\sqrt{n}$, then $n-p \ge \sqrt{n}$, so $\log (n-p) \ge
\frac{1}{2}\log (n)$, hence $\Omega(\log(n-p)) =
\Omega(\log(n))$. Thus the result holds for $p \le n - \sqrt{n}$;
we proceed to consider the case where $p > n-\sqrt{n}$.

We give a filling strategy, that we call \textbf{IncMass}, with
running time $O(p/(n-p))$, that increases the mass of the cups in
$n$ rounds by at least $1/2$, provided that backlog starts below
$O(\log n)$. The filler's strategy for achieving backlog
$\Omega(\log n)$ is to repeatedly use IncMass as many as
$\Theta(n\log n)$ times, terminating if backlog $\Omega(\log n)$
is ever achieved (after which the filler changes to the strategy
of placing a unit of water in the $p$ fullest cups on each
round). The filler's strategy either terminates before finishing,
which only happens if it has achieved backlog $\Omega(\log n)$,
or does not terminate early, in which case the mass of the cups
has increased by $\Omega(n\log n)$, which means
that average fill and also backlog have increased by at least
$\Omega(\log n)$. We now describe IncMass.

Let $S$ denote the set of cups. The filler will maintain a set $U
\subset S$ throughout the algorithm. The algorithm's procedure
will ensure that once a cup enters $U$ its fill never decreases
for the rest of the process (by placing $1$ unit of water in such
a cup each round). Furthermore, $|U|$ will increase by $n-p$ at
each iteration of the process. $U$ is initialized to $\varnothing$.
For each of $\lfloor(p+1) / (n-p)\rfloor$ steps the filler will:
\begin{enumerate}
  \item Distribute $p - |U|$ water equally among the cups in
    $S\setminus U$; each cup receives
    $\frac{p-|U|}{n-|U|}$ fill
  \item Distribute $|U|$ water equally among the cups in $U$;
    each cup receives $1$ fill
\end{enumerate}

Then the emptier must chose $p$ cups to empty from, and hence
$n-p$ cups to \textbf{neglect}, i.e. not empty from. Let $N$ be
the set of neglected cups on this step, with $|N| = n-p$. The
emptier adds all cups in $N \setminus U$ to $U$, and then
adds the $(n-p) - |N\setminus U |$ fullest cups in $S
\setminus U$ to $U$. 

The number of cups in $S\setminus U$ decreases by $n-p$ on each
of the $\lfloor (p+1) / (n-p) \rfloor$ steps, so in the end we
have $|S\setminus U| \ge p+1$.

\begin{clm}
  Any cup $c \in S\setminus U$ at the end of this process has $0$
  fill, provided that the backlog started as at most $O(\log n)$.
\end{clm}
\begin{proof}
On the $i$-th step of the filler's process the fill of a
cup in $S \setminus U$ increases by $(p-|U|)/(n-|U|)$, and then
decreases by $1$ on this round, resulting in a net change of
$$-\frac{n-p}{n-(n-p)\cdot i}.$$
The total amount that the fill of $c$ has changed since
the start of the filler's process is simply the sum of the
amounts that the fill changes on each step, which is
\begin{equation}
  \label{eq:interesting}
\sum_{i=0}^{\big\lfloor \frac{p+1}{n-p}\big\rfloor - 1} \frac{n-p}{n-(n-p)\cdot i}.
\end{equation}
We aim to show that \eqref{eq:interesting} is $\Omega(\log n)$; combined
with the fact that the backlog started as $O(\log n)$ this will
imply that $c$ has fill $0$.

For $p=n-1$ \eqref{eq:interesting} is simply the difference of
harmonic numbers; this prompts the idea of lower bounding
\eqref{eq:interesting} by a difference of harmonic
numbers\footnote{ Let $H_n = 1/1+1/2+\cdots +1/n = \ln n +
\Theta(1)$ denote the $n$-th harmonic number.}.
We lower bound the $i$-th term in the sum by:
$$\frac{n-p}{n-i(n-p)} = \sum_{j=0}^{n-p-1} \frac{1}{n-i(n-p)} \ge \sum_{j=n-i(n-p)}^{n-(i-1)(n-p)-1}\frac{1}{j}.$$
Applying this to \eqref{eq:interesting} we get a difference of
harmonic numbers:
$${\sum_{i=0}^{\big\lfloor\frac{p+1}{n-p}\big\rfloor -
1}}\sum_{j=n-i(n-p)}^{n-(i-1)(n-p)-1} \frac{1}{j}.$$
When $i=0,
j=n-(i-1)(n-p)-1$ we get the smallest term in the sum:
$$\frac{1}{n+(n-p-1)}.$$ 
When $i = \lfloor(p+1) / (n-p)\rfloor - 1, j=n-i(n-p)$ we get the largest
term in the sum, which is at least 
$$\frac{1}{n - \left(\frac{p+1}{n-p} -1\right) (n-p)} =
\frac{1}{2(n-p)-1}.$$
Thus, $c$ has lost fill at least 
$$H_{2n-p-1} - H_{2(n-p-1)} \ge \Omega\left(\log \frac{p}{n-p}\right).$$
Because $p > n-\sqrt{n}$ we have that $c$ has lost fill at least
$$\Omega\left(\log \frac{n-\sqrt{n}}{\sqrt{n}}\right) \ge
\Omega(\log n),$$
as desired. Because the backlog, and hence fill of $c$, started
less than $O(\log n)$ by assumption, $c$ must
now have $0$ fill.
\end{proof}

There are now (at least) $p+1$ cups with fill $0$. The final step
of the filler's procedure is to add $1/(p+1)$ fill to $p+1$ cups with
fill $0$. 
The emptier now must waste resources on emptying one of these
cups with fill below $1$. In total on this round the filler adds
$p$ units of fill to the cups, and then the emptier removes at
most $p-1 + 1/(p+1) \le p-1/2$ water to the cups. Hence the mass
increases by at least $1/2$. As mass is monotonically increasing,
this increase of $\frac{1}{2}$ in mass persists, as desired.

We have shown that IncMass can increase the mass given that
backlog starts below $O(\log n)$. As shown previously, by
repeatedly using IncMass we can achieve backlog $O(\log n)$.

We remark that the running time of IncMass is $O(p/(n-p))$ 
which satisfies
$$\sqrt{n} \le \frac{p}{n-p} \le n.$$
The running time of the full strategy is 
$$O\left(\frac{p}{n-p} n\log n\right) \le O(n^2 \log n).$$
\end{proof}

\section{Conclusion}
Our analysis shows that $\Theta(\log n)$ is a tight bound on
backlog in the $p$-processor cup game on $n$ cups with an
adaptive filler. Finding tight bounds on backlog for an oblivious
filler remains an open question. Currently there is a lower bound
of $\Omega(\log \log n)$, and an upper bound of $O(\log \log n +
\log p)$.

\bibliographystyle{plain}
\bibliography{bib}

\end{document}
