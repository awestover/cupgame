\documentclass[twocolumn]{article}[10pt]
\usepackage[left=1in, right=1in, top=1in, bottom=1in]{geometry}
\usepackage[subtle]{savetrees}

\usepackage{amsthm}
\usepackage{amssymb}
\usepackage{amsmath}
\usepackage{mathtools}
\usepackage{hyperref}
\usepackage{xcolor}
\usepackage{xspace}

\newcommand{\defn}[1]{{\textit{\textbf{\boldmath #1}}}\xspace}
\renewcommand{\paragraph}[1]{\vspace{0.09in}\noindent{\bf \boldmath #1.}} 
\DeclareMathOperator{\E}{\mathbb{E}}
\DeclareMathOperator{\Var}{\text{Var}}
\DeclareMathOperator{\img}{Im}
\DeclareMathOperator{\polylog}{\text{polylog}}
\DeclareMathOperator{\poly}{\text{poly}}
\DeclareMathOperator{\st}{\text{ such that }}
\DeclareMathOperator{\tilt}{\text{tilt}}
\DeclareMathOperator{\fil}{\text{fill}}
\newcommand{\norm}[1]{\left\lVert#1\right\rVert}

\newcommand{\contr}[0]{\[ \Rightarrow\!\Leftarrow \]}
\newcommand{\defeq}{\vcentcolon=}
\newcommand{\eqdef}{=\vcentcolon}

\newtheorem{fact}{Fact}
\newtheorem{definition}{Definition}
\newtheorem{remark}{Remark}
\newtheorem{proposition}{Proposition}
\newtheorem{clm}{Claim}
\newtheorem{lemma}{Lemma}
\newtheorem{corollary}{Corollary}
\newtheorem{theorem}{Theorem}
\newtheorem{conjecture}{Conjecture}

\usepackage{authblk}
\usepackage{fancyhdr}
\pagestyle{fancy}
\fancyhead{}
\fancyfoot{}
\fancyfoot[R]{\thepage}
\renewcommand{\headrulewidth}{0pt}

\title{Oblivious Lower Bound: this time its for real}
\date{\vspace{-5ex}}

\author[1]{\small William Kuszmaul\thanks{Supported by a Hertz fellowship and a NSF GRFP fellowship}}
\author[2]{\small Alek Westover\thanks{Supported by MIT PRIMES}}

\affil[ ]{\footnotesize MIT\textsuperscript{1}, MIT PRIMES\textsuperscript{2}}
\affil[ ]{\textit{kuszmaul@mit.edu, alek.westover@gmail.com}}

\begin{document}
\maketitle

We call an emptier $\Delta$\defn{-greedy-like} if, when there are two cups $c_1, c_2$ with
fills satisfying $\fil(c_1) > \fil(c_2) + \Delta$ the emptier never
empties from $c_2$ without emptying from $c_1$ on the same round. 
Intuitively, a $\Delta$-greedy-like emptier has a $\pm \Delta$ range within it is
allowed to ``not be greedy". Note that a perfectly greedy emptier is $0$-greedy-like.
Greedy and greedy-like emptiers are of great interest.

In the randomized setting we are only able to prove lower bounds for backlog
against $\Delta$-greedy-like emptiers, for $\Delta \le O(1)$; whether or not
our results can be extended to a more general class of emptiers is an
interesting open question. 

Let the \defn{anti-backlog} of a set $S$ of cups be $-\min_{c\in S} \fil(c)$;
note that anti-backlog is non-negative, and is the absolute value of the fill
of the cup with lowest fill. 

Let $R_\Delta = 2 + \Delta$. We say that a cup configuration is
$\Delta$-\defn{smooth} if all cups have fill in $[-2R_\Delta-\Delta,
2R_\Delta+\Delta]$. 

First we prove a key property of greedy-like emptiers.
\begin{proposition}
  Given a cup configuration with backlog and anti-backlog both at most $T$, an
  oblivious filler can, in running time $2T$, achieve a $\Delta$-smooth
  configuration of cups against a $\Delta$-greedy-like emptier.
\end{proposition}

\begin{proof}
  Let $\ell_t$ be anti-backlog on round $t$, $u_t$ be backlog on round $t$. Let
  $L_t$ be the set of cups on round $t$ with fill in $[l_t, l_t+R_\Delta]$, and let
  $U_t$ be the set of cups on round $t$ with fill in $[u_t-R_\Delta, u_t]$.

  It is useful to think of $U_t$ as the union of three disjoint intervals: an
  interval of length $1$, followed by a ``buffer" interval of length $\Delta$,
  followed by another interval of length $1$. Note the key property that if a
  cup with fill in $[u_t-R_\Delta, u_t-R_\Delta+1]$ is emptied from, then all
  cups with fill in $[u_t-1, u_t]$ must also be emptied from, because of the
  ``buffer" interval. The symmetric property holds for $L_t$.

  \begin{clm}
    Because the emptier is $\Delta$-greedy-like we have
    $$U_{t} \subseteq U_{t+1}, L_t \subseteq L_{t+1}.$$
  \end{clm}
  \begin{proof}
    Consider a cup $c\in U_t$.

    If $\fil_{t+1}(c) = \fil_t(c)+1/2$, then clearly $c \in U_{t+1}$, because
    $u_{t+1}\le u_t+1/2$, so $\fil_{t+1}(c)\ge u_{t+1} -R_\Delta$.

    On the other hand, if $\fil_{t+1}(c) = \fil_t(c) - 1/2$, we consider two cases.
    \begin{itemize}
      \item If $\fil_t(c) \ge u_t-R_\Delta +1$, then $$\fil_{t+1}(c) \ge
        u_t-R_\Delta+1/2\ge u_{t+1}-R_\Delta.$$
      \item On the other hand, if $\fil_t(c) < u_t-R_\Delta+1$, then every cup
        with fill in $[u_t-1, u_t]$ must have been emptied
        from. The fullest cup at round $t+1$ is the same as the fullest cup on
        round $t$, because the fills of all cups with fill in
        $[u_t-1, u_t]$ have decreased by $1/2$, and no cup with fill less than
        $u_t-1$ had fill increase by more than $1/2$. Hence $u_{t+1} = u_t -1/2$.
        Thus we again have $\fil_{t+1}(c) \ge u_{t+1}-R_\Delta$.
    \end{itemize}

    The argument for $L_t \subseteq L_{t+1}$ is essentially identical.
  \end{proof}

  {\color{red} (red because this is the pure greedy argument)
  Note that so long as $|U_t| \le n/2$ for all $c\in U_t$ we have
  $\fil_{t+1}(c) = \fil_t(c) - 1/2$. Combined with the fact that no cups ever
  exit $U_t$ after entering, and the restriction on the initial backlog, this
  implies that there is some $t_U \le 2T$, for which for $t\ge t_U$, $|U_t| >
  n/2$. By identical reasoning, there is a time $t_L \le 2T$, for which for $t
  \ge t_L$, $|L_t| > n/2$. Let $t_0 = \max(t_L, t_U)$. For times $t \ge t_0$ we
  must have $U_t \cap L_t \neq \varnothing$. Let a cup $c \in L_t \cap U_t$. We
  have $l_t \le \fil_t(c) \le l_t+1$ and $u_t-1 \le \fil_t(c) \le u_t$.
  Combined, this implies that $u_t - l_t \le 2$. Hence we have the fills of all
  the cups are in $[-2, 2]$.
}

\end{proof}


% First we establish the special case where the filler is purely greedy (i.e.
% $0$-greedy-like). The extension to the general case will be easy.
% \begin{proof}
%   Let $l_t$ be anti-backlog on round $t$, $u_t$ be backlog on round $t$. Let
%   $L_t$ be the set of cups on round $t$ with fill in $[l_t, l_t+1]$, and let
%   $U_t$ be the set of cups on round $t$ with fill in $[u_t-1, u_t]$.

%   We claim that $U_t \subset U_{t+1}$ because the emptier is greedy.
%   Consider a cup $c\in U_t$.
%   \begin{itemize}
%     \item If $c$ is one of the $n/2$ fullest cups, then $\fil_{t+1}(c) =
%       \fil_t(c) -1/2$, and $u_{t+1} \le \max(u_t-1/2, \fil(c)+1/2)$. We know
%       that $\fil_{t+1}(c) \ge u_t -1 -1/2,$ and $\fil_{t+1}(c) \ge \fil_t(c) +
%       1/2 -1$ so in either case of the $\max$ we have $c \in U_{t+1}$.
%     \item On the other hand, if $c$ isn't one of the $n/2$ fullest cups, then
%       $\fil_{t+1}(c) = \fil_t(c) + 1/2 \in [u_t-1/2, u_t+1/2]$, and $u_{t+1}\le
%       u_t + 1/2 $ so also $c \in U_{t+1}$.
%   \end{itemize}

%   Hence we get that\footnote{The reasoning for $L_t$ is identical to the reasoning for $U_t$}
%   $$U_{t+1} \subset U_t, \,\, L_{t+1} \subset L_t.$$

%   Note that so long as $|U_t| \le n/2$ for all $c\in U_t$ we have
%   $\fil_{t+1}(c) = \fil_t(c) - 1/2$. Combined with the fact that no cups ever
%   exit $U_t$ after entering, and the restriction on the initial backlog, this
%   implies that there is some $t_U \le 2T$, for which for $t\ge t_U$, $|U_t| >
%   n/2$. By identical reasoning, there is a time $t_L \le 2T$, for which for $t
%   \ge t_L$, $|L_t| > n/2$. Let $t_0 = \max(t_L, t_U)$. For times $t \ge t_0$ we
%   must have $U_t \cap L_t \neq \varnothing$. Let a cup $c \in L_t \cap U_t$. We
%   have $l_t \le \fil_t(c) \le l_t+1$ and $u_t-1 \le \fil_t(c) \le u_t$.
%   Combined, this implies that $u_t - l_t \le 2$. Hence we have the fills of all
%   the cups are in $[-2, 2]$.

% \end{proof}

% \begin{proof}
%   The filler sets $p=n/2$, and equally distributes fill among the cups.
%   As long as there is some cup with fill above $\Delta+1$ the emptier must
%   empty from a cup with fill at least $1$. Thus, until all cups have fill below
%   $\Delta+1$ the positive tilt of the set of cups decreases by $n/2$ every round.
%   However, the positive tilt also increases, by at most $(n-1)/2$ (which is
%   achieved if there is a single cup with negative fill). The net change in
%   positive tilt is thus a decrease of at least $1/2$.
%   It is impossible for the positive tilt to decrease by $1/2$ for more than
%   $2T$ rounds, hence by some round $i \le 2T$ all cups will have fill at most
%   $\Delta+1$.

%   Furthermore the anti-backlog cannot have started any higher than $T$, and the
%   fill of negative cups increases by $1/2$ each time.

%   {\color{red}I'm pretty close here. More importantly, the proposition is true.
%   At least so I believe. Maybe need to change the range to be a bit wider.}

% \end{proof}

\begin{proposition}
  \label{prop:obliviousBase}
  There exists an oblivious filling strategy in the variable-processor cup game
  on $n$ cups that achieves backlog $\Omega(\log n)$ against a
  $\Delta$-greedy-like emptier (where $\Delta \le O(1)$ is a constant known
  to the filler), with constant probability.
\end{proposition}
\begin{proof}
    Let $A$, the \defn{anchor} set, be a subset of the cups chosen uniformly at
  random from all subsets of size $n/2$ of the cups, and let $B$, the
  \defn{non-anchor} set, consist of the rest of the cups ($|B| = n/2$). Let $h
  = 8\Delta + 8$, and let $h' = 2$. Our strategy is roughly as follows: 
  \begin{itemize}
    \item \textbf{Step 1:} Make a constant fraction of cups in $A$ have fill at
      least $h$ by playing single processor cup games on constant-size subsets
      of $B$. With constant probability we can attain a cup in $B$ with
      constant fill by this method, that we then swap into $A$. By a Chernoff
      bound we get a constant fraction of $A$, say $cn$ cups, to have fill at
      least $h$ with exponentially good probability.
    \item \textbf{Step 2:} Reduce the number of processors to $cn$, and raise
      the fill of $cn$ \emph{known} cups to fill $h'$. 
    \item \textbf{Step 3:} Recurse on the $nc$ cups that are known to have fill
      at least $h'$.
  \end{itemize}

By performing $\Omega(\log n)$ levels of recursion, achieving constant backlog
$h'$ at each step (relative to the average fills), the filler achieves backlog
$\Omega(\log n)$.

We now provide a detailed description of the filling algorithm, to prove that
the results claimed in Step 1 and Step 2 are attainable.

To attain step 2, we smooth the non-anchor set.

Now we detail how to achieve Step 1.

We perform a series of \defn{swapping-process}, which are procedures that we
use to get a new cup in $A$. A swapping-process is composed of a substructure,
repeated many times, which we call a \defn{round-block}; a round-block is a set
of rounds. At the beginning of a swapping-process we choose a round-block $j
\in [n^2]$ uniformly at random from all the round-blocks. The swapping-process
proceeds for $n^2$ round-blocks; on the $j$-th round-block we swap a cup into
the anchor set.

On each of the $n^2$ round-blocks, the filler selects a random subset $C\subset
B$ of the non-anchor cups and plays a single processor cup game on $C$. In this
single-processor cup game the filler employs the classic adaptive strategy for
achieving backlog $\Omega(\log |B|)$ on a set of $|B|$ cups, however modified
because it is an oblivious filler. In particular, the filler's strategy in the
single-processor cup games is to distribute water equally among an \defn{active
set} of cups, and then after the emptier removes water from some cup the filler
removes a random cup from the active set. There is at least constant
probability that this results in the active set having a single cup at the end,
with fill that has increased by at least $1/|B| + 1/(|B|-1) + \ldots + 1/1 \ge
\ln |B|$ since the start of the round-block.

On most round-blocks -- all but the $j$-th -- the filler does nothing with the
cup that it achieves in the active set at the end of the single processor cup
game. However, on the $j$-th round-block the filler swaps the winner of the
single processor cup game into the anchor set.

\begin{clm} \label{clm:reg} 
  Let $q\ge \Omega(1)$ be an appropriately small constant ($q$ is a function of
  $h\le O(1)$). In Case 1, with probability at least $1-e^{-nq^2/1024}$, we
  achieve fill at least $h$ in at least $nq/16$ of the cups in $A$ (i.e. a
  constant fraction of the cups in $A$). In particular, this implies that we
  achieve positive tilt $hnq/16 \ge \Omega(n)$ in $A$.
\end{clm}
\begin{proof}
  Consider a swapping-process where the filler does not perform a
  storing-operation where at least $1/2$ of the cups $c \in B$ have $\fil(c)
  \ge -h$. Note that by assumption there are at least $n/4 - 3/2 \gamma$ such rounds.
 
  Say the emptier \defn{neglects} the anchor set in a round-block if on at
  least one round of the round-block the emptier does not empty from every
  anchor cup. By playing the single-processor cup game for $n^2$ round-blocks,
  with only one round-block when we actually swap a cup into the anchor set, we
  strongly disincentive the emptier from neglecting the anchor set on more
  than a constant fraction of the round-blocks. 

  The emptier must have some binary function, $I(i)$ that indicates whether or
  not they will neglect the anchor set on round-block $i$ if the filler has not
  already swapped. Note that the emptier will know when the filler perform a
  swap, so whether or not the emptier neglects a round-block $i$ depends on
  this information. However, $j$ is the only parameter of the swapping-process,
  so there is no other information that the emptier can use to decide whether
  or not to neglect a round-block, because on any round-block when we simply
  redistribute water amongst the non-anchor cups we effectively have not
  changed anything about the game state. 

  If the emptier is willing to neglect the anchor set for at least $1/2$ of the
  round-blocks, i.e. $\sum_{i=1}^{n^2} I(i) \ge n^2 / 2$, then with probability
  at least $1/4$, $j \in ((3/4) n^2, n^2)$, in which case the emptier neglects
  the anchor set on at least $n^2/4$ round-blocks ($I(k)$ must be $1$ for at
  least $n^2/4$ of the first $(3/4)n^2$ round-blocks). Each time the emptier
  neglects the anchor set the mass of the anchor set increases by at least $1$.
  Thus the average fill of the anchor set will have increased by at least
  $(n^2/2)/(n/2) \ge \Omega(n)$ over the entire swapping-process in this
  case, implying that we achieve the desired backlog. 

  Otherwise, there is at least a $1/2$ chance that the round-block $j$, which
  is chosen uniformly at random from the round-blocks, when the filler performs
  a swap into the anchor set occurs on a round-block with $I(j)=0$, indicating
  that the emptier won't neglect the anchor set on round-block $j$. In this
  case, the round-block was a legitimate single processor cup game on $C_j$,
  the randomly chosen set of $\lceil e^{2h} \rceil$ cups on the $j$-th round.
  Then we achieve fill increase $\ge 2h$ by the end of the round-block with
  probability at least $1/\lceil e^{2h}\rceil!$ -- the probability that we
  correctly guess the sequence of cups within the single processor cup game
  that the emptier empties from. 

  The probability that the random set $C_j \subset B$ contains only cups that
  are among the $n/4$ fullest cups in $B$ is $${n/2 \choose {\lceil e^{2h}
  \rceil}} / {n \choose {\lceil e^{2h}\rceil}} = O(1).$$ Note that because, by
  assumption, at least half of the cups $c \in B$ have $\fil(c) \ge -h$, then
  the $n/4$ fullest cups in $B$ must have fill at least $-h$. If all cups $
  c\in C_j$ have $\fil(c) \ge -h$, then the fill of the cup in the active set
  at the end of the round-block is at least $-h + 2h = h$, if the filler
  guesses the emptier's emptying sequence correctly.

  Say that a swapping-process where at least half of the cups $c\in B$ have
  $\fil(c) \ge -h$ \defn{succeeds} if $C_j$ is a subset of the $n/4$ fullest
  cups in $B$, and if the filler correctly guesses the emptier's emptying
  sequence. Note that if a swapping-process succeeds, then the filler is able
  to swap a cup with fill at least $h$ into $A$. We have shown that there is a
  constant probability of a given swapping-process succeeding. Let $X_i$ be the
  binary random variable indicating whether or not the $i$-th swapping process
  where the filler does not perform a storing-operation where at least half of
  the cups $c\in B$ have $\fil(c) \ge -h$ succeeds. Let $q \ge \Omega(1)$ be
  the probability of a swapping-process succeeding, i.e. $P(X_i=1)$. Note that
  the random variables $X_i$ are clearly independent, and identically
  distributed.

  Clearly $$\E\left[\sum_{i=1}^{n/8} X_i\right] = qn/8.$$ Note that we do not
  use all the $X_i$; we know there must be at least $n/4 - 3/2 \gamma$
  swapping-processes that do not consist of a storing-operation, but only use
  $n/8$ of the $X_i$. We make this choice because the particular constants that
  we get do not matter, and because it substantially simplifies the analysis.
  By a Chernoff Bound (i.e. Hoeffding's Inequality applied to binary random variables),
  $$P\left(\sum_{i=1}^{n/8} X_i\le nq/16\right) \le e^{-nq^2/1024}.$$ That is, the
  probability that less than $nq/16$ of the anchor cups have fill at least $h$ is
  exponentially small in $n$, as desired.

\end{proof}
  
\end{proof}

\begin{lemma}[The Oblivious Amplification Lemma]
   \label{lem:obliviousAmplification}
  Let $f$ be an oblivious filling strategy that achieves backlog $f(n)$ in the
  variable-processor cup game on $n$ cups with constant probability (relative
  to average fill, with negative fill allowed). Let $\delta \in (0,1)$ be a
  parameter. Then, there exists an adaptive filling strategy that, with
  constant probability, either achieves backlog $$f'(n) \ge
  (1-\delta)\Big(f((1-\delta)n) + f((1-\delta)\delta n)\Big)$$ or achieves
  backlog $\Omega(\poly(n))$ in the variable processor cup game on $n$
  cups.
\end{lemma}
\begin{proof}
  
\end{proof}

\end{document}
