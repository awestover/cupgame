\documentclass[twocolumn]{article}[11pt]
\usepackage[left=1in, right=1in, top=1in, bottom=1in]{geometry}

\usepackage{amsthm}
\usepackage{amssymb}
\usepackage{amsmath}
\usepackage{mathtools}
\usepackage{hyperref}
\usepackage{xcolor}

\newcommand{\defn}[1]{{\textit{\textbf{\boldmath #1}}}}
\renewcommand{\paragraph}[1]{\vspace{0.09in}\noindent{\bf \boldmath #1.}} 
\DeclareMathOperator{\E}{\mathbb{E}}
\DeclareMathOperator{\Var}{\text{Var}}
\DeclareMathOperator{\img}{Im}
\DeclareMathOperator{\polylog}{\text{polylog}}
\DeclareMathOperator{\poly}{\text{poly}}
\DeclareMathOperator{\st}{\text{ such that }}
\DeclareMathOperator{\tilt}{\text{tilt}}
\DeclareMathOperator{\fil}{\text{fill}}
\newcommand{\norm}[1]{\left\lVert#1\right\rVert}

\newcommand{\contr}[0]{\[ \Rightarrow\!\Leftarrow \]}
\newcommand{\defeq}{\vcentcolon=}
\newcommand{\eqdef}{=\vcentcolon}

\newtheorem{fact}{Fact}
\newtheorem{definition}{Definition}
\newtheorem{remark}{Remark}
\newtheorem{proposition}{Proposition}
\newtheorem{clm}{Claim}
\newtheorem{lemma}{Lemma}
\newtheorem{corollary}{Corollary}
\newtheorem{theorem}{Theorem}
\newtheorem{conjecture}{Conjecture}

\usepackage{fancyhdr}
\pagestyle{fancy}
\fancyhead{}
\fancyfoot{}
\fancyfoot[R]{\thepage}
\renewcommand{\headrulewidth}{0pt}

\title{On variable-processor cup games (easier results)}
\author{Alek Westover}

\begin{document}
\maketitle

\paragraph{Oblivious Lowerbounds}

\begin{theorem}{Hoeffding's Inequality}
  Let $X_i$ be independent bounded random variabels with $X_i \in [a,b]$. Then,
  $$P\left(\Big|\frac{1}{n} \sum_{i=1}^n (X_i - \E[X_i])\Big|\ge t\right) \le 2\exp\left(-\frac{2nt^2}{(b-a)^2}\right) $$
\end{theorem}

\begin{proposition}
  There exists an oblivious filling strategy in the variabl-processor cup game on $n$ cups that achieves backlog $\Omega(\log n)$ against a smoothed greedy emptier.
\end{proposition}
\begin{proof}
  Let $A$ be the anchor set, randomly chosen, let $B$ be non-anchor set, with
  $|A| = |B| = n/2$. Let $h = \Theta(1)$ be the fill that we will achieve at
  each level of our recursive procedure.
Our strategy to achieve backlog $\Omega(\log n)$ is roughly as follows:
\begin{itemize}
  \item Make each cup in the anchor set have a constant probability of having
    fill at least $h$.
  \item Reduce the number of processors to a constant fraction $nk$ of $n$ and
    raise the fill of $nk$ cups to $h/2$. This step relies on the emptier being
    greedy.
  \item Recurse on the $nk$ cups that are known to have fill $\ge h/2$.
\end{itemize}
We can perfom $\Omega(\log n)$ levels of recursion, achieving constant backlog
at each step (relative to average fill); doing so yields backlog $\Omega(\log
n)$.

Our strategy is somewhat complicated by the possibility of the fill being very concentrated in a few cups. We proceed as follows:

For each anchor cup $i$:
\begin{enumerate}
  \item Chose an index $j \in [n^c]$
  \item For $n^c$ times (for some $c > 2$), we select a random subset $C\subset B$ of the non-anchor
    cups and play a single processor cup game on $C$.
  \item On round $j$ with $1/2$ probability we swap the winner of the single
    processor cup game into the anchor set, and with $1/2$ probability we swap
    a random cup from $B$ into the anchor set.
\end{enumerate}

Say that a cup is \defn{op} if it contains fill $\ge \sqrt{\frac{n}{\log\log
n}}$. If there is ever an op cup, then we win. Note that we don't even need to
know which cup is op because it will take $\Omega(\poly(n))$ rounds for the
emptier to reduce the fill below $\poly(n)$. Hence, we can assume without loss
of generality that no cup is ever op.

% want it to be more like you are globally using either clm1 or clm2. ie its hard to transition between the stages

\begin{clm}
  \label{clm:reg}
  If at least $1/2$ of the non-anchor cups have fill $\ge -h/2$ then the
  contribution to the anchor set has fill $\ge h/2$ with constant probability. 
\end{clm}
\begin{proof}
  Playing the single-processor cup game $n^c$ times, with only one time
  that we actually swap a cup into the anchor set, makes it bad for the emptier
  to ignore the anchor set on a more than a constant fraction of the games.
  In particular, if the emptier neglects the anchor set at least once for more
  than half of the games then the anchor set's average fill will have increased
  by $n^{c-1} \ge \Omega(\poly(n))$. Hence we have the desired backlog.

  Otherwise, we have at least a $1/2$ chance that the round $j$ when we will
  perform a switch into the anchor set occurs on a round when the emptier choses 
  not to neglect the anchor set. In this case, the round was a legitimate
  single processor cup game on $e^h$ cups, and we achieved fill increase $\ge
  h$ by the end of the game with probability at least $1/e^h!$, the probability
  that we correctly guess the sequence of cups within the single processor cup
  game that the emptier would neglect. And we get another factor of $1/2^{e^h}$
  (ish), the probability that the randomly chosen subset of the cups has all
  cups with fill $\ge -h/2$. In this case, which we established happens with
  constant probability, it is the case that the winner of the single processor
  cup game now has fill $\ge h/2$, as desired.
\end{proof}

\begin{clm}
  \label{clm:xtreme}
  Let $X$ be a uniformly randomly selected cup from $B$.
  Let $Y_i$ be the random variable $Y_i=\tilt_+(X)$ where $X$ is a randomly
  selected cup from the non-anchor set at the start of the $i$-th round of
  playing single processor cups games. 
  If less than $1/2$ of the non-anchor cups have fill $\ge -h/2$ fill, then
  with probability at least $1- 1/\polylog(n)$,
  $$\frac{1}{n/2}\sum_{i=1}^{n/2} Y_i \ge h/4.$$
\end{clm}

\begin{proof}
  We assume for simplicity that the average fill of $B$ is $0$. In reality this
  is not the case, but by a Hoeffding bound and the fact that op cups don't
  exist, the fill is really tightly concentrated around $0$, so this is almost
  WLOG.

  Let the positive tilt of a cup $i$ be $\tilt_+(i) \defeq \max(\fil(i), 0)$.
  We have
  $$\E[\tilt_+(X)] = \frac{1}{2}\E[|\fil(X)|] \ge h/2$$
  (because negative tilt is at least $nh/4$ and positive tilt must oppose this).
  
  Let $Y_i$ be the random variable $Y_i=\tilt_+(X)$ where $X$ is a randomly
  selected cup from the non-anchor set at the start of the $i$-th round of
  playing single processor cups games. Note that the $Y_i$ are not really
  independent, but it is probably ok. Note that $0\le Y_i \le n/\lg\lg n$.
  Now we have, by Hoeffding's inequality, that 
  $$P\left(\Big|\frac{1}{n/2} \sum_{i=1}^{n/2} (Y_i - \E[Y_i])\Big|\ge h/4
  \right) \le 2\exp\left(-\frac{n(h/4)^2}{(\sqrt{n/\lg\lg n})^2}\right) $$
  $$P\left(\frac{1}{n/2}\sum_{i=1}^{n/2} Y_i \le h/4\right) \le 1/\polylog(n) $$

\end{proof}

Now we consider two cases based on how many times we must apply Claim
\ref{clm:reg} and Claim \ref{clm:xtreme}. If we must apply Claim
$\ref{clm:reg}$ at least half the time, then we achieve a constant fraction of
the anchor cups with fill at least $h/2$. If on the other hand we must apply
Claim $\ref{clm:xtreme}$ at least half of the time, we have that with
probability $1- 1/\polylog(n)$ the process brings $n\cdot h/8$ positive tilt to
the anchor set as desired. 

  In either case we achieve, with probability at least $1-1/\polylog n$,
  positive tilt at least $hn/k$ in the anchor set.
  Use the positive tilt, with one processors, we can transfer over the fill into $n/k$ cups. 
  (Note, we use one processor because we do not know how many cups the fill is
  concentrated in). The filler repeatedly distributes $1$ unit of fill to each of the $n/k$ cups in succession, and continues until $h/4$ fill has been distributed. We cannot continue beyond this point because we have used up the positive tilt. Now we recurse on this set of $n/k$ cups.

  We can perform $\Omega(\log n)$ levels of recursion, and gain $\Omega(1)$
  fill at each step. Hence, overall, backlog of $\Omega(\log n)$ is achieved.
\end{proof}


\begin{lemma}{The Oblivious Amplification Lemma}
  Given an oblivious filling strategy for achieving backlog $f(k)$ in the
  variable-processor cup game on $k$ cups that succeeds with probability at
  least $1/2$, there exists a strategy for achieving ``amplified" fill $$f'(k)
  \ge \frac{1}{4}(f(k/2) + f(k/4) + f(k/8) + \cdots $$ that succceeds with constant probability.
\end{lemma}
\begin{proof}
  Want to do: same proof as before, but there are \textbf{concerns, about using higher values of $h$:}
\begin{itemize}
  \item dealing with star: problem: need to say WLOG avg fill of A,B is 0 each
    initially solution: no op cups wlog, so if we pick them randomly star holds
    by Hoeffding's. (kinda, bc stuff isnt really independent, can probably swap
    with replacement to fix this tho)
\item dealing with start star: What if C needs to be big because we need big
  backlog? claim: star star isnt a problem beause the base case is the only
  case that needs to explicitely deal with positive and negative fill
\end{itemize}
  
\end{proof}

\begin{corollary}
  There is an oblivious filling strategy for the variable-processor cup game on
  $n$ cups that achieves backlog $2^{\Omega(\sqrt{\log n})}$ in running time
  $O(n)$
\end{corollary}
\begin{proof}
  
\end{proof}

\clearpage
\paragraph{Adaptive Lowerbound}
\begin{proposition}
  There exists an adaptive filling strategy for the variable-processor cup game
  on $k$ cups that achieves backlog $\Omega(\log n)$, where fill is relative to
  the average fill of the cups, with negative fill allowed.
\end{proposition}
\begin{proof}
  Let $h = \frac{1}{4}\log n/2$ be the desired fill. Call a cup be \defn{op} if
  it has fill at least $h$. Once there exists an op cup, the proposition is
  immediately satisfied. Let the \defn{positive tilt} of a cup $i$ with fill
  $v$ be $\max(0,v)$, and let the positive tilt of a set $S$ of cups be the sum
  of the positive tilt of each cup $i\in S$.

  If no cups are op, then positive tilt $< h\cdot n$. Assume for sake of
  contradiction that there are $\ge n/2$ cups with fill $\le -2h$. Then the
  mass of those cups would be $\le -hn $, but there isn't enough positive tilt
  to oppose this. Hence there are $<n/2$ cups with fill $\le -2h$. We set the
  number of processors equal to $1$ and play a single processor cup game on
  $n/2$ cups that have fill at least $-2h$, which must exist as stated. In the
  single processor cup game we distribute water equally among all the cups in
  the active set at each step. Then the emptier will chose some cup to empty.
  If this cup is in our active set we remove it from the active set.
  At the end of this process the active set is non-empty, and any cup in the
  active set has gained fill at least $H_{n/2} \ge \log n/2 = 4h$. Thus such a cup
  has fill at least $2h$ now, so the proposition is statisfied.
\end{proof}

\begin{lemma}{The Adaptive Amplification Lemma}
  Given an adaptive filling strategy for achieving backlog $f(k)$ in the 
  variable-processor cup game on $k$ cups, there exists
  a strategy for achieving ``amplified" fill $$f'(k) \ge \frac{1}{4}(f(k/2) +
  f(k/4) + f(k/8) + \cdots .$$
\end{lemma}
\begin{proof}
  If, at any point in the process that will be described, backlog is greater
  than $f'(k)$, then the filler stops and the Lemma is satisfied as the desired
  backlog having been achieved. Thus we assume without loss of generality for
  the rest of the proof that no cup ever exceeds fill $f'(k)$ during the course
  of our algorithm. That is, we assume that we don't achieve the desired
  backlog until the end of our process.

  The main idea of this analysis is as follows:
  \begin{enumerate}
    \item Using $f$ repeatedly, achieve average fill at least $\frac{1}{2} f(n/2)$ in $n/2$ cups.
    \item Halve the number of processors
    \item Recurse on the $n/2$ cups with high average fill.
  \end{enumerate}

  Let $h_l = f(k/2^l)$; the filler will achieve a set of at
  least $n_l/2 = n/2^l$ cups with average fill at least $h_l / 2$ on the $l$-th
  level of recursion. 

  On the $l$-th level of recursion we will repeat the following at most $n_l/2$ times:
  \begin{itemize}
    \item If the non-anchor set has average fill at least $-h_l/2$ then we
      apply $f$ to the non-anchor set. This gets us fill $-h_l/2 + f(n_l/2) =
      h_l/2$ in some non-anchor cup. We replace the lowest cup in the anchor
      set with this cup. A slight complication with this method is that we are
      anchoring the anchor set, and assuming that the emptier allways empties
      from each anchor cup; this may not be the case. However, the issue can be
      resolved by applying $f$ up to $h_ln_l/4+1$ times. If on all of these
      applications of $f$ the emptier doesn't always empty from each cup in the
      anchor set, then the average fill in the anchor set increases by more
      than $h_l/2$, so the desired fill was achieved. Otherwise, there must a
      time when we apply $f$ and the emptier does empty from each anchor cup
      each time. In this case we actually do achieve fill $-h_l/2 + f(n_l/2)$
      in a non-anchor cup, and swap it into the anchor set, as descibred
      before.
    \item If the non-anchor has average fill lower
      than $-h_l/2$, then anchor set has average fill at least
      $h_l/2$, so we terminate the process.
  \end{itemize}
\end{proof}

\begin{corollary}
  There is an adaptive filling strategy for the variable-processor cup game on $n$ cups that achieves backlog $\Omega(\poly(n))$ in running time $2^{O(\log^2 n)}$
\end{corollary}
\begin{proof}
  Let
  $$f_0(k) = 
  \begin{cases} 
    \log_2 k, & k\geq 1, \\
    0 & \text{else.}
  \end{cases}$$

  Let $f_{m+1} $ be the result of applying The Amplification Lemma to $f_m$. 
  By repeated amplification $\log_2 n^{1/9}$ times we 
  achieve a function $f_{\log_2 n^{1/9}}(k)$ with the property that for $k \geq n,$
  $f_{\log_2 n^{1/9}}(k) \geq 2^{\log_2 n^{1/9}} \log_2 k$. In particular, this gives a filling strategy 
  that when applied to $n$ cups gives backlog $\Omega(n^{1/9}\log_2 n) \ge \Omega(\poly(n))$ as desired.
  To prove this, we prove the following lowerbound for $f_m$ by induction:
  $$f_m(k) \geq 2^m \log_2 k, \text{ for } k \geq (2^9)^m.$$
  The base case follows from the definition of $f_0$. Assuming the property for $f_m$, we get the following:
  $ \text{for } k > (2^9)^{m+1},$
  \begin{align*}
    f_{m+1}(k) &= \frac{1}{2}(f_m(k/2) + f_m(k/4) + \cdots + f_m(k/2^9) + \cdots)\\
  &\geq \frac{1}{2}(f_m(k/2) + f_m(k/4) + \cdots + f_m(k/2^9))\\
  &\geq \frac{1}{2}2^m(\log_2 (k/2) + \log_2(k/4) + \cdots + \log_2(k/2^9))\\
  &\geq \frac{1}{2}2^m(9\log_2 (k) - \frac{9 \cdot 10}{2}) \\
  &\geq 2^{m+1} \log_2(k) ,
  \end{align*}

  as desired. Hence the inductive claim holds, which establishes that $f_{\log_2
  n^{1/9}}$ satisfies the desried condition, which proves that backlog can be
  made $\Omega(\poly(n))$.


\end{proof}








% TODO:
% amplification lemma
% hoeffding bound stuff





% thoughts:

% if there are at least n/2 cups with like fill at least -h/4 then we are golden
% cuz just play on them

% if there are less than n/2 cups with fill at least -h/4
% tehn the negative tilt is at least nh/8

% so so it the positive tilt
% (negative tilt overall = positive tilt overall)



% rough goal: get n/2 cups with avgfill at least h
% problem: what if we cant because there are op cups.
% solution: take the op cups.
% question: when is a cup op? if it has fill at least 1/2 f(n/2) + f(n/4) + ...
% or maybe it should just be 1/2f(n/2)



% we win if there isa cup with fill at  least that sum
% if there are none

% wellll the sum is like less than log n f(n/2) 1/2
% could go down to 4.5 f(n/2) if we really wanted














% thoughts:

% goal: get a cup with fill h = 1/2 f(n/2) + f(n/4) + ...
% subgoals: get n/2 cups with fill 1/2 f(n/2), n/4 cups with fill 1/2 f(n/2) + f(n/4) ...


% case 1: 
% there exists a set of size $n/2^m$ with fill $\frac{1}{2} (f(n/2) + \cdots + f(n/2^m))$
% then yay that was our subgoal, recurse

% case 2: 
% no set of size at least $n/2^m$ with fill at least $\frac{1}{2} (f(n/2) + \cdots + f(n/2^m))$

% well then total positive tilt is less than ... n/2 1/2 f(n/2) + n/4 1/2 f(n/4) + ... < nh/2

% imagene $> n/\sqrt{2}$ of the cups had fill $< -h/\sqrt{2}$
% then negative tilt would be at least $-hn/2$
% but this cant be

% hence $< n/\sqrt{2}$ of the cups have fill $< -h/\sqrt{2}$
% at least $1-1/\sqrt{2}$ of the cups have fill $>-h/\sqrt{2}$

% so yay we can apply the alg to those cups.

% feelsbad




\end{document}
