\documentclass[twocolumn]{article}[10pt]
\usepackage[left=1in, right=1in, top=1in, bottom=1in]{geometry}
\usepackage[subtle]{savetrees}

\usepackage{amsthm}
\usepackage{amssymb}
\usepackage{amsmath}
\usepackage{mathtools}
\usepackage{hyperref}
\usepackage{xcolor}
\usepackage{xspace}

\newcommand{\defn}[1]{{\textit{\textbf{\boldmath #1}}}\xspace}
\renewcommand{\paragraph}[1]{\vspace{0.09in}\noindent{\bf \boldmath #1.}} 
\DeclareMathOperator{\E}{\mathbb{E}}
\DeclareMathOperator{\Var}{\text{Var}}
\DeclareMathOperator{\img}{Im}
\DeclareMathOperator{\polylog}{\text{polylog}}
\DeclareMathOperator{\poly}{\text{poly}}
\DeclareMathOperator{\st}{\text{ such that }}
\DeclareMathOperator{\tilt}{\text{tilt}}
\DeclareMathOperator{\fil}{\text{fill}}
\newcommand{\norm}[1]{\left\lVert#1\right\rVert}

\newcommand{\contr}[0]{\[ \Rightarrow\!\Leftarrow \]}
\newcommand{\defeq}{\vcentcolon=}
\newcommand{\eqdef}{=\vcentcolon}

\newtheorem{fact}{Fact}
\newtheorem{definition}{Definition}
\newtheorem{remark}{Remark}
\newtheorem{proposition}{Proposition}
\newtheorem{clm}{Claim}
\newtheorem{lemma}{Lemma}
\newtheorem{corollary}{Corollary}
\newtheorem{theorem}{Theorem}
\newtheorem{conjecture}{Conjecture}

\usepackage{authblk}
\usepackage{fancyhdr}
\pagestyle{fancy}
\fancyhead{}
\fancyfoot{}
\fancyfoot[R]{\thepage}
\renewcommand{\headrulewidth}{0pt}

\title{Oblivious Lower Bound: this time its for real}
\date{\vspace{-5ex}}

\author[1]{\small William Kuszmaul\thanks{Supported by a Hertz fellowship and a NSF GRFP fellowship}}
\author[2]{\small Alek Westover\thanks{Supported by MIT PRIMES}}

\affil[ ]{\footnotesize MIT\textsuperscript{1}, MIT PRIMES\textsuperscript{2}}
\affil[ ]{\textit{kuszmaul@mit.edu, alek.westover@gmail.com}}

\begin{document}
\maketitle

Call a cup configuration $T$\defn{-flat} if the fill of every cup is in the
interval $[-T, T]$.

An emptying strategy is said to be $(R, t)$\defn{-flattenable} if, given a
$T$-flat cup configuration, the emptier can, in running-time $t(T)$, obtain a
$R$-flat configuration of cups. An emptying strategy is said to be
\defn{flattenable} if it is $(R, t)$-flattenable for $R\le O(1)$ and for $t$
such that $t(T) \le O(1)$ if $T\le O(1)$.

In the randomized setting we are only able to prove lower bounds for backlog
against flattenable emptiers; whether or not our results can be extended to a
more general class of emptiers is an interesting open question. 

However, this class of flattenable emtpiers is alredy of great interest. 
In particular, we will show certain emptying strategies with properties similar
to a greedy emptier are flattenable; in particular, we will show that a greedy
emptier is flattenable.

We call an emptier $\Delta$\defn{-greedy-like} if, when there are two cups
$c_1, c_2$ with fills satisfying $\fil(c_1) > \fil(c_2) + \Delta$ the emptier
never empties from $c_2$ without emptying from $c_1$ on the same round.
Intuitively, a $\Delta$-greedy-like emptier has a $\pm \Delta$ range within it
is allowed to ``not be greedy". Note that a perfectly greedy emptier is
$0$-greedy-like.

We now prove that $\Delta$-greedy-like emptiers (for $\Delta \le O(1)$) are flattenable: 
\begin{proposition}
  \label{prop:greedylikeisflat}
  Given a cup configuration that is $T$-flat, an oblivious filler can, in
  running time $2T$, achieve a $2(2+\Delta)$-flat configuration of cups against
  a $\Delta$-greedy-like emptier. 

  In particular, this implies that a $\Delta$-greedy-like emptier is
  $(R,t)$-flattenable for $R = 2(2+\Delta)$ and $t$ the function $T\mapsto 2T$;
  for $\Delta\le O(1)$, $\Delta$-greedy-like emptiers are flattenable.
\end{proposition}

\begin{proof}
  The filler's sets $p=n/2$ and distributes fill equally amongst
  all cups at every round, in particular placing $1/2$ units of water in each cup.
  Let $\ell_t = \min_{c\in S_t} \fil_{S_t}(c)$, $u_t=\max_{c\in S_t} \fil_{S_t}(c)$. Let
  $L_t$ be the set of cups $c$ with $\fil_{S_t}(c) \in \le l_t+2+\Delta$, and let
  $U_t$ be the set of cups $c$ with $\fil_{S_t}(c) \ge u_t-2-\Delta$.

  There are two ways to think of $U_t$.
  First we can consider $U_t$ as the union of intervals of length $1$,
  $\Delta$, and $1$. Note the key property that if a cup with fill in
  $[u_t-\Delta-2, u_t-\Delta-1]$ is emptied from, then all cups with fills in 
  $[u_t-1, u_t]$ must be emptied from, because the emptier is $\Delta$-greedy-like.
  On the other hand, we can consider $U_t$ as the union of $[u_t-2, u_t]$ and
  $[u_t-\Delta-2, u_t-2]$. This is useful as the interval of width $\Delta$
  serves as a ``buffer". In particular, if there are more than $n/2$ cups
  outside of $U_t$ then all cups in $[u_t-2, u_t]$ must be emptied from because
  the emptier is $\Delta$-greedy-like. $L_t$ is of course completely symmetric to $U_t$.

  First we prove a key property of the sets $U_t$ and $L_t$: once a cup is in
  $U_t$ or $L_t$ it is always in $U_{t'}, L_{t'}$ for all $t' > t$. This
  follows immediately from the following claim:
  \begin{clm}
    \label{clm:dontlosestuff}
    $$U_{t} \subseteq U_{t+1}, L_t \subseteq L_{t+1}.$$
  \end{clm}
  \begin{proof}
    Consider a cup $c\in U_t$.

    If $c$ is not emptied from, i.e. $\fil(c)$ has increased by $1/2$, then
    clearly $c \in U_{t+1}$, because backlog has increased by at most $1/2$, so
    the fill of $c$ must still be within $2+\Delta$ of the backlog on round $t+1$. 

    On the other hand, if $c$ is emptied from, i.e. $\fil(c)$ has decreased by
    $1/2$, we consider two cases.
    \begin{itemize}
      \item If $\fil_{S_t}(c) \ge u_t-\Delta -1$, then, as $u_{t+1} \le u_t+1/2$, 
        $$\fil_{S_{t+1}}(c) \ge u_t-\Delta-1 - 1/2\ge u_{t+1}-\Delta-2.$$
      \item On the other hand, if $\fil_{S_t}(c) < u_t-\Delta-1$, then every cup
        with fill in $[u_t-1, u_t]$ must have been emptied
        from. The fullest cup at round $t+1$ is the same as the fullest cup on
        round $t$, because the fills of all cups with fill in
        $[u_t-1, u_t]$ have decreased by $1/2$, and no cup with fill less than
        $u_t-1$ had fill increase by more than $1/2$. Hence $u_{t+1} = u_t -1/2$.
        Because both the fill of $c$ and the backlog have decreased by the same
        amount, the distance between them is still at most $\Delta+2$, hence
        $c\in U_{t+1}$.
    \end{itemize}

    The argument for $L_t \subseteq L_{t+1}$ is essentially identical.
  \end{proof}

  Now that we have shown that $L_t$ and $U_t$ never lose cups, we will show
  that they eventually gain a substantial number of cups. 

  \begin{clm}
    \label{clm:smallthenbigger}
    As long as $|U_t| \le n/2$ we have $u_{t+1} = u_t -1/2$. Identically, as
    long as $|L_t| \le n/2$ we have $\ell_{t+1} = \ell_t+ 1/2$.
  \end{clm}
  \begin{proof}
    If there are more than $n/2$ cups outside of $U_t$ then there must be some
    cup with fill less than $u_t-\Delta-2$ that is emptied from. Because the
    emptier is $\Delta$-greedy-like this means that the emptier must empty from
    every cup with fill at least $u_t-2$. Thus $u_{t+1} = u_t -1/2$: no cup
    with fill less than $u_t-2$ could have become the fullest cup, and the
    previous fullest cup has lost $1/2$ units of fill. 

    The proof is identical for $L_t$.
  \end{proof}

  By Claim \ref{clm:smallthenbigger} we see that both $|U_t|$ and $|L_t|$ must
  eventually exceed $n/2$ at some times $t_u, t_\ell \le 2T$, by the assumption
  that the initial configuration is $T$-flat. Since by Claim
  \ref{clm:dontlosestuff} $|U_{t+1}|\ge |U_t|$ and $|L_{t+1}| \ge |L_t|$ we
  have that there is some round $t_0 =\max(t_u, t_\ell) \le 2T$ on which both
  $|U_{t_0}|$ and $|L_{t_0}|$ exceed $n/2$. Then $U_{t_0} \cap L_{t_0} \neq
  \varnothing$. Furthermore, the sets must intersect for all $t_0 \le t \le 2T$. 
  In order for the sets to intersect it must be that the intervals
  $[u_t-2-\Delta, u_t]$ and $[\ell_t, \ell_t+2+\Delta]$ intersect. Hence we have that 
  $$\ell_t+2+\Delta \ge u_t-2-\Delta.$$ Since $u_t \ge 0$ and $\ell_t \le 0$
  this implies that all cups have fill in $[-2(2+\Delta), 2(2+\Delta)]$.

\end{proof}

\begin{proposition}
  \label{prop:obliviousBase}
  There exists an oblivious filling strategy in the variable-processor cup game
  on $n$ cups that achieves backlog $\Omega(\log n)$ against a
  $\Delta$-greedy-like emptier (where $\Delta \le O(1)$ is a constant known
  to the filler), with constant probability.
\end{proposition}
\begin{proof}
    Let $A$, the \defn{anchor} set, be a subset of the cups chosen uniformly at
  random from all subsets of size $n/2$ of the cups, and let $B$, the
  \defn{non-anchor} set, consist of the rest of the cups ($|B| = n/2$). Let $h
  = 8\Delta + 8$, and let $h' = 2$. Our strategy is roughly as follows: 
  \begin{itemize}
    \item \textbf{Step 1:} Make a constant fraction of cups in $A$ have fill at
      least $h$ by playing single processor cup games on constant-size subsets
      of $B$. With constant probability we can attain a cup in $B$ with
      constant fill by this method, that we then swap into $A$. By a Chernoff
      bound we get a constant fraction of $A$, say $cn$ cups, to have fill at
      least $h$ with exponentially good probability.
    \item \textbf{Step 2:} Reduce the number of processors to $cn$, and raise
      the fill of $cn$ \emph{known} cups to fill $h'$. 
    \item \textbf{Step 3:} Recurse on the $nc$ cups that are known to have fill
      at least $h'$.
  \end{itemize}

By performing $\Omega(\log n)$ levels of recursion, achieving constant backlog
$h'$ at each step (relative to the average fills), the filler achieves backlog
$\Omega(\log n)$.

We now provide a detailed description of the filling algorithm, to prove that
the results claimed in Step 1 and Step 2 are attainable.

To attain step 2, we smooth the non-anchor set.

Now we detail how to achieve Step 1.

We perform a series of \defn{swapping-process}, which are procedures that we
use to get a new cup in $A$. A swapping-process is composed of a substructure,
repeated many times, which we call a \defn{round-block}; a round-block is a set
of rounds. At the beginning of a swapping-process we choose a round-block $j
\in [n^2]$ uniformly at random from all the round-blocks. The swapping-process
proceeds for $n^2$ round-blocks; on the $j$-th round-block we swap a cup into
the anchor set.

On each of the $n^2$ round-blocks, the filler selects a random subset $C\subset
B$ of the non-anchor cups and plays a single processor cup game on $C$. In this
single-processor cup game the filler employs the classic adaptive strategy for
achieving backlog $\Omega(\log |B|)$ on a set of $|B|$ cups, however modified
because it is an oblivious filler. In particular, the filler's strategy in the
single-processor cup games is to distribute water equally among an \defn{active
set} of cups, and then after the emptier removes water from some cup the filler
removes a random cup from the active set. There is at least constant
probability that this results in the active set having a single cup at the end,
with fill that has increased by at least $1/|B| + 1/(|B|-1) + \ldots + 1/1 \ge
\ln |B|$ since the start of the round-block.

On most round-blocks -- all but the $j$-th -- the filler does nothing with the
cup that it achieves in the active set at the end of the single processor cup
game. However, on the $j$-th round-block the filler swaps the winner of the
single processor cup game into the anchor set.

\begin{clm} \label{clm:reg} 
  Let $q\ge \Omega(1)$ be an appropriately small constant ($q$ is a function of
  $h\le O(1)$). In Case 1, with probability at least $1-e^{-nq^2/1024}$, we
  achieve fill at least $h$ in at least $nq/16$ of the cups in $A$ (i.e. a
  constant fraction of the cups in $A$). In particular, this implies that we
  achieve positive tilt $hnq/16 \ge \Omega(n)$ in $A$.
\end{clm}
\begin{proof}
  Consider a swapping-process where the filler does not perform a
  storing-operation where at least $1/2$ of the cups $c \in B$ have $\fil(c)
  \ge -h$. Note that by assumption there are at least $n/4 - 3/2 \gamma$ such rounds.
 
  Say the emptier \defn{neglects} the anchor set in a round-block if on at
  least one round of the round-block the emptier does not empty from every
  anchor cup. By playing the single-processor cup game for $n^2$ round-blocks,
  with only one round-block when we actually swap a cup into the anchor set, we
  strongly disincentive the emptier from neglecting the anchor set on more
  than a constant fraction of the round-blocks. 

  The emptier must have some binary function, $I(i)$ that indicates whether or
  not they will neglect the anchor set on round-block $i$ if the filler has not
  already swapped. Note that the emptier will know when the filler perform a
  swap, so whether or not the emptier neglects a round-block $i$ depends on
  this information. However, $j$ is the only parameter of the swapping-process,
  so there is no other information that the emptier can use to decide whether
  or not to neglect a round-block, because on any round-block when we simply
  redistribute water amongst the non-anchor cups we effectively have not
  changed anything about the game state. 

  If the emptier is willing to neglect the anchor set for at least $1/2$ of the
  round-blocks, i.e. $\sum_{i=1}^{n^2} I(i) \ge n^2 / 2$, then with probability
  at least $1/4$, $j \in ((3/4) n^2, n^2)$, in which case the emptier neglects
  the anchor set on at least $n^2/4$ round-blocks ($I(k)$ must be $1$ for at
  least $n^2/4$ of the first $(3/4)n^2$ round-blocks). Each time the emptier
  neglects the anchor set the mass of the anchor set increases by at least $1$.
  Thus the average fill of the anchor set will have increased by at least
  $(n^2/2)/(n/2) \ge \Omega(n)$ over the entire swapping-process in this
  case, implying that we achieve the desired backlog. 

  Otherwise, there is at least a $1/2$ chance that the round-block $j$, which
  is chosen uniformly at random from the round-blocks, when the filler performs
  a swap into the anchor set occurs on a round-block with $I(j)=0$, indicating
  that the emptier won't neglect the anchor set on round-block $j$. In this
  case, the round-block was a legitimate single processor cup game on $C_j$,
  the randomly chosen set of $\lceil e^{2h} \rceil$ cups on the $j$-th round.
  Then we achieve fill increase $\ge 2h$ by the end of the round-block with
  probability at least $1/\lceil e^{2h}\rceil!$ -- the probability that we
  correctly guess the sequence of cups within the single processor cup game
  that the emptier empties from. 

  The probability that the random set $C_j \subset B$ contains only cups that
  are among the $n/4$ fullest cups in $B$ is $${n/2 \choose {\lceil e^{2h}
  \rceil}} / {n \choose {\lceil e^{2h}\rceil}} = O(1).$$ Note that because, by
  assumption, at least half of the cups $c \in B$ have $\fil(c) \ge -h$, then
  the $n/4$ fullest cups in $B$ must have fill at least $-h$. If all cups $
  c\in C_j$ have $\fil(c) \ge -h$, then the fill of the cup in the active set
  at the end of the round-block is at least $-h + 2h = h$, if the filler
  guesses the emptier's emptying sequence correctly.

  Say that a swapping-process where at least half of the cups $c\in B$ have
  $\fil(c) \ge -h$ \defn{succeeds} if $C_j$ is a subset of the $n/4$ fullest
  cups in $B$, and if the filler correctly guesses the emptier's emptying
  sequence. Note that if a swapping-process succeeds, then the filler is able
  to swap a cup with fill at least $h$ into $A$. We have shown that there is a
  constant probability of a given swapping-process succeeding. Let $X_i$ be the
  binary random variable indicating whether or not the $i$-th swapping process
  where the filler does not perform a storing-operation where at least half of
  the cups $c\in B$ have $\fil(c) \ge -h$ succeeds. Let $q \ge \Omega(1)$ be
  the probability of a swapping-process succeeding, i.e. $P(X_i=1)$. Note that
  the random variables $X_i$ are clearly independent, and identically
  distributed.

  Clearly $$\E\left[\sum_{i=1}^{n/8} X_i\right] = qn/8.$$ Note that we do not
  use all the $X_i$; we know there must be at least $n/4 - 3/2 \gamma$
  swapping-processes that do not consist of a storing-operation, but only use
  $n/8$ of the $X_i$. We make this choice because the particular constants that
  we get do not matter, and because it substantially simplifies the analysis.
  By a Chernoff Bound (i.e. Hoeffding's Inequality applied to binary random variables),
  $$P\left(\sum_{i=1}^{n/8} X_i\le nq/16\right) \le e^{-nq^2/1024}.$$ That is, the
  probability that less than $nq/16$ of the anchor cups have fill at least $h$ is
  exponentially small in $n$, as desired.

\end{proof}
  
\end{proof}

\begin{lemma}[The Oblivious Amplification Lemma]
   \label{lem:obliviousAmplification}
  Let $f$ be an oblivious filling strategy that achieves backlog $f(n)$ in the
  variable-processor cup game on $n$ cups with constant probability (relative
  to average fill, with negative fill allowed). Let $\delta \in (0,1)$ be a
  parameter. Then, there exists an adaptive filling strategy that, with
  constant probability, either achieves backlog $$f'(n) \ge
  (1-\delta)\Big(f((1-\delta)n) + f((1-\delta)\delta n)\Big)$$ or achieves
  backlog $\Omega(\poly(n))$ in the variable processor cup game on $n$
  cups.
\end{lemma}
\begin{proof}
  
\end{proof}

\end{document}
