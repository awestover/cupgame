First we highlight the concentration inequalities that we will need in the analysis. 

The following theorem is known as Hoeffding's Inequality:
\begin{theorem}
  Let $X_i$ for $i=1,2,\ldots, k$ be independent bounded random variables with
  $X_i \in [a,b]$ for all $i$. Then,
  $$\Pr\left(\Big|\frac{1}{k} \sum_{i=1}^k (X_i - \E[X_i])\Big|\ge t\right) \le
  2\exp\left(-\frac{2kt^2}{(b-a)^2}\right) $$
\end{theorem}
There are also several useful corollaries of Hoeffding's Inequality. 
Firstly, the Chernoff Bound, i.e. Hoeffding's Inequality applied to binary
random variables, is a trivial corollary.
A more interesting corollary is that Hoeffding's Inequality applies to random
variables drawn without replacement from a finite population.
Let $S$ be a finite population, let $X_i$ for $i=1,2\ldots, k$ be chosen
uniformly at random from $S \setminus \{X_1,\ldots, X_{i-1}\}$, and let $Y_i$
for $i=1,2,\ldots, k$ be chosen uniformly at random from $S$.
Note that $\{X_1,\ldots, X_k\}$ represents a sample of $S$ chosen without
replacement, whereas $\{Y_1,\ldots, Y_k\}$ represents a sample with
replacement. Because the $Y_i$ are independent random variables
Hoeffding's Inequality provides a bound on the probability of $\sum_{i=1}^k
Y_i$ deviating from its mean by more than $t$.
The same bound can be given on the probability of $\sum_{i=1}^k X_i$ deviating
from its mean by more than $t$, because the probability of $\sum_{i=1}^k X_i$
deviating from its mean by more than $t$ is at most the probability of
$\sum_{i=1}^k Y_i$ deviating from it's mean by $t$.
Formally we can write this as 
\begin{corollary}
  \label{cor:hoeffdingwreplacement}
  Let $S$ be a finite set with $\min(S) \ge a, \max(S) \le b$, and let $X_i$
  for $i=1,2\ldots, k$ be chosen uniformly at random from $S \setminus
  \{X_1,\ldots, X_{i-1}\}$.
Then 
  $$\Pr\left(\Big|\frac{1}{k} \sum_{i=1}^k (X_i - \E[X_i])\Big|\ge t\right) \le
  2\exp\left(-\frac{2kt^2}{(b-a)^2}\right) $$
\end{corollary}
Hoeffding proved \cref{cor:hoeffdingwreplacement} in his seminal work
\cite{who62} (the result follows from his Theorem 4, combined with Hoeffding's
Inequality for independent random variables).
This result is intuitive as samples drawn without replacement should be more
tightly concentrated around the mean than samples drawn with replacement, which
are more free to vary.

We now proceed with our analysis of oblivious lower bounds.

