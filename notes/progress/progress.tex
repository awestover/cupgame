\documentclass{article}[11pt]
\usepackage[subtle]{savetrees}
\usepackage[left=1in, right=1in, top=1in, bottom=1in]{geometry}

\usepackage{amsthm}
\usepackage{amssymb}
\usepackage{amsmath}
\usepackage{mathtools}

% \usepackage{fancyhdr}
% \pagestyle{fancy}
% \lhead{Alek Westover}
% \rhead{}
\usepackage{hyperref}

\DeclareMathOperator{\E}{\mathbb{E}}
\DeclareMathOperator{\Var}{\text{Var}}
\DeclareMathOperator{\img}{Im}
\DeclareMathOperator{\polylog}{\text{polylog}}
\DeclareMathOperator{\st}{\text{ such that }}
\newcommand{\norm}[1]{\left\lVert#1\right\rVert}
\newcommand{\interior}[1]{%
  {\kern0pt#1}^{\mathrm{o}}%
}

%** VECTOR NOTATION
\newcommand{\mb}{\mathbf}
\newcommand{\x}{\mathbf{x}}
\newcommand{\y}{\mathbf{y}}
\newcommand{\z}{\mathbf{z}}
\newcommand{\f}{\mathbf{f}}
\newcommand{\n}{\mathbf{n}}
\newcommand{\p}{\mathbf{p}}
\renewcommand{\k}{\mathbf{k}}
\renewcommand{\d}{\mathrm{d}} %straight d for integrals
\newcommand{\De}{\Delta}
\renewcommand{\Re}{\mathrm{Re}}
\renewcommand{\Im}{\mathrm{Im}}
\newcommand{\ran}{\mathrm{ran}}

%** SETS
\newcommand{\set}[1]{\mathbb{#1}}
\newcommand{\curly}[1]{\mathcal{#1}}
\newcommand{\goth}[1]{\mathfrak{#1}}
\newcommand{\setof}[2]{\left\{ #1\; : \;#2 \right\}}
\newcommand{\cc}{\subseteq\subseteq}
\newcommand{\R}{\set{R}}
\newcommand{\C}{\set{C}}
\newcommand{\Z}{\set{Z}}
\newcommand{\D}{\curly{D}}
\renewcommand{\S}{\set{S}}
\newcommand{\T}{\set{T}}


\newcommand{\contr}[0]{\[ \Rightarrow\!\Leftarrow \]}
\newcommand{\defeq}{\vcentcolon=}
\newcommand{\eqdef}{=\vcentcolon}

\newtheorem{fact}{Fact}
\newtheorem{definition}{Definition}
\newtheorem{remark}{Remark}
\newtheorem{proposition}{Proposition}
\newtheorem{lemma}{Lemma}
\newtheorem{corollary}{Corollary}
\newtheorem{theorem}{Theorem}

\usepackage{mdframed}
\newmdtheoremenv{q}{Question}


\author{Alek Westover}
\title{State of the Cups}

\begin{document}
\maketitle

\begin{itemize}
  \item \textbf{Single-Processor, adaptive filler: } \\
    Filler: $\Omega(\log n)$ (ignore the touched cup each time, equal water to
    all others, $\frac{1}{n}+\frac{1}{n-1}  + \cdots + \frac{1}{1}$)\\
    Emptier: $O(\log n)$ (inductive proof)
  \item \textbf{Single-Processor, oblivious filler: } \\
    Filler: $\Omega(\log\log n)$ (anchoring???) \\
    Emptier: $O(\log\log n)$ (no more than $\log n$ of the cups ever get fill above constant, play a cup game on the rest of the cups, get $O(\log \log n)$)
  \item \textbf{Multi-Processor, adaptive filler: } \\
    Filler: $\Omega(\log n)$ (For $p < n-\sqrt{n}$ see Bills paper that gets
    $\Omega(\log (n-p))$ which is tight for these small values of $p$ by
    playing a single processor cup game on $n-p+1$ cups and anchoring the other
    $p-1$ cups. For $p>n-\sqrt{n}$ you build the anchor set, adding $n-p$ cups
    to it each time, to get $\log n - \log (n-p)$ backlog)\\
    Emptier: $O(\log n)$ (Bill's complicated paper, generalizes the inductive
    proof for single processor case using skewed averages)
  \item \textbf{Multi-Processor, oblivious filler: } \\
    Filler: $\Omega(\log\log n)$ (anchoring???, HYPOTHESIS: this is not tight!
    we should be able to get $\Omega(\log p + \log \log n)$
    $Pr[\text{Hypothesis is correct}] \approx 0.5$ ) \\
    Emptier: $O(\log\log n + \log p)$  
  \item \textbf{Variable-Processor, adaptive filler: } \\
    Filler: $\Omega(n^{1-\epsilon})$ for any constant $\epsilon >0$ (Amplification Lemma: $f'(p) =
    \frac{1}{2}\cdot(f(p/2) + f(p/4) + \ldots)$) \\
    Emptier: $O(n)$ (this is a conjecture, no legit thoughts on how to prove it yet)
  \item \textbf{Variable-Processor, oblivious filler: } \\
    Filler: $\Omega(\log p + \log \log n)$ (using the superpower you can get
    $\Theta(p)$ cups with known constant fill in them. Recursing on these $\log
    p$ times gives $\log p$ backlog, and we already knew $\Omega(\log\log n)$)
    CONJECTURE: we can probably do better: $\Omega(2^{\sqrt{\log n} /2})$ in
    $\text{poly}(n)$ moves (via careful use of Chernoff bounds)\\
    Emptier: Probably $O(2^{\sqrt{\log n} /2})$ is true (how to prove it? maybe not worth it...)
  
\end{itemize}

\textbf{Current goals:}
\begin{itemize}
  \item Make upper bound and lower bound agree for multi-processor cup game
    with oblivious opponent (i.e. randomized)
  \item Prove better bounds on variable-processor cup game against oblivious
    and adaptive fillers
\end{itemize}

\end{document}

